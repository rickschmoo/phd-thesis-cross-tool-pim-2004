%%%%%%%%%%%%%%%%%%%%%%%%%%%%%%%%%%%%%%%%%%%%%%%%
% CHAPTER 6 -- MAIN STUDY
% RESULTS FROM WM EVAL
% File: tex/main-study-chapter/chapter6-main-study-RESULTS-WM.tex
%%%%%%%%%%%%%%%%%%%%%%%%%%%%%%%%%%%%%%%%%%%%%%%%%%%%%%%%%%%%%%%%%%%
%%%%%%%%%%%%%%%%%%%%%%%%%%%%%%%%%%%%%%%%%%%%%%%%%%%%%%%%%%%%%%%%%%%%%%%%%%%%%%%%%%%%%%%%%%
% Richard Boardman PhD Thesis: Improving Tool Support for Personal Information Management
%%%%%%%%%%%%%%%%%%%%%%%%%%%%%%%%%%%%%%%%%%%%%%%%%%%%%%%%%%%%%%%%%%%%%%%%%%%%%%%%%%%%%%%%%%
%%%%%%%%%%%%%%%%%%%%%%%%%%%%%%%%%%%%%%%%%%%%%%%%%%%%%%%%%%%%%%%%%%%%%%%%%%%%%%%%%%%%%%%%%%
% NATBIB NOTES
%%%%%%%%%%%%%%%%%%%
%\citet{jon90}                ->    Jones et al. (1990) 
%   \citet[chap.~2]{jon90}       ->    Jones et al. (1990, chap. 2)
%   \citep{jon90}                ->    (Jones et al., 1990) 
%   \citep[chap.~2]{jon90}       ->    (Jones et al., 1990, chap. 2) 
%%%%%%%%%%%%%%%%%%%%%%%%%%%%%%%%%%%%%%%%%%%%%%%%%%%%%%%%%%%%%%%%%%%%%%%%%%%%%%%%%%%%%%%%%%
%%%%%%%%%%%%%%%%%%%%%%%%%%%%%%%%%%%%%%%%%%%%%%%%%%%%%%%%%%%%%%%%%%%%%
% THINK
% Where are "`Other qualitative insights"' into PIM are reported.
% WHERE: lack of reflection with respect to PIM
%%%%%%%%%%%%%%%%%%%%%%%%%%%%%%%%%%%%%%%%%%%%%%%%%%%%%%%%%%%%%%%%%%%%%
%%%%%%%%%%%%%%%%%%%%%
% RESULTS ISSUES
%%%%%%%%%%%%%%%%%%%%%
% An illuminating set of results:
% THINK: some "study" results bridge the two Phases 1 and 2 (how to indicate?)
% how best to present results? - as case studies?}, , life events
% Highlight results that are new and those that merely confirm previous results, do they confirm/backup/agree/echo/support or contradict/clash/differ}
% THINK: note objective measures/subjective feedback below}
%	\item \textbf{A - Slice results by users}
%	%%%%%%%%%%%%%%%%%%%%%%%%%%%%%%%%%%%%%%%%%%%%
%	\begin{itemize}
%		\item A Cross-tool profiling of users - slice by user type
%		\item Extent of organization (as previous study)
%	\end{itemize}
%	\item \textbf{B - Slice results by information type/tool}
%	%%%%%%%%%%%%%%%%%%%%%%%%%%%%%%%%%%%%%%%%%%%%%%%%%%%%%%%%%%%%%
%	\begin{itemize}
%		\item Unimportance of bookmarks
%		\item Compare core objective stats - size, structure, growth etc.
%		\item User attitudes - Importance, control etc.
%		\item Comparing PIM practices in the three tools, strategies used in different aspects of PIM
%		\begin{itemize}
%			\item Organization: classication style/level of abstraction
%			\item Frequency of various activities
%		\end{itemize}
%		\item Comparing core characteristics of each information type table
%		\item Implications for unification - documents and stored email most similar (MOVE TO DISCUSSION?)
%	\end{itemize}
%	\item \textbf{D --PIM bugbears}
%	%%%%%%%%%%%%%%%%%%%%%%%%%%%%%%%%%
%		\item Break-down by tool/aspect of PIM
%		\item Long terms effects of problems
%		\item Severity increased by repetition?
%	\item \textbf{E -- Slice by production activity}
%	%%%%%%%%%%%%%%%%%%%%%%%%%%%%%%%%%%%%%%%%%%%%%%%%%%%


%%%%%%%%%%%%%%%%%%%%%%%%%%%%%%%%%%%%%%%%%%%%
%%%%%%%%%%%%%%%%%%%%%%%%%%%%%%%%%%%%%%%%%%%%
\newpage
\section{Overview of Results}
\label{main-study:results:overview}
%%%%%%%%%%%%%%%%%%%%%%%%%%%%%%%%%%%%%%%%%%%%
%%%%%%%%%%%%%%%%%%%%%%%%%%%%%%%%%%%%%%%%%%%%

%%%%%%%%%%%%%%%%%%%%%%%%%%%%%%%%%%%%%%%%%%%%%%%%%%
% Relate to process overview in figure above
%%%%%%%%%%%%%%%%%%%%%%%%%%%%%%%%%%%%%%%%%%%%%%%%%%
A sample of the qualitative data collected for each participant is shown in \textbf{Appendix~\ref{chap:appendices--study-data}} on page~\pageref{chap:appendices--study-data:main-study}. The results from the main study are divided into four sections as follows:
\begin{enumerate}

\item \textbf{Section~\ref{main-study:profiles}} provides a brief overview of participants' profiles.

\item Usage of WM varied dramatically between participants, along with their level of PIM activity. Therefore, \textbf{Section~\ref{main-study:case-studies}}, presents a case study summary of each participant. 

\item \textbf{Section~\ref{main-study:wm-analysis}} focuses on the evaluation component of the study.  Usage of WM is reported, along with participants' positive and negative feedback on the design.

\item Finally, \textbf{Section~\ref{main-study:longitudinal}} reports other findings, not directly related to the WM evaluation, such as changes in organizing strategy.
\end{enumerate}



%%%%%%%%%%%%%%%%%%%%%%%%
% Initial profiling
%%%%%%%%%%%%%%%%%%%%%%%%
%%%%%%%%%%%%%%%%%%%%%%%%%%%%%%%%%%%%%%%
\subsection{Participant Profiles}
\label{main-study:profiles}
%%%%%%%%%%%%%%%%%%%%%%%%%%%%%%%%%%%%%%%
% Probably don't need to go into massive detail here - just state when major difference from ES
% Effectively a repeat of exploratory study.  Cross-tool profiling etc.

%%%%%%%%%%%%%%%%%%%%%%%%%%%%%%%%
% Tool-specific strategies
%%%%%%%%%%%%%%%%%%%%%%%%%%%%%%%%
% The cross-tool profiling reported in  \textbf{Chapter~\ref{chapter:exploratory_study}} was carried out in an earlier stage in the study.

All eight participants actively collected files, email and bookmarks.  As in the exploratory study, behaviour varied between participants, and between tools for individual participants.  A brief summary is presented as follows.  All were active collectors and organizers of document files (average: 94 folders).  In email, seven participants had large inboxes. Three of these combined a large inbox with extensive filing (F1, F2 and F4).  Participant M6 pursued a frequent-filer strategy.  The remaining four participants, F3, M5, M7, M8, had large inboxes and filed only a few messages everyday.  On average, participants had  22 email folders.  Bookmarks were indicated to be of low importance by seven of the eight participants who collected them at a very low rate (average 15 folders).  There was one exception, participant F4, who collected bookmarks extensively, distorting the overall average total of bookmark folders up to 38.
%%%%%%%%%%%%%%%%%%%%%%%%%%
% Cross-tool profiles
%%%%%%%%%%%%%%%%%%%%%%%%%%
Cross-tool profiles were calculated using the method outlined in \textbf{Section~\ref{exp-study:Results-cross-tool-profiling}}: CT1 (3 participants), CT2 (2 participants), and CT3 (3 participants).  They are reported along with profile details in \textbf{Table~\ref{table:main-study:participants}}.
% Questionnaire responses to the extent to which they relied on folders in each tool supported the above.  Average responses were 5.3, 4.1, and 3.1 for files, email and bookmarks respectively.

%%%%%%%%%%%%%%%%%%
% Nature of work
%%%%%%%%%%%%%%%%%%
% ADD: nature of work (M6 as odd one out)
Seven of the eight participants were involved in long-term research, teaching, and administrative activities, reflecting their academic careers.  The one exception was participant M6.  He was as an undergraduate with a highly structured set of short-term activities corresponding to his degree courses . As in the exploratory study, all participants proved to be very open.  None of them restricted access to their personal collections.


%%%%%%%%%%%%%%%%%%%%%%%%%%%%%%%%%%
% Other profile discussion?
%%%%%%%%%%%%%%%%%%%%%%%%%%%%%%%%%%
% M6, M7 very confident -- different reason
% ADD: PIM level of confidence
% ADD: views on integration from questionnaire

%%%%%%%%%%%%%
% openness
%%%%%%%%%%%%%
% Compare with other researchers' experiences~\cite{Bellotti:00,Whittaker-email:96}
% Individual differences? (see below)












%%%%%%%%%%%%%%%%%%%%%%%%%%%%%%%%%%%%%%%%%%%%
%%%%%%%%%%%%%%%%%%%%%%%%%%%%%%%%%%%%%%%%%%%%
\newpage
\section{Participant Case Studies}
\label{main-study:case-studies}
%%%%%%%%%%%%%%%%%%%%%%%%%%%%%%%%%%%%%%%%%%%%
%%%%%%%%%%%%%%%%%%%%%%%%%%%%%%%%%%%%%%%%%%%%
% Need to mention installation issues?

As behaviour varied enormously between the eight participants, case study summaries of each are presented as follows. Each case study summarizes usage of WM, major events (e.g. OS reinstalls), and changes in PIM strategy.  % An ove

% The section is intended to provide a summary only, and more detail is provided in subsequent sections.  

% Changes in PIM strategy are discussed in more detail in \textbf{Section~\ref{main-study:longitudinal}}.

%%%%%%%%%%%%%%%%%%%%%%%%%%%%%%%%%%%%%%%%%%%%%%
% Discuss individual differences in depth?
%%%%%%%%%%%%%%%%%%%%%%%%%%%%%%%%%%%%%%%%%%%%%%
% variety of usage, "life events", core strategies (HL groups)
% Hard to generalize across users
% What I wanted: naturally occuring variations in use. Random events definitely prove that this was real



%%%%%%%%%%%%%%%%%%%%%%%
% Participant 6.1
%%%%%%%%%%%%%%%%%%%%%%%
\subsection*{Participant F1}
% No installation issues.
% Initially: positive
% cross-tool profile CT2 (F1, E2, B2)
% Ongoing incremental additions and structuring.
Participant F1, cross-tool profile CT2 (F1, E2, B2), was a male lecturer at Imperial College London (ICL).  Initially he was highly positive towards WM: \textit{``You only have to create a folder once ...  you create a [file] folder, you create a word document in that folder -- and the next thing you think oh I have to do a web search on the topic.  So you do a web search and you find some interesting websites, but you don't really have to think about where you're going to store those websites because you can put them in the folder that's already there''}.  However, although he ran WM for 57 days, he made relatively little use of it, only mirroring 4 folder creation events.  F1 explained that the low usage was due to the fact that he was changing university and had been away from his computer for long periods of time.  
% newly created folders between all three tools: 2 roles and 2 projects.
% (CHECK). Despite the relatively low use, 
After moving to his new job, he proceeded straight onto stage 6. However, in his closing interview he expressed a desire to continue using WM: \textit{``I would like to create equivalent folders in the file manager [to the `Eyegaze' email folder] but I haven't got time. I haven't got your program here''}.  In the process of moving, he transferred his files and email, but abandoned his bookmarks. This illustrates the relatively higher long-term value of those two collections.  





%%%%%%%%%%%%%%%%%%%%%%%
% Participant 6.2
%%%%%%%%%%%%%%%%%%%%%%%
\subsection*{Participant F2}
% No installation issues. 
% Participant 6.2 - ongoing incremental additions and structuring to all three collections.  
Participant F2, cross-tool profile CT1 (F1, E2, B1), was a male PhD student at ICL.  Out of all the participants, he made the most extensive use of WM -- mirroring 26 creation events over the 120 days he had WM installed.  Of these, 21 were related to a major group project called `BOD' (name changed).  His most common mirroring trajectory was files to email (18 events).   Of the 26 events, he identified 2 mistaken mirrorings, and 5 manual mirrorings (made when WM was not operating). He chose to carry on using WM after the closing interview for an additional 80 days.  His primary reason for using WM was to increase the consistency between different folder structures in support of his projects: \textit{``[WM] helps with project management. For example, I started the `BOD' project by creating a folder in email. Straight away, I was prompted whether I wanted it under documents as well. Therefore it was synchronized from the beginning rather than working in a hotchpotch''}.  He also stated that through the prompting, WM increased his reflection on PIM, resulting in an improved state of organization: \textit{``I think it's forced me to think about my workspace more, and in particular to synchronize between outlook and my files. Before I was not being methodical and I would lose track of things. Now things are more consolidated and structured''}.  Despite his take-up in mirroring, F2 did not make any significant changes in organizing strategy.  An extract from his WET is provided in \textbf{Table~\ref{table:main-study:wet-example}}.

%%%%%%%%%%%%%%%%%%
% OTHER QUOTES
%%%%%%%%%%%%%%%%%%
% Quite disorganized before - previously, I would create a one-off document folder, Now it helps me build a mental model of the mapping between different tools                                              % In an indirect way, forced me to think about why I was organizing my things the way I was doing. Trying to just manage project documents in DOCS/Project Documents area. Moved code elsewhere.            % May be more useful at the start of a new piece of work - or for people just starting out with a new computer, say new PhD students      
% Decisions of when/when not to mirror.
% Many design suggestions. 
% Examples of successful mirrorings, and ones used in the future.
% More folders in EM -> better



%%%%%%%%%%%%%%%%%%%%%%%
% Participant 6.3
%%%%%%%%%%%%%%%%%%%%%%%
\subsection*{Participant F3}
% very negative self-characterization
Participant F3, cross-tool profile CT3 (F1, E3, B2), was a male PhD student at ICL. He was highly organizing-neutral at the start of the study, \textit{``I'm lazy quite frankly. I don't naturally organize.  I'd have a pile as opposed to something filed.  And it's also partially due to the fact I have amazing spatial awareness really - I don't have to file things, I can remember [where they are]''}. Initially, he provided negative feedback with respect to WM, \textit{``Its just not my scene, you need someone else''}.

% Then, New Year reinvention
% Said study had greater influence than the tool
% Including move from My Docs (where WM is) to Desktop
However, 120 days into the study, he made a significant change in his file organizing strategy as part of a New Year resolution \textit{``to be more organized''}. In summary, he reorganized his files by moving his active documents to the desktop and creating a number of new folders there.  His change in strategy is discussed in detail in \textbf{Section~\ref{main-study:results:changes}}.
% Manual mirroring as dual file areas.
% Installation issues: dual docs (NO: after the move)
At that point he also started \textit{manually mirroring} new folders, between the Desktop and email (4 events).  He also used WM to mirror 5 folder creation events between ``My Documents'' and bookmarks.  Participant F3 thus provide a key example of emergent PIM behaviour which would not be clear from a one-off lab study. 

In the closing interview, he suggested that WM may be appropriate for someone setting up folder structures for the first time: \textit{``I'd say you get you'd get some seriously different results if you installed WM for someone with a brand new computer ... I think you'd get a very different dynamic, and you my might even get a completely different usage out of the same person. You installed it after `Downloaded Papers' was made in `My Documents'. If the mirror had opened up when I'd done that I would have said `yes stick it in the Favorites as well' ... but most of my pre-organization had already occurred by then, reasonably quickly after purchasing the computer''}.

% CHANGE, STUDY >> TOOL F3: F3: Do you know what ... the biggest... I don't know...  Overall the tool hasn't done that much, its more the conversations that me and you have. Its weird because I've become much more aware of all of my directory structures - there's a bit more thought trying to go into it and how its organized and stuff - but I think that's probably because of the conversations we have, i.e. being part of the user study than because of the tool.



%%%%%%%%%%%%%%%%%%%%%%%
% Participant 6.4
%%%%%%%%%%%%%%%%%%%%%%%
\subsection*{Participant F4}

% PROFILE
% Abandoned ``sub-collections'' of bookmarks.
Participant F4 was a female PhD student at ICL, with cross-tool profile CT1 (F2, E2, B1).  In stage 1, she complained in depth about not having enough time for PIM, and as a result feeling disorganized.  Despite this lack of time, she was the only participant to collect and structure bookmarks extensively. However, 70\% of her 250 bookmark folders had ``failed'' in terms of containing 2 items or fewer).  She had also restarted her bookmark collection three times, on each occasion storing the old collection in a ``\texttt{old Favourites 200x}'' sub-folder.

% WM installation
She provided positive initial feedback on WM, and made significant usage over the study -- mirroring 13 newly created folders, 8 of which were from bookmarks to files and email.  In particular, she welcomed the increase in consistency between folder structures: \textit{``Now its more consistent - and easier to remember. Good for tracking, makes it easier to remember where things are. It forces/reminds you to organize. When I've got a deadline, I save things anywhere. I plan to organize at a later date but never get round to it''}.
% F4: I did find it useful. Very useful. It does help me.
%F4: Good for keeping things consistent. Reminds me about parallel structures in email and file system   
%F4: It forces/reminds you to organize. When I've got a deadline, I save things anywhere. I plan to organize at a later date but never get round to it. Then maybe I have hassles later finding it... now the folder exists from somewhere else.
Due to the incompatibility of WM with her email client, MS-Outlook Express, she did not mirror to email, but reported that she would have found it useful to do so for many of the folders which related to conference submissions.
% , \textit{``I would mirror these [conference submissions] folders will involve email''}.
% Provide examples of missed mirroring?
In the closing interview, she also reported missing 2 potentially useful mirroring opportunities due to lack of time and dismissing the WM dialog box.

One limitation of WM that she identified was its focus on one area of the file system, and she requested WM support for mirroring between multiple file areas.  However, she also observed that the focus on one area had a key benefit: \textit{``It's good to have everything centralized into one directory. Before I had a problem remembering where to find things. Now it's more consistent - and easier to remember.  Previously I'd create a separate folder for each project or conference - but I'd put it anywhere. Mainly on my D-drive but it could be anywhere''}.





%%%%%%%%%%%%%%%%%%%%%%%
% Participant 6.5
%%%%%%%%%%%%%%%%%%%%%%%
\subsection*{Participant M5}
% large inbox, but frequent filer of some info
% QUOTE: irrationality of PIM
% Major reorganization of files and email 

Participant M5, cross-tool profile CT2 (F2, E2, B2), was a male PhD student at University College London (UCL).  He was one of two participants to make a change in organizing strategy, and cited the study participation as a key factor in reorganizing his files and email.  This change in strategy is discussed in detail in \textbf{Section~\ref{main-study:results:changes}}.
% CHANGE      
% [THIS IS MORE STUDY] M5: One definite effect: is that I have more folders in my email now. Before I had this huge inbox basically. So yeah - my email is definitely more structured as a result. It more or less introduced the concept of folders to my thinking about email organization.        
% SMALL CHANGE 
% M5: the folders that I've created, they only take up really minimal, you know - it might be smaller by 2% or 4% - and does that make a difference? Yes it does because most of the stuff that comes in is day-to-day stuff I deal with today or 2 days later - and the things I extract now are things with a longer due time.

% Two installation issues:
% email (changed) and network drive issues. Changed email tool.
% Network drive: major impact, inhibiting.
He encountered two installation issues with WM.  Firstly, his email tool was Outlook Express which  was incompatible with WM. However, as he had planned before the study, he switched to Outlook before installing WM. Secondly, he stored his personal files on a network drive which was incompatible with WM.  This had an impact on his usage of WM: ``\textit{The tool could not prompt me to mirror anything in Outlook because the only thing it could have prompted me for was the creation of bookmarks, which I don't use!  The ones that could have been useful it couldn't prompt me for because they were in the `H:' drive}''.  
% FILE: SPACE LIMITATION
% Ran out of space in document files - only user to remove stuff from document file collection.
%CHANGE Came round to WM at the end. Good quote to use.
%Another example of emergent behaviour
Also, he was initially negative towards WM, arguing that he made relatively little use of folders in email and bookmarks.
However, over the course of the study he \textit{manually} mirrored 4 folders between files and email, and was more positive by the closing interview: ``\textit{Quite impressively, the idea of mirroring structures in emails and in the folders didn't really cross my mind. I think before I had similarly named folders in both systems but they were never agreed in any way. And now I think the structures are a bit more coherent}''.  This represents a second example of emergent behaviour over time. % for why PIM-tools must be evaluated over the long-term. 

% PROMPTIONG UNANTICI
He also acknowledged the increase in reflection offered by WM:  \textit{``the act of being prompted ... is useful because it reminds you that you should think about how you should organize your stuff. But its also someone saying ``clean up your room''}. However, he interpreted this as a hint to do less PIM: \textit{``When I get the WM message up it reminds me that I'm wasting time with information management, creating folders and putting things in folders!''}.

% Participant M5 made a number of design suggestions included the need for more selective prompting, based on top-level folders only (see \textbf{Section~\ref{main-study:wm-analysis}}).
% PROMPTIONG ANNOYING
%M5: If prompting had worked - it might have increased my rating here but it might have also created my annoyance rating because I create folders so many times more in the file space,                 
%M5: Should not always prompt me, as I said before - it is not necessary to always replicate - it should prompt me in sensible cases. Maybe in higher-level folders, and not in lower level folders.                        



         







%%%%%%%%%%%%%%%%%%%%%%%
% Participant 6.6
\subsection*{Participant M6}
%%%%%%%%%%%%%%%%%%%%%%%

%%%%%%%%%%%%%
% profile
%%%%%%%%%%%%%
Participant M6, cross-tool profile CT1 (F1, E1, B1), was a male MSc. student at ICL.
% change
% No change over course of the study -- very settled. Archiving of email leads to net reduction in collection size. SETTLED.
% His PIM strategies did not change over the course of the study.
% wm
%Installation issues: none. Ideal pro-organizing profile (CT1) but highly negative of WM.
%Good quotes why.
He was initially negative towards WM.  % 
%%%%%%%%%%%%%%%%%%%%%
% bm-files partial
%%%%%%%%%%%%%%%%%%%%%
Although he was highly organized in all three tools, he considered his organizational needs to differ between them.  His files were extensively structured, but although he organized his email extensively, he only did so in a few top-level folders.  However, he observed a partial overlap between files and bookmarks: \textit{``Bookmarks and `My Documents' are very much more related to each other.  If you say in Bookmarks, \texttt{`my phd work'} and then underneath you have \texttt{`centra-fusion'} and \texttt{`cog-robotics'} and \texttt{research-labs} ... the first two you would have in files as well, but the third thing you probably wouldn't as you wouldn't put files in there''}. During the study, he only mirrored one folder creation, \texttt{Loreal}, from email to files, despite having no installation problems.

% he organized his email in a few top-level folders, whilst his files were organized in an extensive, deeply-nested folder structure.
%%%%%%%%%%%%%%%%%%%%%%%%%%%%%%%%%%%%%%%%%%%%%%%%%%%%%%%%%%%%%%%%
% Design suggestions. Email to files only (top-level only)
%%%%%%%%%%%%%%%%%%%%%%%%%%%%%%%%%%%%%%%%%%%%%%%%%%%%%%%%%%%%%%%%
% M6: So it does at the top-level maybe - yeah? The very top-level it does maybe, so if I have a folder phd maybe. Actually it goes one-way. If I create a folder in email there's likely to be a folder, a top-level folder in My Documents and in bookmarks relating to that which then may have sub-levels, or is very likely to have sublevels            
%F6: . Actually it goes one-way. If I create a folder in email there's likely to be a folder, a top-level folder in My Documents and in bookmarks relating to that
%F6 The very top-level it does maybe, so if I have a folder phd maybe. Actually it goes one-way. If I create a folder in email there's likely to be a folder, a top-level folder in My Documents and in bookmarks relating to that which then may have sub-levels, or is very likely to have sublevels.    
%F6: In my opinion it is "one to many", or like "one to a folder structure" really. You know ... its always email (one thing) links to one thing in my documents which does not necessarily have to be top-level, it could be above or could be in my documents directly but its just more likely to have another branches underneath.    
% So you know - they are more closely associated because you just don't do that many levels in email or I don't anyway.
At the end of the study, he suggested that mirroring should be one-way from email to files, for top-level folders only: \textit{``In my opinion it is `one to many', or like `one to a folder structure' really. You know ... its always email (one thing) links to one thing in my documents which does not necessarily have to be top-level, it could be above or could be in my documents directly but it's just more likely to have other branches underneath''}.  Again, this change in opinion regarding WM highlights the benefits of evaluating over the long-term.  M6 made no change in strategy over the course of the study.

% Speculation
% M6: highly-structured set of short-term activities in DOCS< little need to mirror across tools

% Prompts annoying
%M6: Because it kept on popping up and asking me the same silly question. I've had more annoying things - like my computer giving up ... WorkspaceMirror doesn't really irritate me. If I don't want it I just say cancel or just take the ticks out and click OK. Which I did at the beginning but then I realised I could just press cancel which was quicker. So you know I still am in control.Well its not really a problem since its just one cancel. Compared to what Word normally does to you, or Office or Windows - all kinds of harassing boxes - its negligible!  At least it never asked me to contact my system administrator!

%integ dangers
% M6: I would never give it to anyone who wasn't experienced at all ... because again it does a lot for you ... and most people, if they don't know what to do they just say "yes"! So if my mum was using it I would come home in the holidays and there would be 5000  folders, all empty, all her email lying in her inbox - you know what I mean ... You need someone who's at the stage of using a folder structure and understands folders and then yes.  
% F6: don't see the point of mixing the two except for confusing people like my mum, and then annoying me because I have to remind her that she's using the wrong piece of software. Anyone who knows how to use a computer won't use IE for organizing their files. The people who don't know enough get confused because they then don't see a difference between the internet and their own files (also see novice quote).

% flex
%M6: you might have to say "I don't want to create a folder, because that one already exists" ...
%
%Need for more mirroring flexibility/power #1: map to arb existing folders
%M6: Because you might start a project and not have an email folder but you receive so many emails on it that you create an email folder for it later ...
%
%Need for more mirroring flexibility/power #2: or map to new folders in any location
%M6: It would nice if you could browse and say "I want to create the folder at that level". You know a button "browse" where you can select where you want to create a new one. The default being "My Documents" because that'll be the case most of the time.



%%%%%%%%%%%%%%%%%%%%%%%
% Participant 6.7
\subsection*{Participant M7}
%%%%%%%%%%%%%%%%%%%%%%%

%%%%%%%%%%%%
% profile
%%%%%%%%%%%%
% ("ad hoc" organizer) - very settled. SETTLED
% separates created and downloaded
% org-neutral
Participant M7, cross-tool profile CT3 (F2, E3, B3), was a male PhD student at UCL.  He was \textit{organizing-neutral} in all three PIM-tools, and considered filing to be low priority.

% Initially negative - likes the principle but wants more power, But negative anyway
He responded unfavourably to WM from the start, expressing a requirement for a stronger form of integration between the tools: \textit{``My guess is I won't like it. I want something I can use in every tool. I want a tool with everything in it, so this isn't enough to warrant my adoption. I don't need it, as I don't organize with folders particularly. I mean I like the concept, the idea of unifying these working differences. I do think that's necessary and useful but I would want that next step - Version 2 or Version 10 - the complete new file-web and email manager - that would be interesting}''. He also foresaw that empty folders resulting from mirroring would cause a problem for him: \textit{``I'd like to create a folder that would be accessible from all three, rather than just an extra empty folder in the other two so I wouldn't get this massive bloat ... the problem with the mirroring is the empty folders ... and that to me would be annoying and would impact how I work}''.
% F7: Sharing I'd be a lot happier with, because it wouldn't impact at all the way I work currently ... so I'd create a folder but it would be accessible from all three - rather than just an extra empty folder in the other two.  So I wouldn't get this massive bloat ... the problem with the mirroring and the way I work is the lots of empty folders ... and that to me would be annoying and would impact how I work.
% M7: Well sharing means that the different objects (well files and email and bookmarks) actually share the same structure. Whereas mirroring means that the actual folder structure is copied but the actual contents of the folders in the bookmarks isn't visible from the email. For me mirroring is not really useful. Sharing I'd be a lot happier with, because it wouldn't impact at all the way I work currently ... so I'd create a folder but it would be accessible from all three - rather than just an extra empty folder in the other two.
% M7: Personally I don't see the need for mirroring ...I do see the need for sharing but mirroring ... because I don't organize things in a structure. I'm fairly haphazard ... if I want to put a folder there and not really bother .... but I don't specifically don't have to have it in my email management because  I don't have to cope with whatever I'm looking at on the web to be mirrored in my mail. For me mirroring is not really useful

% Therefore hard to evaluate.
% Multiple machines
%Logistical problems
%Also not used as little foldering. 
Despite not having any installation problems,  he made no use of WM during the trial: \textit{``There's been no chance - mainly because it hasn't actually done anything as I haven't actually created any folders. Also, even if I did, I haven't felt the need to mirror them''}. This illustrates the logistical challenge of evaluating PIM-tools via a field trial, the experimenter can not force the participant to use the designed functionality.

In the closing interview, although he had not used WM, he indicated that certain users may find it useful: \textit{``if they used the same kind of folder structures in the different applications ... I might recommend it because I'd see it as being useful for them conceptually to mirror the folder structures in the other environments}''.
% other users
% M7: I might do - especially if they used the same kind of folder structures in the different applications. If they did - I might recommend it because I'd see it as being useful for them conceptually to mirror the folder structures in the other environments. If there was someone who didn't use folders, or didn't work like that - I wouldn't bother.



%%%%%%%%%%%%%%%%%%%%%%%
% Participant 6.8
\subsection*{Participant M8}
%%%%%%%%%%%%%%%%%%%%%%%

Participant M8, cross-tool profile CT3 (F2, E3, B3), was a male research assistant at UCL.  He used the metaphor of a house to describe the different levels of organizing he employed in the three tools: ``\textit{my living room [file space] is really tidy because I've been making the effort to tidy it up. The rest of the house [emails and bookmarks] is a tip because everything I need is in the living room now ... Only certain elements are tidy. Other elements are rotting!}''

% wm
%Installation issues: none. Initially positive, but no usage -- although anticipated in different circumstances.
%Observes hurdle: major investment needed to rearrange document files. Bursty PIM quote
Although he encountered no installation problems with WM, and was initially positive, he made no use of it over stage 4. However, he remained positive about the tool: \textit{``Partially [I haven't used it] because I've been working on another machine.  And partially because I tend to have a structure which I put things in, keep that for a period of time, then have a spring-clean and change the structure. And I haven't been through the process of changing the structure with WorkspaceMirror running.  Its something that I'll do eventually, but I just haven't done my reorganization recently''}.
% M8: (DOCS/EM) well the use case here would be that there would be a document for a particular project or experiment or study in that case and I could just look at all my emails that relate just specifically to this project. I mean, its a different way of doing it to at the moment. At the moment I've got a HIGHERVIEW group that does contain anything about that project - so it would be a finer grain of categorization to organize my emails at that level which might be quite useful in terms of finding things again.
% M8: (DOCS/BM) Same reason for bookmarks - so again the documents are really the primary objects if you like, that need organizing, and then I'd have bookmarks relating to the studies that I was doing   - references, PDFs etc.
% M8: (EM/BM) For that I think of once case where I think it might: say for "web purchases" I might have a "shops" folder [in bookmarks] with links to shops that I've bought things from.
% M8: And it will homogenize that across my PDFs from the web, emails about a particular project, and files about a particular project. I could do that anyway without WorkspaceMirror but I don't ... Why not? Never really thought about it


Participant M8 made a number of design recommendations including the need to limit mirroring to top-level project folders: \textit{``once you've mirrored there [at the top-level] - you might not want to mirror it further down. I think it would work at that level. Here are the projects I'm working on.  Here are the emails about that project. And here are web links related to the project. That makes sense to me''}.

%One of the few participants to archive
%Archiving of email leads to net reduction in collection size. SETTLED



%%%%%%%%%%%%%%%%%%
%%%%%%%%%%%%%%%%%%
% ENDING
%%%%%%%%%%%%%%%%%%
%%%%%%%%%%%%%%%%%%
% Classify user behaviour, case studies to illustrate. 
% ADD Summary of section.
% Nominal clustering of participants based on their usage of WM?
% \textbf{Section~\ref{main-study:wm-analysis}} collates findings relating to the WM evaluation.




\newpage
%%%%%%%%%%%%%%%%%%%%%%%%%%%%%%%%%%%%%%%%%%%%%%
%%%%%%%%%%%%%%%%%%%%%%%%%%%%%%%%%%%%%%%%%%%%%%
\section{WM Evaluation}
% Alt titles:
% functional analysis a la Dumais et al 2001
% Qualitative analysis of results (but supported by objective data)
\label{main-study:wm-analysis}
%%%%%%%%%%%%%%%%%%%%%%%%%%%%%%%%%%%%%%%%%%%%%%
%%%%%%%%%%%%%%%%%%%%%%%%%%%%%%%%%%%%%%%%%%%%%%
%%%%%%%%%%%%%%%%%%%%%%%%%%%%%%%%%%%%%%%%%%%%%%
% THINK: anticipated/unanticipated results
% THINK: case studies
% THINK: relate to user types
%%%%%%%%%%%%%%%%%%%%%%%%%%%%%%%%%%%%%%%%%%%%%%

%%%%%%%%%%%%%%
% lead in
%%%%%%%%%%%%%%
The above case studies surveyed each participant's usage of WM in turn.  This section focuses in depth on the results from the WM evaluation.
%%%%%%%%%%%%%%%%%%%%%%%%%%%%%
% Structure of this section
%%%%%%%%%%%%%%%%%%%%%%%%%%%%%
% and also reports objective data from the With-WM tracking in stage 4. 
Firstly, \textbf{Section~\ref{main-study:results:themes-usage}} summarizes the usage of WM based on the objective data collected in stage 4. Then, \textbf{Section~\ref{main-study:results:themes-feedback}} presents qualitative feedback regarding WM, both positive and negative. Finally, \textbf{Section~\ref{main-study:results:themes-design-recs}} reports the design recommendations suggested by participants.
% Finally, \textbf{Section~\ref{main-study:results:themes-influences}} considers the factors which influenced the take-up of WM.


%%%%%%%%%%%%%%%%%%%%%%%%%%%%%%%%%%%
\subsection{Use of WM}
\label{main-study:results:themes-usage}
%%%%%%%%%%%%%%%%%%%%%%%%%%%%%%%%%%%
% MERGE WITH ABOVE
%%%%%%%%%%%%%%%%%
% Usage of WM
%%%%%%%%%%%%%%%%%
% Number of mirrors (as % of total events)
% Average depth
%%%%%%%%%%%%%%%%%%%%%%%%%%%%%%%%%%%%%%%%%%%%%%%%%%%%%%%%%%%%%%%%
% ADD: increased reliance on filing/folders at end of study?
% 6.2 - docs and email
% 6.4 documents
% M5 - email
%%%%%%%%%%%%%%%%%%%%%%%%%%%%%%%%%%%%%%%%%%%%%%%%%%%%%%%%%%%%%%%%%
% questionnaire responses - was it useful
% Impact on folder overlap?
% Types (orgdims) of mirrored folders
%%%%%%%%%%%%%%%%%%
% USAGE VARIED
%%%%%%%%%%%%%%%%%%

This section provides an overview of the objective data recorded during stage 4.
%%%%%%%%%%%%%%%%%%%%%%%%%%%%%%%%%%%%%%%%%%%%%%%%%%%
% SUMMARY - AVERAGE: DOES NOT REFLECT VARIATION
%%%%%%%%%%%%%%%%%%%%%%%%%%%%%%%%%%%%%%%%%%%%%%%%%%%
% % As noted above, usage of WM varied significantly between the participants (see \textbf{Table~\ref{table:main-study:wm-usage}} for a summary).  
% As indicated by the case studies, WM usage varied significantly between the participants.
In total, 57 \textit{folder creation events} were mirrored (not including test mirrorings performed during the interim interview).  However, the usage of WM varied widely between participants as indicated in the case studies above.  \textbf{Table~\ref{table:main-study:wm-usage}} provides an overview on an individual basis. 

%%%%%%%%%%%%%%%%%
% Usage of WM
%%%%%%%%%%%%%%%%%
% Other aspects to report:
% Number of mirrors (as % of total events)
% Average depth
% Creates versus deletes and renames + questionnaire responses
% Impact on folder overlap?
% Types (orgdims) of mirrored folders
Six of the 8 participants mirrored folder creation events during stage 4:
\begin{itemize}

\item Two participants, F2 and F4 made extensive use of WM, mirroring 26 and 13 newly created folders respectively, over an average of 116 days.  For F2, 21 folders related to his `BOD' project, and the most common source tool was files (20 folder creation events).  The most common organizational dimensions were \textit{project} (12) and \textit{event} (5).

For participant F4, the most common organizational dimensions for mirrored folders were \textit{project} (4), and \textit{event} (6).  The event-based folders related to conferences she was considering submitting papers to.  In contrast with F2, her most common source tool was bookmarks (10 folder creation events).

\item Participant F1, despite being very positive towards WM, only mirrored 4 folder creation events (all \textit{role} or \textit{project}).  He mirrored at least one event from each PIM-tool.  He highlighted his job change as a primary cause of his low PIM activity and WM usage.  

\item Participant M5 \textit{manually} mirrored 4 folders from files to email, with organizational dimensions \textit{project} (3), and \textit{document type} (1). He did not use WM to mirror due to its incompatibility with his Samba-based network drive.

\item After his New Year reorganization, Participant F3 \textit{manually} mirrored 4 folder creation events (3 from the `Desktop' to email, and 1 from `Desktop' to bookmarks).  He employed manual mirroring as WM was not compatible with the `Desktop' area of the file system.  He also used WM to mirror 5 events from files to bookmarks.  His events were based on a wide range of organizational dimensions including \textit{format} (2) and \textit{role} (2).

\item Participant M6 mirrored 1 \textit{project} folder from email to files. 

\end{itemize}

%%%%%%%%%%%%%%%%%%%%%%%%%%%%%%%%%%%%%%%%%
% TABLE: WM USAGE
% TO ADD: mirror events over which time span
% TO CHECK: mirrors/day
% EACH TIME I PASTE IN: CHECK %
% missed/manual/mistaken
% ++ days for F2
%%%%%%%%%%%%%%%%%%%%%%%%%%%%%%%%%%%%%%%%%
\begin{sidewaystable}
\begin{center}
\begin{footnotesize}
\setlength{\extrarowheight}{2pt} % 7 columns
% Table generated by Excel2LaTeX from sheet '4 Usage of WM'
\begin{tabular}{|c|p{1.3cm}|p{1.2cm}|p{1.2cm}|p{1.6cm}|p{2cm}|p{2cm}|p{1.8cm}|p{2.0cm}|p{1.4cm}|p{1.0cm}|p{1.0cm}|}
\hline
  {\bf ID} & {\bf Cross-tool profile} & {\bf Initial attitude to WM} & {\bf Final attitude to WM} & {\bf Installation problems, see also Table~\ref{table:main-study:participants-problems}} & {\bf Level of WM usage} & {\bf Source folder events in stage 4 (creates, deletes, renames)} & {\bf Folder creation events mirrored in stage 4} & {\bf Source tools and trajectories of mirrored events} & {\bf \% of folder creation events mirrored} & {\bf WM usage (\# days)} & {\bf Mirrors per day} \\
\hline
        F1 & CT2 (F1, E2, B2) &   Positive &   Positive &            & Some use before change of job & Files (3, 0, 0), Email (2, 0, 0), BM (0, 0, 0) &          4 & Source: files 2, email 2 & Files 66\%, EM 100\% &         57 &       0.07 \\
\hline
        F2 & CT1 (F1, E2, B1) &   Positive &   Positive &            & Extensive use, ongoing after study & Files (42, 3, 1), Email (7, 2, 0), BM (4, 0, 1) & 26 (5 manual, 2 mistaken) & Source: files 20, email 5, BM 1. Most common traj: ``files to email'' (18) & Files 48\%, EM 71\%, BM 25\% &        120 &       0.22 \\
\hline
        F3 & CT3 (F1, E3, B2) &   Negative &    Neutral &      files & Some use of WM. Started manual mirroring towards end of stage 4 & Files (71, 0, 20), Email (1, 0, 0), BM (1, 0, 2) & 9 (4 manual, 1 mistaken) & Source: files. ``Files to BM'' (5). ``Desktop to BM'' (4) & Files 13\%, EM 0\%, BM 0\% &        143 &       0.06 \\
\hline
        F4 & CT1 (F2, E2, B1) &   Positive &   Positive &      email & Extensive use & Files (23, 0, 0), Email (0, 0, 0), BM (14, 0, 0) &         13 & Source: files 2, BM 11. Most common traj: ``BM to files and email'' (8) & Files 9\%,  BM 79\% &        111 &       0.12 \\
\hline
        M5 & CT2 (F2, E2, B2) &    Neutral &   Positive & files, email & Some mirroring towards end of stage 4 & Files (14, 1, 7), Email (5, 3, 12), BM (0, 0, 0) & 4 (all manual) & Source: from files, to email only & Files 29\%, EM 0\% &         65 &       0.06 \\
\hline
        M6 & CT1 (F1, E1, B1) &   Negative &    Neutral &            &        Low & Files (12, 0, 0), Email (1, 0, 0), BM (0, 0, 0) &          1 & Source: from email, to files only. & Files 0\%, EM 100\% &         18 &       0.06 \\
\hline
        M7 & CT3 (F2, E3, B3) &   Negative &   Negative &      files &       None & Files (0, 0, 0), Email (0, 0, 0), BM (0, 0, 0) &          0 &          - &          - &         16 &       0.00 \\
\hline
        M8 & CT3 (F2, E3, B3) &   Positive &   Positive &            &       None & Files (4, 0, 0), Email (0, 0, 0), BM (0, 0, 0) &          0 &          - & Files 0\% &         20 &       0.00 \\
\hline
{\bf Total} &            &            &            &            &            &            &         57 &            &            &        550 &            \\
\hline
{\bf Average} &          - &          - &          - &          - &          - &            &          7 &          - &            &      68.75 &       0.07 \\
\hline
\end{tabular}  
\end{footnotesize}
\caption{Summary of WM usage}
\label{table:main-study:wm-usage}
\end{center}
\end{sidewaystable}
\normalsize

%%%%%%%%%%%%%%%%%%%%%%%%%%%%%%%%%%%%%
% Creates versus deletes and renames
%%%%%%%%%%%%%%%%%%%%%%%%%%%%%%%%%%%%%
% During stage 4, participants mirrored an average of 8 folder creations.  
One immediate observation is that all mirrored events related to \textit{folder creations}.  No participants used WM to mirror a folder delete or folder rename event, except to test the WM functionality during stage 3.  As reported in \textbf{Chapter~\ref{chapter:design}}, WM was limited to mirroring renaming events within a particular folder.  Renames corresponding to moving a folder between parent folders were not supported by WM, and resulted in a warning message being displayed.  However, no users reported encountering this over the study.
% However, M3 mirrored 2 folder deletes and 4 folder renames \textit{manually}.
Possible reasons for the bias in favour of creations over deletes and renames are discussed in \textbf{Section~\ref{main-study:discussion:evaluation}}. The rest of this section focuses on the mirroring of folder creation events.


%%%%%%%%%%%%%%%%%%%%%
% DEPTH and ORG DIM
%%%%%%%%%%%%%%%%%%%%%
Mirrored folders were high up in most participants' folder structures.  The average depth across all participants was 1.83 (SD: 0.89)\footnote{This was biased in favour of participant F2 who mirrored a number of deeper folders related to his `BOD' project. Without participant F2, the average depth was 1.43 (SD: 0.57).}.  A wide range of organizational dimensions were encountered.  Across all participants, the most common dimensions for mirrored folders were \textit{project} (40\%) and \textit{event} (22\%).  These aggregate figures are biased towards those participants who mirrored more folders (F2 and F4).

%%%%%%%%%%%%%%%%%%%%%%%%%%%%%%%%%%%%%%%%%%%%%%%%%
% TRAJECTORY
% Need to define source and destination tools?
% Mixture of trajectories.
% Source/destination tools F2 (D-E), F1 (DEB), F3 (DB), F4 (BD)
%%%%%%%%%%%%%%%%%%%%%%%%%%%%%%%%%%%%%%%%%%%%%%%%%
% F2 mirrored mostly from files to email, stating that he rarely used bookmarks so it was not worth mirroring to there.   F3 and F4 mirrored between files and bookmarks.  Note that possible trajectories were limited by encountered incompatibilities between WM and participants' PIM-tools.  Participant F4 was unable to mirror to email due to the incompatibility of Outlook Express. In contrast, the 4 events mirrored by participant F1 were based on creation events in all three PIM-tools.  The feedback presented in the next section shows a wide range of opinions from participants in terms of the form mirroring should take.
As noted above, participants employed WM with a range of mirroring trajectories.  \textbf{Table~\ref{table:main-study:trajectories}} summarizes the mirroring trajectories that were observed.  Across all participants, the most common source was files (64\% of all events), followed by bookmarks (21\%), and email (14\%).  The most common trajectories were \textit{``files to email''} (45\% of all events), \textit{``bookmarks to files and email''} (16\% of all events), and \textit{``files to bookmarks''} (13\% of all events). 

%%%%%%%%%%%%%%%%%%%%%%%%%%%%%%%%%%%%%%
% %%%%%%%%%%%%%%%%%%%%%%%%%%%%%%%%%%%%
% FIGURE - Sample from WET % Or a table?
% %%%%%%%%%%%%%%%%%%%%%%%%%%%%%%%%%%%%
%%%%%%%%%%%%%%%%%%%%%%%%%%%%%%%%%%%%%%
%%%%%%%%%%%%%%%%%%%%%%%%%%%%%%%%%%%%%%%%%
% TABLE OF PARTICIPANTS IN MAIN STUDY
%%%%%%%%%%%%%%%%%%%%%%%%%%%%%%%%%%%%%%%%%
\begin{table}[hbtp]
\begin{center}
\begin{footnotesize}
\setlength{\extrarowheight}{2pt}
\begin{tabular}{|p{1.1cm}|p{1.4cm}|p{1.5cm}|p{1.5cm}|p{1.5cm}|p{1.5cm}|p{1.5cm}|p{1.5cm}|}
\hline
{\bf Source PIM-tool} & {\bf \# mirrored creation events} & {\bf Destination: files} & {\bf Destination: email } & {\bf Destination: BM} & {\bf Destination: files and email} & {\bf Destination: files and BM} & {\bf Destination: email and BM} \\
\hline
{\bf Files} &         36 &          - &         25 &          7 &          - &          - &          4 \\
\hline
{\bf Email} &          8 &          3 &          - &          0 &          - &          5 &          - \\
\hline
  {\bf BM} &         12 &          2 &          1 &          - &          9 &          - &          - \\
\hline
\end{tabular}  
\end{footnotesize}
\caption{Trajectories of mirrored folder creation events (all participants)}
\label{table:main-study:trajectories}
\end{center}
\end{table}
\normalsize

% PERCENTAGE OF FOLDER EVENTS MIRRORED  \textbf{Section~\ref{main-study:results:growth}} reports the growth of the three collections over the length of the study.  
% For these five participants, the percentages of folder creation events mirrored was as follows. XX\% of file folder creations were mirrored.  The rates for email and bookmark folder creation events were YY\% and ZZ\% respectively.
% PERCENTAGE OF FOLDER EVENTS MIRRORED  \textbf{Section~\ref{main-study:results:growth}} reports the growth of the three collections over the length of the study.  
For the six participants who used WM, the percentage of total folder creation events that were mirrored varied substantially.  For example, participant F2 mirrored the most folder creation events, 26. Twenty were in files (equivalent to 48\% of all file-based creation events), 5 were in email (71\% of email creation events), and 1 was in bookmarks (25\% of BM creation events). \textbf{Table~\ref{table:main-study:wm-usage}} provides a detailed summary for each participant.% In contrast, participant M6 chose to not mirror any of the 12 file folders he created during stage 4. The one email folder that he mirrored was the only one he created.

%%%%%%%%%%%%%%%%%%%%%%%%%%%%%%
% PEOPLE WHO DIDN'T MIRROR
%%%%%%%%%%%%%%%%%%%%%%%%%%%%%%
The remaining two participants, M7 and M8, made no use of WM.  However, they both ran WM on their computers to test its robustness, and provided qualitative feedback.
%%%%%%%%%%%%%%%%%%%%%%%%%%%%%%%%%%%%%%%%%%%
% Difference between track 1 and track 2
%%%%%%%%%%%%%%%%%%%%%%%%%%%%%%%%%%%%%%%%%%%
% The difference is marked between the participants from \textit{track 1}, compared to those from the second track.
% Track 1 used WM for longer but NB: bugs and interrupted.
Overall, there was a marked difference between the track 1 and track 2 participants, who mirrored an average of 14 folders, and 1.5 folders respectively.  Possible reasons for this variation are discussed in \textbf{Section~\ref{main-study:discussion:evaluation}}.
% However, two of them (M6 and M8) found some aspects of the idea attractive.  % 
% The other four experimented with WM but did not use it in the long-term.  
% Possible reasons for this are discussed in \textbf{Section~\ref{main-study:wm-analysis}}.






















%%%%%%%%%%%%%%%%%%%%%%%%%%%%%%%%%%%
\subsection{Feedback from WM Users}
\label{main-study:results:themes-feedback}
%%%%%%%%%%%%%%%%%%%%%%%%%%%%%%%%%%%

%This section reports the benefits reported from those participants who employed WM., extracts common themes from the case studies, 

%%%%%%%%%%%%%%%%%%%%%%%%%%%%%%%%%%%%%%%%%%%%%
\subsubsection{Pros and Cons of Mirroring}
%%%%%%%%%%%%%%%%%%%%%%%%%%%%%%%%%%%%%%%%%%%%%

% Quotes from F1, F2 and F4. Even F3, F5, change.
% Others did not use it but acknowledged potential benefits (M8)
The core aim of the WM evaluation was to investigate whether the participants would use it to share folder structures between the PIM-tools.  A range of positive and negative feedback was received regarding the current design of WM.

The two heaviest users, F2 and F4, both observed and welcomed the increase in consistency, and described how it helped them manage information for a variety of projects (see case studies above). Both also suggested that mirroring lead to easier navigation, e.g. F2: \textit{``Its easier to navigate with a mirrored structure, compared to three different ones''}.  
%%%%%%%%%%%%%%%%%%%%%%%
% USERS WHO SWITCHED
%%%%%%%%%%%%%%%%%%%%%%%
Three participants (F1, F3, and M5) who performed more limited mirroring also acknowledged the benefits of an increase in consistency.  In addition, participant M8 who did not use WM, imagined that its use would offer benefits: ``\textit{It will homogenize that across my PDFs from the web, emails about a particular project, and files about a particular project. I could do that anyway without WorkspaceMirror but I don't ... Why not? Never really thought about it. The use case here would be that there would be a document for a particular project or experiment or study in that case and I could just look at all my emails that relate just specifically to this project. I mean, it's a different way of doing it to at the moment. At the moment I've got a `ProjectX' group that does not contain anything about that project.  So it would be a finer grain of categorization to organize my emails at that level which might be quite useful in terms of finding things again}''.
% M8: (DOCS/BM) Same reason for bookmarks - so again the documents are really the primary objects if you like, that need organizing, and then I'd have bookmarks relating to the studies that I was doing   - references, PDFs etc.
% M8: (EM/BM) For that I think of once case where I think it might: say for "web purchases" I might have a "shops" folder [in bookmarks] with links to shops that I've bought things from.


%%%%%%%%%%%%%%%%%%%%%%%%%%%%%
% BETWEEN WHICH TOOL
%%%%%%%%%%%%%%%%%%%%%%%%%%%%%
% FILES AND EMAIL								F5 F6 F7 F8
% FILES AND EMAIL AND BOOKMARKS	F1 F2 F3 F4 
In the closing study, 4 participants (F1, F2, F3, F4) indicated that mirroring was useful between all three tools, whilst 4 (F5, F6, F7, F8) indicated that it was most worthwhile between files and emails.  Note that all members of this second group placed little importance on mirroring to their bookmark collections.

%%%%%%%%%%%%%%%%%%%%%%%%%%%%%%%%%%%%%%%%%%
% THINK: benefits in different tools
% INCLUDE? Particular bonus in email.
%%%%%%%%%%%%%%%%%%%%%%%%%%%%%%%%%%%%%%%%%%
% \item  Tentative evidence of leveraging file investment across other tools? (F2)

%%%%%%%%%%%%%%%%%%%%%%%%%%%%%%%%%%%%%%%%%%%%%%%%%%
% However feedback: not 1:1 mapping (see below)
%%%%%%%%%%%%%%%%%%%%%%%%%%%%%%%%%%%%%%%%%%%%%%%%%%%
% However, all eight participants noted that there was not necessarily a one-to-one mapping between the folder structures in the different tools.  This is discussed below.
% In some cases a certain tool requires greater organizational granularity than another tool for a particular cross-tool activity.  
However, a general theme mentioned by all eight participants was that mirroring should not follow a one-to-one mapping between PIM-tools due to differing organizational requirements.  In many cases, participants had more complex organizing requirements in their file collections compared to email and bookmarks, e.g. F5: ``\textit{It doesn't make sense to create an email folder for every single publication so I just have a single submissions folder that goes across the publications and has, let's say,  at the moment maybe 30 entries or so. In my H-drive on the other hand, every single publication is a project  - and deserves its own folder because it consists of much more files than just the five emails. So that's an example where it makes sense that H-drive and email correspond but are not exactly mirrored}''.  

%%%%%%%%%%%%
% CLUTTER
%%%%%%%%%%%%
M7 felt that mirroring would result in clutter due to unused empty folders.  However, comments from several other participants suggested that they felt the benefits of increased consistency between folder structures outweighed the risk of clutter.  For example, F3 welcomed having folders mirrored and ready for use in the future, even if they were not used immediately, \textit{``I haven't written any latex so I haven't used it, but I'm keeping it.  I think it may be useful''}.  

%%%%%%%%%%%%%%%%%%%%%%%%%%%%%
% Top-level folders only
% Less power e.g. high-level categories most effective to share
% backed up by objective data? (F1, F2, F3, F4, M5, M6, M8)
%%%%%%%%%%%%%%%%%%%%%%%%%%%%%
Seven participants suggested that mirroring was particularly appropriate for \textit{top-level folders}, e.g. M8: \textit{``Images related to my project and all the substructure of that project ... it's very difficult to see why you'd want to mirror all that.  Once you've mirrored there [at the top-level] - you might not want to mirror it further down. I think it would work at that level. Here are the projects I'm working on.  Here are the emails about that project. And here are web links related to the project. That makes sense to me''}.
% M7: First impression good but not clear that it mirrored the whole structure
% Probably wouldn'tt mirror stuff that deep: experiments/study ....
% M6
% QUOTE:F3: Need for top-level only
% In a way My Documents - I could have a top-level one called Alfebitte which branched into presentations, deliverables, you know book chapters, experimental results, programming code - I could do that and that would have then mirrored to email... and probably bookmarks as well. But because... in a way because there was already a folder hierarchy partly made before I started using the FolderMirror... that might you know... yeah, because I just created the experiment results folder but if I'd been creating just an Alfebitte folder I would have mirrored it in all three. But specifically the one with results, there was no need to mirror it. But the top-level idea of the whole project, yes - I would have mirrored.
%%%%%%%%%%%%%%%%%%%%%%%%%%%%%%%%%%%%%%%%%%
% Partial mirroring (different schemas)
%%%%%%%%%%%%%%%%%%%%%%%%%%%%%%%%%%%%%%%%%%
% \item \textit{Selective mirroring} -- A variation to the core desktop functionality would be to mirror only parts of the folder hierarchies between tools (for example, two participants in the initial evaluation stated that mirroring was most appropriate for top-level folders).

\textbf{Section~\ref{main-study:results:themes-design-recs}} reports design suggestions relating to making mirroring more selective.

%%%%%%%%%%%%%%%%%%%%%%%%%%%%%
\subsubsection{Prompting}
%%%%%%%%%%%%%%%%%%%%%%%%%%%%%
%%%%%%%%%%%%%%%
% Prompting (both positive and negative wrt current design) (F3, F4, M6)
%%%%%%%%%%%%%%%
% The participants had a range of views on the existing prompting functionality.  Several found it annoying, and suggested possible alternatives.  \textit{Overhead of prompting.  Participant F4 clicking OK and closing box, and just mirroring to get it out of the way.  In other words, extra overhead is being created.}
% Prompts annoying
% ... WorkspaceMirror doesn't really irritate me. If I don't want it I just say cancel or just take the ticks out and click OK. Which I did at the beginning but then I realised I could just press cancel which was quicker. So you know I still am in control.Well its not really a problem since its just one cancel. Compared to what Word normally does to you, or Office or Windows - all kinds of harassing boxes - its negligible!  At least it never asked me to contact my system administrator!
% M7: I'd create a folder but it would be accessible from all three - rather than just an extra empty folder in the other two.  So I wouldn't get this massive bloat ... the problem with the mirroring and the way I work is the lots of empty folders ... and that to me would be annoying and would impact how I work.
% DESIGN: M6: Another thing there is that I had another folder in my email under phd, and I created a subfolder of that - then it would realise that phd was associated with that in files and then would create the folder beneath it.% Prompting (both positive and negative wrt current design) (F3, F4, M6)
%%%%%%%%%%%%%%%
%%%%%%%%%%%%%%%%%%%%%%%%%%%%%%%%%%%%%%%%%%%%%%%%%%%%
% Dislike of prompting by M7. (see design issues)
%%%%%%%%%%%%%%%%%%%%%%%%%%%%%%%%%%%%%%%%%%%%%%%%%%%%
% In addition, both M6 and M7 disliked prompting as they found it disruptive. M6: \textit{``Because . I've had more annoying things ... like my computer giving up!''}.
Participants had a range of views on the prompts generated by WM.  Three (M5, M6, M7) found it annoying, and suggested possible alternatives.  For example, M5 suggested WM should prompt more selectively: \textit{``It is not necessary to always replicate - it should prompt me in sensible cases. Maybe in higher-level folders, and not in lower level folders''}.  However, M6 noted that it was only a minor problem, \textit{``It kept on popping up and asking me the same silly question ... Well it's not really a problem since its just one cancel. Compared to what Word normally does to you, or Office or Windows - all kinds of harassing boxes - its negligible''}.

% \textit{Overhead of prompting.  Participant F4 clicking OK and closing box, and just mirroring to get it out of the way.  In other words, extra overhead is being created.} 
Three participants (F2, F3, F4) highlighted the extra decision-making introduced by WM -- i.e. deciding whether to mirror a folder event between tools.  This contributed to a number of missed and mistaken mirrorings:

%%%%%%%%%%%%%%%%%%%%%%%%%%%%%%%%%%%%%%%%%%%%%%%%%%%%%%%%%
% \subsubsection{Manual, missed and mistaken mirrorings}
%%%%%%%%%%%%%%%%%%%%%%%%%%%%%%%%%%%%%%%%%%%%%%%%%%%%%%%%%

\begin{itemize}
%%%%%%%%%%%%%%%%%%%%
% missed mirrors
% \item Missed (F3, F4)
%%%%%%%%%%%%%%%%%%%5

\item Both F3 and F4 reported missing 2 useful mirroring opportunities since they did not always have time for making such a decision and just pressed the `cancel' button on the WM dialog, W4: \textit{``It would have been useful to mirror to email but I didn't as there was no time, it [the associated work] was a deliverable''}.  
% F3:  \textit{``I might have just clicked automatically `no' without really thinking about it''}.  

%%%%%%%%%%%%%%%%%%%%%%%
% MISTAKEN MIRRORINGS
%%%%%%%%%%%%%%%%%%%%%%%
\item There were also a number of \textit{mistaken mirrorings} reported by participants F2 and F3 (2 and 1 respectively).  Again, participant F3 highlighted rushed mirroring decisions to be the key cause, in which he just clicked the `OK' button on the WM dialog box.

\end{itemize}

% Not all mirrorings were ``plain vanilla''. 
%%%%%%%%%%%%%%%%%%%%%%%%%%%%%%%%%%%%%%%%%
% MANUAL mirrors -- move to qualitative?
% \item Manual (F2, F3)
%%%%%%%%%%%%%%%%%%%%%%%%%%%%%%%%%%%%%%%%%
Three participants (F2, F3, and M5) reported carrying out \textit{manual mirroring}.  F2 did so when WM was not running.  The other two performed manual mirroring due to WM incompatibilities (see \textbf{Table~\ref{table:main-study:participants-problems}}). F3 manually mirrored 4 folders from the `Desktop' to his bookmarks, and M5 manually mirrored 2 folders from his network drive to email.


%%%%%%%%%%%%%%%%%%%%%%%%%%%%%%%%%%%%%%%%
\subsubsection{Getting Started with WM}
%%%%%%%%%%%%%%%%%%%%%%%%%%%%%%%%%%%%%%%%

%%%%%%%%%%%%%%%%%%%%%%%%%%%%%%%%%%%%%%%%%%%%%%%%%
% Difficulty of use when setting up (F3 and M8). 
%%%%%%%%%%%%%%%%%%%%%%%%%%%%%%%%%%%%%%%%%%%%%%%%%
Participants F3 and M8 noted difficulties in accommodating WM within an existing personal information environment where differing file, email and bookmark folder structures have already been developed.  Participant M8 noted that he would need to perform a significant reorganization of his workspace to make WM worthwhile. Although he reported planning to do so, he had not got round to it over the course of the study. 
% Therefore best if use when setting up. Therefore recommendations for certain types of other people
Participant F3 suggested that WM would be most appropriate to people setting up a new computer, \textit{``I'd say you get you'd get some seriously different results if you installed WM on someone with a brand new computer about to start to using it ... I think you'd get a very different dynamic, and you might even get a completely different usage out of the same person ... Most of my pre-organization had already occurred by then, reasonably quickly after purchasing the computer''}.

%%%%%%%%%%%%%%%%%%%%%%%%%%%%%%%%%%%%%%%%%%%%%%%%%%%%%
% Other recommendations: RE: installation issues
%%%%%%%%%%%%%%%%%%%%%%%%%%%%%%%%%%%%%%%%%%%%%%%%%%%%%
% M5: ... well maybe to people who don't use the network drive, and rely on bookmarks more ...  








%%%%%%%%%%%%%%%%%%%%%%%%%%%%%%%%%%%
% \subsection{``Negative'' feedback}
% \subsection{Negative feedback from non-users of WM}
\subsubsection{Reasons for not using WM}
\label{main-study:results:themes-neg-feedback}
% this is more discussion than results
%%%%%%%%%%%%%%%%%%%%%%%%%%%%%%%%%%%

Three participants, M6, M7 and M8 made very little use of WM, each for a different set of reasons. Their cases are summarized as follows.

%%%%%%%%%%%%%%%%%%%%%%%%%%%%%%%%%%%%%%%%%%%%%%%%%%%%%%%%%%%%%%%%%%%%%%%%%%%%%%%%5
% M6: Need flexibility, mapping is not 1:1 (see design recommendations below)
%%%%%%%%%%%%%%%%%%%%%%%%%%%%%%%%%%%%%%%%%%%%%%%%%%%%%%%%%%%%%%%%%%%%%%%%%%%%%%%%%
Although participant M6 was profiled as \textit{pro-organizing} in all three collections, he had very different organizing behaviour in each.  In files he had a deep hierarchy containing 235 folders, whilst his email only contained a 10 top-level folders.  He argued for the unidirectional mirroring of top-level folders from email to files only.

%%%%%%%%%%%%%%%%%%%%%%%%%
% M7: Participant M7
%%%%%%%%%%%%%%%%%%%%%%%%%
%%%%%%%%%%%%%%%%%%%%%%%%%%%%%%%%%%
% Problems with unused folders as footnote
% Not a prob: F1, F4
% F3: I haven't written any latex so I haven't used it But I'm keeping it. I think it may be useful. It will be used - I'm pretty certain of that!
%%%%%%%%%%%%%%%%%%%%%%%%%%%%%%%%%%
Participant M7 was initially unfavourable towards WM, and demanded more powerful integration based on folder-sharing, \textit{``I'd create a folder but it would be accessible from all three - rather than just an extra empty folder in the other two.  So I wouldn't get this massive bloat ... the problem with the mirroring and the way I work is the lots of empty folders ... and that to me would be annoying and would impact how I work''}.  In any case, during stage 4, he carried out little foldering, stating that he had instead been working on paper.  % In contrast to the other participants, he also stated that unused folders would not offer consistency, but instead clutter.
% M7: Personally I don't see the need for mirroring ...I do see the need for sharing but mirroring ... because I don't organize things in a structure. I'm fairly haphazard ... if I want to put a folder there and not really bother .... but I don't specifically don't have to have it in my email management because  I don't have to cope with whatever I'm looking at on the web to be mirrored in my mail. 




%%%%%%%%
% M8
%%%%%%%%
The primary reason offered by participant M8 for not using WM, was that he had performed very little PIM on his main work computer, as he had instead been using another computer for experimental work.  Although he stated that he could foresee benefits to using WM in the future, he also suggested that he would have to perform significant reorganizing of his files beforehand.  In other words, he foresaw significant overheads in accommodating WM within his existing personal information environment.
% foresaw significant bootstrapping barriers to reorganizing his files in particular.

%%%%%%%%%%%%%%%%%%%%%%%%%%%
% Performance issues
% some implementation issues, e.g. Performance impact for M7
%%%%%%%%%%%%%%%%%%%%%%%%%%%
% In general, apart from early versions of the prototype occasionally crashing, WM performed well. Participant M7 was the only one to note performance issues: \textit{``It just seems incredibly clunky ... its a proof of concept, but if you do plan on making it more available ... The other thing is I suppose its been intrusive in terms that I've had a few performance-related issues''}.



%%%%%%%%%%%%%%%%%%%%%%%%%%%%%%%%%%%%%%%%
\subsubsection{Unanticipated Feedback}
% \label{main-study:results:themes-unanticipated}
%%%%%%%%%%%%%%%%%%%%%%%%%%%%%%%%%%%%%%%%:
% plus-reflection (F2, M5)
Several participants reported benefits and problems due to WM that had not been anticipated by the author.

% : (1) increased reflection, and (2) positive constraint.  In fact both were seen as trade-offs.
% plus-reflection (F2, M5)
% All participants noted that they did not typically think about PIM, e.g. M5: 
Firstly, three participants (F2, F4 and M5) reported that prompting from WM lead to them thinking more about PIM than they would normally.  F4 saw this as a benefit that encouraged her to be more organized.  M5 reported benefits of a very different kind, suggesting that WM encouraged him to spend less time filing, \textit{``When I get the WM message up it reminds me that I'm wasting time with information management, creating folders and putting things in folders''}.  Participant F2 argued that the increase in reflection had both positive and negative consequences, F2: \textit{``WM has made me more aware of the directories in Outlook and on the Hard Drive. I'm certainly thinking more now about how I organize. This is both good and bad. Good because I'm producing a better organization. But bad, because I'm spending more time doing it [organizing]''}. 



% positive constraint (F2, F4, cf. multiple roots)
The second unanticipated benefit was reported by participants F2 and F4. They noted that since WM was limited to monitoring one area of the file system they were implicitly encouraged to store more files there to take advantage of the mirroring facility. F4 described this as follows: \textit{``Its good to have everything centralized into one directory. Before I had a problem remembering where to find things.  Now it's more consistent, and easier to remember''}. % Good for tracking, makes it easier to remember where things are''}.

%%%%%%%%%%%%%%%%%%%%%%%%%%
% Anti-integration-ism
%%%%%%%%%%%%%%%%%%%%%%%%%%
% Surprising: not for novice users.
% See anti-integrationism discussion (F3 and M6 and M8)
% was that some participants suggested that \textit{integration was not necessarily a positive thing}.  
The final theme concerned \textit{problems due to too much integration between PIM-tools}.  Firstly, participant M8 observed that integrated ``application suites'' such as Netscape Communicator were overly complex, \textit{``I did use Netscape for a while and I found it kind of annoying ... it just seemed like too many functions in one application''}.  In terms of WM, participant F3 predicted that the increased integration offered by WM may cause problems for novice users, F3: \textit{``One more thing - talking about novice users ... say my mum. If you gave her your tool I'd worry that she'd have a massive retrieval problem. She saves things and then lose them and then calls me up and asks me where they are. And I need to direct her to them. She'll say `I saved it in X' but if it all looks the same ... Maybe it's an extreme example but maybe integration can go too far for novice users as well, that's all I'm saying. It could be a problem if all the folders look the same - cos they actually may give hopeless users like her a clue as to where she is''}.

Participant M6 also made a similar comment: \textit{``I would never give it to anyone who wasn't experienced at all ... because again it does a lot for you ... and most people, if they don't know what to do they just say `yes'!  So if my mum was using it I would come home in the holidays and there would be 5000 folders, all empty, all her email lying in her inbox - you know what I mean ... You need someone who's at the stage of using a folder structure and understands folders''}.
% F6: don't see the point of mixing the two except for confusing people like my mum, and then annoying me because I have to remind her that she's using the wrong piece of software. Anyone who knows how to use a computer won't use IE for organizing their files. The people who don't know enough get confused because they then don't see a difference between the internet and their own files (also see novice quote).







%%%%%%%%%%%%%%%%%%%%%%%%%%%%%%%%%%%
\subsection{Design Recommendations}
\label{main-study:results:themes-design-recs}
%%%%%%%%%%%%%%%%%%%%%%%%%%%%%%%%%%%
% THINK: bring up now or motivate based on evaluation feedback}
% QUOTE: design is never-ending/open-ended (Carroll)}
% THINK: CONSIDER MOVE TO CONCLUSION
% Design space of extensions and variations, although initially think evaluation of core design:
% Were these echoed by main study participants?

This section presents the design suggestions made by the participants over the course of the evaluation. All participants, whether they responded positively or negatively to WM, were helpful in making suggestions.

%%%%%%%%%%%%%%%%%%%%%%%%%%%%%%%%%%%%%%%%%%%%%%%%%%%%%
% Mapping is not one-to-one (F1, F2, F4, M6, M8). 
% Requests for more flexibility
% Requests for more flexibility, but understanding that this would increase complexity of the tool.
% Link to dislike by some of more integration?  
%%%%%%%%%%%%%%%%%%%%%%%%%%%%%%%%%%%%%%%%%%%%%%%%%%%%%

%%%%%%%%%%%%%%%%%%%%%%%%%%%%%%%%%%%%%%%%%%%%%%%%%%%%%
\subsubsection{Customizing mirroring functionality}
%%%%%%%%%%%%%%%%%%%%%%%%%%%%%%%%%%%%%%%%%%%%%%%%%%%%%


%%%%%%%%%%%%%%%%%%%%%%
% MORE FLEXIBILITY
%%%%%%%%%%%%%%%%%%%%%%
A number of suggestions were made for making the mirroring process more selective, or making it customizable:

\begin{itemize}

\item As noted above, seven participants suggested that mirroring should focus on top-level folders only, e.g. M5: \textit{``It should not always prompt me ... it should prompt me in sensible cases. Maybe in higher-level folders, and not in lower level folders''}.


%%%%%%%%%%%%%%%%%%%%%%%%%%%%%%%%%%%%%%%%%%%%%
%Need for more mirroring flexibility/power
% #2: map to new folders in any location
%%%%%%%%%%%%%%%%%%%%%%%%%%%%%%%%%%%%%%%%%%%%%
\item Three participants expressed the need to mirror a new folder between different locations in distinct folder structures, e.g. M6: \textit{``It would nice if you could browse and say `I want to create the folder at that level'. You know a button `browse' where you can select where you want to create a new one.''}.
% The default being "My Documents" because that'll be the case most of the time

%%%%%%%%%%%%%%%%%%%%%%%
% Mirror names only
%%%%%%%%%%%%%%%%%%%%%%%
\item Two suggested the option to give a mirrored folder a different name in each PIM-tool.
% Several participants wanted more flexibility regarding mirroring, such as selecting where a folder would be mirrored to, and whether to change its name. % -- i.e. to mirror folder names but allow flexibility in terms of their location.  

%%%%%%%%%%%%%%%%%%%%%%%%%%%%%%%%%%%%%%%%%%%%%%
%Need for more mirroring flexibility/power
% #1: map to arb existing folders
%%%%%%%%%%%%%%%%%%%%%%%%%%%%%%%%%%%%%%%%%%%%%%
\item Two wanted to be able to associate a newly-created folder with another already existing folder in another tool, M6: \textit{``You might have to say `I don't want to create a folder, because that one already exists' ... You might start a project and not have an email folder, but you receive so many emails on it that you create a folder for it later''}.

%%%%%%%%%%%%%%%%%%%%%%%%
% One-way mirroring
% Flex: certain directions only: M6 in the extreme. 
%%%%%%%%%%%%%%%%%%%%%%%%
\item Participant M6 saw the need for a \textit{one-way mirror} from email to files, limited to top-level folders only, \textit{``if you create a folder in email you will nearly always have, or will create, a folder in my documents''}.

\end{itemize}



%%%%%%%%%%%%%%%%%%%%%%%%%%%%%%%%%%%%%%%%%%%%%%%%%%%%
\subsubsection{Extending mirroring functionality}
%%%%%%%%%%%%%%%%%%%%%%%%%%%%%%%%%%%%%%%%%%%%%%%%%%%


%%%%%%%%%%%%%%%%%%%%%%%%%%%%%%%%%%%%%%%%%%%%
% Flexibility: multiple roots (F2, F4)
% ADD diagram?
%%%%%%%%%%%%%%%%%%%%%%%%%%%%%%%%%%%%%%%%%%%%
% F4: BUT I said this previously. I prefer several root files, not just one papers file. So it has improved in the sense that everything's under one directory and everything's not that large... but I think when it gets a lot bigger and over time again I'll have the same difficulty finding the correct directory or the files that I need]
% F4: For example 
% NB: SCALABILTY OF CENTRALIZATION I said this previously. I prefer several root files, not just one papers file. So it has improved in the sense that everything's under one directory and everything's not that large... but I think when it gets a lot bigger and over time again I'll have the same difficulty finding the correct directory or the files that I need . ++ No, its helped again - I feel the problem that the problem will be there over time because this one root directory will get so big and I'll need to search my files in there. But now it has put them all in one place and I can find them. And it is sort of constraining me to put them in that one place cos it sort of pops up and asks me where to... the name of the directory.
%%%%%%%%%%%%%%%%%%%%%%%%%%%%%%%%%%%%%%%%%%%%%%%%%
% Support multiple roots in the file system
%%%%%%%%%%%%%%%%%%%%%%%%%%%%%%%%%%%%%%%%%%%%%%%%%
% \item \textit{Handle multiple file roots} -- Several participants stated that it would be useful to mirror certain types of folders between different parts of the file system (e.g. mirroring project folders between the area used for writing reports, and the area used for programming).
% \textit{F5: want multiple document space not just one such as My Documents}
WM was limited to monitoring one user-nominated area of the file system. Three participants (F2, F3 and F4) expressed the need to monitor more of the file system (e.g. files on different drives or disk partitions).  Furthermore, two of these (F2 and F4) raised the potential of mirroring between different parts of the file system, e,g, F4: \textit{``It would be good for you to have separate top-level folders within documents.  I could do with one for 'projects', one for 'papers' with categories mirrored between them''}. 

%%%%%%%%%%%%%%%%%%%%%%%%%%%%%%%%%%%%%%%%%%%%%%%%%
% Recognise duplicate folders and prompt user
% Intelligence: e.g. auto-recognition of duplicate folders (F3)
%%%%%%%%%%%%%%%%%%%%%%%%%%%%%%%%%%%%%%%%%%%%%%%%%
%%%%%%%%%%%%%%
% History
% Mirror folders but only those used in a particular context are displayed
%%%%%%%%%%%%%%
Two participants talked about how the mirroring scheme could be made ``more intelligent'' by automatically mapping a new folder in one PIM-tool to one with a similar name elsewhere, e.g. F3: \textit{``If I was duplicating a folder that I didn't need to duplicate, like I did there - it could have reminded me. And then I would have gone `ah', instead of creating a new folder and arsing around like that, what I should have been doing was using that one''}.  One participant suggested that folder-mirroring could be combined with a \textit{history-based mechanism}. This would lessen the potential clutter due to unused folders, by hiding  unused folders by default.  % Such folders, but they would be available to seed folder names in a ``new folder'' dialogue.

%%%%%%%%%%%%%%%%%%%%%%%%%%%%%%%%%%%%%%
% Extension to other local PIM-tools 
%%%%%%%%%%%%%%%%%%%%%%%%%%%%%%%%%%%%%%
%%%%%%%%%%%%%%%%%%%%%%%%%%%%%%%%%%%%%%%%%%%%%%%%%%%%%%%%%%%
% Extension to rest of personal information environment
%%%%%%%%%%%%%%%%%%%%%%%%%%%%%%%%%%%%%%%%%%%%%%%%%%%%%%%%%%%
% One can imagine a consistent organizational structure mirrored throughout a user's personal information environment.
Participant F1 argued that mirroring could be extended to other PIM-tools local to the desktop, such as photo managers. He also raised the possibility of applying folder-mirroring to the organizational schemes used to manage software applications in MS-Windows, such as the ``Start Menu'', the ``Program Files'' folder, and clusters of program icons on the desktop
\footnote{Interestingly, no participants suggested applying folder mirroring around the wider personal information environment beyond the local computer, such as: (1) to a user's other computers, (2) to other devices involved in PIM such as PDA devices, and (3) to on-line PIM-tools such as web-mail. Within the web domain, one can also imagine the application of mirroring to ``web portals'' that offer a suite of PIM-tools. For instance, providers such as Yahoo!, Google and MSN offer a range of PIM tools (e.g. email, task list, contacts, document storage, notes). Currently, each PIM-tool enables the management of a distinct collection of personal information which must be separately structured.  The application of folder mirroring to other domains is discussed in \textbf{Chapter~\ref{chapter:conclusion}}.}.

Finally, participant F4 was interested in ``collaborative mirroring'' -- sharing folder structures between users, \textit{``How about institutional templates?  The department could define starting points for PhD students. A PhD student might want their supervisor's folders.   My Mum might for instance like Elton John's folder structure or Schumacher's!''}.

%%%%%%%%%%%%%%%%%%%%%%%%%%%%%%%%%%%%%%%%%%%%%%%%%%%
\subsubsection{Additional integration}
%%%%%%%%%%%%%%%%%%%%%%%%%%%%%%%%%%%%%%%%%%%%%%%%%%%
%%%%%%%%%%%%%%%%%%%%%%%%%%
% Add-on functionality
% Working towards more cross-tool functionality (direct traversal, auto-linking of attachments)
% These included support for cross-tool navigation (e.g. enabling traversal between mirrored folders via a context-menu option),
% better handling for email attachments (e.g. automatic saving of document/bookmark attachments in mirrored folders)
%project management-like facilities (e.g. cross-tool high-level functionality such as ``start project'' and ``archive project''). 
%%%%%%%%%%%%%%%%%%%%%%%%%%
% \textit{Use as a foundation for more advanced cross-tool functionality} -- As discussed in \textbf{Section~\ref{design:feasibility-study}}, several trial users were interested in the extra functionality which could be built on top of folder-mirroring.    In addition, file attachments could be automatically filed in a matching file folder on saving of an email message in an email folder. As well as mirroring the creation, deletion and renaming folders, a ``cross tool archiving'' facility could be provided, in which all files, email and bookmarks stored within a particular folder could be moved en-masse to an archival area.
Several suggestions were made regarding using WM as a platform on which to add extra integration mechanisms.  
%%%%%%%%%%%%%%%%%%%%%%%%%%%%%%%%%%%%%%%%%%%%%%%%%%%%%%%%%%%%%%%%%%%%%
% Add-on functionality, e.g. Increased support for attachments
%%%%%%%%%%%%%%%%%%%%%%%%%%%%%%%%%%%%%%%%%%%%%%%%%%%%%%%%%%%%%%%%%%%%%
Two participants (F1 and M6) explained how mirroring could enable improved support for attachments, e,g, F1: \textit{``If I got an email with an attachment and I dragged that email into the MERL [email] folder. Then the attachment should be filed in the file system in the MERL folder''}.  Since mirrored folders relating to a particular activity are implicitly linked in terms of their location, two participants suggested that it would be straightforward to provide a facility to directly navigate between equivalent folders in different tool contexts.

        

%%%%%%%%%%%%%%%%%
% Archive quote
%%%%%%%%%%%%%%%%%
% \textit{F1: it would be useful if after a while a folder in the web bookmarks could be archived � not completely deleted, but if the project has finished it would be archived or something like that. It would go somewhere � like a central folder like "COMPLETED PROJECTS" where you get all your emails, all your files and all your bookmarks together. }
Three participants (F1, F2, and F4) wanted increased support for cross-tool project management.  Two participants wanted a ``cross-tool start-project'' facility, e.g. 6.1: \textit{``Sometimes a project is started after an exchange of emails and so usually you keep the emails in your inbox ... after a while it becomes clear that the project is going to emerge and then what I'll do is create a new folder in my email, copy all those emails in there and that gets mirrored to the other things ... that would be good. But sometimes I say we've got this new project which I want to start and I'm going to start it now. This is more or less the same as you get in Visual Studio where you say `new project' and it creates all the necessary folders for you. Maybe you ought to have that on the desktop''.}
%%%%%%%%%%%%%%%%%%%%%%%%%%%%%%%%%%%%%%%%%%%%%%%
% Cross-tool project support, e.g. archiving
%%%%%%%%%%%%%%%%%%%%%%%%%%%%%%%%%%%%%%%%%%%%%%%
Two (F1 and F4) were also interested in cross-tool archiving, e.g. F1: \textit{``If the project has finished it would be archived or something like that. It would go somewhere - like a central folder like `Completed Projects' where you get all your emails, all your files and all your bookmarks together''}.                    % F4: actually you've got archive there, but it doesn't really allow you to archive - perhaps does. I haven't used it. It probably could. Again, I wouldn't want them under the active root folder. Would it allow me to do that? 
Furthermore, two participants suggested using templates for setting up multiple folders and files for new projects, e.g. \textit{F4: ``You could have a project template to set up a sub-tree structure in one action. Example standard files would include project plan, document templates''}.

Participant M7 was adamant that the current level of integration offered by WM was not enough and he wanted a more advanced mechanism based on folder-sharing: \textit{``This isn't enough to warrant my adoption. Well sharing means that the different objects (well files and email and bookmarks) actually share the same structure. Whereas mirroring means that the actual folder structure is copied but the actual contents of the folders in the bookmarks isn't visible from the email. For me mirroring is not really useful. Sharing I'd be a lot happier with, because it wouldn't impact at all the way I work currently ... so I'd create a folder but it would be accessible from all three - rather than just an extra empty folder in the other two''}.
% F7: Sharing I'd be a lot happier with, because it wouldn't impact at all the way I work currently ... so I'd create a folder but it would be accessible from all three - rather than just an extra empty folder in the other two.  So I wouldn't get this massive bloat ... the problem with thmust be see mirroring and the way I work is the lots of empty folders ... and that to me would be annoying and would impact how I work.
% M7: Personally I don't see the need for mirroring ...I do see the need for sharing but mirroring ... because I don't organize things in a structure. I'm fairly haphazard ... if I want to put a folder there and not really bother .... but I don't specifically don't have to have it in my email management because  I don't have to cope with whatever I'm looking at on the web to be mirrored in my mail. For me mirroring is not really useful

Such requests for additional integration stand in contrast to the dangers of over-integration reported in the earlier section, ``Unanticipated feedback''.

The evaluation results presented in this section are discussed in \textbf{Section~\ref{main-study:discussion:evaluation}}.



%%%%%%%%%%%%%%%%%%%%%%
% OTHER DISCUSSION TO PLACE
%%%%%%%%%%%%%%%%%%%%%%%
%\subsection{Other things to discuss maybe}
%
%\begin{itemize}
%
%\item Increased reliance on filing/folders
%
%\item Intrusion on users (F3, F4, M5, M6)
%\item Analyse the key claims inherent in the folder-mirroring principle
%
%\item Consider variation across users
%
%\item Consider variation across tools
%
%\item Consider influence on various aspects of PIM.  How did WorkspaceMirror influence user strategies?
%
%\end{itemize}






%%%%%%%%%%%%%%%%%%%%%%%%%%%%%%%%%
% FIN THESIS Chapter 6 MAIN STUDY Results: WM EVALUATION
%%%%%%%%%%%%%%%%%%%%%%%%%%%%%%%%%