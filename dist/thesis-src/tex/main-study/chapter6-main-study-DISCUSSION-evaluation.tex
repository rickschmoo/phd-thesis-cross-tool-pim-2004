%%%%%%%%%%%%%%%%%%%%%%%%%%%%%%%%%%%%%%%%%%%%%%%%
% CHAPTER 6 -- DISCUSSION
% Conclusion
% File: tex/main-study-chapter/chapter6-main-study-DISCUSSION.tex
%%%%%%%%%%%%%%%%%%%%%%%%%%%%%%%%%%%%%%%%%%%%%%%%
%%%%%%%%%%%%%%%%%%%%%%%%%%%%%%%%%%%%%%%%%%%%%%%%%%%%%%%%%%%%%%%%%%%%%%%%%%%%%%%%%%%%%%%%%%
% Richard Boardman PhD Thesis: Improving Tool Support for Personal Information Management
%%%%%%%%%%%%%%%%%%%%%%%%%%%%%%%%%%%%%%%%%%%%%%%%%%%%%%%%%%%%%%%%%%%%%%%%%%%%%%%%%%%%%%%%%%
%%%%%%%%%%%%%%%%%%%%%%%%%%%%%%%%%%%%%%%%%%%%%%%%%%%%%%%%%%%%%%%%%%%%%%%%%%%%%%%%%%%%%%%%%%
% NATBIB NOTES
%%%%%%%%%%%%%%%%%%%
%\citet{jon90}                ->    Jones et al. (1990) 
%   \citet[chap.~2]{jon90}       ->    Jones et al. (1990, chap. 2)
%   \citep{jon90}                ->    (Jones et al., 1990) 
%   \citep[chap.~2]{jon90}       ->    (Jones et al., 1990, chap. 2) 
%%%%%%%%%%%%%%%%%%%%%%%%%%%%%%%%%%%%%%%%%%%%%%%%%%%%%%%%%%%%%%%%%%%%%%%%%%%%%%%%%%%%%%%%%%

%%%%%%%%%%%%%%%%%%%%%%%%%%%%%%%
%%%%%%%%%%%%%%%%%%%%%%%%%%%%%%%
\newpage
\section{Discussion}
\label{main-study:discussion}
%%%%%%%%%%%%%%%%%%%%%%%%%%%%%%%
%%%%%%%%%%%%%%%%%%%%%%%%%%%%%%%
% KEEP ALL MAIN-STUDY DISCUSSION LOCAL!
% Where to discuss overall success of dual-purpose study (here or in 7?)
% Methods: data analysis, substantiating claims made in analysis of evaluation results
% What does it all (the study/evaluation results) mean?. Questions raised.
% Moving from evaluation of design and study results to theory
%  The results are related to the rest of this thesis, and to other research in the field.
%%%%%%%%%%%%%%%%%%%%%%%%%%%%%%%%%%%%%%%%%%%%%%%%%%%%%%
% DESIGN/EVAL: Discussion of design/evaluation findings.
% The implications from the evaluation of WorkspaceMirror are discussed: (1), Specific implications - possible refinements for WorkspaceMirror
%%%%%%%%%%%%%%%%%%%%%%%%%%%%%%%%%%%%%%%%%%%%%%%%%%%%%%
% STUDY: Discussion of findings from empirical work (exploratory study and main study).
% These are related to to the ``snapshot'' findings reported in \textbf{Chapter~\ref{chapter:exploratory_study}}. A focus is taken on the nature of the changes in PIM strategy, that were observed.  
% This section discusses the two sets of results from the field trial. 
Firstly,  \textbf{Section~\ref{main-study:discussion:evaluation}} discusses the results from the evaluation of WM.  Then, \textbf{Section~\ref{main-study:discussion:study}} discusses the longitudinal results presented in  \textbf{Section~\ref{main-study:longitudinal}}.

% \newpage
%%%%%%%%%%%%%%%%%%%%%%%%%%%%%%%%%%%%%%%%%%%
\subsection{Discussion of Evaluation Findings} 
\label{main-study:discussion:evaluation}
%%%%%%%%%%%%%%%%%%%%%%%%%%%%%%%%%%%%%%%%%%
%%%%%%%%%%%%%%%%%%%%%%%%%%%%%%%%%%%%%%%%%%%%%%%%%%%%%%%%%%%%%%%%%%%%%%%%%%%%%%%%%%%%%%%%%%
% NEED to relate evaluation to original design aims
%%%%%%%%%%%%%%%%%%%%%%%%%%%%%%%%%%%%%%%%%%%%%%%%%%%%%%%%%%%%%%%%%%%%%%%%%%%%%%%%%%%%%%%%%%
% MOVE TO MAIN DISCUSSION?
% Changing design versus changing user behaviour (what can institutions do?)
%%%%%%%%%%%%%%%%%%%%%%%%%%%%%%%%%%%%%%%%%%%%%%%%%%%%%%%%%%%%%%%%%%%%%%%%%%%%%%%%%%%%%%%%%%
%%%%%%%%%%%%%%%%%%%%%%%%%%%%%%%%%%%%%%%%%%%%%%%%
% TO PLACE:
% There may also be inter-dependencies between the tools to be taken into consideration.}
%%%%%%%%%%%%%%%%%%%%%%%%%%%%%%%%%%%%%%%%%%%%%%%%

%%%%%%%%%%%%%%%%%%%%%%%%%%%
% START: INTRO: EVAL DISC
%%%%%%%%%%%%%%%%%%%%%%%%%%%
% HERE Firstly, findings specific to WM.  Discuss WM-specific evaluation.  Towards redesign? Success of incremental approach
This section discusses the evaluation results which were reported in \textbf{Sections~\ref{design:feasibility-study}}, \textbf{\ref{main-study:case-studies}}, and~\textbf{~\ref{main-study:wm-analysis}}.  
%%%%%%%%%%%%%%%%%%%
% OVERALL SUCCESS
% positive spin
%%%%%%%%%%%%%%%%%%%
% Overall success of evaluation: good.  Many good insights.  However in terms of WM efficacy: only a measured success, or is this still too optimistic?  Mixed results in terms of WM evaluation
%%%%%%%%%%%%%%%%%%%%%%%%%%%%%%%%%%%%%%%%%%%%%%%%%%%%%%%%%%%%%%%%%%%%%%%%%%%%%%%%%%%%%%%%%%
% THINK: during evaluation, need to relate to:
% original design aims and rationale?}
% Evaluation should focus on overall potential rather than short-term design/implementation headaches -- or at least clearly differentiate}
%%%%%%%%%%%%%%%%%%%%%%%%%%%%%%%%%%%%%%%%%%%%%%%%%%%%%%%%%%%%%%%%%%%%%%%%%%%%%%%%%%%%%%%%%%
% THINK: NB: not a comparative study}
%%%%%%%%%%%%%%%%%%%%%%%%%%%%%%%%%%%%%%%%%%%%%%%%%%%%%%%%%%%%%%%%%%%%%%%%%%%%%%%%%%%%%%%%%%
% HARD TO CONCLUDE: idiosyncratic responses -- therefore hard to draw a general conclusion
% The eight participants responded in a number of ways as described in the case studies presented in \textbf{Chapter~\ref{chapter:main-study}}.  General discussion points are extracted as follows. 
%%%%%%%%%%%%%%%%%%%%%%%%%%%%%%%%%%%%%%%%%%%%%%%%%%%%%%%%%%%%%%%%%%%%%%%%%%%%%%%%%%%%%%%%%%
Although the responses to the tool were mixed, it is argued that the evaluation of WM was a success on several counts.  Firstly, the author succeeded in rolling out a prototype PIM tool to eight participants, integrating it into their existing desktop environments, and obtaining feedback over an average of 69 days.  Furthermore, this is one of the few examples of PIM-tool evaluations performed to date.  A key design aim of WM was to explore the potential to share folder structures between PIM-tools.  Based on the range of feedback received -- some positive, some negative, and some unanticipated -- it is argued that the study achieved this aim.



%%%%%%%%%%%%%%%%%%%%%%%%%%%%%%%%%%%%%%%%%%%%%%%
% \subsubsection{METH: Methodological limitations}
%%%%%%%%%%%%%%%%%%%%%%%%%%%%%%%%%%%%%%%%%%%%%%%
% the following methodological limitations are acknowledged:
%%%%%%%%%%%%%%%%%%%%%%%%%%%%%%%%%%%%%%%%%%%%%%%%%%%%%%%%%%%%%%%%%%%%%%%%%%%%%%%%%%%%%%%%%%
% EVAL SUCCESS: evaluation methodology, success of evaluation method, methodological recommendations (e.g. handling/beware of individual differences (diverse user group) (PLACEMENT: here or method discussion?)
%%%%%%%%%%%%%%%%%%%%%%%%%%%%%%%%%%%%%%%%%%%%%%%%%%%%%%%%%%%%%%%%%%%%%%%%%%%%%%%%%%%%%%%%%%
% Methodological issues are discussed in greater depth in \textbf{Chapter~\ref{chapter:conclusion}}.
% ADD pros and cons of natural and objective studies.
%%%%%%%%%%%%%%%%%%%%%%%%%%%%%%%
% Possible friendship bias
%%%%%%%%%%%%%%%%%%%%%%%%%%%%%%%
However, it is acknowledged that like most previous PIM studies, the selection of participants was not ideal.  Only a small number of participants took part, all of whom were technically experienced. 
Ideally, similar studies should be carried out with a wider range of users, including some without strong technical experience. Such follow-up studies are outside the scope of the work reported here in this thesis.  
% POSSIBLE BIAS The author acknowledges the potential experimental bias from using friends and colleagues to evaluate software. This was partially balanced out by asking the participants to be completely honest.
In addition, the participants were all known to the author.  This was a deliberate precaution, taken to alleviate privacy concerns.  However, the author acknowledges the potential experimental bias from using colleagues to evaluate software.  To counter this, participants were encouraged to be open and honest with their feedback.  The range of positive and negative feedback suggests that this was accomplished.






%%%%%%%%%%%%%%%%%%%%%%%%%%%%%%%%%%%%%%%%%%%%%%%%%%%%%%%%%%%%%%%%%%%%%%%%%%%%%%%%%%%%%%%%%%
%%%%%%%%%%%%%%%%%%%%%%%%%%%%%%%%%%%%%%%%%%%%%%%%%%%%%%%%%%%%%%%%%%%%%%%%%%%%%%%%%%%%%%%%%%
%%%%%%%%%%%%%%%%%%%%%%%%%%%%%%%%%%%%%%%%%%%%%%%%%%%%%%%%%%%%%%%%%%%%%%%%%%%%%%%%%%%%%%%%%%
%%%%%%%%%%%%%%%%%%%%%%%%%%%%%%%%%%%%%%%%%%%%%%%%%%%%%%%%%%%%%%%%%%%%%%%%%%%%%%%%%%%%%%%%%%


%%%%%%%%%%%%%%%%%%%%%%%%%%%%%%%%%%%%%%%%%%%%%%%%%
\subsubsection{Variation in WM Usage}
%%%%%%%%%%%%%%%%%%%%%%%%%%%%%%%%%%%%%%%%%%%%%%%%%
%%%%%%%%%%%%%%%%%%%%%%%%%%%%%%%%
% MOVE TO DISCUSSION: Present as disappointing?  
%%%%%%%%%%%%%%%%%%%%%%%%%%%%%%%%
% due to much feedback, and many lessons - many of them unanticipated.  
% IDIO RESPONSES 
As described in the case studies in \textbf{Section~\ref{main-study:case-studies}}, the eight participants varied in how they responded to WM.  % General discussion points are extracted as follows. 
Overall, the usage of WM, as indicated by the objective data reported in \textbf{Section~\ref{main-study:results:themes-usage}}, was lower than the author had envisaged before the study.  Only two participants (F2 and F4) mirrored more than 10 folder creation events each.  Three other participants, F1, F3 and M5, mirrored 4 or 5 folders (F3 also performed some mirroring manually).  The remaining three (M6, M7 and M8) did little or no mirroring.

% SOME USERS extensive usage. Could make decisions on how to use WM. Four made extensive use of folder mirroring (>10 mirrorings)\footnote{Note that one of these employed manual mirroring techniques instead of using WM.  This was due to the inability of WM to mirror to the ``Desktop''.}, and three made some use (between 1 and 10 mirrorings).  
% , including several unanticipated themes
% In retrospect it is argued that the evaluation of WM was a success due to the wide range of qualitative feedback obtained. 
It is argued that the variation in usage was a natural consequence of a design intervention in such a complex and idiosyncratic activity as PIM.  The range of uptake emphasises the fact that the evaluation took place in ``natural conditions'' -- participants were not forced into using WM, but were asked to use it as seemed appropriate within the context of their normal everyday PIM.  Indeed, it can argued that the wide range of responses is useful as an illustration of the idiosyncratic nature of PIM. 


% Key issues -- organizing sparse. And some users did no PIM!
%%%%%%%%%%%%%%%%%%%%%%%%%%%%%%%%%%%%%%%%%%%%%%%%%%%
%\subsection{Influences on take-up/success of tool}
%\label{main-study:results:themes-influences}
%%%%%%%%%%%%%%%%%%%%%%%%%%%%%%%%%%%%%%%%%%%%%%%%%%%
% Several other factors appeared to influence how participants responded to WM, as follows:
The following factors may have contributed to the low usage of WM by some participants:
% Use of WM was impacted by various longitudinal issues.  
\begin{itemize}

%%%%%%%%%%%%%%%%%%%%%%%%%%%%%%
% The length exposed to WM
% but note: track 1 more buggy
%%%%%%%%%%%%%%%%%%%%%%%%%%%%%%
% Primarily, the  may have been a factor in determining the extent to which it was used.  
\item \textit{Length of time which participants were exposed to WM} -- As noted in \textbf{Section~\ref{main-study:results:changes}}, users can be slow to change management strategies and when they do so, changes may be of an incremental nature.  Adopting a new tool, such as WM, may involve adaptation of strategies by the user.  It is acknowledged that the length of the study may not have been long enough for some participants to adjust to WM.  In particular, participant M8, stressed the need to reorganize his files and email to make full use of WM.  This emphasises the importance of evaluating PIM-tools over the long-term.  Despite this, all participants indicated that they had been exposed to WM for enough time to evaluate it.



%  Those users who did more foldering, made more use of WM (STATS).  
% The sporadic nature of organizing.
% WM usage was also impacted by the slow rate of organization.  If users are not creating folders, there is little need for WM.
% Furthermore, two participants (M7 and M8) performed relatively little organizing on the computers on which WM was installed.
\item \textit{Frequency of folder events} -- As indicated by the data in \textbf{Section~\ref{main-study:results:growth}}, organizing (the creation of new folders) is a sporadic activity. This meant that participants only had limited opportunities to try mirroring out.  Furthermore, two of the participants who did not use WM (M7 and M8), carried out little PIM on the installation platform during stage 4.  M8 suggested that if he had not been working on another machine, he would have made more use of WM.  This indicates the challenge of carrying out evaluations that involve discretionary activities in a natural environment.  It also indicates one of the strengths of the objective study approach, where it can be ensured that users try out desired functionality.  However, despite organizing activity being relatively infrequent, all participants indicated that they had had enough exposure to WM to give feedback.

\item \textit{Unforeseen events} -- Another consequence of a field-study based evaluation was that a number of external factors influenced usage of WM. For example, participant F1 changed job, and consequently performed less PIM than normal.


%%%%%%%%%%%%%%%%%%%%%%%%%%%%%%%%%%%%%%%%%%%%%%%%%%%%%%%
% Installation problems
% bugs, installation issues, success of play acting
%%%%%%%%%%%%%%%%%%%%%%%%%%%%%%%%%%%%%%%%%%%%%%%%%%%%%%%
% \item Finally, a number of tool incompatibilities were encountered whereby it was not possible for participants to try out the full functionality of WM (see \textbf{Figure~\ref{table:main-study:participants-problems}}).  % For instance, participant F4 used Outlook Express as her email client which it was not possible to mirror folders to and from. 
\item \textit{Installation issues} -- A range of incompatibilities were encountered between WM and participants' PIM-tools (see \textbf{Table~\ref{table:main-study:participants-problems}} on page~\pageref{table:main-study:participants-problems}).  In particular, M5 suggested that he may have made more use of WM if it had been compatible with his network drive.  This lead both him and F3 to successfully take up \textit{manual mirroring}.  It is envisaged that both participants would have made more use of WM without these technical problems.

%%%%%%%%%%%%%%%%%%%%%%%%%%%%%%%%%%%%%
% VARIATIONS IN ORG GRANULARITY
%%%%%%%%%%%%%%%%%%%%%%%%%%%%%%%%%%%%
% Style of PIM (M8, M6), similarity of PIM needs between different tools. Nature of information being manage?  Nature of org needs in each tool?
\item \textit{Differences in organizational requirements between PIM-tools} -- Participants organized different types of information in different ways, which resulted in a major impact on the utility of folder mirroring.  In particular, M6 used WM very little since he had very different organizational requirements in each tool. This issue is discussed in more depth below.

\end{itemize}


%%%%%%%%%%%%%%%
% TRACK 2
%%%%%%%%%%%%%%%
% On average less time to use.  Also worse initial attitude.
%\begin{itemize}
%	\item Different exposure
%	\item Different initial attitudes?
%\end{itemize}
\textbf{Section~\ref{main-study:results:themes-usage}} highlighted that the \textit{track 2} participants made significantly less use of WM compared to the \textit{track 1} participants (an average of 1.25 versus 12.75 folder creation events mirrored).  The following factors contributed to this anomaly: (1) differing organizational requirements between tools for participant M6, and (2) a lack of PIM activity for participants M7 and M8.  Furthermore, M5, M6 and M7 had negative attitudes towards WM initially, although M5 and M6 performed some mirroring towards the end of the study.   Finally, the \textit{track 2} users were exposed to WM for less time (an average of 30 days, compared to 108 days for \textit{track 1}), meaning they had less time to mirror folder events. It is envisaged that both participants M5 and M8 may have made more use of WM if the study had been longer.

%%%%%%%%%%%%%%%%%%%%%%%%%%%%%%%%%%%%%%%%
% USAGE: Why all creates? MOVE TO RESULTS?
%%%%%%%%%%%%%%%%%%%%%%%%%%%%%%%%%%%%%%%%
%\begin{itemize}
%	\item Few deletes and renames
%	\item Lack of trust, not mirrors.  Dangerous?
%\end{itemize}
% : (1) users did very few deletions and renames, (2) , (3) lack of trust, danger danger.
All mirrored events corresponded to \textit{folder creation events}.  A key reason for this was that the participants performed few \textit{folder deletion} or \textit{folder renaming events} during stage 4 as indicated in \textbf{Table~\ref{table:main-study:wm-usage}} on page~\pageref{table:main-study:wm-usage}.  Participants F3 and M5 performed most of these types of events during the reorganizations they performed.  However, most of the events did not result in prompts by WM because they occurred in areas not monitored by WM, e.g. M5's network drive.
% However, the deleted and renamed folders related to folders As discussed in \textbf{Section~\ref{main-study:discussion:study}}, maintenance (deletion) and reorganization were performed irregularly.  
Secondly, as reported in \textbf{Chapter~\ref{chapter:design}}, the WM functionality for mirroring folder renaming events was incomplete.  This meant that events which involved moves between parent folders were not mirrored.  Thirdly, participants may have been wary of this functionality.  In the initial interview, one commented that mirroring deletes was ``dangerous technology''.

%%%%%%%%%%%%%%%%%%%%%%%%%%%%%%%%%%%%%%%%%%%%%%%%%%%%%%%%%%%%%%%%%%%%%%%%%%%%%%%%%%%%%%%%%%
%%%%%%%%%%%%%%%%%%%%%%%%%%%%%%%%%%%%%%%%%%%%%%%%%%%%%%%%%%%%%%%%%%%%%%%%%%%%%%%%%%%%%%%%%%
%%%%%%%%%%%%%%%%%%%%%%%%%%%%%%%%%%%%%%%%%%%%%%%%%%%%%%%%%%%%%%%%%%%%%%%%%%%%%%%%%%%%%%%%%%
%%%%%%%%%%%%%%%%%%%%%%%%%%%%%%%%%%%%%%%%%%%%%%%%%%%%%%%%%%%%%%%%%%%%%%%%%%%%%%%%%%%%%%%%%%

%%%%%%%%%%%%%%%%%%%%%%%%%%%%%%%%%%%%%%%%
\subsubsection{Benefits of Long-term Evaluation}
%%%%%%%%%%%%%%%%%%%%%%%%%%%%%%%%%%%%%%%%

%%%%%%%%%%%
% CHANGES - Emergent behaviour.
%%%%%%%%%%%
%%%%%%%%%%%%%%%%%%%%%%%%%%%%%%%%%%%%%%%%%%%%%%%%%%%%%%%%%%%%%%
% \subsubsection{Did initial attitude predict overall usage?}
%%%%%%%%%%%%%%%%%%%%%%%%%%%%%%%%%%%%%%%%%%%%%%%%%%%%%%%%%%%%%%
% Consider initial attitude on installation
% Were the heaviest users of WM were typically highly positive from the start?
% \textit{Initial attitude to WM} was a partial indication of whether the participant would use WM.  The three heaviest users of WM were initially positive. However, M5 and F3 who made some usage were initially negative.  Also, M8, who was originally positive, ended up making no usage of WM.
% Participants F1, F2 and F4 responded positively to WM, and made extensive use of it.  Participants F3 and M5 responded negatively, but performed some mirroring later in the study\footnote{Note that participant F3 carried out manual mirroring.}.  Participants M6 and M7 responded negatively to WM and made little use.  M8 responded positively. but made little use of WM.
The field study highlighted the benefits of evaluating over the long-term.  Primarily, emergent behaviour was observed for three participants who changed their attitude towards WM over the course of the study.   \textbf{Table~\ref{table:main-study:wm-attitude-usage}} shows a break-down of the participants based on their initial attitude to WM, and and their subsequent usage of WM.
% Four were initially positive (F1, F2, F4, and M8).  Of these, three made extensive use of WM (F1, F2 and F4).  Four participants were initially negative regarding WM (F3, M5, M6, M7). 
Participants F3 and M5 were initially negative but ended up making some use of WM, M5: ``\textit{Quite impressively, the idea of mirroring structures in emails and in the folders didn't really cross my mind. I think before I had similarly named folders in both systems but they were never agreed in any way. And now I think the structures are a bit more coherent}''.  In addition, M8 although positive towards the tool at the start of the study, performed no mirroring. In other words, initial attitude did not define whether WM was used over the long-term.


%%%%%%%%%%%%%%%%%%%%%%%%%%%%%%%%%%%%%%%%%
% TABLE OF PARTICIPANTS IN MAIN STUDY
%%%%%%%%%%%%%%%%%%%%%%%%%%%%%%%%%%%%%%%%%
\begin{table} % [hbtp]
\begin{center}
\begin{footnotesize}
\setlength{\extrarowheight}{2pt}
% Table generated by Excel2LaTeX from sheet '5.5 WM-IMPRESSIONS'
\begin{tabular}{|p{2.5cm}|p{2.5cm}|p{2.5cm}|p{2.5cm}|}
\hline
{\bf Initial Attitude to WM} & {\bf Used WM extensively} & {\bf Made limited use of WM} & {\bf Did not use WM} \\
\hline
{\bf Positive} & F1, F2, F4 &            & \textit{M8 (little PIM, used other computer)} \\
\hline
{\bf Negative} &            & \textit{F3 (manual mirroring, some use of WM), M5 (manual mirroring)} &     M6, M7 \\
\hline
\end{tabular}  
\end{footnotesize}
\caption{Initial attitude to WM, and usage over stage 4 of main study}
\label{table:main-study:wm-attitude-usage}
\end{center}
\end{table}
\normalsize


%%%%%%%%%%%%%%%%%%%%%%%
% EXPOSURE TO WM
%%%%%%%%%%%%%%%%%%%%%%%
% Participants F3, M5, and M6 did not make use of WM until some time after their initial exposure to it.  In fact, all three cases, their attitude to 
% It is acknowledged that the two heaviest users (F2 and F4) had it installed the longest.  However, in their cases, they employed WM from the beginning of the study, suggesting that usage was not purely dependent on exposure to the tool over time. 

%%%%%%%%%%%%%%%%%%%%%%%%%%%%%%%
% UNFORESEEN BENEFITS 
%%%%%%%%%%%%%%%%%%%%%%%%%%%%%%%
% Unforeseen benefits (e.g. plus-reflection, RELATE TO THEORY-DEV).  
Additionally, the field trial unearthed some unforeseen effects of the design intervention for some participants: (1) increased reflection, and (2) the positive side-effect of a user being constrained to one portion of the file system.  It is argued that such results may be less likely to be uncovered in a short-term controlled study.



%%%%%%%%%%%%%%%%%%%%%%%%%%%%%%%%%%%%%%%%%%%%%%%%%%%%%%%%%%%%%%%%%%%%%%%%%%%%%%%%%%%%%%%%%%
%%%%%%%%%%%%%%%%%%%%%%%%%%%%%%%%%%%%%%%%%%%%%%%%%%%%%%%%%%%%%%%%%%%%%%%%%%%%%%%%%%%%%%%%%%
%%%%%%%%%%%%%%%%%%%%%%%%%%%%%%%%%%%%%%%%%%%%%%%%%%%%%%%%%%%%%%%%%%%%%%%%%%%%%%%%%%%%%%%%%%
%%%%%%%%%%%%%%%%%%%%%%%%%%%%%%%%%%%%%%%%%%%%%%%%%%%%%%%%%%%%%%%%%%%%%%%%%%%%%%%%%%%%%%%%%%

%%%%%%%%%%%%%%%%%%%%%%%%%%%%%%%%%%%%%%%%%%%%%%%%%%%%%%%%%%%%%
\subsubsection{Differences in Organizational Requirements between PIM-tools}
%%%%%%%%%%%%%%%%%%%%%%%%%%%%%%%%%%%%%%%%%%%%%%%%%%%%%%%%%%%%%

%%%%%%%%%%%%%%%%%%%%%%%%%
% OTHER DESIGN SUGGESTIONS
%%%%%%%%%%%%%%%%%%%%%%%%%
A wide range of design suggestions were reported in \textbf{Section~\ref{main-study:results:themes-design-recs}}. %, and the most promising are suggested as future work in \textbf{Chapter~\ref{chapter:conclusion}}.
% The evaluation is still considered a success as a wide range of both  was received, many of it unanticipated.   % Furthermore, a range of directions for further design were identified.
This section focuses on the most common area of feedback -- the need to make mirroring more selective.
% Here a focus is taken on the feedback relating to the core functionality of WM: mirroring folder structures between PIM-tools.
% In general, participants were positive regarding the use of folder-mirroring between the file, email and bookmark tools -- \textit{for top-level folders}.  
% Three expressed interest in mirroring the entire folder structures. Five participants were less favourable to mirroring folders from lower levels.  

Several participants welcomed the promise of increased consistency between the folder structures across PIM-tools. However, in the closing interview, seven of the participants suggested that mirroring was most appropriate for top-level folders.
% EXPLAIN: diff org granularities in different tools. 
% many users employ different organizational granularities in different tool contexts.
% This outweighed the importance of his pro-organizing attitude. One possible way of compensating for different PIM styles would be to focus mirroring on top-level `project' folders. 
Furthermore, participants disagreed with respect to whether \textit{all} top-level folders should be mirrored. 
Some argued that if a folder is tool-specific, then it is unnecessary to see that folder in other tool contexts.  Others favoured the increase in consistency, even if some mirrored folders remained unused in some tools.

%%%%%%%%%%%%%%%%%
% TRADE-OFF
%%%%%%%%%%%%%%%%%
% of organization offered by WM, and (2) the resulting impact on the flexibility to organize different types of information in different ways. 
This points to a trade-off between \textit{organizational consistency} (having the same folder structure in different tools) and \textit{cross-tool organizational flexibility} (being able to organize different types of information in different ways). Several participants welcomed the increase in consistency, indicating that it made navigation easier.  However increasing consistency in this way, carries the cost of constraining the user to organize different types of information in the same way.
Overall, most participants favoured flexibility over consistency.  However, seven participants said that mirroring folders made sense in some cases -- typically for top-level folders.  % Furthermore, some participants expressed the need to only mirror some top-level folders -- in those cases where the activity to which it relates involves both tools.  \textit{Also need to think about name?}

A key reason for the bias in favour of flexibility was that there were differences in organizational requirements between tools. For many participants, email and bookmarks tended to be based on shallow, one layer folder structures, whilst files were organized within deep, many-branching structures.  Therefore they saw little need to mirror all low-level file folders to email and bookmarks.  In those other tool contexts, unless the user changes organizing strategy, the low-level folders will not be used\footnote{The author would like to reinforce the difference between two similar terms used in the thesis: \textit{organizing tendencies} listed in the cross-tool profiles in \textbf{Chapter~\ref{chapter:exploratory_study}}, and \textit{organizational requirements}.  For example, M6 was \textit{highly pro-organizing} in both files and email (i.e. he had similar organizing tendencies in each tool).  However, he made little use of WM due to having different organizational requirements in each tool.  In his file collection, he arranged files in a richly-developed folder structure.  However, in email, where he was also pro-organizing, he employed a frequent filer strategy but only required a flat set of folders.}.

% FOOTNOTE: ORG TENDENCY != ORG REQUIREMENT
%\item There was also only a partial correlation with participant's \textit{cross-tool profile}. Again the heaviest users were consistently pro-organizing across files, email and bookmarks in their profile.  However M6 was pro-organizing but did not use it, due to different organizational needs in each tool.  F3 and M5 were less pro-organizing but made some usage of the tool (see \textbf{Table~\ref{table:main-study:wm-usage}}).
%\item Initially pro-WM, CT pro-org: -> use WM: 3 users (F1, F2, F4)
%\item Initially anti-WM, CT org-neut: -> manual: 1 user (F3)
%\item Initially anti-WM, CT org-neut: -> some: 1 user (M5)
%\item Initially anti-WM, CT pro-org: -> no use: 1 user (M6)
%\item Initially anti-WM, CT org-neut: -> no use: 1 user (M7)
%\item Initially pro-WM, CT pro-org: -> no use: 1 user (M8)
% attitude to organizing,  how did it map to usage. Personal characteristics (e.g. need for tidiness), consequent PIM strategies
% Use over time also indicated that there was only a partial correlation with participant's \textit{cross-tool profile}. 
% Note that the organizing strategies calculated in the study do not necessarily reflect consistent organizational requirements.  
% ADD: contrast with heaviest users?
% The heaviest users were consistently pro-organizing across files, email and bookmarks (participants F1, F2, and F4).    % F3 and M5 were less pro-organizing but made some usage of the tool (see \textbf{Table X}. 
% much potential to improve. Particularly need for increased configurability
% {Instead attitude to organizing may be more important than organizing strategies. Personal characteristics (e.g. need for tidiness)}

%%%%%%%%%%%%%%%%%%%%%%
% DESIGN CONCLUSION
%%%%%%%%%%%%%%%%%%%%%%
%%%%%%%%%%%%%%%%%%%%%%%%%%%%%%%
% DESIGN: Formative redesign
%%%%%%%%%%%%%%%%%%%%%%%%%%%%%%%
% Sell conclusions as trade-off: users welcomed consistency, but were conerned as flexibility cost.  
% Therefore need for configurability.  Think about default mode of operation.
% A trade-off can be identified between  Therefore configurability is important.  It is speculated that the default action should be to mirror top-level folders only.  % Those users who are keen on WM are perhaps more likely to configure it on.
% Overall, demonstrates the potential worth of the design So all in all, some integration is good to a certain point
The study indicates that folder-mirroring has potential, especially for top-level mirroring.
% \textit{SUGGESTION Better for long-term projects/consistent organizational 
The formative redesign of WM is outside the scope of this thesis.  However, based on the response from the participants, the next step would certainly be to limit mirroring to top-level folders by default.  Participants varied in terms of the trajectories of mirroring events. Although overall files-to-email was the most common trajectory, several participants (e.g. F3 and F4) mirrored mainly between files and bookmarks.  Therefore, a customization facility to select the PIM-tools between which mirroring occurs would also be worth investigating. 
% Many participants emphasised the need for improved configurability in terms of selecting the tools between which folders should be mirrored.  

%%%%%%%%%%%%%%%%%%%%%%%%%%%%%%%%%%%%%%%%%%%%%%%%%%%%%%%%%%%%%%%%%%%%%%%%%%%%%%%%%%%%%%%%%%
%%%%%%%%%%%%%%%%%%%%%%%%%%%%%%%%%%%%%%%%%%%%%%%%%%%%%%%%%%%%%%%%%%%%%%%%%%%%%%%%%%%%%%%%%%
%%%%%%%%%%%%%%%%%%%%%%%%%%%%%%%%%%%%%%%%%%%%%%%%%%%%%%%%%%%%%%%%%%%%%%%%%%%%%%%%%%%%%%%%%%
%%%%%%%%%%%%%%%%%%%%%%%%%%%%%%%%%%%%%%%%%%%%%%%%%%%%%%%%%%%%%%%%%%%%%%%%%%%%%%%%%%%%%%%%%%


%%%%%%%%%%%%%%%%%%%%%%%%%%%%%%%%%%%%%%%%%%%%%
\subsubsection{Exploring the Limits of Integration}
%%%%%%%%%%%%%%%%%%%%%%%%%%%%%%%%%%%%%%%%%%%%%

% LINK TO DESIGN SECTION: demonstrates that integration good to a point. However this is countermanded by some of the participant feedback in which they noted the benefits some users gain from having clear separation between tools.   The dangers of increased integration are discussed in more detail in \textbf{Section~\ref{discussion:design-guidelines-discussion}}.

% Differing views on integration.  Made author consider pros and cons.  Link to Chapter 7 generalization.
% These views contrast with those outlined above that highlights problems in increasing integration.  In particular, two participants suggested that even the core folder-mirroring functionality could cause problems for some novice users. 
The study participants conveyed a number of views regarding the pros and cons of PIM-tool integration.  Participant M7 advocated more powerful integration based on folder sharing, and a number of other participants requested ``add-on'' functionality on top of WM.  However, a number of participants noted that integration of folder structures may cause problems for less technical users -- making the structures consistent removes one contextual cue of the user's current location.
% INTEGRATION the results The evaluation of WorkspaceMirror as a straw man design is also considered, in terms of the implications for PIM-integration in general as a design genre.
% Design recommendations suggested by the participants to solve the issues listed above are discussed in the next section.
% calls such as these for increased integration, with possible downsides. 
In the next chapter, \textbf{Section~\ref{discussion:design-guidelines-discussion}} weighs up the possible pros and cons of design work aimed at increasing PIM-integration more generally.  % The findings from this study were crucial in making the researcher aware that integration may have downsides.  In contrast, most software is marketed as ``integrated'' -- it is normally assumed to be a good thing.


%%%%%%%%%%%%%%%%%%%%%%%%%%%%%%%%%%%%%%%%%%%%%%%%%%%%%%%%%%%%%%%%%%%%%%%%%%%%%%%%%%%%%%%%%%
%%%%%%%%%%%%%%%%%%%%%%%%%%%%%%%%%%%%%%%%%%%%%%%%%%%%%%%%%%%%%%%%%%%%%%%%%%%%%%%%%%%%%%%%%%
%%%%%%%%%%%%%%%%%%%%%%%%%%%%%%%%%%%%%%%%%%%%%%%%%%%%%%%%%%%%%%%%%%%%%%%%%%%%%%%%%%%%%%%%%%
%%%%%%%%%%%%%%%%%%%%%%%%%%%%%%%%%%%%%%%%%%%%%%%%%%%%%%%%%%%%%%%%%%%%%%%%%%%%%%%%%%%%%%%%%%

%%%%%%%%%%%%%%%%%%%%%%%%%%%%%%%%%%%%%%%%%%%%%%%%%
\subsubsection{Prompting Overheads}
%%%%%%%%%%%%%%%%%%%%%%%%%%%%%%%%%%%%%%%%%%%%%%%%%

Another design issue raised in the evaluation was the intrusive nature of the WM prompts.  Several participants reported not always having time to decide whether to mirror or not.  Consequently they sometimes mirrored mistakenly, or missed occasions when it would have been useful to mirror.  Participant M4 encountered particular problems, explaining how she did not always have time to make a mirroring decision.

%  Not only do they have to decide: (1) whether they should organize, and (2) and how to organize, in other words choose a folder name. They now have to also decide whether to mirror the folder event.  
% It is noted that this is adding to the immediate overhead on the user in terms of the organizational decisions they have to make.  
It is noted that running WM in prompted mode adds to the cognitive overhead during organization.  The user has to make two additional decisions when creating a folder.  In standard tools, users must decide whether to organize, and if so where a folder should be placed, and what it should be named.  With WM, they must also decide: (1) whether to mirror the new folder, and (2) which tools to mirror it to.
% then they avoid similar decisions at a later date (assuming they have created the mirrored folder and remember to use it).  
Of course, if the mirrored folders are put to use in the other tool contexts at a later time, then the extra up-front overhead may be worthwhile, as the user avoids subsequent organizing overheads in those other contexts.  However, it is noted that to take advantage of this, the user must remember that the mirrored folder is available for use.  Another concern is that users may place more weight on short-term, immediate costs over long-term, potential benefits.
% All participants preferred prompted mirroring over an automatic mode of operation.   However it is noted that all the feedback was from technically experienced users who place a greater emphasis on control and flexibility.  
Despite the extra organizing decisions due to the WM mirroring prompts, participants, including M4, wanted to retain control over mirroring.

% Could make decisions on how to use WM. 
% NEED FOR CONTROL CF. NOVICE USERS.


%%%%%%%%%%%%%%%%%%%%%%%%%%%%%%%%%%%%%%%%%%%%%%%%%%%%%%%
\subsubsection{Designing for Non-technical Users}
%%%%%%%%%%%%%%%%%%%%%%%%%%%%%%%%%%%%%%%%%%%%%%%%%%%%%%%

The author believes that the selection of participants (all were technically experienced) had a strong influence on two areas of feedback: (1) the need for high organizational flexibility, and (2) the need for control over mirroring.

%%%%%%%%%%%%%%%%%%%%%%%%%%%%%%%
% DESIGN HYPOTHESIS
%%%%%%%%%%%%%%%%%%%%%%%%%%%%%%%
% Suggest suitability for some users. Can I characterise them? Suitability to novice users (hunch or more?) 
% But conflicts with feedback of some of the parts. Would like to repeat study with novice users.
It is hypothesised that WM in its current form may be most suitable for novice, less technical users who may have less demand for specialized organizational requirements in different tools.  % Furthermore, it is argued that such users may welcome cross-tool organizational consistency and avoid the complex user model resulting from mirroring some folders and not others.  
Furthermore, it is hypothesised that novice users may welcome the simplicity of mirroring all folder events without prompting.  The author still holds this view, despite the comments by participants F3 and M6 regarding possible downsides for novice users from increased consistency.    Ideally, a repeat evaluation should be carried out with less technical users.  However this is outside the scope of this thesis, and is discussed in \textbf{Chapter~\ref{chapter:conclusion}} as possible future work.












%%%%%%%%%%%%%%%%%%%%%%%%%%%%%%%%%%%%%%%%%%%%%%%%%%%%%%%%%%%%%%%%%%%%%%%%%%%%%%%%%%%%%%%%%%
%%%%%%%%%%%%%%%%%%%%%%%%%%%%%%%%%%%%%%%%%%%%%%%%%%%%%%%%%%%%%%%%%%%%%%%%%%%%%%%%%%%%%%%%%%
%%%%%%%%%%%%%%%%%%%%%%%%%%%%%%%%%%%%%%%%%%%%%%%%%%%%%%%%%%%%%%%%%%%%%%%%%%%%%%%%%%%%%%%%%%
%%%%%%%%%%%%%%%%%%%%%%%%%%%%%%%%%%%%%%%%%%%%%%%%%%%%%%%%%%%%%%%%%%%%%%%%%%%%%%%%%%%%%%%%%%

%%%%%%%%%%%%%%%%%%%%%%%%%%%%%%%%%%%%%%%%%%%%%%%%%%%%%%%%%%%%%%
\subsubsection{Success of the Incremental Design Approach}
%%%%%%%%%%%%%%%%%%%%%%%%%%%%%%%%%%%%%%%%%%%%%%%%%%%%%%%%%%%%%%
% DESIGN METH SUCCESS: Success of incremental design approach (PLACEMENT: here or method discussion?)
% out the need for the wholesale importing of data, or tool change-over
%Limit of incremental design. \textbf{BUT: Consider underlying problems - have they been dealt with?}
%e.g. email overload - not really is honest answer. Brainstorm how this could be done. Need to be more ambitious and configurable.  This was not an attempt to create the perfect PIM-tool, but an attempt to explore some issues.  Limitation of incremental design: users want more!
% All the evaluation work was performed by the candidate acting on his own.
% Although incremental design does make evaluation more tractable in terms of both development and analysis.  
% Note that WM did not seek to tackle many of the underlying problems of PIM, such as email overload.  There was some suggestion for some users that they felt more satisfied due to the increased organizational consistency afforded by WM.  However, others such as 
The incremental design approach was successful in allowing the prototype to be added to an existing set of PIM-tools, with a minimum of disruption to the users.  However, the limitation of \textit{local optimization}~\citep{Carroll:00} is acknowledged.  WM was only aimed at dealing with one limited PIM-related problem, that of keeping multiple collections of personal information organized.  In several cases, it was difficult to manage the expectations of some participants who tended to focus on the functionality which WM did not have (e.g. more advanced PIM-integration mechanisms), than on what the presented design offered. Participant M7 wanted more advanced WM-like functionality based on folder-sharing.  Several other participants expressed a desire for a secretary agent, F3: \textit{``that can look through things and work out what they are and where they should go''}.

This highlights a key challenge for designers pursuing an incremental approach: users may not be satisfied with incremental improvements.


%%%%%%%%%%%%%%%%%%%%%%%%
\subsubsection{Summary}
%%%%%%%%%%%%%%%%%%%%%%%%

%%%%%%%%%%%%%%%%%%%%%%%%%%%%%%%%%%
% AIM 1: EVALUATION
%%%%%%%%%%%%%%%%%%%%%%%%%%%%%%%%%%
%The next section moves on to make generalized recommendations for PIM tool design.
% Firstly, it provided the context for the evaluation of the WM prototype presented in \textbf{Chapter~\ref{chapter:design}}.
% This idea is developed in \textbf{Chapter~\ref{chapter:discussion}}.

This section considered the results from the evaluation of the WM prototype.    Although responses to WM were mixed, a range of design recommendations were provided by all participants.  In particular, the majority of participants expressed their support for the mirroring of top-level folders.  These results raised the researcher's awareness of the pros and cons of integration.  In the next chapter, \textbf{Section~\ref{discussion:design-guidelines-discussion}} moves on to present general implications for the design of PIM systems, particularly those aimed at increasing integration. % \textit{In the next chapter,
% \textit{Now THINK: which findings are specific to my tool? Which can be generalized more widely for the Design of PIM Systems?}
% NEXT SECTION Then try to generalize beyond WM in next section. Generalizing WM-specific findings, towards recommendations?
Routes for formative redesign, driven by the evaluation feedback, are outlined as potential future work in \textbf{Chapter~\ref{chapter:conclusion}}.   
%%%%%%%%%%%%%%%%%%%%%%%%%%%%%%
% FIN@ CHAPTER 7 DESIGN/EVAL DISCUSSION
%%%%%%%%%%%%%%%%%%%%%%%%%%%%%%