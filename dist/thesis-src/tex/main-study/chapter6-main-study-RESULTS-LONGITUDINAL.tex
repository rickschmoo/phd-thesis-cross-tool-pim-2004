%%%%%%%%%%%%%%%%%%%%%%%%%%%%%%%%%%%%%%%%%%%%%%%%
% CHAPTER 6 -- MAIN STUDY
% RESULTS FROM LONGITUDINAL DATA
% File: tex/main-study-chapter/chapter6-main-study-RESULTS-LONGITUDINAL.tex
%%%%%%%%%%%%%%%%%%%%%%%%%%%%%%%%%%%%%%%%%%%%%%%%%%%%%%%%%%%%%%%%%%%
%%%%%%%%%%%%%%%%%%%%%%%%%%%%%%%%%%%%%%%%%%%%%%%%%%%%%%%%%%%%%%%%%%%%%%%%%%%%%%%%%%%%%%%%%%
% Richard Boardman PhD Thesis: Improving Tool Support for Personal Information Management
%%%%%%%%%%%%%%%%%%%%%%%%%%%%%%%%%%%%%%%%%%%%%%%%%%%%%%%%%%%%%%%%%%%%%%%%%%%%%%%%%%%%%%%%%%
%%%%%%%%%%%%%%%%%%%%%%%%%%%%%%%%%%%%%%%%%%%%%%%%%%%%%%%%%%%%%%%%%%%%%%%%%%%%%%%%%%%%%%%%%%
% NATBIB NOTES
%%%%%%%%%%%%%%%%%%%
%\citet{jon90}                ->    Jones et al. (1990) 
%   \citet[chap.~2]{jon90}       ->    Jones et al. (1990, chap. 2)
%   \citep{jon90}                ->    (Jones et al., 1990) 
%   \citep[chap.~2]{jon90}       ->    (Jones et al., 1990, chap. 2) 
%%%%%%%%%%%%%%%%%%%%%%%%%%%%%%%%%%%%%%%%%%%%%%%%%%%%%%%%%%%%%%%%%%%%%%%%%%%%%%%%%%%%%%%%%%

%%%%%%%%%%%%%%%%%%%%%%%%%%%%%%%%%%%%%%%%%%%%
%%%%%%%%%%%%%%%%%%%%%%%%%%%%%%%%%%%%%%%%%%%%
\newpage
\section{Other Study Findings}
\label{main-study:longitudinal}
%%%%%%%%%%%%%%%%%%%%%%%%%%%%%%%%%%%%%%%%%%%%
%%%%%%%%%%%%%%%%%%%%%%%%%%%%%%%%%%%%%%%%%%%%
% THINK: relate to case studies???
% THINK: DIAGRAMS to characterise users
% Follow-up to exploratory study, focus on longitudinal aspects
%%%%%%%%%%%%%%%%%%%%%%%%%%%%%%%%%%%%%%%%%%%%%%%%%%%%%%%%%%%%%
% Characterize ``normal'' pre-intervention PIM practices from logitudinal perspective
%%%%%%%%%%%%%%%%%%%%%%%%%%%%%%%%%%%%%%%%%%%%%%%%%%%%%%%%%%%%%
%  (EITHER (a) not related directly to evaluation of WorkspaceMirror from the longitudinal tracking data (both pre and post-intervention). Findings related to PIM over time/longitudinal issues are discussed.
This section presents results from the field trial that were driven by the second objective of the study -- to investigate longitudinal aspects of PIM.
% A key consideration in any field trial is the \textit{typicality} of the study period.
% These include events such as holidays, and software problems.
\textbf{Section~\ref{main-study:results:typicality}} reports the \textit{external factors} that affected PIM behaviour during the study.  Then, \textbf{Section~\ref{main-study:results:growth}} analyses the \textit{growth rates} of the file, email, and bookmark collections. Next, \textbf{Section~\ref{main-study:results:changes}} reports participants' \textit{changes in organizing strategy}.  Finally, \textbf{Section~\ref{main-study:results:reflection}} presents findings highlighting the \textit{background nature of PIM}. % The roles of both the study and design intervention in promoting reflection on PIM are observed.
% lack of reflection participants typically directed towards PIM -- and the increases due to the study participation.


%%%%%%%%%%%%%%%%%%%%%%%%%%
\subsection{External Factors}
\label{main-study:results:typicality}
%%%%%%%%%%%%%%%%%%%%%%%%%%

% Is any period truly typical?
A key consideration in any field trial is the \textit{typicality} of the study period.  
The following external factors were noted as having an influence over participants' PIM activities:
\begin{itemize}

\item \textit{Christmas holidays} -- The study ran from Summer 2002 to Spring 2003, and therefore included the Christmas holidays, as well as other ad-hoc holidays for various participants.

\item \textit{Natural lulls in computer usage} -- Both M7 and M8 noted that the study coincided with a period of time in which they made relatively little use of their primary work computer. % for PIM.  % Such lulls have an impact on the amount of PIM that is performed.

% not ideal, the change meant that he had to discontinue usage of WM, it did provide a case study of what happens during such ``life changes''. Also illustrates that this is real data.
\item \textit{Major life events} -- Two months into the study, participant F1 changed job. However, after a short interruption, he continued to take part. During the move he transferred his file and email collections, and restarted his bookmark collection.


\item \textit{Other computer problems} -- Participants F2 and M6 had to perform complete reinstalls of their operating systems for reasons unrelated to the study or the WM design intervention.  Fortunately neither lost any data, but in each case their file system was recreated, and consequently some date-related metadata was lost.

\end{itemize}

The range of unforeseen events that occurred highlight the challenges of performing long-term field studies. Many of these were beyond the control of the researcher, and were the consequence of working in an uncontrolled real-world context.
% Ideally the researcher wants to increase his or her understanding over a period of time that reflects a typical period of computer usage.  
% However they are the price one pays for doing a study like this in a real-world context.
% author fully acknowledges the design intervention due to the installation of WM halfway through the study.
Additionally, the effect caused by the WM design intervention is acknowledged, and is taken into consideration as a factor that contributed towards changes in organizing strategy.

%%%%%%%%%%%%%%%%%%%%%%%%%%%%%%%%%%%%%%%%%%%%%%%%%%%%%%%%%%%%%%%%%%%%%%%%%%%%%%%%%%%%%%%%%%%%%%%%%%%%%%%%%%%%
%%%%%%%%%%%%%%%%%%%%%%%%%%%%%%%%%%%%%%%%%%%%%%%%%%%%%%%%%%%%%%%%%%%%%%%%%%%%%%%%%%%%%%%%%%%%%%%%%%%%%%%%%%%%
%%%%%%%%%%%%%%%%%%%%%%%%%%%%%%%%%%%%%%%%%%%%%%%%%%%%%%%%%%%%%%%%%%%%%%%%%%%%%%%%%%%%%%%%%%%%%%%%%%%%%%%%%%%%
%%%%%%%%%%%%%%%%%%%%%%%%%%%%%%%%%%%%%%%%%%%%%%%%%%%%%%%%%%%%%%%%%%%%%%%%%%%%%%%%%%%%%%%%%%%%%%%%%%%%%%%%%%%%

%%%%%%%%%%%%%
% GROWTH
%%%%%%%%%%%%%
%%%%%%%%%%%%%%%%%%%%%%%%%%%%%%%%%%%%%%
\subsection{Growth in Collections}
\label{main-study:results:growth}
%%%%%%%%%%%%%%%%%%%%%%%%%%%%%%%%%%%%%%
% THINK: to what extent should this be related to WM?
% Problems: OS reinstalls, holidays etc. 9as above)
% \item ADD: Analyse use of folders over time. Illustrate how much of collection in use? (can I use accessed?) Compare proportion (and location) of active folders. Then can say that most of collection effectively archived in situ. Slow change - lead into import of persistency below
%\begin{itemize}
%	\item proportion old / used
%	\item proportion new / active
%	\item proportion old / inactive
%\end{itemize}

%%%%%%%%%%%%%%%%%%%%%%
% GROWTH RATE METHOD
%%%%%%%%%%%%%%%%%%%%%%
The capture of folder snapshots during the study enabled the calculation of the growth in the three collections.  Changes were calculated as follows:
% \textit{Which files and folders were included?}
\begin{itemize}
\item \textit{Item growth rate} -- The net change in the total number of items in a collection, per day of participation.
\item \textit{Folder growth rate} -- The net change in the total number of folders in a collection, per day of participation.
\end{itemize}

% Note that both rates relate to net changes in numbers of items and folders respectively (i.e. creations minus deletions).  
The growth rates relate to the period of time in which the WorkspaceSnapper tool was installed on the participant's computer. For the \textit{track 2} participants (M5--M8), this equated to the entire study period (stages 1--6).  Tracking varied for the \textit{track 1} participants.  Participants F2, F3 and F4 were monitored from partway through stage 4 until the end of the study.  Participant F1 changed job before monitoring commenced, and so his figures relate to stage 6 after his job change.

The following procedures were employed when measuring the growth rates in each collection:
\begin{itemize}
\item \textit{Files} -- Two participants, F3 and M7,  split their file management between two areas of the file system.  For these two, the two areas were treated as a single combined collection.  For all participants, areas of the personal file collection used solely for code development, simulations, temporary storage, and downloaded software were not included.  In these cases, only the root folder and the items stored there were counted.

\item \textit{Email} -- Except for the `Inbox', default folders and items contained therein were excluded.  For MS-Outlook, these included `Sent Items', `Drafts', `Outbox' and `Trash'.

\item \textit{Bookmarks} -- Again, default folders and items contained therein were excluded. In MS-Internet Explorer, these included `Media', and `Channels'.

\end{itemize}

The intention was to compare the growth rates for the primary collections of files, email and bookmarks.  It is acknowledged that the figures do not take into account other collections: all participants stated that they had personal files, email and bookmarks on other computers (e.g. web-based email collections).  However, the aim in this study was to obtain an indicative comparison based on the primary collections only.  \textbf{Table~\ref{table:main-study:growth-rates-indiv}} summarizes average changes in collection size.  Each collection is discussed in turn below.

%%%%%%%%%%%%%%%%%%%%%%%%%%%%%%%%%%%%%%%%%%%%%%%%%%%%%%%
% TABLE: changes in PIM collections over Phase 2
%%%%%%%%%%%%%%%%%%%%%%%%%%%%%%%%%%%%%%%%%%%%%%%%%%%%%%%
%\begin{table}[hbtp]
%\begin{center}
%\begin{footnotesize}
%\setlength{\extrarowheight}{2pt}
% \begin{tabular}{|c|p{2.5cm}|p{2.5cm}|c|c|c|c|c|p{2.5cm}|}
% Table generated by Excel2LaTeX from sheet '6 MAIN STUDY CHANGES'
%\begin{tabular}{|p{2.5cm}|p{2cm}|p{2cm}|p{3cm}|p{3cm}|}
%\hline
%{\bf Collection} & {\bf \# folders created per day (average)} & {\bf \# items added per day (average)} & {\bf Total change \# folders (average over monitored period, average \% change in brackets)} & {\bf Total change \# items (average over monitored period, average \% change in brackets)} \\
%\hline
%Files (all participants, n=8) &       0.35 &       5.92 & 100 (+98.6\%) & 1764 (+164.6\%) \\
%\hline
%Email (non-archivers only, n=4) &       0.06 &       5.28 & 12.75 (+64.5\%) & 1551 (+75.6\%) \\
%\hline
%Bookmarks (all participants, n=8) &       0.03 &        0.2 & 9.75 (+41.9\%) & 60.5 (+41.8\%) \\
%\hline
%\end{tabular}  
%\end{footnotesize}
%\caption{Average growth rates of file, email and bookmark collections}
%\label{table:main-study:growth-rates}
%\end{center}
%\end{table}
%\normalsize


%%%%%%%%%%%%%%%%%%%%%%%%%%%%%%%%%%%%%%%%%%%%%%%%%%%%%%%
% TABLE: changes in PIM collections over Phase 2
%%%%%%%%%%%%%%%%%%%%%%%%%%%%%%%%%%%%%%%%%%%%%%%%%%%%%%%
% CHECK COLUN ITALICIZED AS APPROPRIATE
\begin{table}[hbt]
\begin{center}
\begin{footnotesize}
\setlength{\extrarowheight}{2pt}
% Table generated by Excel2LaTeX from sheet '6 MAIN STUDY CHANGES'
\begin{tabular}{|p{1.6cm}|p{1.7cm}|p{1.7cm}|p{1.8cm}|p{1.8cm}|p{1.7cm}|p{1.7cm}|}
\hline
{\bf Participant} & {\bf \# file folders added per day (average)} & {\bf \# file items added per day (average)} & {\bf \# email folders added per day (average)} & {\bf \# email items added per day (average)} & {\bf \# BM folders added per day (average)} & {\bf \# BM items added per day (average)} \\
\hline
        F1 &       0.17 &       2.05 &       0.07 &       6.08 &       0.01 &       0.06 \\
\hline
        F2 &       0.22 &       7.43 &       0.12 &       6.14 &       0.06 &       0.08 \\
\hline
F3 (archived files) & {\it 0.06 (archiver)} & {\it -0.19 (archiver)} &       0.03 &       3.05 &       0.04 &       0.17 \\
\hline
        F4 &       0.06 &       1.79 &          0 & {\it not recorded} &       0.13 &          1 \\
\hline
M5 (archived files) & {\it 0.01 (archiver)} & {\it 5.97 (archiver)} &       0.08 &        9.5 &       0.01 &       0.11 \\
\hline
M6 (archived email) &        0.4 &       9.58 & {\it 0.01 (archiver)} & {\it -4.26 (archiver)} &          0 &       0.03 \\
\hline
        M7 &       0.13 &       4.14 &       0.01 &       5.83 &          0 &       0.08 \\
\hline
M8 (archived email) &       0.12 &       1.67 & {\it 0.04 (archiver)} & {\it -21.02 (archiver)} &          0 &       0.09 \\
\hline
   \textbf{Average} &       \textbf{0.15} &       \textbf{4.06} &       \textbf{0.05} &       \textbf{0.76} &       \textbf{0.03} &       \textbf{0.20} \\
\hline
        SD &       0.12 &       3.35 &       0.04 &      10.52 &       0.05 &       0.32 \\
\hline
       Minimum &       0.01 &      -0.19 &       0.00 &     -21.02 &       0.00 &       0.03 \\
\hline
       Maximum &       0.40 &       9.58 &       0.12 &       9.50 &       0.13 &       1.00 \\
\hline
     Notes & {\it Average without archivers [n=6]: 0.18 (SD 0.13)} & {\it Average without archivers [n=6]: 4.44 (SD 3.11)} & {\it Average without archivers [n=6]: 0.05, (SD 0.05)} & {\it Average without archivers [n=5]: 6.12(SD 2.29)} &            &            \\
\hline
\end{tabular}   
\end{footnotesize}
\caption{Growth rates of file, email and bookmark collections}
\label{table:main-study:growth-rates-indiv}
\end{center}
\end{table}
\normalsize

%%%%%%%%%%%%%%%%%%%%%%%%%%%%%%%%%%%%%%%%
\subsubsection{Growth of File Collections}
%%%%%%%%%%%%%%%%%%%%%%%%%%%%%%%%%%%%%%%%

The average growth rate for files was 0.15 folders, and 4.06 items, per day of participation.  File collections increased in size (in terms of folders and items) for all but one of the participants (F3).  For F3, the number of files decreased due to the archiving he performed, transferring a number of items to his home computer and network drives.    M5 also archived to save on space on his network drive.  Despite his archiving, M5's collection increased in terms of both files and folders over the course of the study.

Other participants reported that they performed archiving very rarely, e.g. M7: \textit{``I've done one spring-clean in the past year ... my past experience with computers is that I tend not to''}.   In contrast to F3 and M5, many archived material (such as websites) \textit{into} their file collections, which contributed towards the high growth rate.  One mentioned reason for not archiving material out of collections was that it made items harder to find at a later date. % This runs counter to the traditional definition of archiving as the removal of items from a collection, e.g.~\citep{barreau:95}.
Average growth rates without the two archivers were 0.18 for folders, and 4.44 for items.  


%%%%%%%%%%%%%%%%%%%%%%%%%%%%%%%%%%%%%%%%
\subsubsection{Growth of Email Collections}
%%%%%%%%%%%%%%%%%%%%%%%%%%%%%%%%%%%%%%%%

Due to technical difficulties it was not possible to collect message counts for participant F4.  This was due to the incompatibility of WS with the storage format in her MS-Outlook Express email tool\footnote{M5 also started the study using the same email tool, but switched to MS-Outlook two weeks into stage 2.}.  Across the remaining seven participants, average email growth was 0.76 for items.  The low item growth rate was due to two participants (M6 and M8) who archived an average of 3600 messages out of their collections in one-off tidying sessions.  Without the non-archivers, item growth rate increased to 6.12 per day.

Compared to files, a smaller average \textit{folder} growth rate was observed (0.05 compared to 0.15 folders per day for files). All participants except one (M5) had greater folder creation rates in files compared to email, reflecting a greater emphasis on filing in the file context.  M5's file folder growth rate was impacted by his file archiving activity.
%This was accompanied by a very large \textit{turnover} in messages. The overall net change in items was positive or negative depending on whether the participant archived messages out of the collection:
%\begin{itemize}
%\item \textit{Archivers} -- The overall change in the number of messages was negative for two participants (M6 and M8) who archived an average of 3600 messages out of their collections in one-off archiving during the study.  
%\item \textit{Non-archivers} -- Growth in messages was positive for the other four participants who did not archive in this way.  Note that only the non-archivers (F1, F2, F3, and M7) are included in the email data in \textbf{Table~\ref{table:main-study:growth-rates-indiv}}.
%\end{itemize}



%%%%%%%%%%%%%%%%%%%%%%%%%%%%%%%%%%%%%%%%
\subsubsection{Growth of Bookmark Collections}
%%%%%%%%%%%%%%%%%%%%%%%%%%%%%%%%%%%%%%%%

For most participants, bookmark collections grew very slowly.  Participant F4 was the one exception.  She collected bookmarks extensively, having a daily growth rate of 0.13 folders, and 1 item per day.  

The remaining seven participants had a much lower growth rate in bookmarks compared to the other collections.  The average rates for these seven were 0.02 per day for folders, and 0.09 for items.   % Seven of the participants collected bookmarks very slowly.  For these participants the average growth rates were 



%%%%%%%%%%%%%%%%%%%%%%%%%%%%%%%%%%%%%%%%
% \subsubsection{Comparison between collections}
%%%%%%%%%%%%%%%%%%%%%%%%%%%%%%%%%%%%%%%%
% MOVED TO DISCUSSION



%%%%%%%%%%%%%%%%%%%%%%%%%%%
% LINK ON
%%%%%%%%%%%%%%%%%%%%%%%%%%%
% \textit{Blah de blah blah}
Later in this chapter, \textbf{Section~\ref{main-study:discussion:study}} discusses these findings in depth.  


%%%%%%%%%%%%%%%%%%%%%%%%%%%%%%%%%%%%%%%%%%%%%%%%%%%%%%%%%%%%%%%%%%%%%%%%%%%%%%%%%%%%%%%%%%%%%%%%%%%%%%%%%%%%
%%%%%%%%%%%%%%%%%%%%%%%%%%%%%%%%%%%%%%%%%%%%%%%%%%%%%%%%%%%%%%%%%%%%%%%%%%%%%%%%%%%%%%%%%%%%%%%%%%%%%%%%%%%%
%%%%%%%%%%%%%%%%%%%%%%%%%%%%%%%%%%%%%%%%%%%%%%%%%%%%%%%%%%%%%%%%%%%%%%%%%%%%%%%%%%%%%%%%%%%%%%%%%%%%%%%%%%%%
%%%%%%%%%%%%%%%%%%%%%%%%%%%%%%%%%%%%%%%%%%%%%%%%%%%%%%%%%%%%%%%%%%%%%%%%%%%%%%%%%%%%%%%%%%%%%%%%%%%%%%%%%%%%

%%%%%%%%%%%%%
% CHANGES
%%%%%%%%%%%%%
%%%%%%%%%%%%%%%%%%%%%%%%%%%%%%%%%%%%%%
\subsection{Changes in Organizing Strategy}
\label{main-study:results:changes}
%%%%%%%%%%%%%%%%%%%%%%%%%%%%%%%%%%%%%%



%%%%%%%%%%%%%%%%%%%%%%%%%%%%%%%%%%%%%%%%%%%%%%%%%%%%%%%%%%%%%%%%%%%%%%%%%%
% Drive from change reports in phase 1 + reported changes from profile
%%%%%%%%%%%%%%%%%%%%%%%%%%%%%%%%%%%%%%%%%%%%%%%%%%%%%%%%%%%%%%%%%%%%%%%%%%
% Relate to findings in Phase 1: \textit{The study had an immediate "self-auditing" influence on the behaviour of most participants. Many indicated that taking part in the study had caused them to think more about PIM than they normally would, and to plan future strategy changes.
% Twelve performed ad-hoc tidying during the interviews, e.g. deleting files they had forgotten about.
% Types of changes involved those that had happened in the past, and those planned for the future.
During the exploratory study reported in \textbf{Chapter~\ref{chapter:exploratory_study}}, fourteen participants mentioned historical changes in how they managed information.  A key aim of the study reported in this chapter was to track and investigate changes in management strategies as they happened.  Firstly, reports made in stage 1 regarding  historical changes are described.  Then, changes in strategy that happened during the main study (i.e. during stages 2--6) are reported.

%%%%%%%%%%%%%%%%%%%%%%%%%%%%%%%%%%%%%%
\subsubsection{Reports of Historical Changes}
%%%%%%%%%%%%%%%%%%%%%%%%%%%%%%%%%%%%%%

%%%%%%%%%%%%%%%%%%%%%%%%%%%%%%%%%%%%%%%%%%%%%%
% AND REFER TO CHANGES IN MAIN STUDY ITSELF
%%%%%%%%%%%%%%%%%%%%%%%%%%%%%%%%%%%%%%%%%%%%%%
Five participants mentioned historical changes from before the study.
%%%%%%%%%
% FILES
%%%%%%%%%
% M8: historical change
% previous change: I went through a phase of completely working on the desktop of my laptop but that's another story, an attempt to work in a different way ... putting everything on the desktop, but it gets very cluttered ... I ended up just starting again, I don't do that anymore. [PIM/CHANGE]
In the file collection, two participants (F2 and M8), reported significant increases in organizing tendency, e.g. M8: \textit{``I went through a phase of completely working on my desktop but it gets very cluttered''}.  In contrast, F4 reported filing more in the past when she \textit{``had more time''}.
%. I went through a phase of completely working on the desktop of my laptop but that's another story, an attempt to work in a different way ... putting everything on the desktop, but it gets very cluttered ... I ended up just starting again, I don't do that anymore

%%%%%%%%%
% EMAIL
%%%%%%%%%
In email, participant M6 reported an increase in organizing tendency, \textit{``I used to have them all in my inbox - and I didn't know who I'd answered - well I do know who I've answered but its like emails would disappear off the end of the list and you never see them again. I got sick of worrying about that. Now I get an email, I put it in the right folder when I've answered it - and when I get too many in the inbox I know that I need to answer some''}. 
% How? - strategy? Friends, family, PhD, `keep for reference' (that's stuff that I definitely need to look at again at some stage. For example I got an email once giving me the number I need to call to access my one-tel account from NTL so I put it in there in case I need it again, or registration details go in there. If you order something and you get an email, it goers in there. Very much rely on email.''}.

%%%%%%%%%%%%%
% BOOKMARK
%%%%%%%%%%%%%
Three participants (F1, F4, M8) reported changes in the bookmark context, which involved the abandonment of substantial numbers of folders.  Whilst F1 and F4 continued to file within new folder structures, M8 abandoned filing altogether, \textit{``I started off with a pre-set idea of where things should go, but when the information came along it didn't go anywhere, so I ended up just creating lots of folders and then realised there was no point doing that and stopped''}.
% , e.g. participant M8 reported abandoning all folders.
%What kind of categories did you have? I've tried different approaches ... I've tried I don't know - Work, Pleasure, Shopping ... I've tried topic areas stuff - stuff about knowledge management or something ... difficult to know what category, what they should be. At the end of the day I've given up ...
% Hierarchy before? Again I've been through different things - I started with a fairly deep structure to start with and its got flatter as I went on and in the end I just gave up as it was a one level hierarchy. And then it became a single folder.
% Why the transition? Erm ... I just found that I was bookmarking things ... I started off with a structure that made sense and then I'd be bookmarking something that didn't fit into that structure so I'd have to extend it and change it, so it became wider and flatter. It was really an idea - it started off with a preset idea of where things should go - but when the information came along it didn't go anywhere so I ended up just creating lots of folders and then realised there was no point doing that and stopped.


%%%%%%%%%%%%%%%%%%%%%%%%%%%%%%%%%%%%%%%%%%%%%%%%%%%%%%%%%%%%%%%%%%%%%%%%%%%
\subsubsection{Identifying Strategy Changes during the Study}
%%%%%%%%%%%%%%%%%%%%%%%%%%%%%%%%%%%%%%%%%%%%%%%%%%%%%%%%%%%%%%%%%%%%%%%%%%%

\textbf{Section~\ref{main-study:method:data-analysis}} described how a workspace evolution transcript was constructed for each participant.  This transcript, combined with analysis of the interview data, allowed the monitoring of organizing strategies over time.
Based on this analysis by the author, no major shifts in organizing tendency were observed, along the lines of those discussed in~\cite{ob:97}.  However, in stage 6, when the participants were asked whether they had changed their organizing strategies, two said yes (F3 and M5).  In other words, it was primarily left to participants to self-identify changes.  % This was supported by observations on the part of the researcher. % \textit{

The eight participants could be split into two groups depending on whether they reported changing strategy or not: (1) \textit{non-changers}, and (2) \textit{changers}.




%%%%%%%%%%%%%%%%%%%%%%%%%%%%%%%%
\subsubsection{Non-changers}
%%%%%%%%%%%%%%%%%%%%%%%%%%%%%%%%

In the closing interview, six participants said that their strategies had not changed over the course of the study: F1, F2, F4, M6, M7 and M8. This group of \textit{non-changers} included participants with a range of organizing tendencies.  Whether pro-organizing or not, existing strategies were seen to be satisfactory or not worth changing:
\begin{itemize}
\item Four of the non-changers (F1, F2, F4 and M6)  remained broadly pro-organizing in all 3 tools. Note that this group included participant F1 who changed job during the study.  In the move, he transferred his file and email collection, organizing them within the same folder structure as before.  With bookmarks, he abandoned his existing collection, and started again from scratch.  However, since his strategy was pro-organizing as before, it was not classed as a change in strategy.

\item Participant M8 expressed the desire to reorganize his files and start using WM.  However, he observed that the effort in reorganizing his files was too high to be worth performing, \textit{``Although the system I use at the moment is broken, I know how it works. To take on board a new system, I'd basically have to deal with 2 systems for a while''}. % M8 predicted future changes but had not got round to making them yet.  % F8: partially because working on another machine, partially because I tend to have a structure which I put things in, keep that for a period of time, then have a spring-clean and change the structure. And I haven't been through the process of changing the structure with WorkspaceMirror.  Its something that I'll do eventually, but I just haven't done my reorganization recently.  
% F8: I'd like to go through this process of reorganization and use it ... Its something that I'll do eventually, but I just haven't done my reorganization recently.  I think its something that would be useful if that had happened, and I'd think right - gotta do it. Its something for me that I think needs a wholesale rethink - its very difficult to flow into, from my existing work habits.
% Over the course of the study the way you manage information has changed at all? No. Are they fine? I think its to do with the fact that although the system I use at the moment is broken, I know how it works. To take on board a new system, for you know, partially - whilst still having all my stuff in the broken system, is quite a cognitive overhead if you like. I'd basically have to deal with two systems for a while until time passes and I've got everything in the new system and the old stuff becomes obsolete. So either I sit down and go through the old stuff - "OK I'm going to reorganize it and put it in the new system" or I have to struggle with two systems which is kind of the point of the study but ... how you make that transition is a difficult question.

\item Participant M7 remained broadly organizing-neutral in all three tools. In fact, he only organized items within the file collection over the course of the study. Unlike M8, he was highly satisfied with his strategies and felt no need to be more organized.  
\end{itemize}


%%%%%%%%%%%%%%%%%%%%%%%%%%%
\subsubsection{Changers}
%%%%%%%%%%%%%%%%%%%%%%%%%%%

Two participants reported changing strategy over the course of the study: F3 and M5.

%%%%%%%%%%%%
% F3
%%%%%%%%%%%%
% MY DOCS +3 file folders, 3 mirrored to em, move 2 within my docs, moved 6 to desktop (old work to `WorkArchive' and `Downloaded Papers' to root)
% DESKTOP +7 file folders (including `Thesis', `WIP' and `WorkArchive'
Participant F3 (CT3: F1/E3/B2) reorganized his files during stage 4, two months into the study. He moved many of his working files onto his desktop: \textit{``I'd feel more comfortable using My Documents as an archive position and the Desktop as my working area''}.  Until then, he had pursued a predominantly \textit{file-on-creation} strategy under `My Documents'.  He also stated that the change involved an increase in his level of organizing.  During the reorganization, he created 7 folders on the `Desktop', and moved 6 across from `My Documents'.  When asked about the reorganization, he highlighted two factors that contributed to the change: (1) the need to separate active files for synchronization with his laptop, and (2) the greater influence of the interviews over our design intervention: ``\textit{Overall the tool hasn't done that much, its more the conversations between me and you. It's weird because I've become much more aware of all my directory structures}''.  Later, he mentioned a third additional factor, \textit{``I think society puts certain pressures upon people to think that being organized, being slim, being certain things are good''}. This perceived social pressure might explain why most of the reorganizing happened in the context of a New Year resolution: \textit{``I feel the need to reinvent myself, to get some good working practices together, to stop drinking, stop smoking, fix bike, and organize my computer''}.



%%%%%%%%%%%%%%%%%%%%%%%%%%%%%%%%%%%%%%%%%%%%%%%%%%
% M5
% Planned for a while
% Study >> tool
% Activities shifting from phys to dig
%%%%%%%%%%%%%%%%%%%%%%%%%%%%%%%%%%%%%%%%%%%%%%%%%%
% FILES CHANGE STAGE 1
% stage 1: NEW: 1 AAA OLD 2 "/Outlook archives 3 BT Stuff 4 chi03/DC 5 chi03/presentation
% moved many old to AAA OLD
%%%%%%%%%%%%%%%%%%%%%%%%%%%%%%%%%%%%%%%%%%%%%%%%%%
% EMAIL: 
% NEW: 1 Old 2 "/Sent-pre nov02 3 AA Submit here 4 See and Do 5 Accepted Final Versions
% Major reorg of old stuff into Old
%%%%%%%%%%%%%%%%%%%%%%%%%%%%%%%%%%%%%%%%%%%%%%%%%%
% FILES CHANGE STAGE 2
% stage 2: NEW: 1 "hview" (link to system folder) 3 "sep 02 - shiny happy study" 4 "sep 02 - shiny happy study/data & analysis (most in hview)" 5 "sep 02 - shiny happy study/data & analysis (most in hview)/for paper" 6 "sep 02 - shiny happy study/presentation" 7 "teaching03" 8 "teaching03/3c12" 9 "teaching03/c362" 10 "teaching03/2b16"
% deleted 1, renamed/moved 5
%%%%%%%%%%%%%%%%%%%%%%%%%%%%%%%%%%%%%%%%%%%%%%%%%%
Participant M5 (CT2: F2/E2/B2) was the second participant to report a change.  He reorganized both his files and email two weeks into stage 2.  He identified the reorganizations as a significant change in strategy, towards being more organized.  In each collection he moved completed project folders under an `old' folder: \textit{``in the top level I have all the projects I have done in my PhD and they have become too many''}.  He also created a number of new top-level folders in each tool.  He stated that participation in the study was a major factor in the reorganization -- although he had been planning the changes for some time, previously he had been put off by the effort involved: \textit{``I went through the mental workload of categorizing things as important or not important [in the study], so whilst this information is fresh in my memory I might as well just use it''}. He also reported an increased reliance on filing for a task which had previously been paper-based: \textit{``the `submit here' [folder] is the most significant change because that is the first time that I archived stuff to remind me ... it was paper-based before''}. 

%%%%%%%%%%%%%%%%%%%%%%%%%%%%%%%
% \subsubsection{Mini Discussion}
%%%%%%%%%%%%%%%%%%%%%%%%%%%%%%%

It should be noted that for both F3 and M5 the changes did not amount to global changes in overall organizing tendency.  Instead they were incremental changes, representing subtle shifts in how certain types of information were organized.  However, for both participants, the changes, though subtle, were worthwhile, M5: \textit{``the [email] folders that I've created, they only take up 2\% or 4\% [of the inbox]. [Does that make a difference?] Yes, because most of the stuff that comes in is day-to-day stuff I deal with today, the things I extract now are things with a longer due time''}).
%%%%%%%%%%%%%%%%%%%%%%%%%%%%%%%%%%%%%%%%%
% link to settled/unsettled behaviour
%%%%%%%%%%%%%%%%%%%%%%%%%%%%%%%%%%%%%%%%%
\textbf{Section~\ref{main-study:discussion:study}} discusses these findings in more depth.






%%%%%%%%%%%%%%%%%%%%%%%%%%%%%%%%%%%%%%%%%%%%%%%%%%%%%%%%%%%%%%%%%%%%%%%%%%%%%%%%%%%%%%%%%%%%%%%%%%%%%%%%%%%%
%%%%%%%%%%%%%%%%%%%%%%%%%%%%%%%%%%%%%%%%%%%%%%%%%%%%%%%%%%%%%%%%%%%%%%%%%%%%%%%%%%%%%%%%%%%%%%%%%%%%%%%%%%%%
%%%%%%%%%%%%%%%%%%%%%%%%%%%%%%%%%%%%%%%%%%%%%%%%%%%%%%%%%%%%%%%%%%%%%%%%%%%%%%%%%%%%%%%%%%%%%%%%%%%%%%%%%%%%
%%%%%%%%%%%%%%%%%%%%%%%%%%%%%%%%%%%%%%%%%%%%%%%%%%%%%%%%%%%%%%%%%%%%%%%%%%%%%%%%%%%%%%%%%%%%%%%%%%%%%%%%%%%%


%%%%%%%%%%%%%%%%%%%%%%%%%%%%%%%%%%%
\subsection{Background Nature of PIM}
\label{main-study:results:reflection}
%%%%%%%%%%%%%%%%%%%%%%%%%%%%%%%%%%%

This final set of results presented here relate to the background, supporting nature of PIM.  This theme emerged during the content analysis of the qualitative interview data.

%F2: Certainly thinking more now about how I organize
%�	This is both good and bad
%�	Good: producing a better organization...?
%�	Bad: spending more time doing it!
% These results point to the \textit{supporting} nature of PIM, users rarely devote time to reflecting on how they manage information, e.g. by planning and executing changes in strategy. 

Whilst talking about their PIM practices, many of the study participants portrayed PIM as a necessary activity, yet one that was not considered to be ``real work''.  In fact, it was often considered to be a distraction, e.g. M6: \textit{``There is never enough time to manage stuff, and whatever time you do spend on it is often wasted ... its so easy to get distracted, I need to do some email, I need to organize this''}.

When asked how their time could be better spent, participants typically responded that they should be doing \textit{``real work''} rather than managing their information.  For example, participant F2 identified \textit{``writing papers''} as what he should be doing rather than managing files.

%%%%%%%%%%%%%%
% SHORT-TERM
%%%%%%%%%%%%%%
%%%%%%%%%%%%%%%%%%%%%%%%%%%%
% SATISFICING
%%%%%%%%%%%%%%%%%%%%%%%%%%%%
% The study provided several observations of the satisficing nature of PIM: users often do just enough to get by
Several participants highlighted the short-term, satisficing nature of PIM, e.g. F3: \textit{``I put things wherever is easiest, wherever is quickest''}.  Similarly, M5 observed, \textit{``It [my file organization] could definitely be improved further but it's a trade-off between time I'm willing to spend, and having something that works''}.  Participant F4 referred to how she typically had little time for PIM, meaning that it was often carried out in a rush, \textit{``When I've got a deadline, I save things anywhere. I plan to organize at a later date but never get round to it''}.  Later in the interview she mentioned that the resulting mess made her feel untidy and embarrassed.  Participant F2 also commented on problems resulting from his short-term approach to file management, \textit{``You  start working with something, create a new dedicated area, and then forget about it. Really I might be more wary ... Generally things start as a small project, when I'm not expecting very much, not expecting the  project to last very long. But then I have to live with the consequences of my poor short-term organization''}.  % Other participants noted the embarrassment they felt due to the lack of time they devoted to organization.

%F2 RESEARCHER: when you look at your folders, how do you feel?
%I know its not amazing
%�	I know I need to improve, but its sufficient you know? I'm happy. [PIM/SUPPORT/SATISFICING]
%RESEARCHER: why not improve?
%Low priority [PIM/SUPPORT]
%RESEARCHER: what's the priority?
%Writing papers! [PIM/SUPPORT]


%%%%%%%%%%%%%%%%%%%%%
% AUDITING EFFECT
%%%%%%%%%%%%%%%%%%%%%
As in the exploratory study, the field trial clearly had an ``auditing effect'' on the participants. Most indicated that taking part in the study had caused them to think more about PIM than they normally would, and to plan future strategy changes.  Both the study and the design intervention were mentioned as factors.

%%% F2: 

%%% F3: SATISFICING: Its strange � I would say I'm purposefully out of control in a weird way! I could but decide not to � I put things wherever is easiest, wherever is quickest.

%%% M5: 

%%% M: > SUPPORTING General problems in email: (1) don't have enough time, and (2) waste too much time on it! Well not always as a waste but email has one stupid thing � I'm thinking of doing one thing, not looking at email until the evening when I've finished the work. How often does it happen that you come into work in the  morning at 9 or 10, and you actually mention to go to lunch without actually doing any work. And you come back from lunch and there's another 5 replies in the inbox so there's more emailing to do! So you start working at 3 o' clock even though you came in at 9!

% I wish I could spend less time when I don't feel like working.
% Its so easy to get distracted: "I need to do some email, I need to organize this" 

%%%%%%%%%%%%%%%%%%%%%%%%%%%%%%%%%%%%%%%%
\subsubsection{Influence of the Study}
%%%%%%%%%%%%%%%%%%%%%%%%%%%%%%%%%%%%%%%%

Most performed some ad-hoc tidying during the interviews such as rediscovering files they had forgotten about, and filing or deleting them.  For example, participant F1 commented: \textit{``I also keep one file here, I don't know why, hmm ... Ah, I know why, that's the AVI2000 paper which was originally archived on my new hard disk but when I was writing another paper.  I needed that one so I copied it back and put it there temporarily. But I'd forgotten about it, so now its still there. Everything's got a reason - eventually''}.  Many participants deleted or organized a number of folders and items in the interviews, or made notes to do it later.

% REFLECTION BENEFITS
One interview highlighted the potential benefits from the additional time spent reflecting on his collections, F1: \textit{``Now this `ITN' folder, what's that? I don't know ... oh that gives me an idea, maybe we can approach ITN with this [funding application] although they are content-providers''}.
% WAS ITN A PROJECT? Hmmmm ... Bob got somebody's name - we were interested in the big screen on the wall and we found out that ITN were doing something similar ... so we contacted them - we had a few emails, I was hoping that it would grow into something more but it hasn't 

The two participants who made strategy changes (see \textbf{Section~\ref{main-study:results:changes}}), highlighted the increased reflection on PIM due to the study as a main factor in making those changes.  For example, participant F3 commented: \textit{``Overall the tool hasn't done that much, its more the conversations between me and you. It's weird because I've become much more aware of all of my directory structures -- there's a bit more thought trying to go into it and how it's organized and stuff -- but I think that's probably because of the conversations we have, i.e. being part of the user study than because of the tool ... I've been definitely trying harder to be neat.  But that's not because I'm trying to look nicer and smarter or something for the study, that's not what I'm saying ... it's that I've become more aware of its effect on productivity''}.


%%%%%%%%%%%%%%%%%%%%%%%%%%%%%%%%%%%%%%%%
\subsubsection{Influence of the Design Intervention}
%%%%%%%%%%%%%%%%%%%%%%%%%%%%%%%%%%%%%%%%

Several participants also reported that the prompting mechanism in WM made them think more about PIM, e.g. F2: \textit{``I think it's useful. It helped me put some order in my directory structure and it helped me to think about management a bit more. [Researcher: what do you mean by order?] In an indirect way, it forced me to think about why I was organizing my things the way I was doing. Here, I am trying to just manage project documents in the `Project Documents' area [of files]. I've moved the code elsewhere''}.  However he then observed the possible distraction from his work in spending more time thinking about PIM.

Participant M5 also acknowledged the increase in reflection offered by WM: \textit{``The act of being prompted ... is useful because it reminds you that you should think about how you should organize your stuff. But it's also someone saying ``clean up your room''}. However, instead of obeying the order, he interpreted it as a hint that he was doing less PIM, \textit{``When I get the WM message up it reminds me that I'm wasting time with information management, creating folders and putting things in folders''}.

\textbf{Section~\ref{main-study:discussion:study}} discusses the results presented in this section in more depth.

%%%%%%%%%%%%%%%%%%%%%%%%%%%%%%%%%%%%%%%%%%%%%%%%%%%%%%%%%%%%%%%%%%%%%%%%%%%%%%%%%%%%%%%%%%%%%%%%%%%%%%%%%%%%
%%%%%%%%%%%%%%%%%%%%%%%%%%%%%%%%%%%%%%%%%%%%%%%%%%%%%%%%%%%%%%%%%%%%%%%%%%%%%%%%%%%%%%%%%%%%%%%%%%%%%%%%%%%%
%%%%%%%%%%%%%%%%%%%%%%%%%%%%%%%%%%%%%%%%%%%%%%%%%%%%%%%%%%%%%%%%%%%%%%%%%%%%%%%%%%%%%%%%%%%%%%%%%%%%%%%%%%%%
%%%%%%%%%%%%%%%%%%%%%%%%%%%%%%%%%%%%%%%%%%%%%%%%%%%%%%%%%%%%%%%%%%%%%%%%%%%%%%%%%%%%%%%%%%%%%%%%%%%%%%%%%%%%


%%%%%%%%%%%%%%%%%%%%%%%%%%%%%%%%%%%%
% THINK: MOVE TO DISCUSSION?
%%%%%%%%%%%%%%%%%%%%%%%%%%%%%%%%%%%%
% \subsubsection{Highlight nature of PIM}
%%%%%%%%%%%%%%%%%%%%%%%%%%%%%%%%%%%%
% \textit{TODO: Add signposting to conclusion and discussion.}
% User attitudes to PIM: Useful/waste of time
% Intermingling of user roles in PIM tools
% Highlight nature of PIM - ongoing, supporting, fundamental, irrational

% ADD Lack of reflection?