\newpage
%%%%%%%%%%%%%%%%%%%%%%%%%%%%%%%%%%%%
\section{Three Perspectives on PIM}
\label{discussion:theoretical-framework}
%%%%%%%%%%%%%%%%%%%%%%%%%%%%%%%%%%%%
% \textbf{Section~\ref{discussion:supporting}} moves on to consider PIM as a \textit{supporting} activity.  Finally, \textbf{Section~\ref{discussion:ongoing}} discusses the \textit{ongoing} nature of PIM.
%%%%%%%%%%%%%%%%%%%%%%%%%%%%%%%%%%%%%%%%%%%%%%%%%%%%%%%%%%%%%%%%%%%%%%%%%%%%%%%%%%%%%%%%%%%%%%%%%%%%%%%%%%%%
%	\item Complex activity -- and many problems.  Highly personal -- can't go to the help desk for support!!  Its your own responsibility
%%%%%%%%%%%%%%%%%%%%%%%%%%%%%%%%%%%%%%%%%%%%%%%%%%%%%%%%%%%%%%%%%%%%%%%%%%%%%%%%%%%%%%%%%%%%%%%%%%%%%%%%%%%%
%\textbf{Nature of PIM: a complex beast!!} Complex, multi-faceted area. Even with scoping, lots of data.
% Highlighted difficulty of analysis/constructing a user model/requirements and issues involved (e.g. complexity of PIM, individual Differences)
%%%%%%%%%%%%%%%%%%%%%%%%%%%%%%%%%%%%%%%%%%%%%%%%%%%%%%%%%%%%%%%%%%%%%%%%%%%%%%%%%%%%%%%%%%%%%%%%%%%%%%%%%%%%

This section discusses three properties of PIM which emerged over the course of this research:
\begin{enumerate}
%%%%%%%%%%%%%%%%%%%%%%%%%%%%%%%%%%%%%%%%%
% 2. CROSS-TOOL DISCUSSION
%%%%%%%%%%%%%%%%%%%%%%%%%%%%%%%%%%%%%%%%%
% Barreau's PIM framework~\citep{barreau:95} is extended to encompass this view of PIM.
\item \textbf{Section~\ref{discussion:cross-tool}} considers PIM as a \textit{cross-tool} activity, one which is distributed across multiple tools such as files, email and bookmarks.  

%%%%%%%%%%%%%%%%%%%%%%%%%%%%%%%%%%%%%%%%%
% 4. SUPPORTING DISCUSSION
%%%%%%%%%%%%%%%%%%%%%%%%%%%%%%%%%%%%%%%%%
% The framework from the previous section is extended to reflect the relationship between PIM and a user's production activities.
\item \textbf{Section~\ref{discussion:supporting}} discusses PIM as a \textit{supporting} activity, and considers the relationship between it and a user's production activities. 

%%%%%%%%%%%%%%%%%%%%%%%%%%%%%%%%%%%%%%%%%
% 3. ONGOING DISCUSSION
%%%%%%%%%%%%%%%%%%%%%%%%%%%%%%%%%%%%%%%%%
% Two perspectives on PIM are highlighted for analyzing PIM: (1) 
% An extension to B�lter's model of strategy changes is proposed~\citep{ob:97}.
\item \textbf{Section~\ref{discussion:ongoing}} considers the \textit{ongoing} nature of PIM, viewing it from two longitudinal perspectives: (1) as a series of discrete events, (2) as a thread of continuous activity.

\end{enumerate}

% develops a theoretical framework reflecting three aspects of the nature of PIM that emerged from this work:  (1) PIM as a \textit{cross-tool} activity, (2) PIM as a \textit{supporting} activity, and (3) PIM as an \textit{ongoing} activity.  
% The next three section considers each perspective in turn, and incrementally build up a theoretical framework based on the model proposed in~\citet{barreau:95}.  


%%%%%%%%%%%%%%%%%%%%%%%%%%%%%%%%%%%%
% COMBINATION OF PERSPECTIVES
%%%%%%%%%%%%%%%%%%%%%%%%%%%%%%%%%%%%
% A number of design and methodological implications are derived based on each theoretical perspective. In addition, a number of promising directions are identified for the attention of future research.
%  which is used to provide a commentary on the work presented in the thesis.  
% A theoretical framework is built up by extending Barreau's PIM framework to encompass these perspectives. 
% Barreau's PIM model is extended to form a theoretical framework reflecting these three themes. The framework is constructed by the incremental extension of  over the three sections. The framework is referred to in subsequent chapters of this thesis to discuss directions for future work. 
% Each section presents evidence for the view of PIM it discusses, in terms of related findings from earlier in the thesis.  
Each perspective is motivated using findings from earlier chapters, and is then used to 
incrementally extend the conceptual framework of PIM from \textbf{Section~\ref{bg:pim-activity-cf}}.  
\textbf{Section~\ref{discussion:towards-theory}} discusses how the three perspectives indicate routes for future theory development in the area.   Within this chapter, the three perspectives are used to structure the discussion in subsequent sections.  % , and also indicate potential routes for future theory development.  Indeed, it is argued that each perspective indicates an area of theory that has been under-represented in the PIM literature, and it is argued that they are of high importance to designers and researchers.


%%%%%%%%%%%%%%%%%%%%%%%%%%%%%%%%
% \subsection{Aspects of PIM}
%%%%%%%%%%%%%%%%%%%%%%%%%%%%%%%%
%%%%%%%%%%%%%%
% OHTER: Idiosyncratic
%%%%%%%%%%%%%%
% Apply: methodological recommendations, eval findings
% Individual differences illustrated by responses to tool evaluation and range of strategies.
%%%%%%%%%%%%%%%%%%%%%%%%%%%%%%%%%%%%%%%%%%%
% Identify limitations of Barreau model
%%%%%%%%%%%%%%%%%%%%%%%%%%%%%%%%%%%%%%%%%%%
%The following limitations of Barreau's model are highlighted:
%\begin{itemize}
%\item Does not reflect PIM as a \textit{cross-tool} activity 
%\item Does not reflect PIM as an \textit{ongoing} activity 
%\item Does not reflect PIM as a \textit{supporting} activity 
%\item Does not reflect PIM as an \textit{idiosyncratic} activity.
%\item Does not reflect PIM as an \textit{irrational} activity.
%\item Barreau's break-down into four sub-tasks not necessarily that clear-cut, also inter-relationship between sub-activities is unclear.  In terms of definition, importance or frequency. In actual fact, quite fuzzy. % MOVE FROM CH4
%\item Use data to derive descriptive vocabulary for talking about personal information (e.g. working/non-working versus archived, personal versus system, own versus shared). Need for improvements in descriptive terminology for talking about PIM, including types of personal information
%\end{itemize}

%%%%%%%%%%%%%%
% Irrational
%%%%%%%%%%%%%%
% EVIDENCE: Provide quotes
% DISCUSS: uestion assumptions of efficiency-based approaches and optimization in e.g. info foraging, Balter, Kirsh, 
% METH REC: usability measures, Implications for evaluation
% DESIGN RECS: Implications for ``on-demand'' PIM -- its not just about efficiency, its also how users feel

%%%%%%%%%%%%%%%%%%%%%%%%%%%%%%%%%%%%%%%%%%%%%%%%%%%%%%%%%%%%%%%%%%%%%%%%%%%%%%%%%%%%%%%%%%%%%%%%%%%%%%%%%%%%%%%%%
%%%%%%%%%%%%%%%%%%%%%%%%%%%%%%%%%%%%%%%%%%%%%%%%%%%%%%%%%%%%%%%%%%%%%%%%%%%%%%%%%%%%%%%%%%%%%%%%%%%%%%%%%%%%%%%%%
%%%%%%%%%%%%%%%%%%%%%%%%%%%%%%%%%%%%%%%%%%%%%%%%%%%%%%%%%%%%%%%%%%%%%%%%%%%%%%%%%%%%%%%%%%%%%%%%%%%%%%%%%%%%%%%%%
%%%%%%%%%%%%%%%%%%%%%%%%%%%%%%%%%%%%%%%%%%%%%%%%%%%%%%%%%%%%%%%%%%%%%%%%%%%%%%%%%%%%%%%%%%%%%%%%%%%%%%%%%%%%%%%%%
