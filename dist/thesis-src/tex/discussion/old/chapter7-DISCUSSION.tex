%%%%%%%%%%%%%%%%%%%%%%%%%%%%
% CHAPTER 7 - DISCUSSION
%%%%%%%%%%%%%%%%%%%%%%%%%%%%
%%%%%%%%%%%%%%%%%%%%%%%%%%%%%%%%%%%%%%%%%%%%%%%%%%%%%%%%%%%%%%%%%%%%%%%%%%%%%%%%%%%%%%%%%%
% Richard Boardman PhD Thesis: Improving Tool Support for Personal Information Management
%%%%%%%%%%%%%%%%%%%%%%%%%%%%%%%%%%%%%%%%%%%%%%%%%%%%%%%%%%%%%%%%%%%%%%%%%%%%%%%%%%%%%%%%%%

%%%%%%%%%%%%%%%%%%%%%%%%%%%%%%%%%%%%%%%%%%%%%%%%%%%%%%%%%%%%%%%%%%%%%%%%%%%%%%%%%%%%%%%%%%
% NATBIB NOTES
%%%%%%%%%%%%%%%%%%%
%\citet{jon90}                ->    Jones et al. (1990) 
%   \citet[chap.~2]{jon90}       ->    Jones et al. (1990, chap. 2)
%   \citep{jon90}                ->    (Jones et al., 1990) 
%   \citep[chap.~2]{jon90}       ->    (Jones et al., 1990, chap. 2) 
%%%%%%%%%%%%%%%%%%%%%%%%%%%%%%%%%%%%%%%%%%%%%%%%%%%%%%%%%%%%%%%%%%%%%%%%%%%%%%%%%%%%%%%%%%

%%%%%%%%%%%%%%%%%%%%%%%%
% DISCUSSION NOTES
%%%%%%%%%%%%%%%%%%%%%%%%
%%%%%%%%%%%%%%%%%%%%%%%%%%%%%%%%%%%%%%%%%%%%%%%%%%%%%%
% interpret results in light of prior assumptions
%%%%%%%%%%%%%%%%%%%%%%%%%%%%%%%%%%%%%%%%%%%%%%%%%%%%%%
% Relate to overall PhD aims?} What am I trying to do here?}
% What it all means.  What I learned/gained
% What does it all (the study/evaluation results) mean?. Questions raised.
%%%%%%%%%%%%%%%%%%%%%%%%%%%%%%%%%%%%%%%%%%%%%%%%%%%%%%
% Options for placement of discussion:
% Chapter 4 -- post exploratory study, pre-design: discussion of exploratory study findings, extension of initial conceptual framework, formation of extended research agenda. The research agenda from Chapter 3 is reframed in terms of the extended cross-tool conceptual framework. The second part of the thesis puts the developed cross-tool methodology into practice.
% Chapter 7 -- post empirical work: discussion of overall findings, theory-building
%%%%%%%%%%%%%%%%%%%%%%%%%%%%%%%%%%%%%%%%%%%%%%%%%%%%%%
% Major concerns: challenges/issues/methods to clarify for discussion chapter:
%�	Method of theory-building
%�	Form of deliverable
% how do these insights relate/conflict to those from earlier study?. Can I frame as learning?}
%%%%%%%%%%%%%%%%%%%%%%%%%%%%%%%%%%%%%%%%%%%%%%%%%%%%%%
			

THINKS:
\begin{itemize}
\item NEED: Links back to findings
\item NEED: Links forward to conclusion (e.g. future work)
\item NEED: Consider: conflicts/changes from earlier work (e.g. consideration that integration may not be a good thing!) % Data and discussion here appear to raise potential conflicts with earlier findings - need to think carefully about how to frame them (Angela: \textit{"`OK to admit that you learned something"'}
\end{itemize}

%%%%%%%%%%%%%%%%%%%%%%%%%%%%%%%%%%%%%%
\section{Introduction}
\label{disc:discussion-introduction}
%%%%%%%%%%%%%%%%%%%%%%%%%%%%%%%%%%%%%%

PIM SIG stages of discussion:
\begin{itemize}
\item What is PIM?
\item What is PIM's current state?
\item What is the potential of PIM?
\item What are the challenges/obstacles of working on PIM?
\item promising directions
\item next steps.
\end{itemize}

%%%%%%%%%%%%%%%
% LEAD-IN INTRO
%%%%%%%%%%%%%%%
% Relate to previous chapters}. This chapter builds on ...
This section discusses the findings presented in the main study in \textbf{Chapter~\ref{chapter:main-study}} PLUS REST OF THESIS\footnote{\textit{This draft of Chapter 7 DISCUSSION was printed \today}}.

Four main areas of discussion are being considered by the author:
\begin{enumerate}

\item Discussion of findings from empirical work (exploratory study and main study).

\item Discussion of design/evaluation findings. The implications from the evaluation of WorkspaceMirror are discussed: (1), Specific implications - possible refinements for WorkspaceMirror, (2) General implications - implications for design aimed at improving integration between PIM tools.
	
Implications for design -- towards design recommendations. Can I generalise (from local findings to general design genre?) (Carroll's theory building \#1). Build model/theory out of the evaluation data (Carroll's theory building \#2)

\item Methodological issues. Reflection on my experiences. Recommendations for future evaluation work

\item Model/theory building from study and evaluation findings, e.g. extend cross-tool framework with findings from evaluation
\end{enumerate}

	

%%%%%%%%%%%%%%
% STRUCTURE
%%%%%%%%%%%%%%
\textbf{Chapter~\ref{chapter:discussion}} is structured as follows.
Firstly, \textbf{Section~\ref{disc:methodological-discussion}} offers a critical analysis of the research methodologies used in the thesis.
Then, \textbf{Section~\ref{disc:study-discussion}} discusses the empirical findings from the studies reported in \textbf{Chapters~\ref{chapter:exploratory_study}} and \textbf{\ref{chapter:main-study}}.
Next, \textbf{Section~\ref{disc:evaluation-discussion}} discusses the evaluation of the WM prototype, and considers recommendations that can be made for PIM-integration work in general.

% \textbf{Section~\ref{disc:theory-discussion}} discusses routes for developing theory on the basis of the findings of the studies and WM evaluation.
Finally, \textbf{Section~\ref{disc:chapter-summary}} concludes the chapter.


\newpage
%%%%%%%%%%%%%%%%%%%%%%%%%%%%%%%%%%%%%%
\section{Discussion of Study Findings}
\label{disc:study-discussion}
%%%%%%%%%%%%%%%%%%%%%%%%%%%%%%%%%%%%%%


%%%%%%%%%%%%%%%%%%%%%%%%%%%%%%%%%%%%%%%%%%%%%%%%%%%%%%%%%%%%%%%%%%%%
% General findings provided by the empirical work are discussed:
%%%%%%%%%%%%%%%%%%%%%%%%%%%%%%%%%%%%%%%%%%%%%%%%%%%%%%%%%%%%%%%%%%%%
This section discusses the study findings, culminating in those presented in \textbf{Chapter~\ref{chapter:main-study}}. 
\begin{itemize}
\item \textit{Present evidence -- and come up with possible interpretation/explanation}
\item Build on initial exploratory study
\item Use to drive methodology?
\item \textit{THINK: Need to think what are implications for design/further research?  (specific to WorkspaceMirror and general)}
\item Key gain equals cross-tool longitudinal View of PIM Practice. Key insights gained and consider implications for design:
\end{itemize}

Aspects to discuss:
\begin{itemize}
\item Multiple strategies
\item Nature of PIM (link to model), e.g. cross-tool view
\item Longitudinal issues
\end{itemize}

%%%%%%%%%%%%%%%%%%%%%%%%%%%%%%%%%%%%%%%%
% STRATEGY VARITATION - REPEAT?
%%%%%%%%%%%%%%%%%%%%%%%%%%%%%%%%%%%%%%%%
%%%%%%%%%%%%%%%%%%%%%%%%%%%%%%%%%%
\subsection{Multiple strategies}
%%%%%%%%%%%%%%%%%%%%%%%%%%%%%%%%%%

Users employ a range of strategies (EXPLORATORY CONFIRMATION). One user - multiple strategies depending on type of information being managed. Users are not ``messies'' OR ``Tidies'' -- they tend to be a bit of both, depends on where you're looking.
\begin{itemize}
\item Variegated/multi-faceted strategies - the combination strategies I observed are discussed
\item \textit{Develop messy/tidy theme - is messy good?  What is messy? Drag in Mr. Men and that Economist article}
	\item Merge Ch4 and Ch6 findings
	\item Previous classifications extended (or is this done in Ch4?  IDEA: move here from Ch4)
	\item Levels of variation are discussed.
	\item Key factors contributing to variation are identified.
	\item Implications for design (NOTE: link to design recommendations below.
\end{itemize}

%%%%%%%%%%%%%%%%%%%%%%%%%%%%%%%%%%%%%%
% OTHER REPEATS FROM CHAPTER 4
%%%%%%%%%%%%%%%%%%%%%%%%%%%%%%%%%%%%%%
%	\item Model of typical user. Identify key problems.
%	\item Cross-comparison of information types (contribution). Table. Yet tools are the same.
%	\item Similarity of archived email and the document space.
%	\item What were the main factors influencing the different tools/strategies?


%%%%%%%%%%%%%%%%%%%%%%%%%%%%%%%%%%
\subsection{Nature of PIM}
%%%%%%%%%%%%%%%%%%%%%%%%%%%%%%%%%%

See theory stuff below:
\begin{itemize}
	\item Individual differences
	\item e.g. Cross-tool
	\item \textit{Question assumptions of efficiency-based approaches and optimization in e.g. info foraging, Balter, Kirsh}
	\item \textit{Do users see personal workspace as intrinsically different to remote spaces?}
	\item Paradox of the supporting tool.  Trying to improve a tool that distracts people from what they're meant to be doing!
\end{itemize}


%%%%%%%%%%%%%%%%%%%%%%%%%%%%%%%%%%
\subsection{Longitudinal issues}
%%%%%%%%%%%%%%%%%%%%%%%%%%%%%%%%%%

Aspects to discuss:
\begin{itemize}

\item Growth

\item Changes

\item Reflection, nature of PIM (link to model)

\item Link to further work

\end{itemize}

%%%%%%%%%%%%%%%%%%%%%%%%%%%%%%%%%%%%%
% Summary of longitudinal issues
%%%%%%%%%%%%%%%%%%%%%%%%%%%%%%%%%%%%%
Longitudinal issues - issues that emerged over time are discussed. Increased understanding of PIM as an ongoing activity
\begin{itemize}

\item Changes in PIM strategy over time. Relative influence of study and tool on changes: \textit{Our study also provided insight into long-term issues regarding PIM. Although substantial historical changes were reported, including both increases and decreases in organizing tendency - the changes we observed in Phase 2 were relatively subtle pro-organizing adjustments to existing strategies. We did not observe any "global" changes in strategy along the lines of those discussed in [2], e.g. no-filer to spring-cleaner. Our experiences in Phase 2 point to the need to evaluate PIM designs over the long-term, as strategies may take a long time to evolve.}

\item Lack of reflection (related to changes in strategy observed in main study). Acceptance of problems, get worse in background:\textit{ Although the observed changes were subtle, participants found them beneficial. However, the supporting nature of PIM means that users rarely devote time to planning and executing changes in strategy. Users may benefit from increased reflection with respect to PIM, so as to receive the same benefits that resulted from the "self-auditing" effect of the study. As an alternative to redesigning tools to promote reflection (e.g. providing statistics on time spent filing and searching), organizations could also play a part here. Typically, organizations are more concerned with knowledge management and other strategic IT - whilst PIM is left to the individual. Nevertheless, PIM is a key aspect of employees' activities and has the potential to cause frustration and waste time. Organizations could publicize PIM-related issues, and encourage employees to self-diagnose problems to improve their PIM effectiveness. However, managers should take care not to be overly prescriptive, or interfere with individuals' preferred style. The supporting nature of PIM leads to a second dilemma for users and organizations alike: time spent thinking about PIM may result in distraction from production tasks. Tools and organizations must help the user to balance PIM and the production tasks that it supports.}

\item Design implication: importance of persistency: \textit{The folder hierarchy is often criticized for not being easily adaptable to fast-changing user needs, and requirements for dynamic views of personal information are often emphasized in PIM design, e.g. [8]. Our findings suggest a contrasting perspective: the slow-changing nature of the hierarchy may benefit users by promoting familiarity with the personal information environment. Such familiarity in turn supports location-based finding for which users expressed a clear preference. We thus highlight persistence as an often overlooked, yet desirable design goal.}


\end{itemize}


Changes in PIM strategy over time:
\begin{itemize}
	\item The influences of both design intervention and study participation on changes in user strategy are discussed
	\item Two types of user are presented: changers and non-changers
	\item It is noted that changes were relatively minor (towards improved model of changes in PIM strategies)
	\item B�lter's model of user strategy changes is extended (influencing factors are discussed)
	\item Changes in strategy are related to user experience: settled/unsettled strategies (for users of various organizing dispositions)
\end{itemize}

%%%%%%%%%%%%%%%%%%%%%%%%%%
\subsection{Summary}
%%%%%%%%%%%%%%%%%%%%%%%%%%

THINK: how to link study discussion to other aspects of discussion and conclusions





\newpage
%%%%%%%%%%%%%%%%%%%%%%%%%%%%%%%%%%%%%%%%%%%
\section{Discussion of Design/Evaluation Findings}
\label{disc:evaluation-discussion}
%%%%%%%%%%%%%%%%%%%%%%%%%%%%%%%%%%%%%%%%%%%
% Changing design versus changing user behaviour (what can institutions do?)

Aspects of design/evaluation to discuss:
\begin{itemize}

\item HOWTO

\item WM-specific, towards redesign? Success of incremental approach

\item Generalizing WM-specific findings, towards recommendations?

\item Design/evaluation methodology (link to method discussion below)
	
\end{itemize}

%%%%%%%%%%%%%%%%%%%%%%%%%%%%%%%%%%%%%%%%%%%%%%%%%%%
\subsection{HOWTO: discuss design/evaluation}
%%%%%%%%%%%%%%%%%%%%%%%%%%%%%%%%%%%%%%%%%%%%%%%%%%%
In this section the data on WM evaluation is considered. Firstly, findings specific to WM.  Then try to generalize beyond WM.
\begin{itemize}

\item \textit{Evaluation should focus on overall potential rather than short-term design/implementation headaches -- or at least clearly differentiate}

\item \textit{Question with respect to conclusions -- which are specific to my tool? Which can be generalized more widely for the Design of PIM Systems?}

\item \textit{THINK: NB: not a comparative study}
\end{itemize}


%%%%%%%%%%%%%%%%%%%%%%%%%%%%%%%%%%%%%%%%%%%%%%%%%
\subsection{Findings specific to WorkspaceMirror}
%%%%%%%%%%%%%%%%%%%%%%%%%%%%%%%%%%%%%%%%%%%%%%%%%
\begin{itemize}
	\item Success of incremental approach
	\item Overall - a measured success, or is this too optimistic?
	\item Unforeseen uses
	\item Many design suggestions. Much possible future work ... relate to extension ideas in Chapter 5.
	\item Suitability for some users. Can I characterise them? Suitability to novice users (hunch or more?)
	\item \textbf{BUT: Consider underlying problems - have they been dealt with?}
	\begin{itemize}
		\item e.g. email overload - not really is honest answer
	\end{itemize}
\end{itemize}

%%%%%%%%%%%%%%%%%%%%%%%%%%%%%%%%%%%%%%%%%%%%%%%%%%%%%%%%%%%%%%%%
\subsection{General Implications for the Design of PIM systems}
%%%%%%%%%%%%%%%%%%%%%%%%%%%%%%%%%%%%%%%%%%%%%%%%%%%%%%%%%%%%%%%%

Design implications are discussed:
\begin{itemize}
	\item Paradox of the supporting tool.  Trying to improve a tool that distracts people from what they're meant to be doing!
	\item Implications for future work directed towards the design of more coherent, integrated PIM technologies are discussed.
	\item Consider both PIM-general and PIM-integration
	\item Changing design versus changing user behaviour (what can institutions do?)
	\item Individual differences
\end{itemize}

General Implications for the Design of PIM systems. What do the results mean for design of future PIM systems/OS developers? Firstly focus on PIM-general:
\begin{itemize}
	\item More tools -- to reflect variation in users? So many users -- so few tools.  Encourage variation in tools to match variation in practice.
	\item Universal Usability~\cite{Shneiderman:00}
	\item Promotion of reflection
\end{itemize}

Now consider PIM-integration:
\begin{itemize}
	\item View of integration: from ``bolt-on'' to ``core-aspect''
	\item These study results suggest counter-argument to unification - benefits of applications. Along lines of Norman's Information Appliance view~\cite{norman:98} - simple and specialized.
	\item Potential to unify may be limited - do users really want to manage spam in the same space that they are dealing with important documents? Many different forms of content which may have different management strategies.
	\item Potential to unify in terms of information type (ephemeral/archived). Relate to parts of collections.
	\item Identity Management~\cite{liberty:03,myservices:01}
\end{itemize}

Design in general
\begin{itemize}
	\item Incremental -- success of the approach
	\item Subtle changes, big effect?
\end{itemize}

%%%%%%%%%%%%%%%%%%%%%%%%%%
\subsection{Summary}
%%%%%%%%%%%%%%%%%%%%%%%%%%

THINK: how to link study discussion to other aspects of discussion and conclusions











\newpage
%%%%%%%%%%%%%%%%%%%%%%%%%%%%%%%%%%%
\section{Methodological Discussion}
\label{disc:methodological-discussion}
%%%%%%%%%%%%%%%%%%%%%%%%%%%%%%%%%%%

Start with comments on the study reported in previous chapter.  Then generalize towards general recommendations for work in this field. Methodological aspects to discuss:
\begin{itemize}
\item Exploratory Study
\item Design
\item Main Study
\item Main Study/evaluation
\end{itemize}

%%%%%%%%%%%%%%%%%%%%%%%%%%%%%%%%%%%%%%%%
\subsection{Comments on main study (study/evaluation)}
%%%%%%%%%%%%%%%%%%%%%%%%%%%%%%%%%%%%%%%%

Specific comments on study/evaluation methodology:
\begin{itemize}

	\item Exploratory. Basis for future work
	\item Choice of users. Number of users. Bias?

	\item Relative success of interviews over diary
	
	\item Success (pros/cons) of cross-tool approach
	
	\item Success (pros/cons) of study/evaluation dual-purpose research vehicle. Organizational challenges
	
	\item Methodological recommendations (move to conclusion?)
	\begin{itemize}
		\item Enough time to study long-term activity like PIM? This raises the question ...
		\item Success of different forms of data (esp diary)
		\item Triangulation in WET
	\end{itemize}

\end{itemize}

%%%%%%%%%%%%%%%%%%%%%%%%%%%%%%%%%%%%%%%%%%%%%%%%%%%%%%%
\subsubsection{Impact of the Study - ecological validity}
%%%%%%%%%%%%%%%%%%%%%%%%%%%%%%%%%%%%%%%%%%%%%%%%%%%%%%%
	\begin{itemize}
		\item Relative influences of study and design on practice
		\item Impact of the study on user activity
		\item Hawthorne-like effect/or ecological validity - study changed behaviour. Questions over ecological validity. Always going to be a factor - how I chose to reduce impact
		\begin{itemize}
			\item Pro: increase in visibility of PIM
			\item Con: not as realistic as one might have hoped
		\end{itemize}
	\end{itemize}



%%%%%%%%%%%%%%%%%%%%%%%%%%%%%%%%%%%%%%%%%%%%%%%%%%%%%%%%%%%%%%%%%%%%%%%%%%%
\subsection{Methodological Recommendations: how to do research in this area}
%%%%%%%%%%%%%%%%%%%%%%%%%%%%%%%%%%%%%%%%%%%%%%%%%%%%%%%%%%%%%%%%%%%%%%%%%%%
% Lessons which other researchers can take away ...

When making recommendations, I need to think about:
\begin{itemize}
\item Study
\item Design
\item Evaluation
\item All PIM/cross-tool in particular
\item THINK: base recommendations on my experiences
\item THINK: base recommendations on nature of PIM, and based on that derive methodological implications
\end{itemize}

%%%%%%%%%%%%%%%%%%
% EValuation
%%%%%%%%%%%%%%%%%%
%%%%%%%%%%%%%%%%%%%%%%%%%%%%%%%
\subsubsection{Evaluation}
%%%%%%%%%%%%%%%%%%%%%%%%%%%%%%%
Implications for evaluation methodology
\begin{itemize}
\item The challenge of evaluating PIM tools (and especially cross-tool designs) is discussed.
What have we learned in evaluating WorkspaceMirror? Can we make evaluation recommendations?
\item The limitations of current HCI methodology for dealing with cross-tool issues are discussed. The approach outlined in the Reference Task Agenda (Whittaker et al. 2000) is discussed in this context.
\item Individual differences
\end{itemize}

%%%%%%%%%%%%%%%%%%%%%%%%%%%%%%%%%%%%%
\subsubsection{Cross-tool issues}
%%%%%%%%%%%%%%%%%%%%%%%%%%%%%%%%%%%%%
Methodological extensions are proposed for HCI design and research in order to better incorporate cross-tool issues:
\begin{itemize}

\item Cross-tool design - implications are discussed for design methodology, for describing cross-tool designs, and for making claims about those designs. Balance cross-tool with simplifying incremental

\item Cross-tool evaluation - implications are discussed for the evaluation of cross-tool designs.
\end{itemize}


%%%%%%%%%%%%%%%%%%%%%%%%%%
\subsection{Summary}
%%%%%%%%%%%%%%%%%%%%%%%%%%

THINK: how to link study discussion to other aspects of discussion and conclusions


\newpage
%%%%%%%%%%%%%%%%%%%%%%%%%%%%%%%%
\section{Theory discussion}
\label{disc:theory-discussion}
%%%%%%%%%%%%%%%%%%%%%%%%%%%%%%%%

This section proposes theory and suggest routes for developing further theory.  Aspects to discuss:
\begin{itemize}

\item HOWTO

\item Nature of PIM: activity model (form model of PIM, develop Barreau's model). Relation to supporting activities

\item Nature of PIM: balance model: A model of PIM and relationship to production activities in terms of the "need for balance" (user needs to do enough PIM to support. Relate to the need for satisficing

\item PIM changes.  PIM strategy types (settled, changing, stable/unstable)
	
\end{itemize}

%%%%%%%%%%%%%%%%%%%%%%%%%%%%%%%%%%%%
\subsection{HOWTO: How How How to build theory?}
%%%%%%%%%%%%%%%%%%%%%%%%%%%%%%%%%%%%
What is the appropriate route to theory?
\begin{itemize}
\item Need justification/rationale
\item Need linkage from previous chapters
\item Carroll's view of design/evaluation as theory building is one option (2 methods)
\item Grounded theory analysis of main study data is another
\end{itemize}

%%%%%%%%%%%%%%%%%%%%%%%%%%%%%%%
\subsection{Nature of PIM}
%%%%%%%%%%%%%%%%%%%%%%%%%%%%%%%
% nature of PIM, and based on that derive methodological implications

%%%%%%%%%%%%%%%%%%%%%%%%%%%%%%%%%%%%%%%%%%%%%%%%
% Aspects of PIM that we have gained greater understanding into (BEYOND OVER TIME)
%%%%%%%%%%%%%%%%%%%%%%%%%%%%%%%%%%%%%%%%%%%%%%%%
%The nature of PIM - the aspects of the nature of PIM exposed by my work are discussed
%�	Key aspects to discuss include: cross-tool, ongoing, supporting (relationship to production activities)
%�	The paradoxes of PIM of presented ("there is never enough time to do PIM, and whatever time you do spend on it is often wasted")
%�	Implications for methodology are discussed
%%%%%%%%%%%%%%%%%%%%%%%%%%%%%%%%%%%%%%%%%%%%%%%%
%The nature of PIM - the aspects of the nature of PIM exposed by my work are discussed
%	\item �	Key aspects to discuss include: cross-tool, ongoing, supporting (relationship to production activities), idiosyncratic
%		\item Irrational - efficiency not really a factor
%		\item Supporting, lack of reflection (how tool and study in general prompted reflection).  The paradoxes of PIM of presented ("there is never enough time to do PIM, and whatever time you do spend on it is often wasted")
%		\item Barreau's break-down into four sub-tasks not necessarily that clear-cut
%		\begin{itemize}
%			\item In terms of definition, importance or frequency. In actual fact, quite fuzzy
%		\end{itemize}
%		\item Bugbears
%		\begin{itemize}
%			\item Satisficing nature of PIM. Becomes a problem, knock-on effect on productivity. Build-up over time.	
The conceptual framework outlined in Chapter 2 is extended to encompass findings from the exploratory study and main study - forming my own view of PIM:
\begin{enumerate}

\item PIM as a supporting activity - the framework is extended to encompass the relationship between PIM and the production activities it supports. Discuss in terms of model of PIM and relationship to production activities in terms of the "need for balance" (user needs to do enough PIM to support production activities, but not too much to become distracted)

\item Link this to types of production activities (below)

\item PIM as an ongoing activity - the framework is extended to reflect longitudinal issues, e.g. changes in PIM strategies over time, acquisition, organization, maintenance, retrieval as ongoing inter-related processes

\item PIM as a cross-tool activity - the framework is extended to reflect the workspace as a high-level PIM system, containing lower-level PIM systems within distinct tools. This view is contrasted with the traditional treatment of specific PIM tools as distinct PIM systems.

\item \textit{Secondly, we are extending Barreau's framework [3] to reflect the cross-tool, supporting nature of PIM. Barreau conceptualized the computer as a single abstract PIM system, whereas from our data it is clear that current PIM-tools constitute a set of parallel yet inter-related systems. We also seek to modify the framework to capture the influence of production tasks in determining PIM needs.}

\item Individual differences

\end{enumerate}


%%%%%%%%%%%%%%%%%%%%%%%%%%%%%%%%%%%%%
% TOPLACE: Extraction of production tasks
%%%%%%%%%%%%%%%%%%%%%%%%%%%%%%%%%%%%%
% True cross-tool tasks (with primary tools?)
% Tool-dominated tasks which involved "`excusrsions"' to other tools
% Intra-tool tasks
%%%%%%%%%%%%%%%%%%%%%%%%%%%%%%%%%%%%%%%%%%%%%%
\subsection{Types of production activity}
%%%%%%%%%%%%%%%%%%%%%%%%%%%%%%%%%%%%%%%%%%%%%%
\begin{itemize}
\item Fully cross-tool/high-level, e.g. ISIS
\item Cross-tool but tool-specific, e.g. more organiztional needs in one tool
\item Tool-specific	
\end{itemize}


%%%%%%%%%%%%%%%%%%%%%%%%%%%%%%%%%%%%%%%%%%%%%%
\subsection{From nature of PIM, to model/framework}
%%%%%%%%%%%%%%%%%%%%%%%%%%%%%%%%%%%%%%%%%%%%%%

HOWTO: link the previous section and the next?

%%%%%%%%%%%%%%%%%%%%%%%%%%%%%%%%%%%%%
\subsection{Applying the theory}
%%%%%%%%%%%%%%%%%%%%%%%%%%%%%%%%%%%%%
The extended framework/model is used to discuss various aspects of PIM:
\begin{itemize}
	\item Analysis of problems faced by users. Identify sub-problems that may be able to be solved by PIM integration (e.g. the compartmentalization of technology formats).
	\item Current PIM integration techniques are discussed from this view. Alternative routes for improving integration are considered. The concept of a cross-tool artefact is presented.
\end{itemize}

Use of new model (Nature of PIM or just part of it, e.g. the cross-tool view), to make recommendations:
\begin{itemize}
\item Use to derive methodological implications
\item Use to model activities
\item Use to analyze current tools, why they fail?
\item Use to design tools
\item use to evaluate tools
\end{itemize}

%%%%%%%%%%%%%%%%%%%%%%%%%%
\subsection{Summary}
%%%%%%%%%%%%%%%%%%%%%%%%%%

THINK: how to link study discussion to other aspects of discussion and conclusions




\newpage
%%%%%%%%%%%%%%%%%%%%%%%%%%%%%%%
%%%%%%%%%%%%%%%%%%%%%%%%%%%%%%%
\section{Discussion Conclusion}
\label{disc:chapter-summary}
%%%%%%%%%%%%%%%%%%%%%%%%%%%%%%%
%%%%%%%%%%%%%%%%%%%%%%%%%%%%%%%
% Conclusions from the chapter are presented and contributions towards overall thesis are summarized.


%%%%%%%%%%%%%%%%%%%%
% Recap findings
%%%%%%%%%%%%%%%%%%%%
\textbf{Chapter~\ref{chapter:discussion}} presented a discussion of the findings from the main study.

%%%%%%%%%%%%%%%%%%%%%
% CONTRIBUTIONS
%%%%%%%%%%%%%%%%%%%%%
TODO: Set out contributions of this chapter towards overall thesis.
%%%%%%%%%%%%%%%%%%%%%%%%%%%%%%%%%%%%%%%%%%%%%%%%%%%%%%%%%%%%%%%%%%%%%%

%%%%%%%%%%%%%%%%%%%%%%%%%%%%%%%
% Signposting
% oving on up: summing up}
%%%%%%%%%%%%%%%%%%%%%%%%%%%%%%%
Set the scene for the concluding chapter, Chapter~\textbf{Chapter~\ref{chapter:conclusion}} ...

\textit{This draft of Chapter 7 DISCUSSION was printed \today}

