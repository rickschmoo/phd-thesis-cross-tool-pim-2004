
%%%%%%%%%%%%%%%%%%%%%%%%%%%%%%%%%%%%%%%%%%%%%%%%%%%%%%%%%%%%%
%%%%%%%%%%%%%%%%%%%%%%%%%%%%%%%%%%%%%%%%%%%%%%%%%%%%%%%%%%%%%
%%%%%%%%%%%%%%%%%%%%%%%%%%%%%%%%%%%%%%%%%%%%%%%%%%%%%%%%%%%%%
%%%%%%%%%%%%%%%%%%%%%%%%%%%%%%%%%%%%%%%%%%%%%%%%%%%%%%%%%%%%%
%%%%%%%%%%%%%%%%%%%%%%%%%%%%%%%%%%%%%%%%%%%%%%%%%%%%%%%%%%%%%
% SAY HOW PERSPECTIVES EVOLVED OVER COURSE OF THE RESEARCH
%%%%%%%%%%%%%%%%%%%%%%%%%%%%%%%%%%%%%%%%%%%%%%%%%%%%%%%%%%%%%
The first perspective provided the initial driver behind the thesis work -- to investigate PIM as a cross-tool phenomenon.  The second two perspectives were developed over the course of the research programme, mainly based on experiences during the main study reported in \textbf{Chapter~\ref{chapter:main-study}}.  


%%%%%%%%%%%%%%%%%%%%%%%%%%%%%%%%%%%%%%%%%%%%%%%%%%%%%%%%%%%%%
%%%%%%%%%%%%%%%%%%%%%%%%%%%%%%%%%%%%%%%%%%%%%%%%%%%%%%%%%%%%%
%%%%%%%%%%%%%%%%%%%%%%%%%%%%%%%%%%%%%%%%%%%%%%%%%%%%%%%%%%%%%
%%%%%%%%%%%%%%%%%%%%%%%%%%%%%%%%%%%%%%%%%%%%%%%%%%%%%%%%%%%%%
%%%%%%%%%%%%%%%%%%%%%%%%%%%%%%%%%%%%%%%%%%%%%%%%%%%%%%%%%%%%%
% ONGOING: WHAT IS A STRATEGY
%%%%%%%%%%%%%%%%%%%%%%%%%%%%%%%%%%%%%%%%%%%%%%%%%%%%%%%%%%%%%
MOVE TO CHAPTER SEVEN? When researchers talk about PIM strategies, they are describing \textit{average behaviour over the long-term}.  As discussed in the previous section, users may employ different strategies for different types of information (e.g. filing information related to some production tasks but not others).  In addition, the word ``strategy'' has the connotation that users devote time to selecting that strategy.  However in many cases, PIM strategies may not be pre-determined, but simply the amalgam of many spur-of-the-moment discrete events.


%%%%%%%%%%%%%%%%%%%%%%%%%%%%%%%%%%%%%%%%%%%%%%%%%%%%%%%%%%%%%
%%%%%%%%%%%%%%%%%%%%%%%%%%%%%%%%%%%%%%%%%%%%%%%%%%%%%%%%%%%%%
%%%%%%%%%%%%%%%%%%%%%%%%%%%%%%%%%%%%%%%%%%%%%%%%%%%%%%%%%%%%%
%%%%%%%%%%%%%%%%%%%%%%%%%%%%%%%%%%%%%%%%%%%%%%%%%%%%%%%%%%%%%
%%%%%%%%%%%%%%%%%%%%%%%%%%%%%%%%%%%%%%%%%%%%%%%%%%%%%%%%%%%%%
%%%%%%%%%%%%%%%%%%%%%%
% EXISTING INTEGRATION
%%%%%%%%%%%%%%%%%%%%%%
%%%%%%%%%%%%%%%%%%%%%%%%%%%%%%%%
% So what is integration?
% Tools are often described as being integrated, what does that mean?  
% Talk about different forms of integration between tools.
%%%%%%%%%%%%%%%%%%%%%%%%%%%%%%%%
CURRENT INTEGRATION: Some integration mechanisms have also been developed over the years despite the focus on the tool-specific perspective. Integration mechanisms offer partial solutions to the cross-tool needs of users.  For instance the Windows ``Send-to'' mechanism allows a user to transfer an item information from one PIM-tool directly to another.



%%%%%%%%%%%%%%%%%%%%%%%%%%%%%%%%%%%%%%%%%%%%%%%%%%%%%%%%%%%%%
%%%%%%%%%%%%%%%%%%%%%%%%%%%%%%%%%%%%%%%%%%%%%%%%%%%%%%%%%%%%%
%%%%%%%%%%%%%%%%%%%%%%%%%%%%%%%%%%%%%%%%%%%%%%%%%%%%%%%%%%%%%
%%%%%%%%%%%%%%%%%%%%%%%%%%%%%%%%%%%%%%%%%%%%%%%%%%%%%%%%%%%%%
%%%%%%%%%%%%%%%%%%%%%%%%%%%%%%%%%%%%%%%%%%%%%%%%%%%%%%%%%%%%%
% General recommendations from empirical data for INTEGRATION
%%%%%%%%%%%%%%%%%%%%%%%%%%%%%%%%%%%%%%%%%%%%%%%%%%%%%%%%%%%%%
%%%%%%%%%%%%%%%%%%%%%%%%%%%%%%%%%%%%%%%%%%%%%%%%%%%%%%%%%%%%%
\subsubsection{General recommendations from empirical data}
%%%%%%%%%%%%%%%%%%%%%%%%%%%%%%%%%%%%%%%%%%%%%%%%%%%%%%%%%%%%%

%%%%%%%%%%%%%%%%%%%%%%%%%%
% FUTURE WORK: METHODOLOGICAL INSIGHT
%%%%%%%%%%%%%%%%%%%%%%%%%%
TO METHOD: Cross-tool studies, such as the ones reported in this thesis can provide an empirical foundation for such design by highlighting: (1) synergies between tools that can be exploited to improve integration, and (2) differences between tool usage that may indicate barriers to integration. 

%%%%%%%%%%%%%%%%%%%%%%%%%%%%%%%
% Multiple strategies again
%%%%%%%%%%%%%%%%%%%%%%%%%%%%%%%
% MULT-STRATS: 
MULTIPLE STRATS: The previous section considered the implications of the multiple strategies finding for PIM design in general.  Multiple strategies also has an impact on the design of PIM-integration.  Multiple strategies must still be respected at the cross-tool level: PIM strategies vary significantly between tools for many individuals, e.g. participants tended to organize files more extensively than emails or bookmarks. Thus it is envisaged that integration designs which force the users to employ specific strategies across all technological formats may fail (e.g. total organization or no-organization).  There may still be a need for specialized interface support for each type of information.


%%%%%%%%%%%%%%%%%%%%%%%%%%
% org dims
%%%%%%%%%%%%%%%%%%%%%%%%%%
% STUDY FINDINGS: Org-dims: crits of some approaches (need for flexibility)
% ORG-DIMS:
ORG-DIMS: Several proposed PIM-unification systems have been founded on a pre-eminent organizational dimension, e.g. role or project.  \textbf{Chapter~\ref{chapter:exploratory_study}} noted the range of dimensions employed, both within and across different tools.  The data indicates that email contains more contact-based folders, whilst bookmark folders are mainly interest-based. This variety suggests users may be constrained by integration designs that are based on specific types of concept, such as project as in~\citep{Kaptelinin:03}. It is argued that integration mechanisms should not constrain users to particular organizational dimensions.  However one route may be to share projects and role categories across multiple tools, but also give the user the ability to organize specific information types in arbitrary ways.


%%%%%%%%%%%%%%%%%%%%%%%%%%%%%%%%%%%%%%%%%%%%%%%%%%%%%%%%%%%%%%%%%%%%%%%%%%%%%%%%%%%%%%%%%%%%%%%%%%%%%%%%%%%%%%%%
%%%%%%%%%%%%%%%%%%%%%%%%%%%%%%%%%%%%%%%%%%%%%%%%%%%%%%%%%%%%%%%%%%%%%%%%%%%%%%%%%%%%%%%%%%%%%%%%%%%%%%%%%%%%%%%%
%%%%%%%%%%%%%%%%%%%%%%%%%%%%%%%%%%%%%%%%%%%%%%%%%%%%%%%%%%%%%%%%%%%%%%%%%%%%%%%%%%%%%%%%%%%%%%%%%%%%%%%%%%%%%%%%
%%%%%%%%%%%%%%%%%%%%%%%%%%%%%%%%%%%%%%%%%%%%%%%%%%%%%%%%%%%%%%%%%%%%%%%%%%%%%%%%%%%%%%%%%%%%%%%%%%%%%%%%%%%%%%%%

%%%%%%%%%%%%%%%%%%%%%%%%%%%%%%%%%%%%%%%%%%%%%%%%%%%%%%%%%%%%%
%%%%%%%%%%%%%%%%%%%%%%%%%%%%%%%%%%%%%%%%%%%%%%%%%%%%%%%%%%%%%
%%%%%%%%%%%%%%%%%%%%%%%%%%%%%%%%%%%%%%%%%%%%%%%%%%%%%%%%%%%%%
%%%%%%%%%%%%%%%%%%%%%%%%%%%%%%%%%%%%%%%%%%%%%%%%%%%%%%%%%%%%%
%%%%%%%%%%%%%%%%%%%%%%%%%%%%%%%%%%%%%%%%%%%%%%%%%%%%%%%%%%%%%
% PIM design in general}
%%%%%%%%%%%%%%%%%%%%%%%%%%%%%%%%%%%%%%%%%%%%%%%%%%%%%%%%%%%%%
%%%%%%%%%%%%%%%%%%%%%%%%%%%%%%%%%%%%%%%%%%%%%%%%%%%%%%%%
\subsection{PIM design in general}
\label{disc:design-guidelines-discussion-pim-general}
%%%%%%%%%%%%%%%%%%%%%%%%%%%%%%%%%%%%%%%%%%%%%%%%%%%%%%%%

%%%%%%%%%%%%%%%%%%%%%%%%%%%%%%%%%%%%%%%%%%%%%%%%%%%%%%%%%%%%
% THINK: go through all study/evaluation results to check
%%%%%%%%%%%%%%%%%%%%%%%%%%%%%%%%%%%%%%%%%%%%%%%%%%%%%%%%%%%%
A number of design implications and recommendations follow from the study and evaluation results for PIM design in general, not just integration mechanisms.

% METH-RECS:  Advice for tool-specific designers and cross-tool designers: (1) TS view (does your tool interface with other tools, does it support user's production tasks?), and (2) CT view.

% \textit{Flexibility is also important here.  Important in unification, but also in specific tools.}
The previous section emphasised the importance of retaining the flexibility to manage different types of information in different ways.  This section moves on to highlight that flexibility is also an important issue in the tool-specific context. 


%%%%%%%%%%%%%%%
% ADD IDIO
%%%%%%%%%%%%%%%
%%%%%%%%%%%%%%%%%%%%%%%%%%%%%
% MULT STRATS
%%%%%%%%%%%%%%%%%%%%%%%%%%%%%
% STUDY FINDINGS: multiple strategies/individual differences: need customizability, flexibility. 
% Previous work has also noted multiple strategies in the context of paper archives, where people tend to combine filing and piling strategies [15]. Our findings suggest that much user behaviour does not map onto earlier strategy classifications in email and bookmarks [1,2,13].
% Although such classifications offer useful abstractions of PIM practice, they exaggerate the extremes � portraying users as either messy or tidy, filers or no-filers. We have attempted to classify behaviour in more detail to take account of multiple strategies.
%%%%%%%%%%%%%%%%%%%%%%%%%%%%%
% SUPPORT DIFFERENT USERS
%%%%%%%%%%%%%%%%%%%%%%%%%%%%%
CHAPTER 4: IDIO: PIM is highly idiosyncratic.  Different people manage information in many unique ways. 
Future design work must take account of the variation in strategies \textit{between users} by providing the flexibility to manage in different ways.  Tools typically do this now -- one contributing factor to functionality bloat.

%%%%%%%%%%%%%%%%%%%%%%%%%%%%%
% MULT STRATS FOR ONE USER
%%%%%%%%%%%%%%%%%%%%%%%%%%%%%
CHAPTER 4: MULT STRATS: In addition, \textbf{Chapter~\ref{chapter:exploratory_study}} noted that many people employ multiple organizing strategies within file, email and bookmark collections.  The finding was reinforced in \textbf{Chapter~\ref{chapter:main-study}}.  This underlines the challenge of PIM design to ensure interface \textit{flexibility} in terms of strategies supported.  Thus, as well as catering for individual differences between users, designers must also allow for individual user�s multiple strategies.
Future design work must take account of the variation in strategies \textit{within tools for a particular user} by providing the flexibility to manage different information in distinct ways. For instance, a tool should give users the ability to organize a subset of items as required, whilst not penalizing those users who do not want to organize at all.
% Critique of desktop google.
A move to a purely search-based interface, a ``desktop google'', has been advocated by some.  However it is argued here that even users who depend heavily on search may want to organize some items.



%%%%%%%%%%%%%%%
% Org DIMS
%%%%%%%%%%%%%%%
% Org dims In addition our data indicates that email contains more contact-based folders, whilst bookmark folders are mainly interest-based. 
CHAPTER 4: ORG DIMS: \textbf{Chapter~\ref{chapter:exploratory_study}} also highlighted the range of \textit{organizational dimensions} employed by users in all three tools of focus.  This variety suggests users may be constrained by tool-specific interfaces that are based on specific dimensions.  Such interfaces have been proposed as potential approaches to PIM unification, e.g. ~\citet{Kaptelinin:03}, and were criticised in the previous section. Tool-specific designs should also support multi-dimensional organization.   However they may be appropriate in some contexts, e.g. address book interfaces on mobile phones where the emphasis is on simplicity, i.e. depends on retrieval needs.

%%%%%%%%%%%%%%%%%%%%%%%%%%%%%%%%%%%%%%%%%%%%%%%%%%%%%%%%%%%%%%%%%%%%%%%%%%%%%%%%%%%%%%%%%%%%%%%%%%%%%%%%%%%%%%%%
%%%%%%%%%%%%%%%%%%%%%%%%%%%%%%%%%%%%%%%%%%%%%%%%%%%%%%%%%%%%%%%%%%%%%%%%%%%%%%%%%%%%%%%%%%%%%%%%%%%%%%%%%%%%%%%%
%%%%%%%%%%%%%%%%%%%%%%%%%%%%%%%%%%%%%%%%%%%%%%%%%%%%%%%%%%%%%%%%%%%%%%%%%%%%%%%%%%%%%%%%%%%%%%%%%%%%%%%%%%%%%%%%
%%%%%%%%%%%%%%%%%%%%%%%%%%%%%%%%%%%%%%%%%%%%%%%%%%%%%%%%%%%%%%%%%%%%%%%%%%%%%%%%%%%%%%%%%%%%%%%%%%%%%%%%%%%%%%%%

SUPPORTING: The studies reported in this thesis also emphasised the supporting nature of PIM.  PIM is not carried out for its own sake, it is driven by a user's production activities (both work and leisure) that drive their information needs. % Although the observed changes were subtle, participants found them beneficial. 
Due to the supporting nature, \textbf{Chapter~\ref{chapter:main-study}} observed that users rarely devote time to planning and executing changes in strategy. This may lead to the acceptance of non-optimal strategies, and poor user experience in long-term since PIM is such a fundamental aspect of computer-based activity. Many of the main study participants acknowledged the beneficial ``self-auditing'' effect of the study. 
% Promotion of reflection.
%Users may benefit from
%increased reflection with respect to PIM, so as to receive the
%same benefits that resulted from the �self-auditing� effect o
%the study. 
\textbf{Section~\ref{discussion:uxp}} considers that users may benefit from increased reflection on PIM.  One route may be to provide statistics: for example, recording the time spent filing and searching, or counts of unused folders.
% Changing design versus changing user behaviour (what can institutions do?)
% In fact should users be doing more to help themselves?
As an alternative to redesigning tools to promote reflection, organizations could also play a part here. Typically, organizations are more concerned with knowledge management and other strategic IT - whilst PIM is left to the individual. Nevertheless, PIM is a key aspect of employees' activities and has the potential to cause frustration and waste time. Organizations could publicize PIM-related issues, and encourage employees to self-diagnose problems to improve their PIM effectiveness. However, managers should take care not to be overly prescriptive, or interfere with individuals' preferred style.
% but ... Paradox of the supporting tool.  Trying to improve a tool that distracts people from what they're meant to be doing!  Want to improve efficiency, i.e. not time spent
The supporting nature of PIM leads to a second dilemma for users and organizations alike: time spent thinking about PIM may result in distraction from production tasks. Tools and organizations must help the user to balance PIM and the production tasks that it supports.
%Efficiency-based measures may be both good and bad:
%\begin{itemize}
%\item Good: PIM is supporting, user wants to spend as little time as possible on PIM.
%\item Bad: much more than efficiency, have to think about ongoing/long-term satisfaction. See \textbf{Section~\ref{discussion:uxp}}.
%\end{itemize}
