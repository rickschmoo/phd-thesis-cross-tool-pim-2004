%%%%%%%%%%%%%%%%%%%%%%%%%%%%
% CHAPTER 7 - DISCUSSION INTRO
%%%%%%%%%%%%%%%%%%%%%%%%%%%%
%%%%%%%%%%%%%%%%%%%%%%%%%%%%%%%%%%%%%%%%%%%%%%%%%%%%%%%%%%%%%%%%%%%%%%%%%%%%%%%%%%%%%%%%%%
% Richard Boardman PhD Thesis: Improving Tool Support for Personal Information Management
%%%%%%%%%%%%%%%%%%%%%%%%%%%%%%%%%%%%%%%%%%%%%%%%%%%%%%%%%%%%%%%%%%%%%%%%%%%%%%%%%%%%%%%%%%
%%%%%%%%%%%%%%%%%%%%%%%%%%%%%%%%%%%%%%%%%%%%%%%%%%%%%%%%%%%%%%%%%%%%%%%%%%%%%%%%%%%%%%%%%%
% NATBIB NOTES
%%%%%%%%%%%%%%%%%%%
%\citet{jon90}                ->    Jones et al. (1990) 
%   \citet[chap.~2]{jon90}       ->    Jones et al. (1990, chap. 2)
%   \citep{jon90}                ->    (Jones et al., 1990) 
%   \citep[chap.~2]{jon90}       ->    (Jones et al., 1990, chap. 2) 
%%%%%%%%%%%%%%%%%%%%%%%%%%%%%%%%%%%%%%%%%%%%%%%%%%%%%%%%%%%%%%%%%%%%%%%%%%%%%%%%%%%%%%%%%%
%%%%%%%%%%%%%%%%%%%%%%%%
% DISCUSSION NOTES
%%%%%%%%%%%%%%%%%%%%%%%%
% Speculate, Suggest future work
%%%%%%%%%%%%%%%%%%%%%%%%%%%%%%%%%%%%%%%%%%%%%%%%%%%%%%
% Frame as "`intermediate discussion stage"' ...
%			pulling things altogther from all over the thesis
% What it all means.  What I learned/gained
% interpret results in light of prior research/assumptions
% MUST: Link back to findings/experience
% MUST: Links forward to conclusion (e.g. future work)
%%%%%%%%%%%%%%%%%%%%%%%%%%%%%%%%%%%%%%%%%%%%%%%%%%%%%%
%%%%%%%%%%%%%%%%%%%%%%%%%%%%%%%%%%%%%%%%%%%%%%%%%%%%%%
% FRAME AS A STORY
% Consider: conflicts/changes from earlier work (e.g. consideration that integration may not be a good thing!)
% how do these insights relate/conflict to those from earlier study?. Can I frame as learning?}
% Data and discussion here appear to raise potential conflicts with earlier findings - need to think carefully about how to frame them (Angela: \textit{"`OK to admit that you learned something"'}
%%%%%%%%%%%%%%%%%%%%%%%%%%%%%%%%%%%%%%%%%%%%%%%%%%%%%%
%%%%%%%%%%%%%%%%%%%%%%%%%%%%%%%%%%%%%%%%%%%%%%%%%%%%%%
% HOWTO: What is the appropriate route to theory?
% Need justification/rationale (e.g. empirical grounding)
%�	Method of theory-building
%�	Form of theoretical deliverable
% 			Carroll's view of design/evaluation as theory building is one option (2 methods)
% 			Grounded theory analysis of main study data is another
%%%%%%%%%%%%%%%%%%%%%%%%%%%%%%%%%%%%%%%%%%%%%%%%%%%%%%
%%%%%%%%%%%%%%%%%%%%%%%%%%%%%%%%%%%%%%%%%%%%%%%%%%%%%%
% Other things to add:
% Relate ES and MS -- how did MS build on initial ES?
%	Do users see personal workspace as intrinsically different to remote spaces?}
%%%%%%%%%%%%%%%%%%%%%%%%%%%%%%%%%%%%%%%%%%%%%%%%%%%%%%

%%%%%%%%%%%%%%%%%%%%%%%%%%%%%%%%%%%%%%%%%%%%%%%%%%%%%%
%PIM SIG stages of discussion:
%\begin{itemize}
%\item What is PIM?
%\item What is PIM's current state?
%\item What is the potential of PIM?
%\item What are the challenges/obstacles of working on PIM?
%\item promising directions
%\item next steps.
%\end{itemize}
%%%%%%%%%%%%%%%%%%%%%%%%%%%%%%%%%%%%%%%%%%%%%%%%%%%%%%

%%%%%%%%%%%%%%%%%%%%%%%%%%%%%%%%%%%%%%
\section{Introduction}
\label{discussion:discussion-introduction}
%%%%%%%%%%%%%%%%%%%%%%%%%%%%%%%%%%%%%%
 
%%%%%%%%%%%%%%%
% LEAD-IN INTRO
%%%%%%%%%%%%%%%
% This section proposes theory and suggest routes for developing further theory.
%%%%%%%%%%%%%%%%%%%%%%%%%%%%%%%%
% ALT TITLE FOR CHAPTER: \section{Extending the conceptual basis of PIM}
% draws together the findings from the rest of the thesis.  
% FINDINGS -> derive -> DISCUSSION
% DEVELOPING IMPROVED MODELS OF PIM
% A series of discussions are presented that consider the findings in the light of previous work in the field
% Model/theory building from study and evaluation findings
%%%%%%%%%%%%%%%%%%%%%%%%%%%%%%%%
%%%%%%%%%%%%%%%%%%%%%%%%%%%%%%%%%%%%%%%%%%%%%%%%%%%%
% TALK ABOUT EXTENDING BARREAU'S MODEL
%%%%%%%%%%%%%%%%%%%%%%%%%%%%%%%%%%%%%%%%%%%%%%%%%%%%
% The section develops extensions of Barreau's model relating to three aspects of PIM not covered in previous work.  For each aspect, design implications and methodological implications are discussed.
%%%%%%%%%%%%%%%%%%%%%%%%%%%%%%%%%%%%%%%%%
% Some preliminary steps are taken towards developing a model of PIM encompassing three aspects of its nature as uncovered in this research programme: (1) cross-tool, (2) ongoing, and (3) supporting.
%Firstly, \textbf{Section~\ref{disc:theory-discussion}} revisits the limitations of previous theory on PIM.  Three aspects of PIM are identified as in particular need of attention: (1) the \textit{cross-tool} nature of PIM, (2) the \textit{ongoing} nature of PIM, and (3) the \textit{supporting} nature of PIM.  
%The conceptual framework outlined in Chapter 2 is extended to encompass findings from the exploratory study and main study - forming my own view of PIM
%We are extending Barreau's framework~\citep{barreau:95} to reflect the cross-tool, supporting nature of PIM.
%%%%%%%%%%%%%%%%%%%%%%%%%%%%%%%%%%
% BUILDING THEORETICAL FRAMEWORK
%%%%%%%%%%%%%%%%%%%%%%%%%%%%%%%%%%
% These theoretical perspectives are raised based on the substantive findings reported .
% e.g. extend cross-tool framework with findings from evaluation
% The framework is based on three important aspects of the nature of PIM that have been highlighted in carrying out this research.
% The three perspectives, each of which has been under-represented in earlier work, are as follows:
% Each section discusses PIM from a theoretical perspective following the experiences of the author in pursuing this programme of research.  
This chapter discusses the substantive findings from \textbf{Chapters~\ref{chapter:exploratory_study}} to \textbf{\ref{chapter:main-study}}.  The discussion is made up of the following four sections:

\begin{itemize}

\item \textbf{Section~\ref{discussion:theoretical-framework}} develops a theoretical framework which reflects three perspectives of PIM that have been highlighted over the course of this research: 1) PIM as a \textit{cross-tool} activity, (2) PIM as a \textit{supporting} activity, and (3) PIM as an \textit{ongoing} activity.   Each perspective illustrates a future direction for theory development in this area.


%%%%%%%%%%%%%%%%%%%%%%%%%%%
% DESIGN IMPLICATIONS
%%%%%%%%%%%%%%%%%%%%%%%%%%%
%%%%%%%%%%%%
% General implications/guidelines/recommendations - implications for design aimed at improving integration between PIM tools.
% Implications for design -- towards design recommendations. Can I generalise (from local findings to general design genre?) (Carroll's theory building \#1). Build model/theory out of the evaluation data (Carroll's theory building \#2)
%%%%%%%%%%%%
\item \textbf{Section~\ref{discussion:design-guidelines-discussion}} revisits the evaluation of WorkspaceMirror (WM), and uses the perspective of PIM as a supporting activity to interpret the results.  Based on this analysis, and empirical findings in the thesis, implications for the wider design genre of PIM-integration are considered. % a number of recommendations are made for the design  mechanisms.  %makes recommendations for future design work, with a particular focus on that directed at improving PIM integration.  Recommendations draw on the above theoretical discussion, and the substantive findings from earlier chapters.

%%%%%%%%%%%%%%%%%%%%%%%%%%%%%%%%%%
% METHODOLOGY IMPLICATIONS
% Methodological issues. Reflection on my experiences. 
%%%%%%%%%%%%%%%%%%%%%%%%%%%%%%%%%%
\item \textbf{Section~\ref{discussion:methodological-discussion}} presents a series of methodological recommendations for carrying out work in this area.  The framework from \textbf{Section~\ref{discussion:theoretical-framework}} is used to structure the recommendations.

%%%%%%%%%%%%%%%%%%
% MODEL OF UXP
%%%%%%%%%%%%%%%%%%
% INCLUDE TRANSITIONS FROM PEOPLE IN STUDY
% The need for improved theory in this area is highlighted in \textbf{Section~\ref{discussion:methodological-discussion}}. The next two sections propose models to explain some of the observations made in this thesis. \textbf{Section X} develops a cross-tool model of PIM strategies and how they change over time.
\item \textbf{Section~\ref{discussion:uxp}} explores possible qualitative measures of \textit{PIM user experience} that could be used as evaluation metrics.  \textbf{Section~\ref{discussion:uxp-settled}} considers one aspect of negative user experience in depth, dissatisfaction with management strategies.  The results from the main study in \textbf{Chapter~\ref{chapter:main-study}} are used to provide examples of such dissatisfaction.

%Two aspects of negative PIM user experience are discussed in depth: (1) \textit{``unsettledness'' in PIM strategy}, and (2) \textit{distraction from work activities}. % builds on the earlier discussions of PIM as an ongoing, supporting activity, to build a model of user experience over time.  The model is used to explain some of the observations made in \textbf{Chapter~\ref{chapter:main-study}}.  Suggestions are made for improved evaluation measures.

\end{itemize}

%%%%%%%%%%%%
% CONCLUSION (REQUIRED?)
%%%%%%%%%%%%
% Finally, \textbf{Section~\ref{discussion:chapter-summary}} concludes the chapter with a summary of key findings.

%%%%%%%%%%%%%%%%%%%%%%%%%%%%%%
% FIN@ CHAPTER 7 DISCUSSION INTRO
%%%%%%%%%%%%%%%%%%%%%%%%%%%%%%














