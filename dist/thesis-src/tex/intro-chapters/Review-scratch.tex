%%%%%%%%%%%%%%%%%%%%%%%%%%%%%%%%%%
% \subsection{What is needed from HCI?}
%%%%%%%%%%%%%%%%%%%%%%%%%%%%%%%%%%
% Scope for contributions} in all these areas:
%	\item How do they fit together?
%	\item Use this to derive my research aims and approach.
%	\item However warning: challenges of working in this area
% THINK: How is HCI failing PIM users? Why is HCI failing PIM users?
% THINK: How is HCI failing PIM designers?
% THINK: Focus here on cross-tool dimension? Summarize my claim/approach}
\textbf{Chapter~\ref{chapter:review}} identifies four areas of progress required from HCI research to provide a more effective knowledge base for both PIM design in general, and PIM-integration specifically:
\begin{itemize}
		\item To provide increased understanding of PIM in terms of user needs and problems, to act as requirements for design.
		\item To provide empirically-grounded, evaluated prototypes of improved PIM-tools. % (invent the future)
		\item To provide improved theoretical apparatus for modelling user's PIM behaviour, and predicting the efficacy of new interfaces.
		\item To provide improved methods for evaluating PIM-tools. % Need for different methods? Our methods are not attuned to this kind of activity
\end{itemize}




%%%%%%%%%%%%%%%%%%%%%%
%% THE STATE OF HCI
%%%%%%%%%%%%%%%%%%%%%%
%%%%%%%%%%%%%%%%%%%%%%%%%%%%%%%
%%%%%%%%%%%%%%%%%%%%%%%%%%%%%%
\subsection{The State of HCI}
\label{hci-state}
%%%%%%%%%%%%%%%%%%%%%%%%%%%%%%
%%%%%%%%%%%%%%%%%%%%%%%%%%%%%%%

HCI-PROBS: HCI is a young, interdisciplinary, applied research field containing many competing research paradigms and methods~\citep{Sutcliffe:00,rogers:04}.
The following problematic issues have been identified within HCI. All have impact on research into personal information management. 

%%%%%%%%%%%%%%%%%%%%%%%%%%%%%%%%%%%%%%%%%
%% Theory-practice gap:
%% 1. how do convey research results to designers in meaningful form
%% 2. the issue of design leading theory
%%%%%%%%%%%%%%%%%%%%%%%%%%%%%%%%%%%%%%%%%%
THEORY-PRACTICE-GAP: A primary concern in HCI is the so-called \textit{theory/practice gap} (or research/practice gap)~\citep{Sutcliffe:00,rogers:04}. Firstly there is the question of how to convey the results of HCI research to practitioners in a useful form~\citep{Sutcliffe:00}. 
Secondly there is the issue of research being left behind by design. Carroll and Rosson~\citep{jc-cycle:92} observe that this issue is not peculiar to HCI!  The traditional view is that theory (in the form of scientific deduction) should lead practice, but in reality often not the case. In many cases design leads whilst theory catches up over time. A prime example of this can be seen in terms of personal information management. PIM-tools are ubiquitous and used by millions of people, whilst research into PIM has been left far behind. There is concern in some corners that HCI research may be irrelevant to HCI practitioners who pursue their craft without need for theory or strict methods. 

%%%%%%%%%%%%%%%%%%%%%%%%%%%%%%%%%%%%% 
%% over-focus on radical invention
%%%%%%%%%%%%%%%%%%%%%%%%%%%%%%%%%%%%%
%% 
RADICAL-INVENTION: In terms of design it has been noted that much HCI research is focused on ``radical invention'' rather than incremental systematic science~\citep{newman:94,Whittaker-rta:00}. Whittaker \textit{et al}. suggest this may be due to the incredible growth rate in new information technology offering new potential for human-computer interaction. They go on to state that the consequence of this revolutionary bent is that many everyday core computing tasks remain under-researched.
It has also been noted that relatively little technology developed by HCI researchers has been taken up commercially~\citep{Whittaker-rta:00}.
















%%%%%%%%%%%%%%%%%%%%%%%%%%%%%%%%%%%%%
\subsubsection{HCI Research Paradigms}
%%%%%%%%%%%%%%%%%%%%%%%%%%%%%%%%%%%%%

%% competing schools of thought
% Many observations have been noted of the state of flux in HCI (for example Rogers, Sutcliffe etc.).
Sasse~\citep{Sasse:97} notes the lack of consensus towards methodology in HCI research.  She identifies three main research traditions in HCI -- perhaps a function of the newness of the field. Each paradigm proposes key methods and the forms that research-derived knowledge should take. 
% They differs in terms of how they envisage the theory/practice gap, forms of knowledge, key methods, and foundational disciplines.

%%%%%%%%%%%%%%%%%%%
%% BASIC SCIENCE
%%%%%%%%%%%%%%%%%%%
%% IM too high-level cognitively, very complex, not efficiency-based, not a practical approach
SCIENCE PARADIGM: The first view of HCI is driven by basic-science and emphasizes foundations from cognitive psychology. Main view and advocates include Newell and Card. Research methods are based on scientific models and hypothesis testing. However such approaches have not been applied to PIM beyond low-level studies of classification behaviour (e.g. ~\citep{Dumais:86}). PIM is a long-term activity and involves many cognitive processes such as classification, recall and recognition~\citep{ml:88}. Therefore in terms of this work, the appropriateness of the ``scientific'' paradigm is questioned.

%%%%%%%%%%%%%%%%%
% DESIGN/CRAFT
%%%%%%%%%%%%%%%%%
DESIGN PARADIGM: The second view of HCI research is as design/craft~\citep{Carroll-cycle:91,Carroll-review:97}. HCI research based on this paradigm in effect mirrors the design practice it hopes to influence. Carroll and Rosson~\citep{jc-cycle:92} describe the process of HCI as a \textit{task-artifact cycle} (see \textbf{Figure ~\ref{fig:TaskArtefactCycle}}). Can be used as a framework for developing knowledge about artifact genres based on experiences during design and evaluation. A craft-based view of HCI research, reminiscent of the Empiricist Tradition (no need for theory)~\citep{landauer:95}.
%% Methods build science in the extant practice. Knowledge in implementation.  Artifact-theory. Designed artifacts manifest knowledge, can be made explicit through claims analysis and enriched through evaluation.  Typically not articulated.  Validate via grounding in theory or evaluation.
%% Arguably a more realistic view of the user~\citep{jc-paradox:87} than cognitive models? 
The design-based research paradigm has been criticised for being craft-level and non-scientific (Dowell and Long), non-generalisable, non-verifiable. However Carroll has proposed a defence of the paradigm, including artifact-theory and evaluation as theory-building, and the debate continues. %% Response to this: , Carroll's design-science. Trade-off pragmatism!

%%%%%%%%%%%%%%%%%
% ENGINEERING
%%%%%%%%%%%%%%%%%
ENGINEERING PARADIGM: The final view of HCI research as engineering science enabling design based on engineering principles (Dowell and Long). This paradigm is rooted in the assumption that efficiency is a key evaluation criteria. However within highly-interactive long-term tasks such as PIM it is argued that subjective factors are also crucial. Also no work from this paradigm has been carried out, meaning there is nothing to build on.

PARADIGM NOTES: It should be noted that this is an incomplete break-down of HCI which can be extended to include contributions from areas such as ethnography. It is the view of the author that each paradigm can make useful contributions in this research area.
% However it is necessary to select a research method to follow in this research programme. %  -- see \textbf{Section~\ref{review:research-agenda}}.
%%%%%%%%%%%%%%%
% SIGNPOSTING
%%%%%%%%%%%%%%%
% The rest of this chapter aims to survey the methods and contributions relevant to personal information management.  Firstly \textbf{Section~\ref{review:pim-research-overview}} presents an overview of the main bodies of research that have been carried out.

%%%%%%%%%%%%%%%%%%%%%%%%%%%%%%%%%%%
%%%%%%%%%%%%%%%%%%%%%%%%%%%%%%%%%%%
%%%%%%%%%%%%%%%%%%%%%%%%%%%%%%%%%%%
%%%%%%%%%%%%%%%%%%%%%%%%%%%%%%%%%%%


















%%%%%%%%%%%%%%%%%%%%%%%%%%%%%%%%%%%%%
\subsubsection{Misc studies}
%%%%%%%%%%%%%%%%%%%%%%%%%%%%%%%%%%%%%

%%%%%%%%%%%%%%%%%%%%%%%%%%%%%%%
%% OVERALL STATE OF RESEARCH
%%%%%%%%%%%%%%%%%%%%%%%%%%%%%%%%%%%%%%
%% THINK: how do the three fit together?}
%%%%%%%%%%%%%%%%%%%%%%%%%%%%%%%%%%%%%%
% Newman's food chain?
%%%%%%%%%%%%%%%%%%%%%%%%%%%%%%%
NON-SYSTEMATIC: Although the importance of PIM is not disputed, there has been a lack of systematic research carried out in the area~\citep{Whittaker-rta:00}. Whittaker et al. note the consequences of the lack of systematicity: for example researchers must start from scratch when developing task definitions and user requirements.

COMPARE-PIM-WITH-IR: PIM can be compared with other important tasks such as information retrieval which have had a large amount of research attention. Although there has been work on PIM, the amount carried out does not reflect the importance of the activity. Also what has been carried out is non-systematic.

%%%%%%%%%%%%%%%%%%%%%%%%%%%%%%%%%%%%%%%%%%%
%%%%%%%%%%%%%%%%%%%%%%%%%%%%%%%%%%%%%%%%%%%
\section{Review of Empirical Work}
%%%%%%%%%%%%%%%%%%%%%%%%%%%%%%%%%%%%%%%%%%%



%%%%%%%%%%%%%%%%%%%%%%%%%%%%%%%%%%%%%%%%%%%
%%%%%%%%%%%%%%%%%%%%%%%%%%%%%%%%%%%%%%%%%%%
\section{Review of technology}
%%%%%%%%%%%%%%%%%%%%%%%%%%%%%%%%%%%%%%%%%%%




%%%%%%%%%%%%%%%%%%%%%%%%%%%%%%%%%%%%%%%%%%%


%%%%%%%%%%%%%%%%%%%%%%%%%%%%%%%%%%%%%%%%%%%%
\subsubsection{Criticisms of the Hierarchy}
%%%%%%%%%%%%%%%%%%%%%%%%%%%%%%%%%%%%%%%%%%%%


% Hierarchical classification is a difficult task, with users encountering many
%psychological overheads~\cite{ml:88}.  Lansdale focuses on the problems of
%classification on storing items, and issues of memory on retrieving them. He
%observes that classification is a cognitively difficult task with users having
%to deal with problems of ambiguous and overlapping categories. Users often
%employ a satisficing strategy to compensate for the overheads of
%classification such as the use of piles~\cite{tm:83}. Such loose
%classification facilitates additional retrieval cues such as location, time
%and appearance. However such strategies are rarely scalable (ibid). Lansdale
%also identifies two psychological processes that are used when retrieving
%information: (1) recall-directed search (to home in on a group of items), and
%(2) recognition-based scanning (selecting a particular item). Retrieval
%systems are not tuned to the abilities of the human memory, as identified by
%much work in cognitive psychology, such as our ability to remember general
%meanings better than specific details. Users are forced to remember specific
%filenames and locations, whilst human memory is better at handling contextual
%information such as time and colour.

%Many of the studies of real-world personal information management have noted
%the large amount of effort that users often devote to filing information (e.g.
%Bellotti and Smith 2000). Several researchers identify a trade-off between the
%time investment of up-front organization, versus the costs of not being able
%to find items in subsequent retrieval (B�lter 2000, Whittaker and Hirschberg
%2001). Indeed many users choose not to use personal classification schemes,
%and instead rely on implicit metadata and search mechanisms for finding
%information . When users do file, they typically exhibit satisficing behaviour
%and ``just do enough to get by~\cite{barreau:95}, leading to problems such as
%premature filing and failed folders~\cite{Whittaker-email:96}.










%%%%%%%%%%%%%%%%%%%%%%%%%%%%%%%%%%%%%%%%%%%%%%%%%%%%
% PIM as MANAGING COMPLEXITY - MOVE TO CHAPTER 2?
%%%%%%%%%%%%%%%%%%%%%%%%%%%%%%%%%%%%%%%%%%%%%%%%%%%%
% Other researchers have considered PIM as an attempt to manage the complexity of the world, and by offloading some amount of that management overhead onto the environment~\citep{norman:93}. 
%%%%%%%%%%%%%%%%%%%%%%%%%%%%%%%%%%%%%%
%\subsection{Scientific Foundations}
%\label{review:theory-foundations}
%%%%%%%%%%%%%%%%%%%%%%%%%%%%%%%%%%%%%%
%%%%%%%%%%%%%%%%%%%%%%%%%%%%%%%%%%%%%%%
% Overview of basic science findings
%%%%%%%%%%%%%%%%%%%%%%%%%%%%%%%%%%%%%%%
% provided direction for the design of PIM technology. % : (1) human memory, and (2) classification. % surveyed in \textbf{Section~\ref{review:pim-technological-review}}.
%%%%%%%%%%%%%%%%%%%%%%%%%%
% Studies of human memory
%%%%%%%%%%%%%%%%%%%%%%%%%%
% The first set of basic science findings findings relate to studies of human memory.  
PIM systems are often considered as adjuncts to human memory~\citep{ml:92}. However, Lansdale notes the irony of the situation whereby people store information in PIM-systems to avoid having to remember it -- and then must remember where they stored it to access it again.
% work from the field of autobiographical/episodic memory (Tulving) relevant to the processes of recall and recognition used in retrieving information from any system. Typically recognition is 
\citet{ml:92} survey findings from cognitive psychology that relate to human memory. They observe several results that indicate a preference of recognition over recall since it places less overhead on the user to remember up-front details of the item required.

LOW-LEVEL: Two areas of ``basic science'', Cognitive Psychology and Classification Research, have documented the cognitive capabilities of users that are relevant to PIM design.









%%%%%%%%%%%%%%%%%%%%%%%%%%%%%%%%%%%%
\section{Cross-tool limitations}
%%%%%%%%%%%%%%%%%%%%%%%%%%%%%%%%%%%%
MOVE TO FINAL DISCUSSION: Exceptions include~\citep{Bellotti:00} but even this focused on email.  More understanding of PIM is required from a cross-tool perspective is required. How do PIM strategies vary across distinct collections such as files and email? Do individuals have highly different needs in each collection? 

%%%%%%%%%%%%%%%%%%%%%%%%%%%%%%%%%
\section{Empirical grounding}
%%%%%%%%%%%%%%%%%%%%%%%%%%%%%%%%%
The authors observed the difficulties in using what they termed fundamental research such as cognitive psychology in high-level design systematically.  They note that many such findings are from lab contexts and may therefore not apply in a natural setting. In addition it may be hard to distinguish particular effects in everyday usage. 

