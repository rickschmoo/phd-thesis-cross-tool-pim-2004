\newpage
%%%%%%%%%%%%%%%%%%%%%%%%%%%%%%%%%%%%%%%%%%%%
%%%%%%%%%%%%%%%%%%%%%%%%%%%%%%%%%%%%%%%%%%
\section{Discussion} %  of Prior Research}
\label{review:discussion}
%%%%%%%%%%%%%%%%%%%%%%%%%%%%%%%%%%%%%%%%%%
%%%%%%%%%%%%%%%%%%%%%%%%%%%%%%%%%%%%%%%%%%%%
% phrase: used to argue the potential
% Emphasise theory/practice gap
% PIM has surged ahead of theory (i.e. its a successful application)
%% HCI left behind
% Tone down critcisms and frame as `open research questions'
%%%%%%%%%%%%%%%%%%%%%%%%%%%%%%%%%%%%%%%%%

%%%%%%%%%%%%%%%
% SIGNPOSTING
%%%%%%%%%%%%%%%
%summarizes key limitations of previous work, and motivates 
%, before a focus is taken on research concerned with integration.
%portrayed as a break in the Task-Artefact cycle. Recommendations for future research are proposed.
This section draws together the three areas of research -- empirical studies, design and theory -- discussed in \textbf{Sections~\ref{review:pim-empirical-review}}, \textbf{~\ref{review:pim-technological-review}}, and \textbf{~\ref{review:pim-theory-review}}.  The overall state of HCI knowledge concerned with the PIM application domain is summarized in \textbf{Section~\ref{review:discussion:critical-analysis}}.  Limitations in the area of PIM-integration are identified, and used to motivate the thesis research agenda in \textbf{Section~\ref{review:research-agenda}}.  Finally, \textbf{Section~\ref{review:methodology}} details the selection of methodology for the thesis.


%%%%%%%%%%%%%%%%%%%%%%%%%%%%%%%%%
\subsection{Critical Analysis of Previous Research}
\label{review:discussion:critical-analysis}
%%%%%%%%%%%%%%%%%%%%%%%%%%%%%%%%%
%% SUMMARY: frame situation within PIM-research (creation of a systematic knowledge base) as a break in the task-artefact cycle. Emphasis: reuse}.
% empirical-contribution-discussion
% Thus researchers starting work in the field need to start work from scratch. 
% Whittaker \textit{et al}. argue that this basic groundwork is essential for the development of theory in the area. 
In this chapter, two main engines of research progress have been identified: \textit{empirical studies} and \textit{technology design}. However, despite the body of previous work in each area, the author echoes the affirmation by~\citep{Whittaker-rta:00} that PIM has not received the attention it merits as a fundamental computer-based activity.  In terms of empirical studies, \textbf{Section~\ref{empirical-contribution-discussion}} highlighted a lack of attention to longitudinal aspects of PIM, and to the needs of non-technical users.  \citet{Whittaker-rta:00} argue that this insufficient empirical grounding has contributed to the lack of consensus on descriptive vocabulary, task decompositions, evaluation metrics, or the key problems that need to be solved.  

\textbf{Section~\ref{technological-contribution-discussion}} surveyed previous technological prototyping in the research domain, with a particular focusing on that offering increased PIM-integration.  Two key problems were highlighted.  Much of this body of work can be criticised in terms of making a strong contribution to the HCI knowledge base due to: (1) a lack of grounding in empirical requirements, and (2) a lack of evaluation. This situation can be portrayed as a \textit{break in the task/artefact cycle}~\citep{Carroll-cycle:91}. Studies of user practices are not providing firm grounding for design, which is in turn not being systematically evaluated.

% When evaluation and requirements exist they have been inconsistent and therefore it is hard to compare between designs. It is also observed that existing methods are very application-focused and enforce compartmentalization. %% tidy this waffle

%%%%%%%%%%%%%%%%%%%%%%%%%%%%%
% WHY POOR EPISTEMIC STATE
%%%%%%%%%%%%%%%%%%%%%%%%%%%%%
% Secondly PIM is a hard phenomenon to study. It is long-term and highly idiosyncratic meaning that traditional methods of cognitive-model based analysis such as GOMS are not applicable. Generally there is a lack of theory to build on, and a lack of appropriate evaluation methods. 
Several reasons can be suggested for this poor epistemic state. Firstly, PIM may be seen as ``an area of old technology''.  \citet{Carroll-review:97} discusses the strong research interest on collaborative technology, and the author speculates that PIM may seem a backward, unchallenging area to some researchers.  Secondly, there is a lack of theory and methodology concerning complex high-level activities such as PIM.  \citet{ad:01} and \citet{Whittaker-rta:00} note the lack of appropriate evaluation metrics for evaluating PIM designs. For instance, traditional measures of usability may be inappropriate for evaluating interfaces that support discretionary activities such as PIM.  Thirdly, PIM is a multi-faceted, ongoing, and highly idiosyncratic activity, and may be seen as too challenging an area.  Perhaps this explains the much larger amount of work carried out towards the lower-level area of information retrieval.   In the meantime, progress in developing PIM technology in the commercial arena, is out-stripping the level of knowledge in the research community.  This phenomenon is not specific to PIM and has been termed the \textit{research/practice gap}~\citep{Carroll:00,rogers:04}.  % in the software development arena far out-strips that in HCI, and the evaluation of innovative PIM-tools is left to the market-place (for example no published research on Windows Longhorn~\citep{winfs:03} has appeared).

% One suggested route forward has been proposed by~\citet{Whittaker-rta:00}.  They call for the refocusing of research around \textit{reference tasks}, core everyday tasks such as those involved in PIM.  They describe how such a research focus has proved useful in other research fields.  They call for the definition of tyasks, descriptive vocabulary, task decomposition into main components, metrics for evaluation. This would allow objective comparison of different interaction techniques and enable systematic theory development.

%% Summarise as broken task/artefact cycle Use Task/Artifact Cycle~\citep{Carroll-cycle:91,jc-cycle:92} as a perspective - cycle broken
%%%%%%%%%%%%%%%%%%%%%%%%%%%%%%
% OVERALL: LACK OF ATTENTION
%%%%%%%%%%%%%%%%%%%%%%%%%%%%%%

%%%%%%%%%%%%%%%%%%%%%%%%%%%%%%%%%%%%%%%%%%%%%%%
\subsection{Research Agenda}
\label{review:research-agenda}
%%%%%%%%%%%%%%%%%%%%%%%%%%%%%%%%%%%%%%%%%%%%%%%

% Much has been written on HCI's recent turn \textit{towards the social}, e.g.  \textit{``The portrait of a solitary user finding and creating information in a computer became background to the portrait of several people working together at a variety of times and places''}~\citep{Carroll-review:97}.  Although the recent trend \textit{towards the social} has opened up many interesting areas for research, it has also contributed towards many core everyday computer-based activities such as PIM being under-researched~\citep{Whittaker-rta:00}.  

Turning the pages of recent HCI conferences reveals a wealth of current research directed at collaborative and social interfaces, and other ``cutting edge'' areas such as virtual reality and intelligent agents, but very little on important everyday activities such as PIM.  This thesis seeks to help readdress this balance by taking a conscious turn back towards fundamental, everyday user problems.  After surveying the field, the author developed a particular interest in the problems caused by a user's information being fragmented across a set of distinct PIM-tools.  The PIM-integration genre has attempted to provide solutions to these problems by providing functionality which bridges multiple tools.  However, as noted above, much of this work can be criticised for: (1) a lack of evaluation, and (2) insufficient empirical grounding.

% One area of particular interest to the researcher is the PIM-integration design genre, which has attempted to deal with 

% in the other direction, away from collaborative activities back towards the individual user and a focus on the most personal of activities, PIM.  
% careful PIM: has collaborative aspects


%%%%%%%%%%%%%%%%%%%%%%%%
% ORIENTING INTERESTS
%%%%%%%%%%%%%%%%%%%%%%%%
% The author's initial interests which he brought to the research are described before the initial research objectives are presented.
%%%%%%%%%%%%%%%%%%%%%%%%
% Interest in DESIGN
%  As well as the general interest in integration, another key interest of the author was in design -- in actually implementing the fruits of his research in a concrete artefact which could then be evaluated. 
%% Need to pursue integration-design in an incremental manner, need for requirements, 
%% Balance what I want to do with what is pragmatically possible (what I'm best-equipped to do)

%%%%%%%%%%%%%%%%%%%%%%%%%%%%%%%%%%
% \subsection{Objectives}
%%%%%%%%%%%%%%%%%%%%%%%%%%%%%%%%%%

%%%%%%%%%%%%%%%%%%%%%%%%%%%%
% AIMS/GOALS/OBJECTIVES
%%%%%%%%%%%%%%%%%%%%%%%%%%%%
% Guided by the interests described above, and building on the limitations of previous work in the area of improving integration -- the research was inherently cross-tool: i.e. not focused on a specific application.
% The thesis objectives are defined as follows: (1) to build on current understanding, (2) to provide guidance for designers.
This high-level aim of this research programme was to provide guidance for the designers of PIM-integration mechanisms.  Three key research objectives were identified:
\begin{enumerate}

%%%%%%%%%%%%%%%%%%%%%%%%%%%%%%%%%%%%%%%%%%%%%%%%%%%%%%%%%%%%%%%%%%
% EMPIRICAL : Objective 1: To build on current understanding}
%%%%%%%%%%%%%%%%%%%%%%%%%%%%%%%%%%%%%%%%%%%%%%%%%%%%%%%%%%%%%%%%%%
% Requirements for design.
% provide evidence/description
% Under-researched. Accept observations of relative lack of attention that PIM has received to date within HCI and take steps towards alleviating it. Need for more understanding of core phenomena. Emphasise grounding in empirical data. Empirical stance adopted
% % There is a lack of understanding of fundamental everyday computer-based activities  such as PIM. Importance of grounding in user experiences with the current generation of PIM artefacts (concern for ecological validity).  Understand the world. Study real practice. Use to develop understanding of PIM. 
%%%%%%%%%%%%%%%%%%%%%%%%%%%%%%%%%%%%%%%%%%%%%%%%%%%%
% The first objective was to provide improved understanding of personal information management, with a particular focus on investigating user needs and issues relating to integration between PIM-tools.
%%%%%%%%%%%%%%%%%%%%%%%%%%%%%%%%%%%%%%%%%%%%%%%%%%%%
%%%%%%%%%%%%%%%%%%%%%%%%%%
% ENVISAGED CONTRIBUTION
% The planned method to achieve this objective was to carry out a cross-tool study. The study would provide increased cross-tool understanding of PIM, and so provide an empirical foundation for cross-tool design work aimed at improving PIM integration.
%%%%%%%%%%%%%%%%%%%%%%%%%%%%%%%%%%%%%%%%%%%%%%%%%%%%
\item \textit{To develop increased understanding of PIM behaviour and user needs} -- In order to provide a firm empirical foundation for PIM-integration design work, the author planned to investigate user behaviour and needs beyond the boundaries of specific tools.
%%%%%%%%%%%%%%%%%%%%%%
% LEAD -> THEORETICAL
%%%%%%%%%%%%%%%%%%%%%%%
A secondary aim was to develop theoretical models to describe and explain empirical observations.


%%%%%%%%%%%%%%%%%%%%%%%%%%%%%%%%%%%%%%%%%%%%%%%%%%%%%%%%%%%%%%%%%%%%%%%%%%
% DESIGN/EVALUATION : Objective 2: To provide guidance for designers}
%%%%%%%%%%%%%%%%%%%%%%%%%%%%%%%%%%%%%%%%%%%%%%%%%%%%%%%%%%%%%%%%%%%%%%%%%%%%%
% improved tool support for PIM.
% \textit{The author embarked on the work with a strong interest in design, and a background in engineering, which lead to the intention to create an interactive artefact. Commitment to simplifying?  Interest in non-professional/non-technical users?~\citep{web-good-easy:01}}
% Propose technology and evaluate, contrast with more ambitious programs to revolutionize the desktop.
% To design and evaluate new PIM technology in practice (real users, real data, real tools, real environment, over time). Artefact generation. Yet need to do so systematically and contribute to HCI knowledge-base.
%%%%%%%%%%%%%%%%%%%%%%%%%%%%%%%%%%%%%%%%%%%%%%%%%%%
% THE OBJECTIVE
% The second objective was to produce a set of design recommendations regarding the design of improved PIM tools, particularly design work aimed at improving PIM integration.
% This objective was to be achieved through design/evaluation experience.
% The intention was to use findings from the initial study to form grounded design rationale to direct the design.
%%%%%%%%%%%%%%%%%%%%%%%%%%%%%%%%%%%%%%%%%%%%%%%%%%%
% THE METHOD
%%%%%%%%%%%%%
% Due to the lack of appropriate methodology, a parallel aim was to explore issues related to the evaluation of PIM tools, particularly those aimed at improving PIM integration. 
%%%%%%%%%%%%%%%%%%%%%%%%%%%%%%%%%%%%%%%%%%%%%%%%%%%
\item \textit{To propose, implement and evaluate an empirically-grounded means of PIM-integration} -- The author embarked upon the research programme with a keen interest in developing a novel PIM-integration mechanism.  A key interest was to improve upon the limitations of previous design-centred research, by emphasising empirical grounding and evaluation, and thus making a more systematic contribution to HCI knowledge.
% However, in contrast to most previous design work in the area, the intention was t
 
 %%%%%%%%%%%%%%%%%
% METHODOLOGICAL
%%%%%%%%%%%%%%%%%
% Methodological: Explore potential of cross-tool approach for both researchers and designers of cross-tool approach. Evaluation: explore ways of doing effective evaluation.  To explore issues related to the evaluation of PIM designs, particularly those directed at improving PIM integration.
% regarding appropriate methodology
% devise appropriate methodology to support the above objectives of investigating PIM, and designing and evaluating PIM-integration mechanisms.  Furthermore, based on these experiences, to produce a set of recommendations regarding the design of PIM interfaces, particularly those aimed at improving PIM integration.
\item \textit{To devise methodological recommendations for future research and design work } -- The final objective was to provide methodological guidance for future work in the area of PIM-integration (and PIM more generally), based on the experience gained in pursuing this course of research.  \citet{Whittaker-rta:00} note the lack of appropriate methodology such as evaluation metrics.

\end{enumerate}

% The next section discusses the methodological approach employed in the thesis to achieve the above objectives.


%%%%%%%%%%%%%%%%%%%%%%%%%%%%%%%%%%%%%%%%
\subsection{Selection of Methodology}
\label{review:methodology}
%%%%%%%%%%%%%%%%%%%%%%%%%%%%%%%%%%%%%%%%
% This section collates a summary of the methods used in personal information management research. These findings are used to justify the selection of methods used in this thesis.
% Previous empirical work has mostly been based on an ethnomethodological approach, although objective studies have also been carried out -- mostly in terms of task-specific evaluations of new designs. There is no agreed understanding of core problems.
%% \item Requirements gathering: Studies -- ethnographic/fieldwork techniques. Main engine.
% Theory-building based on empirical data has been rare and there has is no agreed methodological approach. In some ways this is consistent with the state of affairs in HCI.
%% Theory-building: Cognitive modelling~\citep{ob:00}. Rare.
% There is no generally accepted design methodology, or approach for describing design rationale.
% Design methods remain controversial in terms of design rationale. 
%In general the design work has been radical/revolutionary, whilst incremental (evolutionary, cumulative) design work is seen as more scientific by many researchers such as Kuhn~\citep{newman:95,Carroll:00}. In some cases design has been based on specific empirical findings~\citep{Bellotti:00,Bellotti:03,bn:98}. The author is not aware of any collaborative design being carried out in the area. In other cases it has been based on specific theoretical principles~\citep{ml:92}. However the majority of design work has been based on designers' intuition and anecdotal reports of user problems. This is not a problem in itself, however the additional lack of evaluation is.
% There is clearly scope for theory to guide design work and further studies.However it has been noted that HCI is not well equipped with appropriate techniques for dealing with PIM.  For example, Dillon~\citep{ad:01} notes the lack of appropriate evaluation methodology for evaluating interfaces that support discretionary activities (activities with no immediate goal) such as PIM.  Suggested approaches include \textit{Reference Tasks} - limiting context to a specific user-group/tool/task such as information retrieval~\citep{Whittaker-rta:00}.  But there are inherent limitations in focusing on one sub-task in this way, when one considers the inter-relationship between the different aspects of PIM - e.g. retrieval and previous organization.  %% THIS NEEDS TIDYING
%%%%%%%%%%%%%%
% Evaluation
%%%%%%%%%%%%%% In addition, there is no consensus on evaluation metrics~\citep{Whittaker-rta:00}. This means that in those cases where evaluations have been carried out, it is hard to compare between designs. In many cases, evaluations have not been performed at all.
%% What other methodological lessons have been learned? e.g. Import of individual differences

Methodology has been defined as \textit{``something that avoids you having to think too hard about how to do something''}~(anon.). % However the lack of methodological consensus in the area means that the author had to think very hard about how to do everything!
%%%%%%%%%%%%%%%%%%%%%%%%%
% Classic HCI dilemma
%%%%%%%%%%%%%%%%%%%%%%%%%
However, the selection of appropriate research methodology is a classic HCI dilemma.  As an interdisciplinary research field, HCI offers many research approaches and methodologies~\citep{Sasse:97}.
%%%%%%%%%%%%%%%%%%%%%%%%%%%%%%%%%%%%%%%%%%%%%%%%%%
% Selection of design-centered methodology
%%%%%%%%%%%%%%%%%%%%%%%%%%%%%%%%%%%%%%%%%%%%%%%%%%
% , chosen as it matches the two \textbf{orienting commitments} of this research:
% Blomberg et al.~\citep{blomberg:96} similarly advocate a design-centred approach through two complementary commitments design and research to support each other and improve~\citep{blomberg:96}.
% Design-based research is selected as an appropriate methodology to enable the candidate to experience design issues at first hand. Turn towards design science route. Promise of design-based research methodology. Subscribe to design-based research methodology. Outline how to do research by this approach. Outline how to add to HCI KB according to~\citep{jc:00}. Recap pros and cons and how I will account for them.
%% A view that my research takes up with gusto.  Need to get hands dirty
The methodology employed in this thesis is heavily influenced by the \textit{design-based} research paradigm, as advocated in~\citet{Carroll:00}.  Carroll describes how design can be employed as a research method to achieve two complementary goals: (1) to \textit{understand the world} through the process of gathering design requirements, and (2) to \textit{improve the world} by creating a new artefact.  He contrasts this applied research paradigm (literally ``research through design'') with traditional perspectives on design as a \textit{craft}, or design as \textit{the object of research}.  Carroll argues how the designed artefact can be interpreted as a theory, a set of claims regarding how a particular situation of concern can be improved.  Theory development, the validation of the designers' claims, is enabled through the subsequent evaluation of the design, a crucial stage of the research process.  This assessment of the strengths and weaknesses of a specific design may then be generalizable to a wider design genre. The task-artefact cycle~\citep{Carroll-cycle:91} forms a backdrop to the research approach: the study of a task context provides the requirements for the design of an artefact, which is then in turn evaluated in the context of the original task.  Evaluation also provides an opportunity for further empirical discovery (understanding of the world).


% The evaluation of the tool then enables further investigation of the nature of the activity.
% The end-product of the research is the designed artefact itself, grounded in its empirical and theoretical motivation, and post-design evaluation.

%%%%%%%%%%%%%%%%%%%%%%%%%%%%%%%
% APPROPRIATENESS OF APPROACH
%%%%%%%%%%%%%%%%%%%%%%%%%%%%%%%
% A key concern in HCI is the so-called \textit{theory/practice gap} (or research/practice gap)~\citep{Sutcliffe:00,rogers:04}.  There is concern that the products of much HCI research may be irrelevant to HCI practitioners who pursue their craft without need for theory or strict methods. 
% The traditional view of research is that theory leads practice, 
% in reality often not the case. In many cases design leads whilst theory catches up over time.
The approach was seen to be highly compatible with the author's desire to produce a novel PIM-integration mechanism, whilst also supporting the exploration of user behaviour, and theoretical development.  Also, it was envisaged that by experiencing design issues at first hand, the author was more likely to produce findings in a form of relevance to practitioners. % . A key concern in HCI is the so-called \textit{theory/practice gap} (or research/practice gap)~\citep{Sutcliffe:00,rogers:04}, whereby the products of much HCI research can be irrelevant to designers' needs in the real-world.


% Therefore In this paradigm, research is performed through a standard iterative-design process: the investigation of user needs to establish design requirements, the design of interactive artefacts to meet those needs, and the evaluation of the design in real-world use. 



%%%%%%%%%%%%%%%%%%%%%%%%%%%%%%%%%%%%%%%%%%%%%%%%%%%%%%%%%%%%
%% NB: need to make case for the selected research method
%%%%%%%%%%%%%%%%%%%%%%%%%%%%%%%%%%%%%%%%%%%%%%%%%%%%%%%%%%%%
%% Make the case for my chosen methods. The choice of methods to be employed in the research are outlined. The main approaches to HCI research in general are discusseda above. Prior HCI research methods employed in the PIM application domain are surveyed. 
%%%%%%%%%%%%%%%%%%%%%%%%%%%%%%%
% \subsubsection{Research Methodology}
%%%%%%%%%%%%%%%%%%%%%%%%%%%%%%%
%%%%%%%%%%%%%%%%%%%%%%%%%%%%%%%%%%%%%%%%%%%%%%
% Walkthrough the stages of the methodology
%%%%%%%%%%%%%%%%%%%%%%%%%%%%%%%%%%%%%%%%%%%%%%
% employs a 3-stage , centered on a process of user-centred design~\citep{} consisting of three stages: (a) an exploratory study, (b) design work, (c) follow-up evaluation and further study.
% The selected method is a design-based research methodology and can be considered as an instance of standard HCI practice~\citep{Whittaker-rta:00}: (1) interviews users to understand needs, (2) develop system to meet needs, and (3) evaluate it to see if it meets those needs.
\newpage
Specifically, the research reported in subsequent chapters is centred on a 3-stage \textit{user-centred design} methodology:

\begin{enumerate}

%%%%%%%%%%%%%%
% FIRST STAGE
%%%%%%%%%%%%%%
%, managed within the file system, email tool, and web browser respectively.
% DIANE: not too much detail here
% Tio provide evidence/description
% Requirements will be gathered through a cross-tool study to build cross-tool understanding of PIM, and so provide an empirical foundation for cross-tool design work aimed at improving PIM integration. The study methodology will consist of semi-structured interviews as used in previous empirical work in this area. The aim is to compare how individuals manage a range of types of personal information, to investigate the effectiveness of current forms of integration, and so as support the generation of empirically-grounded ideas of how further integration can be provided.
\item \textit{Requirements gathering} -- The research is empirically grounded in an exploratory study, reported in \textbf{Chapter~\ref{chapter:exploratory_study}}.  The study was aimed at both developing more understanding of PIM, and establishing requirements for subsequent design.  % \textbf{Chapter~\ref{chapter:exploratory_study}} details the study which compares PIM practices across 3 collections of personal information: personal files, email, and web bookmarks. 

%%%%%%%%%%%%%%%%%%
% SECOND STAGE
%%%%%%%%%%%%%%%%%%
\item \textit{Design and prototyping} -- Findings from the exploratory study are used to motivate the design and implementation of a prototype PIM-integration mechanism in \textbf{Chapter~\ref{chapter:design}}. In order to facilitate systematic evaluation, and cause minimum disruption to users, the design route is \textit{incremental} rather than revolutionary~\citep{newman:95}. % The prototype was proposed as a research vehicle to enable the investigation of general issues relating to PIM integration during a field study.
% The design was prototyped so as to embody specific design hypotheses that express intended improvements.
% The findings from the exploratory study will provide the grounding for the design of an interface providing enhanced PIM integration. 
% the design may be a new form of integration or modify an existing form
%% contrast with unification systems that unify and propose a new organizational paradigm. I just do one

%%%%%%%%%%%%%%%%%%
% THIRD STAGE
%%%%%%%%%%%%%%%%%%
\item \textit{Evaluation} -- \textbf{Chapter~\ref{chapter:main-study}} reports the field-study based evaluation of the prototype. \citet{ml:92} notes the importance of evaluating PIM technology over time. As well as evaluating the proposed design, the field study also enabled the investigation of long-term user behaviour such as changes in strategy over time. 
% Note that the field study also facilitates the investigation of long-term aspects of PIM such as changes in strategy over time. Such longitudinal aspects of PIM have received little attention to date.
% Since no evaluations of this type have been carried out, an important part of this work is the development of appropriate methodology, both for evaluating PIM-tools in general, but also for evaluating means of integration.
% Finally the main study also allowed the investigation of appropriate methodology for evaluating PIM tools.
% Do these really go together? Maybe bring out in the discussion - rather than as an up-front aim!
% Finally the effectiveness of the design will be investigated through a field study-based evaluation.
%% to assess impact of the design intervention

% \item The iterative process of study, design and evaluation provided a platform for theory-building.

\end{enumerate}

% This three-step research agenda will be portrayed as an excursion around the task-artefact cycle. \textbf{Section~\ref{review:discussion}} discusses how much previous PIM research breaks this cycle.
% The next section summarizes the work presented over subsequent chapters, and details the research contributions offered.





%%%%%%%%%%%%%%%%%%%%%%%%%%%%%%%%%%%%%%%%
%\subsection{Justification of Methodology}
%\label{review:methodology}
%%%%%%%%%%%%%%%%%%%%%%%%%%%%%%%%%%%%%%%%
%In this section, recommendations are made for future direction in research into personal information management.
% Personal information management is considered in general, before the section focuses on research into improving integration between PIM-tools.
%%%%%%%%%%%%%%%%%%%%%%%
% Interest in INTEGRATION
%%%%%%%%%%%%%%%%%%%%%%%
%At an early stage, the author expressed an interest in design efforts aimed at improving integration between PIM-tools. A lack of integrated, applicable body of knowledge on integration was identified in much of this chapter. There is a lack of descriptive vocabulary for discussing such technology, and its usage has not been researched. An initial conceptual framework was presented in \textbf{Chapter~\ref{chapter:bg}} but this is clearly just a first step.
% be more specific
%The majority of \textbf{Section~\ref{review:pim-technological-review}} was devoted to a survey of this work, much of which can be criticized for a lack of empirical/theoretical foundation and a lack of evaluation.  Several steps were identified in \textbf{Section\ref{review:research-recommendations}} for making progress in this area.
%The promise of new technology in this area from the commercial sector means that artefacts in use will continue to move beyond continued research understanding. This is an important area of user activity that needs increased and more systematic research understanding to guide future design work.
%In conclusion, four recommendations are made by the author to enable HCI research to provide more effective knowledge to designers seeking to improve \textit{integration} between PIM tools in a more scientific way: 
%\begin{enumerate}
%\item A descriptive vocabulary must be defined so that different forms of integration can be described using consistent terminology. This goes hand in hand with the theoretical groundwork talked about by Whittaker \textit{et al}. and discussed above.
%\end{enumerate}
%%%%%%%%%%%%%%%
% SIGNPOSTING
%%%%%%%%%%%%%%%
% This section summarized the current state of research and outlined areas for improvement. \textbf{Section~\ref{intro:aims}} sets out a research agenda for this thesis within this context. % Here, \textbf{Section~\ref{review:conclusion}} concludes the chapter and states the contributions towards the overall thesis.
%% Moving on towards how I can take a step towards this in my PhD ...

%%%%%%%%%%%%%%%%%%%%%%%%%%%%%%%%%%%
% \newpage
%%%%%%%%%%%%%%%%%%%%%%%%%%%%%%%%%%%%%%%%%%%%%%%%%%%%%%%
%%%%%%%%%%%%%%%%%%%%%%%%%%%%%%%%%%%%%%%%%%%%%%%%%%%%%%
% MOVED-TO/MERGED-WITH CHAPTER 1
%\section{A Research Agenda for this Thesis}
%\label{review:research-agenda}
%%%%%%%%%%%%%%%%%%%%%%%%%%%%%%%%%%%%%%%%%%%%%%%%%%%%%%
%%%%%%%%%%%%%%%%%%%%%%%%%%%%%%%%%%%%%%%%%%%%%%%%%%%%%%%
%% MOVING ON: emphasise huge area, need to focus}


%%%%%%%%%%%%%%%%%%%%%%%%%%%%%%%%%%%
%%%%%%%%%%%%%%%%%%%%%%%%%%%%%%%%%%%
%%%%%%%%%%%%%%%%%%%%%%%%%%%%%%%%%%%
%%%%%%%%%%%%%%%%%%%%%%%%%%%%%%%%%%%
% \newpage
%%%%%%%%%%%%%%%%%%%%%%%%%%%%%%%%%%%
%\section{Conclusion}
%\label{review:conclusion}
%%%%%%%%%%%%%%%%%%%%%%%%%%%%%%%%%%%
%In this section conclusions from \textbf{Chapter~\ref{chapter:review}} are presented. 
% Contributions include a critical review of the literature and the analysis of key limitations, a taxonomy of PIM technology design research relating to integration, and the presentation of recommendations for future research in the area. Analysis of limitations of previous research lead to the development of the initial thesis objectives presented in \textbf{Section~\ref{intro:aims}}.
% \textbf{Chapter~\ref{chapter:exploratory_study}} presents an exploratory study of personal information management, the first step on that research agenda.

\textbf{Chapter~\ref{chapter:exploratory_study}} now moves on to report the first stage of the design-centred methodology employed in this thesis, an exploratory study that compares how people manage three types of personal information: files, email and bookmarks.

%%%%%%%%%%%%%%%%%%%%%%%%%%%%%%%%%
% FIN THESIS CHAPTER 3: LITREVIEW
%%%%%%%%%%%%%%%%%%%%%%%%%%%%%%%%%
% \textit{This draft of Chapter 3 LITREVIEW was printed \today}