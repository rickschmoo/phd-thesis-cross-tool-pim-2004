%%%%%%%%%%%%%%%%%%%%%%%%%%%%%%%%%
\section{TOOL-SPECIFIC SCRATCH}
%%%%%%%%%%%%%%%%%%%%%%%%%%%%%%%%%

% \textbf{Sections~\ref{exp-study:Results1-WTBU-email}}, \ref{exp-study:Results1-WTBU-files}, and \ref{exp-study:Results1-WTBU-bookmarks} report observed behaviour regarding files, email and bookmarks respectively.  

%%%%%%%%%%%%%%%%%%%%%%%%%%%%%%%%%%%%%%%%%%%%%%%%%%%%%%%%%%%%
\subsection{Characterizing file management strategies}
\label{exp-study:Results1-WTBU-files}
%%%%%%%%%%%%%%%%%%%%%%%%%%%%%%%%%%%%%%%%%%%%%%%%%%%%%%%%%%%%

%%%%%%%%%%%%%%%%%%%%%%%%%
% intro and structure
%%%%%%%%%%%%%%%%%%%%%%%%%
This section reports participants' file management behaviour.
All 25 participants actively maintained collections of files, which were highly prized.
Many users expressed the pride they felt towards their document collections that they had built up over many years. Many users also felt that there was no reason to rationalize them \textit{(P31: "`Some of them I'll need again. Some of them I might need again. Some of the things I'm quite proud of ... I spent a lot of time doing it - why should I throw it away?  It doesn't cost me anything!"' CHANGE)}.

%%%%%%%%%%%%
%% RET - files
%%%%%%%%%%%%
Generally participants reported a strong preference for browsing-based retrieval over search, confirming previous findings in~\cite{bn:95}. We highlight the two types of browsing: browsing of folders/desktop icons, using user-defined locational metadata, and (2) sorting/scanning items ordered by user-defined metadata (e.g. "`name"') or implicit metadata (e.g. "`date created"'). In files, users employed a combination of both types - browsing to a folder, and then sorting items within it. Many participants also used the desktop to store work in progress or temporary files.

%%%%%%%%%%%%%%%%%%%%%%%%%%%%%%%%%%%%%%%%%%%%%%%%%%%%%%%%%%
\subsubsection{Classifying File Management Strategies}
%%%%%%%%%%%%%%%%%%%%%%%%%%%%%%%%%%%%%%%%%%%%%%%%%%%%%%%%%%

All 25 participants carried out extensive organization of document files, and had at least 5 document folders (average 60). All 25 employed a predominantly \textit{file-on-creation} strategy - filing most documents as they created them in a particular folder. However two groups (D1 and D2) were identified based on the extent to which participants stored work-in-progress and/or temporary data as unfiled items on the desktop or in a root folder:

\begin{itemize}
	\item D1 users filed the vast majority of documents on creation. Unfiled documents tended to be left by accident - except for a few deliberately placed work-in-progress files (<25 unfiled items, average 12).
	\item D2 users filed the majority of documents on creation like the D1 users, but also maintained a significant subset of unfiled items (>=25 unfiled items, average 133). \textit{Most D2 users transferred unfiled items to a folder on completion of the relevant activity, or during spring-cleans.} 
\end{itemize}


The location of working/active files varied between the two groups. For the D1 users, working files were distributed within active folders, whilst for the D2 users working files tended to be on the desktop or in the root folder. However even for the D2 users the number of unfiled items was a very small proportion of the overall collection. \textbf{Table~\ref{table:exp-study:file_mment_strats}} summarizes the observed strategies. 



%%%%%%%%%%%%%%%%%%%%%%%%%%%%%%%%%%%%%%%%%%%%%%%%%%%%%%%%%%
% TABLE: A classification of file management strategies}
%%%%%%%%%%%%%%%%%%%%%%%%%%%%%%%%%%%%%%%%%%%%%%%%%%%%%%%%%%
\begin{table}[h]
\begin{center}
\begin{footnotesize}
\setlength{\extrarowheight}{2pt}
\begin{tabular}{|c|p{4cm}|c|c|c|}
\hline
{\bf Strategy} & {\bf Description} & {\bf \# Users} & {\bf Average \# Folders} & {\bf Average \# unfiled items} \\
\hline
 {\bf D1 } & Extensive file-on-creation (<25 unfiled items) &         14 & 50 (SD = 31)  & 12 (SD = 7) \\
\hline
 {\bf D2 } & Less extensive file-on-creation (>= 25 unfiled items) &         11 & 47 (SD = 31) & 133 (SD = 132) \\
\hline
           & {\it Total check} &   {\it 25} &            &            \\
\hline
\end{tabular}  
\end{footnotesize}
\caption{Classification of observed file management strategies [n=25]}
\label{table:exp-study:file_mment_strats}
\end{center}
\end{table}
\normalsize

%%%%%%%%%%%%%%%%%%%%%%%%%%%%%%%%%%%%%%%%%%%%%%%%%%%%%%%%%%%%%%%%%%%%%%
\subsubsection{Organizational Dimension Analysis of File Folders}
%%%%%%%%%%%%%%%%%%%%%%%%%%%%%%%%%%%%%%%%%%%%%%%%%%%%%%%%%%%%%%%%%%%%%%

The most common organizational dimensions for file folders are listed in \textbf{Table~\ref{table:exp-study:file-orgdims}}. % and \textbf{Figure ADD}
The three most common dimensions for document file folders were \textit{Project} (e.g. ``Term-paper''), \textit{Role} (e.g. ``Teaching'') and \textit{Document Class} (e.g. ``reports''), representing percentages of 29\%, 17\% and 14\% respectively. 
\textit{The roughly even split between these three types suggests that users would be constrained by a particular subordinate dimension as proposed in tools such as UMEA~\cite{Kaptelinin:03} (NO-NO-NO!)}.

%%%%%%%%%%%%%%%%%%%%%%%%%%%%%%%%%%%%%%%%%%%%%%%%%%%%%%%
% TABLE: SUMMARY OF ORG-DIMS FOR FILES
% in tables/ch2/exp-study-tables.xls
%%%%%%%%%%%%%%%%%%%%%%%%%%%%%%%%%%%%%%%%%%%%%%%%%%%%%%%
\begin{small}
\begin{table}
\begin{center}
\begin{footnotesize}
\setlength{\extrarowheight}{2pt}
\begin{tabular}{|c|c|c|c|}
\hline
{\bf Rank} & {\bf Dimension} & {\bf Count} &   {\bf \%} \\
\hline
         1 & {\bf Project} &        317 &       29\% \\
\hline
         2 & {\bf Class of document} &        185 &       17\% \\
\hline
         3 & {\bf Role} &        148 &       14\% \\
\hline
         4 & {\bf Contact} &         84 &        8\% \\
\hline
         5 & {\bf Topic / Interest} &         72 &        7\% \\
\hline
         6 & {\bf Format} &         58 &        5\% \\
\hline
         7 & {\bf Event} &         47 &        4\% \\
\hline
         8 & {\bf Temporary} &         45 &        4\% \\
\hline
         9 & {\bf Version control} &         38 &        4\% \\
\hline
        10 & {\bf Geographic location} &         26 &        2\% \\
\hline
        11 & {\bf General} &         22 &        2\% \\
\hline
        12 & {\bf Time} &         18 &        2\% \\
\hline
        13 & {\bf Backup} &         17 &        2\% \\
\hline
        14 & {\bf Others (<1\%)} &          8 &        1\% \\
\hline
           & {\bf Total} &       1085 &            \\
\hline
\end{tabular}  
\end{footnotesize}
\caption{Observed organizational dimensions in files [n=25]}
\label{table:exp-study:file-orgdims}
\end{center}
\end{table}
\end{small}


%%%%%%%%%%%%%%%%%%%%%%%%%%%%%%%%%%%%%%%%%%%%%%%%%%%%%%%%%%%%
\subsection{Characterizing email management}
\label{exp-study:Results1-WTBU-email}
%%%%%%%%%%%%%%%%%%%%%%%%%%%%%%%%%%%%%%%%%%%%%%%%%%%%%%%%%%%%

%%%%%%%%%%%%%%%%%%%%%%%%%
% intro and structure
%%%%%%%%%%%%%%%%%%%%%%%%%
This section reports the observed email management behaviour. \textbf{Table~\ref{table:chapter3_email_and_browser_tools}} summarizes the range of email tools that were encountered.
Email collections were valued relatively less than files, but most users noted the sentimental and professional value of a subset of their messages. (\textit{P24: "`I keep them to make sure I've got one thing from them to reply to (for their address). Also it's nice that the person has written. And sometimes they are literally amusing!"'}).


In email, most users indicated that retrieval was focused on sorting/scanning the inbox. Location-based browsing of folders was less common.  Search was used more commonly in email than the other collections, but was seen as a last resort by most users (P25: "I usually know exactly where I'm going and what I'm looking for. If I search I wouldn't necessarily know the exact keyword to search for. And it takes a long time - if you know where you're going, browsing is a lot quicker than using the search").

%%%%%%%%%%%%%%%%%%%%%%%%%%%%%%%%%%%%%%%%%%%%%%
\subsubsection{Classifying Email Management Strategies}
%%%%%%%%%%%%%%%%%%%%%%%%%%%%%%%%%%%%%%%%%%%%%%


The author attempted to classify the 25 participants' email management strategies  using previous classifications of user behaviour from the literature~\cite{Whittaker-email:96,ob:97}. Whittaker \& Sidner observed 3 types of strategies for managing email: \textit{frequent filer}, \textit{spring cleaner} and \textit{no-filer}~\cite{Whittaker-email:96}. B�lter extended this classification by dividing the \textit{no-filer} class into \textit{folderless cleaner} and \textit{folderless spring-cleaner}, depending on whether old items are deleted from the inbox on a daily basis~\cite{ob:97}\footnote{See \textbf{Section~\ref{ch2:pim-empirical-review}} for more discussion of previous classifications of approaches to managing email}.
%% \\ 
However it was found that much of the observed behaviour did not map well onto these classifications. Of the 25 users who collected email, only 10 mapped onto the previously proposed classes of user behaviour. 2 participants were \textit{non-filers} (folderless spring-cleaners~\cite{ob:97}), and 8 participants were \textit{frequent filers} who cleaned their inbox everyday. No participants matched the \textit{spring-cleaner} class (users who clean their inbox only periodically). The remaining 15 participants did not fit any of the classifications presented in earlier research.  



The remaining participants had large inboxes (>75 items, average 1137 items).  Whittaker \& Sidner identify a large inbox as an attribute of \textit{no-filers} and \textit{spring-cleaners}. However the behaviour of the remaining participants did not match either of these classes. The remaining 15 participants filed a \textit{subset} of incoming emails straight away, typically those of high long-term value, e.g. e-commerce receipts. In addition they deleted low-value emails such as spam straight away.  Other emails were left in the inbox, which was occasionally spring-cleaned or purged. 
%%  
The author proposes the term \textit{multiple-strategy users} to designate this class of users who employed an intermediate combination of the frequent filer, spring cleaner, and no-filer strategies. The multiple-strategy users could be split into two groups depending on the proportion of incoming emails that they manually filed on a daily basis:

\begin{itemize}

\item The first group reported that they filed 5 or more emails everyday  (\textit{P25: "I'd like to manage as and when I receive them but I don't. I do it periodically, maybe daily - ten minutes a day just to categorize the things that are important for that day. 10 or 15 a day I'll categorize or put them in a folder. The rest of them I think "`oh I'll get round to doing that at some stage - but I don't normally! However I did spend an hour and half on a train last week tidying my emails because I was bored. I reduced my inbox by about 1500 so I was quite pleased by that"'}). This first group had many folders like the frequent-filers (average 39.9), but much larger inboxes (average 1002) reflecting the smaller proportion of messages that were filed.

\item The second group reported that they filed only a minority of messages (less than 5) on a daily basis, often for a very specific purpose, (P31: "I have a folder for registrations. I've got other folders - I don't even know what they are. The vast majority is just a big long list."). This second group had much fewer folders than the first group, reflecting a lower reliance on folders for managing mail.
	
\end{itemize}


The author proposes a new classification of email management strategies based on the four observed strategies. The classification is summarized in Table~\ref{table:exp-study:email_mment_strats} along with the characteristics of each class. 

%%%%%%%%%%%%%%%%%%%%%%%%%%%%%%%%%%%%%%%%%%%%%%%%%%%%%%%%%%
% TABLE: A classification of email management strategies}
%%%%%%%%%%%%%%%%%%%%%%%%%%%%%%%%%%%%%%%%%%%%%%%%%%%%%%%%%%
\begin{table}[h]
\begin{center}
\begin{footnotesize}
\setlength{\extrarowheight}{2pt}
\begin{tabular}{|c|p{3cm}|c|c|c|}
% Table generated by Excel2LaTeX from sheet 'Email STRATS'
\hline
{\bf Strategy} & {\bf Description} & {\bf \# Users} & {\bf Average \# Folders} & {\bf Average Inbox size} \\
\hline
 {\bf E1 } & Frequent filers - small inbox &          7 &       56.1 &       25.6 \\
\hline
 {\bf E2 } & Multiple strategy users - large inbox, file >=5 items everyday &         12 &       42.4 &     1001.6 \\
\hline
 {\bf E3 } & Multiple strategy users - large inbox, file < 5 items everyday &          4 &        3.8 &     1251.3 \\
\hline
 {\bf E4 } & No filers - large inbox, no filing &          2 &          0 &     1105.5 \\
\hline
\end{tabular}  
\end{footnotesize}
\caption{Classification of observed email management strategies [n=25]}
\label{table:exp-study:email_mment_strats}
\end{center}
\end{table}
\normalsize

%%%%%%%%%%%%%%%%%%%%%%%%%%%%%%%%%%%%%%%%%%%%%%%%%%%%%%%%%%%%%%%%%%%%%%
\subsubsection{Organizational Dimension Analysis of Email Folders}
%%%%%%%%%%%%%%%%%%%%%%%%%%%%%%%%%%%%%%%%%%%%%%%%%%%%%%%%%%%%%%%%%%%%%%

The participants' email hierarchies were analyzed in terms of their organizational dimensions (see \textbf{Section \ref{exp-study:folder-analysis-orgdim}} for an overview of the technique used).

The most common organizational dimensions for email folders are listed in \textbf{Table~\ref{table:exp-study:email-orgdims}}. % and \textbf{Figure ADD}
The most commonly observed dimensions were \textit{Role} (e.g. ``Personal''), \textit{Contact} (e.g. ``Alexis''), \textit{Project} (e.g. ``term-paper''), \textit{Topic/Interest} (e.g. ``Java''), and \textit{Mailing List} (e.g. ``linux-users''), representing percentages of 25, 19, 17, 12 and 11\% respectively. This indicates that the participants rely on a wide range of organizational dimensions when naming email folders. \textit{This in turn suggests that  users would be constrained by an organizational mechanism that constrained them to organizing email along one dominant dimension such as role.}
%% The contact dimension appears relatively low in the list \textit{(SHOULD WE EXPECT IT HIGHER?)}. This may be explained by the fact that users could rely on implicit \textit{Sender} metadata, rather than having to organize messages explicitly based on contact.


%%%%%%%%%%%%%%%%%%%%%%%%%%%%%%%%%%%%%%%%%%%%%%%%%%%%%%%
% TABLE: SUMMARY OF ORG-DIMS FOR EMAIL
% in tables/ch2/exp-study-tables.xls
%%%%%%%%%%%%%%%%%%%%%%%%%%%%%%%%%%%%%%%%%%%%%%%%%%%%%%%
\begin{small}
\begin{table}
\begin{center}
\begin{footnotesize}
\setlength{\extrarowheight}{2pt}
% Table generated by Excel2LaTeX from sheet 'Email Dims'
\begin{tabular}{|c|c|c|c|}
\hline
{\bf Rank} & {\bf Dimension} & {\bf Count} &   {\bf \%} \\
\hline
         1 & {\bf Role} &        192 &       25\% \\
\hline
         2 & {\bf Contact} &        150 &       19\% \\
\hline
         3 & {\bf Project} &        133 &       17\% \\
\hline
         4 & {\bf Topic / Interest} &         90 &       12\% \\
\hline
         5 & {\bf Mailing list} &         89 &       11\% \\
\hline
         6 & {\bf Class of document} &         51 &        7\% \\
\hline
         7 & {\bf General} &         27 &        4\% \\
\hline
         8 & {\bf Event} &         22 &        3\% \\
\hline
         9 & {\bf Temporary} &         13 &        2\% \\
\hline
        10 & {\bf Others} &         16 &        2\% \\
\hline
           & {\bf Total} &        783 &      100\% \\
\hline
\end{tabular}  
\end{footnotesize}
\caption{Observed organizational dimensions in email [n=25]}
\label{table:exp-study:email-orgdims}
\end{center}
\end{table}
\end{small}



%%%%%%%%%%%%%%%%%%%%%%%%%%%%%%%%%%%%%%%%%%%%%%%%%%%%%%%%%%%%
\subsection{Characterizing bookmark management strategies}
\label{exp-study:Results1-WTBU-bookmarks}
%%%%%%%%%%%%%%%%%%%%%%%%%%%%%%%%%%%%%%%%%%%%%%%%%%%%%%%%%%%%

%%%%%%%%%%%%%%%%%%%%%%%%%
% intro and structure
%%%%%%%%%%%%%%%%%%%%%%%%%
This section reports participants' bookmark management behaviour. Bookmarks were of low importance for most participants, confirming previous findings in~\cite{kftf:01}.
However only 1 participant did not collect bookmarks at all, most had small collections which were rarely used. Bookmarks were devalued by the existence of alternative mechanisms for re-accessing websites (e.g. search engines), and their ephemeral nature (\textit{P28: "`I don't trust the stability of web URLs, I would rather download the actual document I'm interested in"' (CHANGE)}). 

On average, bookmark collections were small (tens of items) compared to files and email (thousands of items). \textit{(MOVE TO COMPARE)}


%%%%%%%%%%%%%%%
%% RET -- bookmarks
%%%%%%%%%%%%%%%

Reuse of bookmarks was rare for most participants  - and based on scanning recently added or frequently accessed items. Instead many users preferred to search the web again (P28: "If something is really exciting then I bookmark it (or I used to) - then when I come back to it, if I need it again, I just use google.") Note that even these users did feel the need to save some bookmarks, even if they were never to be used again. We observed a similar phenomenon in email, in particular the automatic filing of messages from mailing lists - i.e. the acquisition of items that are never reused ("P29: emails you do save, 90\% you'll never read again"). Whittaker \& Hirschberg [p] observed similar "irrationality" in paper archives (e.g. people want their own copies of documents that are available on the web).

%%%%%%%%%%%%%%%%%%%%%%%%%%%%%%%%%%%%%%%%%%%%%%%%%%%%%%%%%%
\subsubsection{Classifying Bookmark Management Strategies}
%%%%%%%%%%%%%%%%%%%%%%%%%%%%%%%%%%%%%%%%%%%%%%%%%%%%%%%%%%

The author attempted to classify the 25 participants' bookmark management strategies  using previous classifications of user behaviour from the literature~\cite{da:98}. Abrams et al. proposed the following classification of bookmark management strategies: \textit{no-filer}, \textit{creation-time filer}, \textit{end-of-session filer}, and \textit{sporadic filer}.


10 of the 30 participants who collected bookmarks were \textit{no filers}. The remaining 20 did not map onto the classifications proposed in~\cite{da:98}, and instead were best described as \textit{multiple-strategy} users.
Typically these users filed a subset of items on creation; others were left unfiled and occasionally spring-cleaned or purged (\textit{P28: "`Because the main thing is a complete mess and completely littered with things. I think "`oh this is important, I should bookmark this"'. The only exception to that is when I mirrored these web pages for the experiments, I had a project folder of bookmarks and also I keep a folder with personal homepages."'}) 

%% TO HONE BIGTIME
The multiple-strategy users were divided into two groups, B1 and B2, based on the extent to which they reported filing bookmarks (Table 4). 

%%%%%%%%%%%%%%%%%%%%%%%%%%%%%%%%%%%%%%%%%%%%%%%%%%%%%%%%%%
% TABLE: A classification of bookmark management strategies}
%%%%%%%%%%%%%%%%%%%%%%%%%%%%%%%%%%%%%%%%%%%%%%%%%%%%%%%%%%
% Table generated by Excel2LaTeX from sheet 'BM STRATS'
\begin{table}[h]
\begin{center}
\begin{footnotesize}
\setlength{\extrarowheight}{2pt}
\begin{tabular}{|p{4cm}|p{3cm}|p{3cm}|p{3cm}|} % total 13
\hline
{\bf Bookmark Management Strategy} & {\bf \# Users} & {\bf Average \# Folders} & {\bf Average \# unfiled } \\
\hline
B1 Multiple strategy users -- extensive filing (few unfiled items) &          8 &       48.4 &       20.9 \\
\hline
B2 Multiple strategy users -- partial filing (many unfiled items) &         12 &        9.6 &       36.9 \\
\hline
B3 No-filers &         10 & 3.3 (abandoned) &         70 \\
\hline
\end{tabular}  
\end{footnotesize}
\caption{Classification of observed bookmark management strategies [n=25]}
\label{table:exp-study:bm_mment_strats}
\end{center}
\end{table}
\normalsize





%%%%%%%%%%%%%%%%%%%%%%%%%%%%%%%%%%%%%%%%%%%%%%%%%%%%%%%%%%%%%%%%%%%%%%
\subsubsection{Organizational Dimension Analysis of Bookmark Folders}
%%%%%%%%%%%%%%%%%%%%%%%%%%%%%%%%%%%%%%%%%%%%%%%%%%%%%%%%%%%%%%%%%%%%%%

The most common organizational dimensions for bookmark folders are listed in \textbf{Table~\ref{table:exp-study:bookmark-orgdims}}. % and \textbf{Figure ADD}
In contrast to the document file and email collections, web bookmark collections were dominated by one particular dimension: \textit{Topic}. Example topic-based folders that were encountered included ``Star Trek'', ``Cooking'' and ``Java''.
The \textit{Topic} dimension accounted for 55\% of the categories, agreeing with the findings of Gottlieb \& Dilevko~\cite{gd:01} who noted that a majority of classificatory decisions were dependant on usage of topic-based content metadata.
\begin{itemize}
	\item \textit{THINK: It is worth noting the special meaning of ``Document class'' with web bookmarks - equivalent to that of website function, e.g. search engine or portal.} Eh?
\end{itemize}

%%%%%%%%%%%%%%%%%%%%%%%%%%%%%%%%%%%%%%%%%%%%%%%%%%%%%%%
% TABLE: SUMMARY OF ORG-DIMS FOR BOOKMARKS
% in tables/ch2/exp-study-tables.xls
%%%%%%%%%%%%%%%%%%%%%%%%%%%%%%%%%%%%%%%%%%%%%%%%%%%%%%%
\begin{small}
\begin{table}
\begin{center}
\begin{footnotesize}
\setlength{\extrarowheight}{2pt}
\begin{tabular}{|c|c|c|c|}
\hline
{\bf Rank} & {\bf Dimension} & {\bf Count} &   {\bf \%} \\
\hline
         1 & Topic / Interest &        135 &       55\% \\
\hline
         2 & Class of document &         32 &       13\% \\
\hline
         3 &    Project &         18 &        7\% \\
\hline
         4 &       Role &         17 &        7\% \\
\hline
         5 &    Contact &         15 &        6\% \\
\hline
         6 &    General &         14 &        6\% \\
\hline
         7 &      Event &          5 &        2\% \\
\hline
         8 & Others (<1\%) &          6 &        2\% \\
\hline
         9 &     Format &          3 &        1\% \\
\hline
           &      Total &        245 &      100\% \\
\hline
\end{tabular}  
\end{footnotesize}
\caption{Observed organizational dimensions in bookmarks [n=25]}
\label{table:exp-study:bookmark-orgdims}
\end{center}
\end{table}
\end{small}
