%%%%%%%%%%%%%%%%%%%%%%%%%%%%%%%%%%%%%%%%%%%%%%%%%%%
\newpage
\section{Results: Changes in Organizing Strategy}
\label{exp-study:comparison-changes}
%%%%%%%%%%%%%%%%%%%%%%%%%%%%%%%%%%%%%%%%%%%%%%%%%%%
% Immediate, Planned, Historical


%%%%%%%%%%%%%%%%%%%%%
% IMMEDIATE
%%%%%%%%%%%%%%%%%%%%%
The study had an immediate ``self-auditing'' influence on the behaviour of most participants.
Many participants rediscovered items they had lost, and twelve performed ad-hoc tidying during the interviews, e.g. filing or deleting files they had forgotten about.
% Self-audit: rediscover info, duplicated, failed, misfiled, tofile, old, todelete, useful 

% Changes included ones made during the studies, planned

%%%%%%%%%%%%%%%%%%%%%
% FILES
%%%%%%%%%%%%%%%%%%%%%
Fourteen participants reported historical strategy changes from before the study. Five reported historical changes in file strategy, all of which involved increases in organization, e.g. P5: \textit{``Now I've got a set of folders and create a new one if I've got too many unfiled.  Historically I use to be less organized and everything was unfiled.  I still have to search for this using type or date metadata''}.

%%%%%%%%%%%%%%%%%%%%%
% EMAIL
%%%%%%%%%%%%%%%%%%%%%
In email, seven participants reported historical changes in organizing strategy  -- three increases and four decreases in filing tendency.  Several participants also reported major incidents, such as a hard disk failure which lead to the loss of an email collection.  One example decrease in organizing tendency was as follows, e.g. P12: \textit{``I used to have lots of folders for each sub-project [of a main research area] but there just wasn't enough time to manage them. In an ideal world there'd be a rich structure ... and the hierarchy is now flattened and simplified''}.

%%%%%%%%%%%%%%%%%%%%%
% BM
%%%%%%%%%%%%%%%%%%%%%
In the case of bookmarks, six historical changes were reported: one increase, and five decreases in organization (e.g. abandoning all folders).

%%%%%%%%%%%%%%%%%%%%%%%%%%
\subsection{Discussion}
%%%%%%%%%%%%%%%%%%%%%%%%%%

%%%%%%%%%%%%%%%%%%%%%
% BOTH INC AND DEC
%%%%%%%%%%%%%%%%%%%%%
Note that both increases and decreases in organizing tendency were observed. This stands in contrast to previous work which has emphasized decreases in organizing tendency, e.g. the abandoning of folder structures~\citep{ob:97}. 

%%%%%%%%%%%%%%%%%%%%
% MAJOR INCIDENTS
%%%%%%%%%%%%%%%%%%%%
% However participants were typically settled in their choice of management strategy	

Additionally, many indicated that taking part in the study had caused them to think about PIM more than normal, causing them to plan future changes.  However, the snapshot nature of the nature meant that it was not possible to track these changes over time.
%%%%%%%%%%%%%%%%%%%%%%%%%%%%%%%%%%%%%%%%%%%%%%%%%%%%%%%%%%%%
% LINK TO CAHPTER 6
%%%%%%%%%%%%%%%%%%%%%%%%%%%%%%%%%%%%%%%%%%%%%%%%%%%%%%%%%%%%
\textbf{Chapter~\ref{chapter:main-study}} presents a longitudinal study of PIM behaviour, a key objective of which was to track strategy changes. 

% This exploratory study provided the incentive to investigate such issues in more depth.