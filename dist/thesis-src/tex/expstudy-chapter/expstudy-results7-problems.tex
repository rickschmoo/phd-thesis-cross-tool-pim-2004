%%%%%%%%%%%%%%%%%%%%%%%%%%%%%%%
% CHAPTER 4: EXPLORATORY STUDY
% RESULTS 5 - Results: Problems and User Experience
% File: tex/expstudy-chapter/expstudy-results7-problems.tex
%%%%%%%%%%%%%%%%%%%%%%%%%%%%%%%%%%%%%%%%%%%%%%%%
%%%%%%%%%%%%%%%%%%%%%%%%%%%%%%%%%%%%%%%%%%%%%%%%
%%%%%%%%%%%%%%%%%%%%%%%%%%%%%%%%%%%%%%%%%%%%%%%%
%%%%%%%%%%%%%%%%%%%%%%%%%%%%%%%%%%%%%%%%%%%%%%%%%%%%
\newpage
%%%%%%%%%%%%%%%%%%%%%%%%%%%%%%%%%%%%%%%%%%%%%%%%%%%
\section{Results: Problems and User Experience}
\label{exp-study:comparison-problems}
%%%%%%%%%%%%%%%%%%%%%%%%%%%%%%%%%%%%%%%%%%%%%%%%%%%
% NOTES
% * take care not to exaggerate problems
%	* Integrate discussion points (nature of PIM, e.g. reflection/distribution) into here??
%%%%%%%%%%%%%%%%%%%%%%%%%%%%%%%%%%%%%%%%%%%%%%
% ALT STRUCTURE: Key problems in different aspects of PIM
%%%%%%%%%%%%%%%%%%%%%%%%%%%%%%%%%%%%%%%%%%%%%%
% grievances
% the most fundamental/common points raised.  However note that problems varied across participants. %% EXPAND THIS - say everyone mentioned a different problem (for some version control was the bugbear, for others it was naming files
The study provided much evidence of dissatisfaction regarding current PIM interfaces.  
The author was often surprised at the vehemence expressed regarding PIM-related problems, and applied the term ``bugbear'' for recurring problems that frequently affected users.  Since PIM is an ongoing and often repetitive everyday activity, it appeared that even relatively minor short-term grumbles (e.g. inconvenient interface support for naming files) can build up and have a negative impact on ongoing user experience (e.g. perceived level of control).  
% \textit{Aim of design: to improve user experience.  FOCUS: INTEGRATION.  Argue need for more cross-tool design of integration-mechanisms.}

%%%%%%%%%%%%%%%%%%%%%%%%%%%
% Lead onto Cross-tool problems
%%%%%%%%%%%%%%%%%%%%%%%%%%%
%any problems are cross-tool, or at least compounded. \textit{THINK: Mention bugbears?}
% Potential of mismapping betwen activities and tools
% Problems at various contexts: (1) within specific tools (intra-tool problems), (2) the same problem in multiple tools, and (3) problems that bridged multiple tools. This third time of problem is discussed in \textbf{Section~\ref{exp-study:comparison-cross-tool}}.
% The discussion up until now has focused on user problems within particular PIM tools. WE also observed a range of \textit{cross-tool} problems, problems that bridged multiple collections 
A wide range of problems and concerns were raised by participants relating to all PIM sub-activities in all three PIM-tools.  Furthermore, issues varied significantly between participants.  This section highlights some of the issues that are relevant to subsequent work in this thesis.  Three types of problem were identified:
\begin{enumerate}
\item Tool-specific issues that were limited to single tool contexts.
% Tool-specific problems that occurred in multiple tool contexts 
\item Tool-specific issues that occurred repeatedly in multiple distinct tool contexts.
\item Cross-tool issues that bridged multiple tool contexts. 
\end{enumerate}

%%%%%%%%%%%%%%%%%%%%%%%%%%%%%%%%%%%%%%%
\subsection{Tool-specific Issues}
\label{exp-study:tool-specific}
%%%%%%%%%%%%%%%%%%%%%%%%%%%%%%%%%%%%%%%
% Problems that were limited to single tools (tool-specific issues)
%	\item Mention a few of the extreme/far-out examples/outliers of problems (e.g. version control esp docs/email)
%	Acquisition-related problems were focused on inbox management in email. Many participants complained of the time they spent processing messages and deciding which to keep.  
Numerous PIM-problems were reported within each tool collection.  File-related problems included difficulties managing multiple versions of files, and slow search mechanisms.  A key email-related problem was that of ascertaining the value of large numbers of newly-arrived messages.  Common issues in the bookmark context included lack of sorting and search functionality, and the ephemeral nature of websites.

% This section, and subsequent chapters of the thesis focus on problems that 
Since the thesis takes a focus on problems that bridge multiple tools, tool-specific problems are not discussed in more detail.

%%%%%%%%%%%%%%%%%%%%%%%%%%%%%%%%%%%%%%%%%%%%%%%%%%%%%%%%%%%%%%%%%%%%%%%%%%%%%%%%%%%%%%%%%%%%%%%%
%%%%%%%%%%%%%%%%%%%%%%%%%%%%%%%%%%%%%%%%%%%%%%%%%%%%%%%%%%%%%%%%%%%%%%%%%%%%%%%%%%%%%%%%%%%%%%%%
%%%%%%%%%%%%%%%%%%%%%%%%%%%%%%%%%%%%%%%%%%%%%%%%%%%%%%%%%%%%%%%%%%%%%%%%%%%%%%%%%%%%%%%%%%%%%%%%
%%%%%%%%%%%%%%%%%%%%%%%%%%%%%%%%%%%%%%%%%%%%%%%%%%%%%%%%%%%%%%%%%%%%%%%%%%%%%%%%%%%%%%%%%%%%%%%%

%%%%%%%%%%%%%%%%%%%%%%%%%%%%%%%%%%%%%%%%%%%%%%%%%%%%%%%%%%%%%%%%%%%%%%%%%%%%%%
\subsection{Tool-specific Issues that Occurred in Multiple Tools}
\label{exp-study:issues-repeated}
%%%%%%%%%%%%%%%%%%%%%%%%%%%%%%%%%%%%%%%%%%%%%%%%%%%%%%%%%%%%%%%%%%%%%%%%%%%%%%

Other problems were of a tool-specific nature, and were manifested in multiple contexts for many participants.  Examples included difficulties in naming items, and ``anxiety of order''.

%%%%%%%%%%%%%%%%%%%%%%%%%%%%%%%%%%%%%%%%%%%%%
% MULTI-CONTEXT PROBS: ACQUISITION/NAMING
%%%%%%%%%%%%%%%%%%%%%%%%%%%%%%%%%%%%%%%%%%%%%
% ACQUISITION/NAMING: Other acquisition-related problems included being forced to name items in the file system, and poor default naming for bookmarks. Also hard to rename emails.
One common problem that appeared in multiple tool contexts related to the naming of items.  Participants complained of the difficulty of selecting appropriate, meaningful names in their file collections.  One particular bugbear resulted from the attempts of software to offer default names based on a file's initial content (e.g. a report title).  In email, many participants complained of the difficulty in changing message subjects.  Those created by other users were often considered inadequate.  Likewise, in bookmarks participants complained of the poor interface provision for changing the names of newly-created bookmarks.  This was often necessary as by default bookmark names are set to the title of a web page to which they refer.  Several participants observed that web page names were often general to entire web-sites.

%%%%%%%%%%%%%%%%%%%%%%%%%%%%%%%%%%%%%%%%%%%%%
% MULTI-CONTEXT PROBS: ANXIETY OF ORDER
%%%%%%%%%%%%%%%%%%%%%%%%%%%%%%%%%%%%%%%%%%%%%
% ANXIETY OF ORDER: The most common observed organization-related problem was \textit{anxiety of order} e.g. old/failed/duplicate folders, unfiled items
% Many \textit{pro-organizing} participants suffered from an ``anxiety of order''~\citep{levy:01}.
% Classic tree problems (multiple classification, static hierarchy) rarely mentioned (COUNT USERS). Some (failed folders, duplicate folders) mentioned in context of staying tidy 
% What can WE conclude from this.  Users satisfice -- therefore no need for multiple attributes?  
% Postulate: some people have a disposition to keep things tidy.  
Another problem that appeared in multiple tool contexts was that of ``anxiety of order''~\citep{levy:01}.  This describes the tendency for many users to ``feel bad'' for ``being untidy''.  In other words, a perceived failure to manage personal information may seriously dent user's self-image.

Anxiety of order was widespread in the study reported in this chapter.  Many participants felt it necessary to excuse themselves for perceived mess, e.g. Participant P21: \textit{``I'm sorry, those files must have gone there accidently''}.  Anxiety was most extreme in the context of email, where participants emphasized the overheads of managing email, due to the higher (and uncontrolled) creation rate of messages compared to manually created files and bookmarks. % laso exacerbated by spam!).
However, participants also tended to be dissatisfied with the organizational state of document files and bookmarks, especially in terms of old or unfiled items, and failed folders. Dissatisfaction was expressed in terms of guilt, shame, stress, and lack of control, P11: \textit{``I'm really ashamed ... Its such a mess! I have stuff in there that needed organizing ages ago''}.

% Most participants said that they did not have enough time to organize the collections, resulting in a lack of satisfaction regarding their tidiness.  
A particular source of exasperation was the existence of old unfiled items, such as emails in the inbox, and icons on the desktop. Most participants wanted to devote more time to managing their personal information but could not do so due to lack of time or were unwilling to do so because of perceived overheads.

The level of anxiety was influenced by user disposition towards tidiness, with organizing-neutral participants being less affected.  Interestingly, some of the most pro-organizing participants, those who invested a lot of time in filing, remained dissatisfied with the tidiness of their collections. 

Anxiety of order was possibly exacerbated by other ``classic'' classification problems which were reported in all tool contexts by many participants.  These included difficulties classifying items, lack of multiple classification support, failed folders, duplicate folders, and the static nature of the folder hierarchy.  The increasing amount of storage in modern computing devices may be a contributory factor in user dissatisfaction: since (1) they are able to collect more stuff, and (2) there is less pressure to delete information in an ongoing manner.

Only a few participants complained of the impact on their productivity due to time spent organizing, and time spent retrieving items.  However this is subjective and hard to confirm objectively.  Impact of messiness is not clear on retrieval since participants indicated that they could generally find required items. 

% Maintenance problems were rarely mentioned by participants except for the fact that they had little time for archiving etc. Another maintenance-related problem was expressed in terms of not maintaining, e.g. when participants had failed to back-up.

%%%%%%%%%%%%%%%%%%%%%%%%%%%%%%%%%%%%%%%%%%%%%%%%%%%%%%%%%%%%%%%%%%%%%%%%%%%%%%%%%%%%%%%%%%%%%%%%
%%%%%%%%%%%%%%%%%%%%%%%%%%%%%%%%%%%%%%%%%%%%%%%%%%%%%%%%%%%%%%%%%%%%%%%%%%%%%%%%%%%%%%%%%%%%%%%%
%%%%%%%%%%%%%%%%%%%%%%%%%%%%%%%%%%%%%%%%%%%%%%%%%%%%%%%%%%%%%%%%%%%%%%%%%%%%%%%%%%%%%%%%%%%%%%%%
%%%%%%%%%%%%%%%%%%%%%%%%%%%%%%%%%%%%%%%%%%%%%%%%%%%%%%%%%%%%%%%%%%%%%%%%%%%%%%%%%%%%%%%%%%%%%%%%



%%%%%%%%%%%%%%%%%%%%%%%%%%%%%%%%%%%%%%%%%%%%%%%%
\subsection{Issues that Bridged Multiple Tools}
\label{exp-study:issues-cross-tool}
%%%%%%%%%%%%%%%%%%%%%%%%%%%%%%%%%%%%%%%%%%%%%%%%
% Include stuff from CSCW2002 paper
% * Need to relate to conceptual framework of tools and activities in Chapter 1
%	* make sure production/supporting activity are defined
%	* Cross-tool problems as basis for scenarios in next chapter
%%%%%%%%%%%%%%%%%%%%%%%%%%%%%%%%%%%%%%%%%%%%%%%%

A number of problems were observed that bridged multiple tool contexts:
\begin{enumerate}
\item Design inconsistencies between different PIM-tools.
\item The inability to share folder structures between PIM-tools.
\item The fragmentation across PIM-tools of information in a particular technological format.
\item The fragmentation of information related to a particular activity across PIM-tools.
\end{enumerate}

%%%%%%%%%%%%%%%%%%%%%%%%
% INCONSISTENCIES
%%%%%%%%%%%%%%%%%%%%%%%%
% Example functionality which was implemented differently included ``create new folder'' or ``mark this item as important''
% Inconsistencies, interface inconsistency. % Reference to previous work on consistency (\textit{standard result in HCI?})
Annoyance was caused by inconsistencies between different PIM-tools in terms of how they provided equivalent functionality.  One example was the interface used to manipulate the folder structure by changing folder names, or reorganizing the folder structure.  Participants found this particularly irritating between tools from the same vendor.  In other cases, a function was available in one tool, but not in others.  One example was the ability to highlight an item as ``important''.  Email clients such as MS-Outlook provide the ability to ``flag'' an item, whereas file and web bookmark management software typically does not.

%%%%%%%%%%%%%%%%%%%%%%%%%%%
% COMPOUNDED OVERHEADS
%%%%%%%%%%%%%%%%%%%%%%%%%%%
% Evidence that magnify and compound above basic PIM problems. Overhead. Overload. Talk about in terms of forced-split attention, parallel working. Extent to which some users avoid separate management. % \textit{QUOTE: gil?} 
Several participants complained of the need to manage different collections of information separately, noting that it was not possible to share organizational structure between tools.  One went to the lengths of saving email messages as files to avoid having to manage two distinct collections.  When viewed from a cross-tool perspective it is clear that the management overheads that have been reported in specific tools are compounded when multiple PIM tools are considered.

%%%%%%%%%%%%%%%%%%%%%%%%%%%%%%%%%%%%%%%%%%%%%%%%%%%%%%%%%%%%%%%%%%%%%%%%%%
% FRAGMENTATION: OF SPECIFIC TECH FORMATS
% The fragmentation of information of a particular technological format.
%%%%%%%%%%%%%%%%%%%%%%%%%%%%%%%%%%%%%%%%%%%%%%%%%%%%%%%%%%%%%%%%%%%%%%%%%%
% Despite the presence of compartmentalization, users tended to focus their PIM efforts on a \textit{primary collection} (MOVE TO LATER)
Participants also complained that information in some technological formats was fragmented across multiple distinct collections.  For instance, many participants managed files using several parallel mechanisms: (1) within the file system, (2) spatially as desktop icons, and (3) as email attachments. Each mechanism requires separate organization.  This distribution of the management of a particular type of information between distinct PIM-tools has been referred to as \textit{compartmentalization}~\citep{Bellotti:00}.  \textbf{Table~\ref{table:compartmentalization}} summarizes the observations of the compartmentalization of document files, email, and web bookmarks -- both within a single computer, and across the extended personal information environment\footnote{Compartmentalization was also observed for other types of personal information such as contacts and to-do items.  For example, contacts were frequently scattered between email, personal diaries, notebooks, and mobile phones.}.
%%%%%%%%%%%%%%
% RETRIEVAL
%%%%%%%%%%%%%%
Several participants reported that the compartmentalization of files lead to problems of retrieval, especially in the case that they were looking for a particular file and had to search both the file and email collections.% Issues and problems relating to the compartmentalization of document files, email and web bookmarks is discussed in more detail in \textbf{Section~\ref{exp-study:ct}}
% Consequences for version control
%%%%%%%%%%%%%%%%%%%%%%%%%%%%%%%%%%%%%%%%%%%%%
% MULTI-CONTEXT PROBS: RETRIEVAL
%%%%%%%%%%%%%%%%%%%%%%%%%%%%%%%%%%%%%%%%%%%%%
% RETRIEVAL: participants mentioned encountering only occasional problems, although these were highly frustrating. In contrast failure to find things was mentioned less often. % See \textbf{Section~\ref{exp-study:comparison-retrieval}}.

\begin{table}[hbtp]
\begin{center}
\begin{footnotesize}
\setlength{\extrarowheight}{2pt}
\begin{tabular}{|p{2.5cm}|p{3.5cm}|p{3.5cm}|p{3.5cm}|}
\hline
 & {\bf Document File} & {\bf Email} & {\bf Web Bookmark} \\
\hline
{\bf On primary computer} & Document files can also be managed as desktop icons or as email attachments.  & Email typically managed only within email tool. & Web bookmarks often managed as desktop icons or as embedded links within emails.  \\
\hline
{\bf Outside primary desktop computer} & Network drives. Personal document files stored on other computers or devices. & Email stored on other computers or devices. Web-email collections (such as Yahoo! or Hotmail) & Web bookmarks stored on other computers or devices.  \\
\hline
\end{tabular}  
\end{footnotesize}
\caption{Compartmentalization of different types of information}
\label{table:compartmentalization}
\end{center}
\end{table}
\normalsize

%%%%%%%%%%%%%%%%%%%%%%%%%%%%%%%%%%%%%%%%%%%%%%%%%%%%%%%%%%%%%%%%%%%%%%%%%%
% FRAGMENTATION: INFO RELATED TO ONE ACTIVITY
%%%%%%%%%%%%%%%%%%%%%%%%%%%%%%%%%%%%%%%%%%%%%%%%%%%%%%%%%%%%%%%%%%%%%%%%%%
% The fragmentation of information relating to a particular activity.	
% Users complained about the need to coordinate production activities across multiple tools. 
% Several participants mentioned activities which involved PIM-tools.  
Another aspect of fragmentation concerned information relating to a particular user activity such as a project.  A number of participants highlighted difficulties in coordinating multiple PIM-tools in carrying out a particular project.
%  such as starting a project, or finishing a project
One difficulty was encountered in \textit{project management}-related tasks such as starting a new production activity (setting up folders in distinct tools), and finishing a production activity (archiving items in distinct tools).  One participant talked of the difficulties involved in archiving two types of information, P1: \textit{``After the project finished it was all 99\% useless stuff [files and email]. I just wanted to get it out of the way''}. In such cases, it was necessary to perform these actions repeatedly in multiple tools.  This type of fragmentation also impacted retrieval, when a user is not sure if the information related to an activity is stored in an email or a file.

%%%%%%%%%%%%%%%%%%%
% BRAINSTROMING
%%%%%%%%%%%%%%%%%%%
% Add collating examples: idea brainstorming, gathering ideas. % Ad-hoc categories (SJ)
Some participants wanted a facility to gather different types of information within a single interface. One example was \textit{brainstorming} which involved collating information from multiple PIM-tools into her email, P9: \textit{``I like to pull things together here, URLs, notes ... and jumble them up in broad categories. My categories tend to be fairly wide and get quite big.  It's great for brainstorming and ideas. However the cost is that sometimes you can't find things''}.

%%%%%%%%%%%%%%%%%%%%%%%%
% REMINDERS
%%%%%%%%%%%%%%%%%%%%%%%%
% The author speculates that the users have problems staying on top of all these reminders.
% Cross-tool aspects of PIM, Aspects of PIM such as information management and task management can be considered as cross-tool: they cut across a wide range of tools. 
Most participants employed a range of PIM-tools in performing task and time management, e.g. setting reminders in multiple tool contexts such as icons on the desktop, emails in the inbox, and links to websites to visit. Most also made extensive use of physical artefacts such as diaries.  Two participants complained that there was no easy way to collate such reminders together.

%%%%%%%%%%%%%%%%%%%%%%%%%%%%%%%%%%%%%%%%
% Usage of existing forms of integration
%%%%%%%%%%%%%%%%%%%%%%%%%%%%%%%%%%%%%%%%
Participants varied in the extent to which they reported using existing integration mechanisms.  The most commonly mentioned was attaching files to an message from within an email tool.  Several also mentioned using the ``Send-to'' mechanism in MS-Windows for attaching files to an email message.  % However note that current forms of integration fall short of solving the above problems and currently users have to coordinate manually across tools (e.g. gathering information of a particular technological format).

%%%%%%%%%%%%%%%%%%%%%%%%%%
\subsection{Discussion}
%%%%%%%%%%%%%%%%%%%%%%%%%%
%%%%%%%%%%%%%%%%%%%%%%%%%%%
% DISCUSSION AT THE END
%%%%%%%%%%%%%%%%%%%%%%%%%%%
% In this thesis, the author argues that PIM is a cross-tool activity, and should be designed for as such. This section discusses a number of findings that provided evidence to support the cross-tool perspective that is developed over the thesis: (1) the usage of existing forms of integration, and (2) problems that involved multiple tools. Such problems may indicate needs for improved integration. % Note that problems specific to particular tools are discussed in \textbf{Section~\ref{exp-study:comparison-problems}}.
The previous two sections illustrate a number of user problems that involve multiple PIM-tools. Firstly, \textbf{Section~\ref{exp-study:issues-repeated}} highlights \textit{tool-specific} problems which appeared in multiple tools.  Secondly, \textbf{\ref{exp-study:issues-cross-tool}} highlights a number of \textit{cross-tool} issues.  Such problems suggest that there is a need for improved integration between PIM-tools. \textbf{Chapter~\ref{chapter:design}} discusses prospective cross-tool design solutions to some of the problems discussed in this section.

% This thesis investigates how a cross-tool design perspective can be applied to solving such problems.  A number of potential solutions are outlined in \textbf{Chapter~\ref{chapter:design}}. 












	


