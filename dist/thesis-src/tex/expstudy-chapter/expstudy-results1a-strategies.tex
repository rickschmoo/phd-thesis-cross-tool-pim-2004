%%%%%%%%%%%%%%%%%%%%%%%%%%%%%%%
% CHAPTER 4: EXPLORATORY STUDY
% 	RESULTS 2 - Organizational Dimensions
% File: tex/expstudy-chapter/expstudy-results1a-strategies.tex
%%%%%%%%%%%%%%%%%%%%%%%%%%%%%%%%%%%%%%%%%%%%%%%%
%%%%%%%%%%%%%%%%%%%%%%%%%%%%%%%%%%%%%%%%%%%%%%%%
%%%%%%%%%%%%%%%%%%%%%%%%%%%%%%%%%%%%%%%%%%%%%%%%

%%%%%%%%%%%%%%%%%%%%%%%%%%%%%%%
% CHAPTER 4: EXPLORATORY STUDY
% 	RESULTS 2 - Cross-tool profiling
% File: tex/expstudy-chapter/expstudy-results2-cross-tool-profiling.tex
%%%%%%%%%%%%%%%%%%%%%%%%%%%%%%%%%%%%%%%%%%%%%%%%
%%%%%%%%%%%%%%%%%%%%%%%%%%%%%%%%%%%%%%%%%%%%%%%%
%%%%%%%%%%%%%%%%%%%%%%%%%%%%%%%%%%%%%%%%%%%%%%%%
\newpage
\section{Results: Comparing Organizing Strategies}
\label{exp-study:Results-org-strategies}
%%%%%%%%%%%%%%%%%%%%%%%%%%%%%%%%%%%%%%%%%%%%%%%%
%%%%%%%%%%%%%%%%%%%%%%%%%%%%%%%%%%%%%%%%%%%%%%%%

The previous section provided a high-level comparison of PIM behaviour between files, email and bookmarks.  This section presents a more detailed comparison of organizing strategies.  \textbf{Section~\ref{exp-study:cross-tool-profiling}} describes the approach that was employed.
\textbf{Sections~\ref{exp-study:Results-org-strategies-files}}, \textbf{\ref{exp-study:Results-org-strategies-email}} and \textbf{\ref{exp-study:Results-org-strategies-bookmarks}} classify participants' behaviour in each PIM-tool in turn.  Then \textbf{Section~\ref{exp-study:Results-cross-tool-profiling}} reports the cross-tool profiling in which strategies were compared across tools.

% \textbf{Section~\ref{exp-study:Results-org-strategies}} reports the comparison of  on a user-by-user basis. % Key findings include the observation of the multiple PIM strategies employed by many participants in the 3 tools.  New classifications of behaviour are offered in each tool based on participants' reported filing strategies.
%Method described in . Two stages:
%\begin{enumerate}
%\item Characterize organizing behaviour in each PIM-tool in turn
%\item Compare organizing behaviour between the PIM-tool for each participant to investigate consistency 
%\end{enumerate}

% %%%%%%%%%%%%%%%%%%%%%%%%%%%%%%%%
% TOOL-SPECIFIC FINDINGS
%%%%%%%%%%%%%%%%%%%%%%%%%%%%%%%%
% Although not the main aim of the study, several incremental tool-specific contributions are offered including the new classifications of filing strategies reflecting the multiple strategies observed in each tool. 

%%%%%%%%%%%%%%%%%%%%%%%%%%%%%%%%%%%%%%%%%%%%%
\subsection{File Organizing Strategies}
\label{exp-study:Results-org-strategies-files}
%%%%%%%%%%%%%%%%%%%%%%%%%%%%%%%%%%%%%%%%%%%%%
% SCRATCH: Twenty-three of the twenty-five participants had document file folders in active use~\footnote{An active folder was defined as one containing items that were used in the last week/on a regular basis.}. Most of these participants pursued a \textit{file-on-creation} strategy with their document files: as items were created, users habitually filed them away (either using the file manager or the file dialog within editors such as MS Word).
% The most common profile was to organize most files as they were created, but to leave a small portion unfiled on the desktop or at the root level.

Since no classifications of file management strategies had been proposed in previous work, the author developed one from scratch based on participants' strategy descriptions. Three strategies were identified and are reported in \textbf{Table~\ref{table:exp-study:file_classification}}.

%%%%%%%%%%%%%%%%%%%%%%%%%%%%%%%%%%%%%%%%%%%%%%%%%%%%%%%%%%
% TABLE: A classification of file management strategies}
%%%%%%%%%%%%%%%%%%%%%%%%%%%%%%%%%%%%%%%%%%%%%%%%%%%%%%%%%%
% THINK: add overall average?
\begin{table}[hbtp]
\begin{center}
\begin{footnotesize}
\setlength{\extrarowheight}{2pt}
\begin{tabular}{|p{4cm}|c|c|c|}
% Table generated by Excel2LaTeX from sheet 'FS STRATS'
\hline
{\bf Strategy} & {\bf \# Users} & {\bf Average \# Folders} & {\bf Average \# unfiled items} \\
\hline
F1 total filers: file majority of items on creation. &         16 & 50 (SD: 29, min: 12, max: 122)  & 14 (SD: 9, min: 0, max: 30) \\
\hline
F2 extensive filers: file extensively, but leave many items unfiled. &          7 & 58 (SD:29, min: 5, max: 108) & 132 (SD: 137, min:  31, max: 340) \\
\hline
F3 occasional filers: file occasionally, leave most items unfiled, have few folders. &          2 & 5 (SD: 0, min: 5, max: 5) & 240 (SD: 127, min: 150, max: 330) \\
\hline
\end{tabular}  
\end{footnotesize}
\caption{Classification of observed file management strategies [n=25]}
\label{table:exp-study:file_classification}
\end{center}
\end{table}
\normalsize

23 of the 25 participants were \textit{pro-organizing}, and had extensive folder structures.  They could be divided into two groups (F1 and F2) based on the extent to which they employed a \textit{file-on-creation} strategy (filing new items immediately). F1 participants employed a predominantly file-on-creation strategy, and tended only to leave items unfiled by accident, except for a few temporarily placed work-in-progress files. F2 participants filed the majority of items on creation, but also managed a large unfiled subset of working/ephemeral items. F2 participants filed these items on completion of the relevant task, or during a spring-clean. Thus the location of ephemeral/working files varied between the two groups. F1 participants distributed them around active folders, whilst F2 participants left many unfiled. However, even for the F2 participants, unfiled items were a small proportion of their total collection.

The remaining two participants (group F3) were \textit{organizing-neutral}.  They filed less extensively, and stated that filing was not a priority.  In contrast most F1/F2 participants said that being organized was an important (though not always achievable) goal. 

%%%%%%%%%%%%%%%%%%%%%%%%
% MULTIPLE STRATEGIES
%%%%%%%%%%%%%%%%%%%%%%%
In addition. most participants occasionally performed spring-cleaning of their file collections. Note that most of the participants could not be described as filers or non-filers.  Instead, they employed \textit{multiple-strategies} -- filing some items on creation, leaving some unfiled, and carrying out occasional spring-cleaned. Multiple strategies were similarly observed in the other collections. % The next two sections also identify multiple strategies in the context of email and bookmarks respectively.






%%%%%%%%%%%%%%%%%%%%%%%%%%%%%%%%%%%%%%%%%%%%%%
\subsection{Email Organizing Strategies}
\label{exp-study:Results-org-strategies-email}
%%%%%%%%%%%%%%%%%%%%%%%%%%%%%%%%%%%%%%%%%%%%%%
% The author proposes a new classification of email management strategies based on the four observed strategies. The classification is summarized in Table~\ref{table:exp-study:email_classification} along with the characteristics of each class. 
% THINK: add overall average?

The author attempted to categorize participants' behaviour using previous classifications of email organizing behaviour~\citep{Whittaker-email:96,ob:97}. However, this was only a partial success. The sample included 2 \textit{no-filers} (folderless spring-cleaners), and 7 \textit{frequent filers} - but no \textit{spring-cleaners} (participants who only clean their inbox periodically).  The remaining 16 participants had large inboxes (>75 items, average 1137), like the no-filers and spring-cleaners in~\citep{Whittaker-email:96}, however their reported strategies did not match these classifications. They filed some new emails immediately (typically those of perceived long-term value such as e-commerce receipts), and deleted low-value spam. Other messages were left in the inbox, which was occasionally spring-cleaned.  In other words, as in files, they employed \textit{multiple strategies} - a combination of frequent filer, spring cleaner, and no-filer, e.g. P25: \textit{``I'd like to manage as and when I receive them but I don't. I do it periodically - 10 minutes a day just to categorize the things that are important. 10 or 15 I'll categorize ... the rest of them I think oh I'll get round to doing that at some stage - but I don't normally. However I did spend an hour on a train last week tidying my emails because I was bored. I reduced my inbox by about 1500''}.

A new classification was developed based on participants' strategy descriptions (see  \textbf{Table~\ref{table:exp-study:email_classification}}). The 16 \textit{multiple-strategy} participants could be divided into two sub-groups, E2 and E3, based on the extent to which they reported manually filing new messages on a daily basis. E2 participants filed many emails everyday, whilst E3 participants only filed a few (<5) messages of particular long-term value, P31: \textit{``I have a folder for registrations. I've got other [unused] folders - I don't even know what they are. The vast majority [of email] is a big long list''}. E1/E2 participants were pro-organizing, whilst E3/E4 participants considered it to be less important.



%%%%%%%%%%%%%%%%%%%%%%%%%%%%%%%%%%%%%%%%%%%%%%%%%%%%%%%%%%
% TABLE: A classification of email management strategies}
%%%%%%%%%%%%%%%%%%%%%%%%%%%%%%%%%%%%%%%%%%%%%%%%%%%%%%%%%%
\begin{table}[hbtp]
\begin{center}
\begin{footnotesize}
\setlength{\extrarowheight}{2pt}
\begin{tabular}{|p{4cm}|c|c|c|}
% Table generated by Excel2LaTeX from sheet 'EM STRATS'
\hline
{\bf Strategy} & {\bf \# Users} & {\bf Average \# Folders} & {\bf Average \# unfiled items} \\
\hline
E1 frequent filers: file or delete most incoming messages everyday. &          7 & 56 (SD: 62, min: 3, max: 181)  & 26 (SD: 15, min: 7, max: 50) \\
\hline
E2 extensive filers: try to file many messages everyday.  &         12 & 42 (SD: 24, min: 8, max: 91) & 1002 (SD: 1497, min: 87, max: 5577) \\
\hline
E3 partial filers: file only a few (<5) messages everyday.  &          4 & 4 (SD: 3, min: 0, max: 6) & 1251 (SD: 1254, min: 205, max: 3000) \\
\hline
E4 no-filers: do not file any messages. &          2 & 0 (SD: 0, min: 0, max: 0) & 1106 (SD: 1265, min: 211, max: 2000) \\
\hline
\end{tabular}  

\end{footnotesize}
\caption{Classification of observed email management strategies [n=25]}
\label{table:exp-study:email_classification}
\end{center}
\end{table}
\normalsize



 





%%%%%%%%%%%%%%%%%%%%%%%%%%%%%%%%%%%%%%%%%%%%%%
\subsection{Bookmark Organizing Strategies}
\label{exp-study:Results-org-strategies-bookmarks}
%%%%%%%%%%%%%%%%%%%%%%%%%%%%%%%%%%%%%%%%%%%%%%
%Bookmarks tended to be less structured than the other two collections with most items being left in a chronologically ordered list.  Less than half the participants had web bookmark folders in active use. 
% Several participants indicated that both web bookmarks and email had less value than document files and so were less worth organizing.

The author attempted to map participants' behaviour onto an existing classification~\citep{da:98}. However, as with email, the previous classification did not reflect the observed behaviour, and another new classification was developed (see \textbf{Table~\ref{table:exp-study:bookmark_classification}}).  Only 8 participants matched a previous classification, that of ``no filer''. 

%%%%%%%%%%%%%%%%%%%%%%%%%%%%%%%%%%%%%%%%%%%%%%%%%%%%%%%%%%
% TABLE: A classification of bookmark management strategies}
%%%%%%%%%%%%%%%%%%%%%%%%%%%%%%%%%%%%%%%%%%%%%%%%%%%%%%%%%%
% THINK: add overall average?
\begin{table}[hbtp]
\begin{center}
\begin{footnotesize}
\setlength{\extrarowheight}{2pt}
\begin{tabular}{|p{4cm}|c|c|c|} % total 13
\hline
{\bf Strategy} & {\bf \# Users} & {\bf Average \# Folders} & {\bf Average \# unfiled items} \\
\hline
B1 extensive filing: file many bookmarks as they are created or at the end of browsing session &          6 & 31 (SD: 16, min: 13, max: 55)  & 24 (SD: 19, min: 10, max: 40) \\
\hline
B2 partial filing: file bookmarks sporadically  &         10 & 10 (SD: 7, min: 3, max: 24) & 35 (SD: 32, min: 7, max: 120) \\
\hline
B3 no-filers: never file, all folders abandoned. &          8 & 1 (SD: 2, min: 0, max: 5) & 71 (SD: 67, min: 4, max: 200) \\
\hline
B4: non-collector &          1 &    {\bf -} &    {\bf -} \\
\hline
\end{tabular}  
\end{footnotesize}
\caption{Classification of observed bookmark management strategies [n=25]}
\label{table:exp-study:bookmark_classification}
\end{center}
\end{table}
\normalsize

The remaining 16 active collectors of bookmarks instead employed \textit{multiple strategies}. They filed a subset of bookmarks on creation, leaving others unfiled, often as reminders, until they were spring-cleaned or simply abandoned, \textit{P12: ``The main thing is a mess and completely littered with things. The only exception is when I mirrored web pages for experiments. Also I keep a folder with homepages''}. The multiple-strategy participants were divided into two groups, B1 and B2, based on the extent to which they reported filing new bookmarks on creation. Organization was of lower priority for the B2 participants who had fewer folders and more unfiled bookmarks.

% Note that in all three tools, most participants employed \textit{multiple-strategies}.


%%%%%%%%%%%%%%%%%%%%%%%%%%%%%%%%%%%%%%%%%%%%%%%%%%%%%%%%%%%%%%%%%%%%%%%%%%%%%%%%%%%%%%%%%%%%%%%%
%%%%%%%%%%%%%%%%%%%%%%%%%%%%%%%%%%%%%%%%%%%%%%%%%%%%%%%%%%%%%%%%%%%%%%%%%%%%%%%%%%%%%%%%%%%%%%%%
%%%%%%%%%%%%%%%%%%%%%%%%%%%%%%%%%%%%%%%%%%%%%%%%%%%%%%%%%%%%%%%%%%%%%%%%%%%%%%%%%%%%%%%%%%%%%%%%
%%%%%%%%%%%%%%%%%%%%%%%%%%%%%%%%%%%%%%%%%%%%%%%%%%%%%%%%%%%%%%%%%%%%%%%%%%%%%%%%%%%%%%%%%%%%%%%%


%%%%%%%%%%%%%%%%%%%%%%%%%%%%%%%%%%%%%%%%%%%%%%%%
%%%%%%%%%%%%%%%%%%%%%%%%%%%%%%%%%%%%%%%%%%%%%%%%
\subsection{Cross-tool Profiling}
\label{exp-study:Results-cross-tool-profiling}
%%%%%%%%%%%%%%%%%%%%%%%%%%%%%%%%%%%%%%%%%%%%%%%%
%%%%%%%%%%%%%%%%%%%%%%%%%%%%%%%%%%%%%%%%%%%%%%%%

%%%%%%%%%%%%%%%%%%%%%%%%%
% intro and structure
%%%%%%%%%%%%%%%%%%%%%%%%%
% *** Earlier section: compared PIM strategies (acquisition, organization, maintenance and retrieval) across tools.  high-level similarities and differences
% *** Previous studies have attempted to classify users based on their PIM strategies within particular tools, particularly with respect to filing strategies~\citep{da:98,Whittaker-email:96}. Why did they do this?
% *** Rationale: complex set of strategies and data. Attempt to summarize
%	*** The intention here was to investigate dependency on filing from a workspace-wide/cross-tool perspective
% Participants were profiled in terms of which tools they reported making significant filing effort in.
% based on the extent to which users devote effort to organizing the three types of information. The specific measure employed was how consistent   participants were in terms of reported filing behaviour. % The method used is discussed in \textbf{Section~\ref{exp-study:cross-tool-profiling}}. 
% *** Here WE attempt to classify users in terms of their \textit{cross-tool} approach to filing.  WE propose a new user classification based on the extent to which they organized multiple types of information within folders.  
This section presents the results from the cross-tool profiling analysis.  The aim of this analysis was to investigate the consistency of each participant's approach to organizing files, email and bookmarks. % This section goes beyond the analysis in \textbf{Section~\ref{exp-study:Results-comparison}} by attempting to build up a \textit{cross-tool} profile on a user-by-user basis, rather than generalizing across participants. 

%%%%%%%%%%%%%%%%%%%%%%%%%%%%%%%%%%%%%%%%%%%%%%%%%%%%%%%
% TABLE: table of tool-specific strategies
%%%%%%%%%%%%%%%%%%%%%%%%%%%%%%%%%%%%%%%%%%%%%%%%%%%%%%%
\begin{table}[btp]
\begin{center}
\begin{footnotesize}
\setlength{\extrarowheight}{2pt}
\begin{tabular}{|c|c|c|c|c|}
\hline
        \textbf{Participant} &       \textbf{File strategy} &         \textbf{Email strategy} &         \textbf{Bookmark strategy} & \textbf{Cross-tool profile} \\
\hline
         P1 &         F1 &         E2 &         B2 &        CT2 \\
\hline
         P2 &         F1 &         E2 &         B1 &        CT1 \\
\hline
         P3 &         F2 &         E2 &         B3 &        CT2 \\
\hline
         P4 &         F1 &         E3 &         B3 &        CT3 \\
\hline
         P5 &         F2 &         E2 &         B2 &        CT2 \\
\hline
         P6 &         F3 &         E3 &         B3 &        CT4 \\
\hline
         P7 &         F3 &         E4 &         B3 &        CT4 \\
\hline
         P8 &         F1 &         E3 &         B2 &        CT3 \\
\hline
         P9 &         F2 &         E2 &         B3 &        CT2 \\
\hline
        P10 &         F1 &         E4 &         B3 &        CT3 \\
\hline
        P11 &         F1 &         E1 &         B1 &        CT1 \\
\hline
        P12 &         F1 &         E1 &         B3 &        CT2 \\
\hline
        P13 &         F1 &         E2 &         B2 &        CT2 \\
\hline
        P14 &         F1 &         E1 &         B1 &        CT1 \\
\hline
        P15 &         F2 &         E2 &         B1 &        CT1 \\
\hline
        P16 &         F1 &         E2 &         B2 &        CT2 \\
\hline
        P17 &         F1 &         E1 &         B1 &        CT1 \\
\hline
        P18 &         F1 &         E2 &         B1 &        CT1 \\
\hline
        P19 &         F1 &         E2 &         B2 &        CT2 \\
\hline
        P20 &         F1 &         E2 &         B2 &        CT2 \\
\hline
        P21 &         F1 &         E1 &         B2 &        CT2 \\
\hline
        P22 &         F1 &         E1 &         B3 &        CT2 \\
\hline
        P23 &         F2 &         E1 &         B2 &        CT2 \\
\hline
        P24 &         F2 &         E3 &         B2 &        CT3 \\
\hline
        P25 &         F2 &         E2 &         B3 &        CT2 \\
\hline
\end{tabular}  
\end{footnotesize}
\caption{Table of participants' tool-specific strategies, and cross-tool profile}
\label{table:exp-study:tool-specific-classifications}
\end{center}
\end{table}
\normalsize

%%%%%%%%%%%%%%%%%%%%%%%%%%%%%%%%%%%%%%%
% \subsubsection{Method}
%%%%%%%%%%%%%%%%%%%%%%%%%%%%%%%%%%%%%%%
%%%%%%%%%%%%%%%%%%%%%%%%%%%%%%%%%%%%%%%%%%%
% 1. Table of the tool-specific strategies
%%%%%%%%%%%%%%%%%%%%%%%%%%%%%%%%%%%%%%%%%%%
%%%%%%%%%%%%%%%%%%%%%%%%%%%%%%%%%%%%%%%%%%%
% 2. Identification of cross-tool profile
%%%%%%%%%%%%%%%%%%%%%%%%%%%%%%%%%%%%%%%%%%%
The middle three columns of \textbf{Table~\ref{table:exp-study:tool-specific-classifications}} list the three \textit{tool-specific} strategies for each participant.  A cross-tool profile was then identified for each participant by collating the three strategies as a \textit{3-tuple}, e.g. \texttt{F1/E2/B2} for Participant P1. Across the twenty-five participants, fourteen unique tuples were identified.
%%%%%%%%%%%%%%%%%%%%%%%%%%%%%%%%%%%%%%%%%%%%%%%%%%%%%%%%%%%%%%%%%%%%%%%%%%%%%%%%%%%%%%%%%%%%
% 3. Classification of each set of tool-specific strategies based on extent of organizing
%%%%%%%%%%%%%%%%%%%%%%%%%%%%%%%%%%%%%%%%%%%%%%%%%%%%%%%%%%%%%%%%%%%%%%%%%%%%%%%%%%%%%%%%%%%%
The \textit{cross-tool profiles} were then clustered based on the following criterion:
\begin{quote}
\textit{In which of the three collections were the participants pro-organizing? (i.e. in which collections did they report making significant organizing effort?)}
\end{quote}

The first step of clustering the cross-tool profiles was to classify each set of tool-specific strategies as either \textit{pro-organizing} (involving high organizing effort) or \textit{organizing-neutral} (involving low organizing effort). This process was necessarily subjective since the nature of PIM in each tool varies, along with the objective criteria used to define the tool-specific strategy classifications.   Several classifications were attempted, from which the one shown in \textbf{Table~\ref{table:exp-study:strategy-classification}} emerged as the best match for the data. % THIS IS GOING TO NEED SOME ATTENTION! (-;

%%%%%%%%%%%%%%%%%%%%%%%%%%%%%%%%%%%%%%%%%%%%%%%%%%%%%%%
% TABLE: classifying the tool-specific strategies
%%%%%%%%%%%%%%%%%%%%%%%%%%%%%%%%%%%%%%%%%%%%%%%%%%%%%%%
%\begin{tabular}{|c|c|}
%\hline
%{\bf Tool-specific strategies} & {\bf Level of organizing effort} \\
%\hline
%F1/F2, E1/E2, B1 & ``Pro-organizing'', strategies that involve high organizing effort \\
%\hline
%F3, E3/E4, B2/B3 & ``Organizing-neutral'', strategies that involve low organizing effort \\
%\hline
%\end{tabular}  
\begin{table}[btp]
\begin{center}
\begin{footnotesize}
\setlength{\extrarowheight}{8pt}
\begin{tabular}{|c|c|c|c|}
\hline
{\bf Level of organizing effort} & {\bf Files} & {\bf Email} & {\bf Bookmarks} \\
\hline
\multicolumn{ 1}{|l|}{``Pro-organizing'', strategies that involve high organizing effort} &         F1 &         E1 &            \\
\hline
\multicolumn{ 1}{|l|}{"} &         F2 &         E2 &         B1 \\
\hline
\hline
\multicolumn{ 1}{|l|}{``Organizing-neutral'', strategies that involve low organizing effort} &         F3 &         E3 &         B2 \\
\hline
\multicolumn{ 1}{|l|}{"} &            &         E4 &         B3 \\
\hline
\end{tabular}  
\end{footnotesize}
\caption{Cross-tool profiling schema}
\label{table:exp-study:strategy-classification}
\end{center}
\end{table}
\normalsize

%%%%%%%%%%%%%%%%%%%%%%%%%%%%%%%%%%%%%%%
% Clustering Results
%%%%%%%%%%%%%%%%%%%%%%%%%%%%%%%%%%%%%%%
Based on this classification of the tool-specific strategies, four clusters of cross-tool profiles were identified, CT1-CT4 (see \textbf{Table~\ref{table:exp-study:file-cross-tool-profiles}}).

%%%%%%%%%%%%%%%%%%%%%%%%%%%%%%%%%%%%%%%%%%%%%%%%%%%%%%%
% TABLE: Cross-tool profiles
% check '#' and '&'
%%%%%%%%%%%%%%%%%%%%%%%%%%%%%%%%%%%%%%%%%%%%%%%%%%%%%%%
\begin{table}[btp]
\begin{center}
\begin{footnotesize}
\setlength{\extrarowheight}{8pt}
% Table generated by Excel2LaTeX from sheet 'CROSS-TOOL PROFILING'
\begin{tabular}{|l|c|c|}
\hline
{\bf Cross-tool profile} & {\bf \# Users} & {\bf \% Users} \\
\hline
  CT1: pro-organizing in all 3 tools (e.g. F1/E1/B1) &          6 &       24\% \\
\hline
  CT2: pro-organizing in files \& email only &         13 &       52\% \\
\hline
  CT3: pro-organizing in files only (e.g. F2/E3/B3) &          4 &       16\% \\
\hline
  CT4: organizing-neutral in all tools &          2 &        8\% \\
\hline
\end{tabular}  
\end{footnotesize}
\caption{Four user groups identified from the clustering of the cross-tool profiles [n=25]}
\label{table:exp-study:file-cross-tool-profiles}
\end{center}
\end{table}

Six participants were \textit{pro-organizing} in all three tools (profile CT1), meaning that they reported making significant organizing effort consistently across all three collections. The most common CT1 profile was F1/E1/B1 (three participants).  Thirteen participants were \textit{pro-organizing} in files and email only (profile CT2), with F1/E2/B2 being the most common CT2 profile (five participants).  Many of the CT1 and CT2 users had a significant level of folder overlap where similar folder labels were used in different collections (see \textbf{Section~\ref{exp-study:Results-folder-overlap}}).
% \textit{There was a tendency for some of the users to focus on one hierarchy to a greater extent (``I'm primarily email driven'').

Four were \textit{pro-organizing} in files only (profile CT3).
% F2/E3/B3 and F1/E3/B2 were the most common CT3 profiles (two participants each).
Two described the file system as being the most important part of their workspace, compared to email and web bookmarks which were not seen as worth organizing.  Several cited lack of time, and the perceived effort involved in developing folder structures, as the reason for not structuring the other collections. Some went to elaborate lengths to avoid having to organize multiple types of information. For instance, P4 organized email messages in her file collection as Word documents, and organized them within file folders rather than developing another set of email folders.
% Subject 4: \textit{``I save everything in my file system (EXPAND)''} 

Two participants were \textit{organizing-neutral} in all tools (profile CT4). They made use of no folders beyond those provided by default in the email tool (e.g. \texttt{Inbox}). Both had created folders in the past (P6: 4 file folders, 4 email folders; and P7: 4 file folders) but these were no longer in use. Their files, emails and web bookmarks were all managed as unstructured lists. Both users relied on sorting mechanisms based on implicit metadata and the occasional use of search. In addition one of the users made extensive use of spatial arrangements of icons on the desktop to manage documents, which had become very cluttered. Both users mentioned that they occasionally misplaced items but this inconvenience outweighed the perceived overhead of organizing items into folders, e.g. P6: \textit{``I've got better things to do than organise my stuff.''}

%%%%%%%%%%%%%%%%%%%%%%%%%%%%%%%%%%%%%%%%%%%%%%%%%%%%%%%%%%%%
% TO ADD: WHY WAS THIS DONE? WHAT IS THE MAIN CONCLUSION?
%%%%%%%%%%%%%%%%%%%%%%%%%%%%%%%%%%%%%%%%%%%%%%%%%%%%%%%%%%%%
It is acknowledged that this comparison of organizing strategies is at a high-level, based on overall organizing tendency. However it makes an important point: that most participants (17 participants, cross-tool profiles CT2 and CT3) reported employing different levels of organizing in different tools. Note that there was a strong tendency for participants to organize files more extensively than emails or bookmarks.

%%%%%%%%%%%%%%%%%%%%
\subsection{Discussion}
\label{exp-study:Results-cross-tool-profiling-discussion}
%%%%%%%%%%%%%%%%%%%%

The results presented over the previous four sections illustrate the multiple strategies employed by users when they organize information.  \textbf{Sections~\ref{exp-study:Results-org-strategies-files}}, \textbf{\ref{exp-study:Results-org-strategies-email}} and \textbf{\ref{exp-study:Results-org-strategies-bookmarks}} highlighted that many users employ multiple strategies in different \textit{tool-specific} contexts.  Furthermore, the cross-tool analysis in \textbf{Section~\ref{exp-study:Results-cross-tool-profiling}} indicates that PIM strategies also vary significantly \textit{between} tools for many individuals.  In other words, multiple strategies can be identified at both tool-specific and cross-tool levels of analysis for many participants.  \textbf{Section~\ref{exp-study:discussion:multiple-strategies}} develops a model of organizing strategies to describe these observations.

% Previous work has not taken such cross-tool variation into account.  The results presented in this paper focus on variations in organizing strategy, e.g. 





%%%%%%%%%%%%%%%%%%%%%%%%%
% Lower-level analysis - link to other related parts of the chapter!
%%%%%%%%%%%%%%%%%%%%%%%%%
% This section has classified PIM strategies in terms of extent/style of filing, and allow the high-level comparison of behaviour between the three tools.  Lower-level strategy variation was also observed between tools in terms of the types of folders created, and how folders were arranged. For example, P17 classified both the email and files related to one of her main projects extensively. However, whilst she kept all the project email in one top-level folder, she had a hierarchy of project file folders for different versions of a report, and other types of file. As a first step towards exploring low-level variation in filing behaviour between the tools, \textbf{Section~\ref{exp-study:Results-org-dims}} compares the tools in terms of the types of folders that are developed in each.
% WE analysed participants' folder structures to investigate the concepts employed to name folders. % Aggregate results are presented in \textbf{Section~\ref{exp-study:Results-org-dims}}. 
% Different types of information -- different organizational strategies (e.g. MP3s cf. source code)

% In the next section, the results of the folder overlap analysis are presented.

%%%%%%%%%%%%%%%%%%%%%%%%%%%%%%%%%%%%%%%%%%%
%% END RESULTS2-CROSS-TOOL-PROFILING/CHAPTER 4 EXP STUDY
%%%%%%%%%%%%%%%%%%%%%%%%%%%%%%%%%%%%%%%%%%%





