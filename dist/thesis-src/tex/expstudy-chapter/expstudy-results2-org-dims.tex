%%%%%%%%%%%%%%%%%%%%%%%%%%%%%%%
% CHAPTER 4: EXPLORATORY STUDY
% 	RESULTS 2 - Organizational Dimensions
% File: tex/expstudy-chapter/expstudy-results2-cross-tool-profiling.tex
%%%%%%%%%%%%%%%%%%%%%%%%%%%%%%%%%%%%%%%%%%%%%%%%
%%%%%%%%%%%%%%%%%%%%%%%%%%%%%%%%%%%%%%%%%%%%%%%%
%%%%%%%%%%%%%%%%%%%%%%%%%%%%%%%%%%%%%%%%%%%%%%%%
\newpage
\section{Results: Analysis of Organizational Dimensions}
\label{exp-study:Results-org-dims}
%%%%%%%%%%%%%%%%%%%%%%%%%%%%%%%%%%%%%%%%%%%%%%%%
% INCLUDE PIE CHART FIGURE? 
% Check tables and the numbers in text match up
% Explain 'others' in tables
% Provide examples in each table
%	Does not take into account different numbers of active folders in different tools
%%%%%%%%%%%%%%%%%%%%%%%%%%%%%%%%%%%%%%%%%%%%%%%%

This section reports findings from the analysis of the file, email and bookmark folder structures in terms of organizational dimensions. \textbf{Section~\ref{exp-study:Results-org-strategies}} presented classifications of organizing strategies for files, email and bookmarks in terms of extent/style of filing. % This \textit{between-user/within-tool} analysis allows the high-level comparison of behaviour between the three tools.
However, lower-level variation was also observed between tools in terms of the \textit{types of folders created}, and \textit{how folders were arranged}.  For example, P17 organized \textit{both} the email and files related to one of her main projects extensively.  However, although she organized both types of information extensively, she organized them in different ways.  Whilst she kept all the email in one top-level folder, she had a hierarchy of project folders for different types of files (e.g. those relating to different versions of a report).  As a first step towards exploring low-level variation in filing behaviour between the tools, participants' folder structures were analysed to investigate the \textit{organizational dimensions} used to manage information (the concepts employed to name folders). The method used is detailed in \textbf{Section~\ref{exp-study:folder-analysis-orgdim}}, and the coding scheme that was used to label folders is shown in \textbf{Figure~\ref{table:exp-study:dimensions}}.

Most participants employed a wide range of organizational dimensions in each collection.  Therefore, due to space limitations, results are presented in aggregate form as follows.  

%%%%%%%%%%%%%%%%%%%%%%%%%%%%%%%%%%%%%%%%%%%%%%
\subsection{Files}
%%%%%%%%%%%%%%%%%%%%%%%%%%%%%%%%%%%%%%%%%%%%%%

The identified organizational dimensions for \textit{file} folders are listed in \textbf{Table~\ref{table:exp-study:file-org-dims}}. 
The three most common dimensions for document file folders were \textit{Project} (e.g. \texttt{Term-paper}), \textit{Role} (e.g. \texttt{Teaching}) and \textit{Document Class} (e.g. \texttt{reports}), representing folder percentages of 29\%, 17\% and 14\% respectively.  The wide range of organizational dimensions indicate that participants employed many types of folders when managing information.


%%%%%%%%%%%%%%%%%%%%%%%%%%%%%%%%%%%%%%%%%%%%%%%%%%%%%%%
% TABLE: SUMMARY OF ORG-DIMS FOR FILES
% in tables/ch2/exp-study-tables.xls
%%%%%%%%%%%%%%%%%%%%%%%%%%%%%%%%%%%%%%%%%%%%%%%%%%%%%%%
\begin{table}[btp]
\begin{center}
\begin{footnotesize}
\setlength{\extrarowheight}{2pt}
\begin{tabular}{|c|c|c|c|}
% Table generated by Excel2LaTeX from sheet 'FS Dims'
\hline
{\bf Rank} & {\bf Dimension} & {\bf Count (aggregated across all participants)} &   {\bf \%} \\
\hline
         1 &    Project &        317 &       29\% \\
\hline
         2 & Class of document &        185 &       17\% \\
\hline
         3 &       Role &        148 &       14\% \\
\hline
         4 &    Contact &         84 &        8\% \\
\hline
         5 & Topic / Interest &         72 &        7\% \\
\hline
         6 &     Format &         58 &        5\% \\
\hline
         7 &      Event &         47 &        4\% \\
\hline
         8 &  Temporary &         45 &        4\% \\
\hline
         9 & Version control &         38 &        4\% \\
\hline
        10 & Geographic location &         26 &        2\% \\
\hline
        11 &    General &         22 &        2\% \\
\hline
        12 &       Time &         18 &        2\% \\
\hline
        13 &     Backup &         17 &        2\% \\
\hline
        14 & Others (<1\%) &          8 &        1\% \\
\hline
           &      Total &       1085 &      100\% \\
\hline
\end{tabular}  
\end{footnotesize}
\caption{Organizational dimensions in files (aggregated across participants [n=25])}
\label{table:exp-study:file-org-dims}
\end{center}
\end{table}

%%%%%%%%%%%%%%%%%%%%%%%%%%%%%%%%%%%%%%%%%%%%%%
\subsection{Email}
%%%%%%%%%%%%%%%%%%%%%%%%%%%%%%%%%%%%%%%%%%%%%%

% As for files, participants' email hierarchies were analysed in terms of their organizational dimensions (see \textbf{Section \ref{exp-study:folder-analysis-orgdim}} for an overview of the technique used). 

The most common organizational dimensions for \textit{email} folders are listed in \textbf{Table~\ref{table:exp-study:email-org-dims}}.
The most commonly observed dimensions were \textit{Role} (e.g. \texttt{Personal}), \textit{Contact} (e.g. \texttt{Alexis}), \textit{Project} (e.g. \texttt{term-paper}), \textit{Topic/Interest} (e.g. \texttt{Java}), and \textit{Mailing List} (e.g. \texttt{linux-users}), representing percentages of 25, 19, 17, 12 and 11\% respectively. This indicates that the participants rely on a wide range of organizational dimensions when naming email folders. This in turn suggests that  users would be constrained by an organizational mechanism that constrained them to organizing email along one dominant dimension such as role.

Interestingly, the \textit{contact} dimension appears relatively low in the list. This may be explained by the fact that users could rely on implicit \texttt{Sender} metadata, rather than having to organize messages explicitly based on contact.


%%%%%%%%%%%%%%%%%%%%%%%%%%%%%%%%%%%%%%%%%%%%%%%%%%%%%%%
% TABLE: SUMMARY OF ORG-DIMS FOR EMAIL
% in tables/ch2/exp-study-tables.xls
%%%%%%%%%%%%%%%%%%%%%%%%%%%%%%%%%%%%%%%%%%%%%%%%%%%%%%%
\begin{table}[btp]
\begin{center}
\begin{footnotesize}
\setlength{\extrarowheight}{2pt}
% Table generated by Excel2LaTeX from sheet 'Email Dims'
\begin{tabular}{|c|c|c|c|}
\hline
{\bf Rank} & {\bf Dimension} & {\bf Count (aggregated across all participants)} &   {\bf \%} \\
\hline
         1 &       Role &        192 &       25\% \\
\hline
         2 &    Contact &        150 &       19\% \\
\hline
         3 &    Project &        133 &       17\% \\
\hline
         4 & Topic / Interest &         90 &       12\% \\
\hline
         5 & Mailing list &         89 &       11\% \\
\hline
         6 & Class of document &         51 &        7\% \\
\hline
         7 &    General &         27 &        4\% \\
\hline
         8 &      Event &         22 &        3\% \\
\hline
         9 &  Temporary &         13 &        2\% \\
\hline
        10 & Others (<1\%) &         16 &        2\% \\
\hline
           &      Total &        783 &      100\% \\
\hline
\end{tabular}  
\end{footnotesize}
\caption{Organizational dimensions in email (aggregated across participants, [n=25])}
\label{table:exp-study:email-org-dims}
\end{center}
\end{table}

%%%%%%%%%%%%%%%%%%%%%%%%%%%%%%%%%%%%%%%%%%%%%%
\subsection{Bookmarks}
%%%%%%%%%%%%%%%%%%%%%%%%%%%%%%%%%%%%%%%%%%%%%%

The most common organizational dimensions for \textit{bookmark} folders are listed in \textbf{Table~\ref{table:exp-study:bookmark-org-dims}}.
In contrast to the document file and email collections, web bookmark collections were dominated by one dimension, that of \textit{Topic}. Example topic-based folders that were encountered included \texttt{Star Trek}, \texttt{Cooking} and \texttt{Java}.
The \textit{Topic} dimension accounted for 55\% of folders. This ties in with the findings of \citet{gd:01} who noted that a majority of classificatory decisions in bookmarks were dependant on topic-related factors.

Note the special meaning of \textit{Document class} in the web bookmark context. The document in question related to that of the referenced website, rather than the bookmark itself. In other words, \textit{document class} was equivalent to that of website function, e.g. ``search engines''.

%%%%%%%%%%%%%%%%%%%%%%%%%%%%%%%%%%%%%%%%%%%%%%%%%%%%%%%
% TABLE: SUMMARY OF ORG-DIMS FOR BOOKMARKS
% in tables/ch2/exp-study-tables.xls
%%%%%%%%%%%%%%%%%%%%%%%%%%%%%%%%%%%%%%%%%%%%%%%%%%%%%%%
\begin{table}[btp]
\begin{center}
\begin{footnotesize}
\setlength{\extrarowheight}{2pt}
\begin{tabular}{|c|c|c|c|}
\hline
{\bf Rank} & {\bf Dimension} & {\bf Count (aggregated [n=25])} &   {\bf \%} \\
\hline
         1 & Topic / Interest &        135 &       55\% \\
\hline
         2 & Class of document &         32 &       13\% \\
\hline
         3 &    Project &         18 &        7\% \\
\hline
         4 &       Role &         17 &        7\% \\
\hline
         5 &    Contact &         15 &        6\% \\
\hline
         6 &    General &         14 &        6\% \\
\hline
         7 &      Event &          5 &        2\% \\
\hline
         8 & Others (<1\%) &          6 &        2\% \\
\hline
         9 &     Format &          3 &        1\% \\
\hline
           &      Total &        245 &      100\% \\
\hline
\end{tabular}  
\end{footnotesize}
\caption{Organizational dimensions in bookmarks (aggregated across participants, [n=25])}
\label{table:exp-study:bookmark-org-dims}
\end{center}
\end{table}

%%%%%%%%%%%%%%%%%%%%%%%%%%%%%%%%%%%%%%%%%%%%%%
\subsection{Discussion}
%%%%%%%%%%%%%%%%%%%%%%%%%%%%%%%%%%%%%%%%%%%%%%
%%%%%%%%%%%%%%%%%%%%%%%%%%%%%%%%%%%%%%%%%%%%%%%%%%%%%%%%%%%%
% TO ADD: WHY WAS THIS DONE? WHAT IS THE MAIN CONCLUSION?
%%%%%%%%%%%%%%%%%%%%%%%%%%%%%%%%%%%%%%%%%%%%%%%%%%%%%%%%%%%%
% METHOD did figures hold up for specific users as well?}

The data indicates that participants employed a wide variety of organizational dimensions both within a particular collection, and across different collections.  Note that being aggregated results, the results tend to reflect the organizational dimensions manifested by those participants who tended to create more folders in a particular tool.  However, it is argued that they are adequate to illustrate broad trends across the tools.

%%%%%%%%%%%%%%%%%%%%%%%%%%%%%%%%%%%
% Summary of tool-specific data
%%%%%%%%%%%%%%%%%%%%%%%%%%%%%%%%%%%
The most common types of file folder were \textit{project} (short-term activities, e.g. \texttt{ucl presentation}) 34\%, \textit{document class} (e.g. \texttt{letters}) 17\%, and \textit{role} (long-term activities, e.g. \texttt{teaching}) 9\%. The most common types for email folders were \textit{role} 22\%, project 20\%, \textit{contact} (e.g. \texttt{bill}) 18\%, \textit{topic/interest} (e.g. \texttt{linux}) 11\%, and \textit{mailing list} 11\%. For bookmarks, the most common types were \textit{topic/interest} 61\%, \textit{document class} 10\%, \textit{project} 6\%, and \textit{contact} 6\%. 

% Note that being aggregated results, the results tend to reflect the organizational dimensions manifested by those participants who tended to create more folders in a particular tool.  However, it is argued that they are adequate to illustrate broad trends across the tools.

%%%%%%%%%%%%%%%%%%%%%%%%%%%%%%%%%%%%%%%%%%%%%%%%%%%%%%%%%%%
% Similarities and differences in tool-specific make-up
%%%%%%%%%%%%%%%%%%%%%%%%%%%%%%%%%%%%%%%%%%%%%%%%%%%%%%%%%%%
The file and email folder structures had broadly similar dimensional make-ups. Both are dominated by \textit{project} and \textit{role}, which account for 49\% of file folders, and 42\% of email folders respectively. This similarity in terms of organisational dimensions suggests that the file and email classification schemes are potentially more suited to unification. In contrast only 15\% of web bookmarks are made up of \textit{role} and \textit{project}. The similar nature of files and emails, relative to web bookmarks may be a contributory factor here - they both represent actual documents, in contrast to web bookmarks that are references or pointers. In addition files and emails are both owned by the individual concerned, whilst bookmarks refer to remotely managed websites outside their control. Certain dimensions only appeared in certain tool contexts: for instance \textit{Mailing List} was email-specific. % One exception.

%%%%%%%%%%%%%%%%%%%%%%%%%%%
% Aggregated dimensions
%%%%%%%%%%%%%%%%%%%%%%%%%%%
% Org Dims overall biased by FILES and email (more folders)
\textbf{Table~\ref{table:exp-study:overall-org-dims}} show the organisational dimensions aggregated across all three collections, and across all participants. Three dimensions dominate: \textit{project}, \textit{role} and \textit{topic} at 22, 17 and 14\% respectively. The roughly even split between these three types suggests that users may be constrained by unification based on a particular organizational dimension, such as \textit{roles}~\citep{Shneiderman:94}, \textit{activities}~\citep{Kaptelinin:03} and \textit{contacts}~\citep{Whittaker-contactmap:02b}.  These approaches can be criticised for focusing on one organisational dimension, whilst the results in this section suggest that users employ a range of dimensions. % They would at least cause the users to have to organize their personal information in a different way.
% One limitation of the data presented here (as well as being aggregated across all users) it that it is dominated by file folders, and also email folders to a lesser extent. % Why was analysis not performed on a user-by-user basis?}
%%%%%%%%%%%%%%%%%%%%%%%%%%%%%%%%%%%%%%%%%%%%%%%%%%%%%%%
% TABLE: SUMMARY OF ORG-DIMS WORKSPACE-WIDE
% in tables/ch2/exp-study-tables.xls
%%%%%%%%%%%%%%%%%%%%%%%%%%%%%%%%%%%%%%%%%%%%%%%%%%%%%%%
\begin{table}[btp]
\begin{center}
\begin{footnotesize}
\setlength{\extrarowheight}{2pt}
\begin{tabular}{|c|c|p{4cm}|c|}
% Table generated by Excel2LaTeX from sheet 'Overall Dims'
\hline
{\bf Rank} & {\bf Dimension} & {\bf Count (aggregated [n=25])} &   {\bf \%} \\
\hline
         1 & {\bf Project} &        468 &     22\% \\
\hline
         2 &       Role &        357 &     17\% \\
\hline
         3 & Topic / interest &        297 &     14\% \\
\hline
         4 & Document class &        268 &     13\% \\
\hline
         5 &    Contact &        249 &     12\% \\
\hline
         6 & Mailing list &         90 &      4\% \\
\hline
         7 &      Event &         74 &      4\% \\
\hline
         8 &     Format &         64 &      3\% \\
\hline
         8 &    General &         63 &      3\% \\
\hline
         9 &  Temporary &         58 &      3\% \\
\hline
        10 &   Version  &         43 &      2\% \\
\hline
        11 & Other (<1\%) &         32 &      2\% \\
\hline
        12 & Geographic location &         29 &      1\% \\
\hline
        13 &     Backup &         21 &      1\% \\
\hline
    {\bf } & {\bf Total} & {\bf 2113} & {\bf 100.0\%} \\
\hline
\end{tabular}  
\end{footnotesize}
\caption{Total dimensions (aggregated across all three tools, and all participants, [n=25])}
\label{table:exp-study:overall-org-dims}
\end{center}
\end{table}


% \textbf{Section~\ref{exp-study:discussion:integration}} considers design implications based on these results.
% In the next section, \textit{cross-tool profiling} results are presented.







