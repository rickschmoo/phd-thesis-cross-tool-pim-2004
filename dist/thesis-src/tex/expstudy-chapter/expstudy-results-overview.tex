%%%%%%%%%%%%%%%%%%%%%%%%%%%%%%%
% CHAPTER 4: EXPLORATORY STUDY
%%%%%%%%%%%%%%%%%%%%%%%%%%%%%%%
% Overview of results
%%%%%%%%%%%%%%%%%%%%%%%%%%%%%%%%%%%%%%%%%%%%%%%%
%%%%%%%%%%%%%%%%%%%%%%%%%%%%%%%%%%%%%%%%%%%%%%%%
\newpage
%%%%%%%%%%%%%%%%%%%%%%%%%%%%%%%%%%
\section{Initial Observations and Results Overview}
\label{exp-study:results-overview}
% CHECK ORDERING
% SIGNPOSTING
%%%%%%%%%%%%%%%%%%%%%%%%%%%%%%%%%%

All 25 participants actively collected both files and email. 24 of the 25 also collected bookmarks to some extent, but this collection was consistently considered to be much less important. The PIM-tools used varied between participants. For managing files, most participants used the graphical file manager provided by their operating system.  \textbf{Table~\ref{table:exp-study:user_summary}} shows the operating systems employed by participants to manage their files.  Both linux users and two of the Windows 2000 users also made extensive use of command-line shells.  Many participants also used the desktop to store work in progress or temporary files. \textbf{Table~\ref{table:chapter3_email_and_browser_tools}} summarizes the email tools and web browsers that were encountered.
%%%%%%%%%%%%%%%%%%%%%%%%%%%%%%%%%%%%%%%%%%%%%%%%%%%%%%%%%%%%%%
% files - consider use of graphical explorer/desktop/cmd-line
%%%%%%%%%%%%%%%%%%%%%%%%%%%%%%%%%%%%%%%%%%%%%%%%%%%%%%%%%%%%%%

%%%%%%%%%%%%%%%%%%%%%%%%%%%%%%%%%%%%%%%%%%%%%%%%%%%%%%%%%%
% TABLE: Email tools and Web browsers used by the study participants}
%%%%%%%%%%%%%%%%%%%%%%%%%%%%%%%%%%%%%%%%%%%%%%%%%%%%%%%%%%
\begin{table}[hbtp]
\begin{minipage}{3.0 in}
\begin{footnotesize}
\setlength{\extrarowheight}{2pt}
\begin{tabular}{|c|c|}
\hline
{\bf Email Tool} & {\bf Number of users} \\
\hline
    eudora &          9 \\
\hline
   outlook &          5 \\
\hline
  netscape &          4 \\
\hline
outlook express &          3 \\
\hline
    xfmail &          1 \\
\hline
      pine &          1 \\
\hline
        mh &          1 \\
\hline
   pegasus &          1 \\
\hline
\end{tabular}  
\end{footnotesize}
\end{minipage}
\begin{minipage}{3.0 in}
\begin{footnotesize}
\begin{tabular}{|c|c|}
\hline
{\bf Web browser} & {\bf Number of users} \\
\hline
  netscape &         11 \\
\hline
internet explorer &         14 \\
\hline
\end{tabular}  
\end{footnotesize}
\end{minipage}
\begin{center}
\caption{Email tools and web browsers used by participants}
\label{table:chapter3_email_and_browser_tools}
\end{center}
\end{table}



% One user succinctly summed up the challenge of PIM (P24: "stuff goes into the computer and doesn't come out - it just builds up!").
Participants were highly motivated to talk about PIM - it was an area that was important to them, and a source of problems and frustration.  One participant (P24) succinctly summed up the ongoing challenge of PIM, and the need to organize: \textit{``stuff goes into the computer and doesn't come out - it just builds up''}.  Hearing about these problems at first-hand reinforced the author's belief that this was a compelling real-world problem space that merited more research.
%%%%%%%%%%%%%%%%%%%%%%%%%%%%%%%%%%%%%%%%%%%%%%%
% NB: take care not to exaggerate problems
%%%%%%%%%%%%%%%%%%%%%%%%%%%%%%%%%%%%%%%%%%%%%%%

Despite the researcher's concerns about privacy, all participants were very open, although one joked: \textit{``this is a high-trust exercise!''}. In fact several participants seemed to enjoy ``opening up'', e.g. P25: \textit{``Its like a confessional getting all my computer problems off my chest''}. Only two excluded areas of their workspace for reasons of personal and/or professional confidentiality.  P8 permitted access to his work-related document files only.  Access to his email and web bookmarks was unaffected.  P13 restricted access to portions of his document file and email collections because they contained confidential information relating to personnel management. It is acknowledged that due to these two cases, some of the quantitative results presented in this chapter may be slightly conservative (e.g. average numbers of folders per collection). Aside from these exceptions, the guided tours were unrestricted.  It is envisioned that the high level of openness was due to participants' prior familiarity and trust in the researchers. % This level of openness compares positively with the experiences of other researchers who in similar studies encountered users who restricted access to their personal information for reasons of privacy and/or commercial confidentiality~\citep{Whittaker-email:96,Bellotti:03} % (\textit{ADD GONCALVES STUDY?}).
% It is envisaged that their openness was due to a number of factors, primarily the participants' prior familiarity with the researcher, but also the relative lack of confidentiality in academia compared to the commercial arena. A final factor encouraging openness may be that the study was not concerned with the content of specific items of information. Instead the study focused on high-level strategies and the folder hierarchies.  Users seemed content to have their folder hierarchies recorded. This may indicate that folder names carry less confidential meaning than the content of particular items. 

The study findings are presented over the next six sections, as follows: % PLUS PROBLEMS!

\begin{itemize}

%%%%%%%%%%%%%%%%%%%%%%%%%%%%%%%%%%%
% 1. Comparison BEHAVIOUR between the tools
%%%%%%%%%%%%%%%%%%%%%%%%%%%%%%%%%%%
% \textbf{Section~\ref{exp-study:Results-comparison}} compares PIM practices between the three tools.
\item Firstly, \textbf{Section~\ref{exp-study:Results-comparison}} presents a high-level comparison of user behaviour between the three PIM-tools in terms of four PIM sub-activities: acquisition, organization, maintenance and retrieval.

The next four sections focus on the organizing sub-activity.

%%%%%%%%%%%%%%%%%%%%%%%%%%%
% 2. COMPARE ORG STRATS
%%%%%%%%%%%%%%%%%%%%%%%%%%%
% Finally \textbf{Section~\ref{exp-study:qual_classification}} offers a user classification of the study participants in terms of the extent to which they organize information from a cross-tool perspective.}
 % presents findings from the \textit{between-tools/within-user} analysis (perspective 2 in \textbf{Figure~\ref{fig:exp-study:intra_versus_cross_tool}}).  A \textit{cross-tool profile} of participants' PIM practices is developed in terms of their tendency to organize multiple types of information.  
% focuses on the organizing sub-activity.  Key findings include the characterisation of the strategies used to manage each type of personal information. The combination-strategies employed by many participants are noted, and the limitations of previous classifications of PIM strategies are highlighted. Then \textbf{Section~\ref{exp-study:Results-cross-tool-profiling}} reports the results of the \textit{Between-Tools/Within-User Analysis} which attempted to construct a cross-tool profile on a user-by-user basis. % his section includes a classification of participants' reported filing behaviour in each PIM-tool.
\item \textbf{Section~\ref{exp-study:Results-org-strategies}} reports the classification of organizing strategies in each tool context.  It then moves on to present the findings from the cross-tool profiling, in which strategies were compared between tools for each participant.

%%%%%%%%%%%%%%%%%%%%%%%%%%%%%%%%%%%
% 3. ORG DIMS
%%%%%%%%%%%%%%%%%%%%%%%%%%%%%%%%%%%
\item \textbf{Section~\ref{exp-study:Results-org-dims}} reports the analysis of the organizational dimension make-up of the three PIM-tools. % \textbf{Section~\ref{exp-study:Results-org-dims}} reports the analysis of the folder structures in terms of organizational dimensions.

%%%%%%%%%%%%%%%%%%%%%
% 4. FOLDER OVERLAP
%%%%%%%%%%%%%%%%%%%%%
\item \textbf{Section~\ref{exp-study:Results-folder-overlap}} reports findings from the analysis of folder overlap for those participants who organized multiple types of information.

%%%%%%%%%%%%%%%
% OTHER: CHANGES
%%%%%%%%%%%%%%%
\item \textbf{Section~\ref{exp-study:comparison-changes}} surveys participants' reports of historical and planned changes in their organizing strategy.
% Findings related to problems, changes in strategy and cross-tool integration are also presented.

%%%%%%%%%%%%%%%
% OTHER: UXP
%%%%%%%%%%%%%%%
\item \textbf{Section~\ref{exp-study:comparison-problems}} surveys reported problems relating to PIM.  Both tool-specific and cross-tool problems are discussed.

\end{itemize}

\textbf{Figure~\ref{fig:exp-study:analysis-structure}} on page~\pageref{fig:exp-study:analysis-structure} provides an overview of the different sets of findings and the respective methodology.
A sample of the qualitative data collected for each participant is shown in \textbf{Appendix~\ref{chap:appendices--study-data}} on page~\pageref{chap:appendices--study-data:exp-study}.
 
%%%%%%%%%%%%%%%%%%%%%%%%%%%%%%%%%%%%%%%%%
% Other aspects of PIM covered as well:
% * changes
% * problems/UE
% * integration. Participant comments relating to integration between the three collections are reported 
%%%%%%%%%%%%%%%%%%%%%%%%%%%%%%%%%%%%%%%%%
% Moving on, \textbf{Section~\ref{exp-study:comparison-problems}} surveys reported problems, \textbf{Section~\ref{exp-study:comparison-cross-tool}} presents cross-tool findings, and \textbf{Section~\ref{exp-study:comparison-changes}} surveys reported changes in PIM strategy.

%%%%%%%%%%%%%%%%%%%%%%%%%%%%%%%%%%%%%%%%%%%%%%%%
%\subsection{Initial Observations}
%\label{exp-study:comparison-initial-observations}
%%%%%%%%%%%%%%%%%%%%%%%%%%%%%%%%%%%%%%%%%%%%%%%%
% First, \textbf{Section~\ref{exp-study:comparison-initial-observations}} makes initial observations about the quality of data obtained.
% In the next section, the nature of acquisition is compared between files, email and bookmarks.

%%%%%%%%%%%%%%%%%%%%%%%%%%%%
% END OF RESULTS OVERVIEW
%%%%%%%%%%%%%%%%%%%%%%%%%%%%