%%%%%%%%%%%%%%%%%%%%%%%%%%%%
% CHAPTER 8 - CONCLUSION - Critical reflection
%%%%%%%%%%%%%%%%%%%%%%%%%%%%
%%%%%%%%%%%%%%%%%%%%%%%%%%%%%%%%%%%%%%%%%%%%%%%%%%%%%%%%%%%%%%%%%%%%%%%%%%%%%%%%%%%%%%%%%%%%%
% Richard Boardman PhD Thesis: Improving Tool Support for Personal Information Management
%%%%%%%%%%%%%%%%%%%%%%%%%%%%%%%%%%%%%%%%%%%%%%%%%%%%%%%%%%%%%%%%%%%%%%%%%%%%%%%%%%%%%%%%%%%%%

\newpage
%%%%%%%%%%%%%%%%%%%%%%%%%%%%%%
%%%%%%%%%%%%%%%%%%%%%%%%%%
\section{Critical Review of the Thesis}
\label{conclusion:critical-reflection}
%%%%%%%%%%%%%%%%%%%%%%%%%%
%%%%%%%%%%%%%%%%%%%%%%%%%%%%%%

%%%%%%%%%%%%%%%%%%%%%%%%%%%%%%%%%%%%%%%%%%%%%%%%%%%%%%%%%%%%%%%%%%%%%%%%%%%%%%%%%%%%%%%%%%%%%
% \textit{Study, design and evaluation were all limited to the context of one desktop computer.}
%%%%%%%%%%%%%%%%%%%%%%%%%%%%%%%%%%%%%%%%%%%%%%%%%%%%%%%%%%%%%%%%%%%%%%%%%%%%%%%%%%%%%%%%%%%%%
% \textit{REFER: to chapter centric things as well}
%%%%%%%%%%%%%%%%%%%%%%%%%%%%%%%%%%%%%%%%%%%%%%%%%%%%%%%%%%%%%%%%%%%%%%%%%%%%%%%%%%%%%%%%%%%%%
% \textit{PRIMARY LIMITATION: manpower resources. To be taken account during the consideration of the contributions offered.}
%%%%%%%%%%%%%%%%%%%%%%%%%%%%%%%%%%%%%%%%%%%%%%%%%%%%%%%%%%%%%%%%%%%%%%%%%%%%%%%%%%%%%%%%%%%%%
% \textit{STRESS: new area, therefore ambitious challenges overcome. These outweigh limitations.}
%%%%%%%%%%%%%%%%%%%%%%%%%%%%%%%%%%%%%%%%%%%%%%%%%%%%%%%%%%%%%%%%%%%%%%%%%%%%%%%%%%%%%%%%%%%%%
%State what section has offered: critical appraisal and recommendations based on lesson learned
%This section has assessed the study, design and evaluation methodologies employed in the thesis. 
%Despite limitations, still useful -- set apart from being attempt at cross-tool, longitudinally grounded work.
% \textit{Here discussion of the specifics of my method. Assess/appraise the work reported in this thesis.  Tell about what I have learned/gained. Do methods etc. show promise?}
%%%%%%%%%%%%%%%%%%%%%%%%%%%%%%%%%%%%%%%%%%%%%%%%%%%%%%%%%%%%%%%%%%%%%%%%%%%%%%%%%%%%%%%%%%%%%
%There are several arguments that can be directed/leveled against the work presented in this thesis:
%\begin{itemize}
%	\item Reflections on method. What would I do differently? What improvements would I make?
%	\item Limitation of evaluation results?
%	\item Applicability in the real world
%	\item Angela's cliff - plus I'm sure Angela will suggest a few more!
%	\item \textit{THINK: What are the implications of this now validated thesis?}
%%%%%%%%%%%%%%%%%%%%%%%%%%%%%%%%%%%%%%%%%%%%%%%%%%%%%%%%%%%%%%%%%%%%%%%%%%%%%%%%%%%%%%%%%%%%%
% This section offers a critical assessment/appraisal/analysis of the study, design, and evaluation research methodologies employed in the thesis. Comment on overall methodology and make recommendations about future work in this area.  What lessons have been learned?  
%%%%%%%%%%%%%%%%%%%%%%%%%%%%%%%%%%%%%%%%%%%%%%%%%%%%%%%%%%%%%%%%%%%%%%%%%%%%%%%%%%%%%%%%%%%%%
%%%%%%%%%%%%%%%%%%%
% RECOMMENDATIONS
% Then generalize towards general recommendations for work in this field based on lessons, experience
%%%%%%%%%%%%%%%%%%%
% To make methodological recommendations for future research in this area, based on the lessons learned in pursuing this course of research. Each of the next four sections on study, design and evaluation methods include a subsection containing methodological recommendations.  \textbf{Section X} provides a summary of all the methodological recommendations.
%%%%%%%%%%%%%%%%%%%%%%%%%%%%%%%%%%%%%%%%%%%%%%%%%%%%%%%%%%%%%%%%%%%%%%%%%%%%%%%%%%%%%%%%%%%%%
% This section critically appraises the research presented in this thesis.  
\textbf{Section~\ref{disc:methodological-discussion:overall}} considers the overall success of the design-based research methodology in achieving the research aims.  Then, \textbf{Sections~\ref{disc:methodological-discussion:study}} and  \textbf{\ref{disc:methodological-discussion:design}} consider the strengths and weaknesses of the study, and design/evaluation components on the research.
% The next three sections consider the methods used study, design and evaluation components of the research.  \textbf{Section~\ref{disc:methodological-discussion:study}} considers study methods, \textbf{Section~\ref{disc:methodological-discussion:design}} discusses the incremental design approach, and \textbf{Section~\ref{disc:methodological-discussion:evaluation}} considers evaluation techniques.
% study, design and evaluation methodologies The objective of this section is to critically appraise the study, design and evaluation methods employed over the course of the thesis.  The section is structured based on the three main components of substantive work. Firstly, the overall success of the  is considered in \textbf{Section~\ref{disc:methodological-discussion:overall}}. Then  considers the study methods employed in \textbf{Chapters~\ref{chapter:exploratory_study}} and \textbf{\ref{chapter:main-study}}. \textbf{Section~\ref{disc:methodological-discussion:design}} discusses the incremental design approach, and \textbf{Section~\ref{disc:methodological-discussion:evaluation}} assesses the evaluation approach from \textbf{Chapter~\ref{chapter:main-study}}.  Many of the issues identified in this section form the basis for the future work routes outlined overleaf.




%%%%%%%%%%%%%%%%%%%%%%%%%%%%%%%%%%%%%%%%%%%%%%%%%%%%%%%%%%%%%%%%%%%%%%%%%%%%%%%%%%%%%%%%%%%%%%%%%%%%%%%%%%%%%%%%%
%%%%%%%%%%%%%%%%%%%%%%%%%%%%%%%%%%%%%%%%%%%%%%%%%%%%%%%%%%%%%%%%%%%%%%%%%%%%%%%%%%%%%%%%%%%%%%%%%%%%%%%%%%%%%%%%%
%%%%%%%%%%%%%%%%%%%%%%%%%%%%%%%%%%%%%%%%%%%%%%%%%%%%%%%%%%%%%%%%%%%%%%%%%%%%%%%%%%%%%%%%%%%%%%%%%%%%%%%%%%%%%%%%%
%%%%%%%%%%%%%%%%%%%%%%%%%%%%%%%%%%%%%%%%%%%%%%%%%%%%%%%%%%%%%%%%%%%%%%%%%%%%%%%%%%%%%%%%%%%%%%%%%%%%%%%%%%%%%%%%%


%%%%%%%%%%%%%%%%%%%%%%%%%%%%%%%%%%%%%%%%
\subsection{Design-based Research Approach}
\label{disc:methodological-discussion:overall}
%%%%%%%%%%%%%%%%%%%%%%%%%%%%%%%%%%%%%%%%

%%%%%%%%%%%%%%%%%%%%%%%%%%%%%%%%%%%%%%%%%%%%%%%%%%%%%%%%%%%%%%%%%%%%%%%%%%%%%%%%%%%%%
% Overall success of choice of design-based research, in which research aimed at :
%%%%%%%%%%%%%%%%%%%%%%%%%%%%%%%%%%%%%%%%%%%%%%%%%%%%%%%%%%%%%%%%%%%%%%%%%%%%%%%%%%%%%
% Based on the empirical findings, and experience in evaluating WorkspaceMirror a number of theoretical models were proposed to explain aspects of PIM behaviour. % The next section surveys the contributions made over the thesis.
The use of design as a research vehicle was influenced by Carroll's discussion of the applied research paradigm~\citep{Carroll:00}.  The design-centred methodology was also motivated by the philosophy that if you want to provide guidance to designers, you need get in there and get your hands dirty doing some design work yourself.   In summary, it is argued that this research approach was successful in generating a number of contributions.  As well as producing design requirements, the study component of the research offered significant advances on previous knowledge, by investigating behaviour across multiple PIM-tools.  Furthermore, a novel form of PIM-integration was invented by the author, and although its evaluation resulted in both positive and negative feedback, the potential to share high-level folders between PIM-tools was confirmed.  Although the prototype is not as technologically innovative as previous designs, the author argues that it represents a more complete contribution to HCI knowledge, due to its systematic evaluation.  
%, in which design acts as a research vehicle to achieve two complementary aims: (1) improved understanding of the world, and (2) the development of a new artefact.   

% It can be argued that such an approach enables the gap to be bridged between research and design.  The author also had a strong technical background in software engineering and felt impelled to do some development.
% The study components of the research enabled the development of empirical and theoretical contributions through the investigation of current behaviour.  In parallel, the research also enabled the production of a specific PIM-integration design instance, and through its evaluation, the development of guidelines for the wider design genre.
% This section appraises the overall \textit{design-based research methodology} employed in the thesis.  The method was justified based on the twin aims of the thesis: (1) to develop understanding, and (2) to design and evaluate a new PIM integration mechanism based on that understanding. % Symbiotic relationship between the two. 


%%%%%%%%%%%%%%%
% T/A cycle
%%%%%%%%%%%%%%%
% See \textbf{Figure~\ref{fig:iterative_methodology}} (adapted from Carroll's scenario-based design methodology~\cite{carroll:00}).  \textit{NB: EMPHASISE exploratory study is requirements stage.}
% How can design-based research operate at this complex high-level? Talk about in terms of Design/Action-Science Paradigm and Task/Artefact cycle? -- appropriate? See \textbf{Figure~\ref{fig:TaskArtefactCycle}}. Carroll's \textit{action science} approach~\cite{jc-cycle:92} in which 
% The research can be summarized as an excursion around the Task-Artefact cycle~\citep{jc-cycle:92}.  This entailed the generation of requirements based on solid empirical investigation, the design of a potential solution, and the formative evaluation of the design in the context of the real-world activity. The research also involved theory-building based on the empirical data collected.


%%%%%%%%%%%%%%%%%%%%%%%%%%%%%%%
% RESEARCH CHALLENGE MET HERE:
%%%%%%%%%%%%%%%%%%%%%%%%%%%%%%%
% Retouch challenges, this was an ambitious agenda. 
%  Thus need to narrow scope. But meant that lots of experience gained.
% Many substantial challenges were encountered over the course of the research.
% PIM, the domain of investigation, PIM, is a complex area, as d by
The extended conceptual framework in \textbf{Section~\ref{discussion:theoretical-framework}} illustrates the complexity of PIM as a research domain.  Since the author acted as the sole researcher, performing all studies, design and evaluation work, a number of pragmatic constraints were applied to the research scope. %%%%%%%%%%%%%%%%%%%
% CONSTRAINTS: 
%%%%%%%%%%%%%%%%%%%
% FOCUS ON ONLY THREE TOOLS: Consider other tools (e.g. calendar)
% For example, due to the ambitious research scope, a number of constraints were applied in the face of the limited resources available.
Firstly, the scope of concern was limited to three PIM-tools -- files, email and bookmarks -- on one personal computer in a work context.  Ideally, other related tools such as calendars and to-do lists would have been studied in-depth, as well as PIM activity on other devices.  Furthermore, there is potential for follow-up work outside the work context.  % A key route for future work in \textbf{Section~\ref{conclusion:future-work}} is the widening of the study scope in this way.



%%%%%%%%%%%%%%%%%%%%%%%%%%%%%%%
% Pre-empt limitations and criticisms
%%%%%%%%%%%%%%%%%%%%%%%%%%%%%%%
%\item However limitations as discussed below.
%\item What are the key criticisms that could be levelled against this approach? (e.g. from Sasse thesis).  Provide arguments against them.
%\end{itemize}
%One high-level limitation of the approach is that it covered a lot of areas, rather than tunnelling down deep in one area.  The result of this is that the research offers a set of related contributions around the central theme of PIM.  However, an important resulting benefit for the author was that a wide range of experience was gained.

% Key challenges and limitations of the study, design and evaluation components of the research are discussed over the next three sections.





%%%%%%%%%%%%%%%%%%%%%%%%%%%%%%%%%%%%%%%%%%%%%%%%%%%%%%%%%%%%%%%%%%%%%%%%%%%%%%%%%%%%%%%%%%%%%%%%%%%%%%%%%%%%%%%%%
%%%%%%%%%%%%%%%%%%%%%%%%%%%%%%%%%%%%%%%%%%%%%%%%%%%%%%%%%%%%%%%%%%%%%%%%%%%%%%%%%%%%%%%%%%%%%%%%%%%%%%%%%%%%%%%%%
%%%%%%%%%%%%%%%%%%%%%%%%%%%%%%%%%%%%%%%%%%%%%%%%%%%%%%%%%%%%%%%%%%%%%%%%%%%%%%%%%%%%%%%%%%%%%%%%%%%%%%%%%%%%%%%%%
%%%%%%%%%%%%%%%%%%%%%%%%%%%%%%%%%%%%%%%%%%%%%%%%%%%%%%%%%%%%%%%%%%%%%%%%%%%%%%%%%%%%%%%%%%%%%%%%%%%%%%%%%%%%%%%%%

%%%%%%%%%%%%%%%%%%%%%%%%%%%%%%%%%%%%%%%%
\subsection{Study Component}
\label{disc:methodological-discussion:study}
%%%%%%%%%%%%%%%%%%%%%%%%%%%%%%%%%%%%%%%%
% Acknowledge limitations
%%%%%%%%%%%%%%%%%%%%%%%%%%
% OTHER VALIDITY ISSUES
%%%%%%%%%%%%%%%%%%%%%%%%%%
% FOCUS ON SPECIFIC TOOL-SPECIFIC APPS: Focus on specific applications (e.g. outlook versus eudora)
% \textit{The inter-tool comparison reported in \textbf{Chapter~\ref{chapter:exploratory_study}} did not take account of variation in PIM-tools between users.  Further work could limit participation to users of specific tools (e.g. Windows, Outlook, Internet Explorer).}
%%%%%%%%%%%%%%%%%%%%%%%%%%%%%%%%%%%%%%%%%%%%%%%%%%%%%%%%%%%%%%%%%%%%%%%%%%%%%%%%%%%%%%%%%%%%%%%%%%%%%%%%%%%%%%%%%


%%%%%%%%%%
% RECAP
%%%%%%%%%%
% The study methods are discussed as follows.  
%This section considers the study methods employed in \textbf{Chapters~\ref{chapter:exploratory_study}} and \textbf{\ref{chapter:main-study}}.

%%%%%%%%%%%%%%
% RICH-DATA: 
%%%%%%%%%%%%%%
% First strengths and positive aspects. Enabled range of study insights, and dual-purpose evaluation and further unanticipated insights
The studies reported in \textbf{Chapters~\ref{chapter:exploratory_study}} and \textbf{\ref{chapter:main-study}} yielded a rich data set.  % The findings included new insights, as well as extensions of previous work.
% indeed in many ways this thesis only scratches the surface.  
%%%%%%%%%%%%%%%%%%%%%%%%%%%%%%%%
% CROSS-TOOL/LONGIT NEWNESS: 
%%%%%%%%%%%%%%%%%%%%%%%%%%%%%%%%
%Challenge overcome: collect data/evaluate over time (and over three tools, first study to do this). 
% collect data/evaluate over three tools (first study to do this) 
A key achievement was the cross-tool nature of the empirical work which sets the research apart from previous work in this area.  Furthermore, few longitudinal studies have been carried out, and the main study was successful in revealing insights that would not be possible in a shorter-term study (e.g. changes in PIM strategy).  Although the work set out with a focus on PIM-integration, it is noted that many of the contributions relate to PIM in general.  This was a natural consequence of performing a cross-tool investigation, which facilitated both tool-specific and cross-tool insights. It is hoped that the work reported here, as well as building on previous results, has opened up new directions for future work.
% In particular, the empirical findings from \textbf{Chapters~\ref{chapter:exploratory_study}} and \textbf{\ref{chapter:main-study}} offer contributions regarding changes in organizing strategy, and the supporting nature of PIM.  
% Over the course of the thesis, the work shifted away somewhat from a pure focus on PIM-integration.

However, a number of limitations are acknowledged regarding the empirical work.
%%%%%%%%%%%%%%%%%%%%%%%%%%%%%%%%%%%%%%%%%%%%%%
% exploratory. Basis for future work
%%%%%%%%%%%%%%%%%%%%%%%%%%%%%%%%%%%%%%%%%%%%%%
% EXPLORATORY: Clearly, since many of the methods were exploratory in nature, the resulting contributions are indicative, and should be considered the basis for future work.
% \textit{Secondly, limitations and negative aspects are considered.}
% BOTTOM LINE: Defend methods: pragmatics, time and manpower limitations. Bottom line: useful first step, not perfect but basis for future work.
% DATA ANALYSIS LIMITATIONS? Need for more systematic analysis at some stages, for example IRR for organizational dimension analysis etc. Spend time going over results with participants.
The first study was exploratory, and it is acknowledged that the results are preliminary.  Areas that could be revisited, assuming sufficient resources, are the folder-overlap and organizational dimensions analyses.  Routes for improvement include the use of inter-rater reliability in the coding of organizational dimensions, and identification of folder overlap. Further validation of results would also be desirable through reflective interviews with participants.

Other limitations are noted in the main study methodology.  The author acknowledges that his design intervention had a significant influence on participants' behaviour, e.g. changes in strategy.  This was acknowledged in the results section, and was an unavoidable consequence of the dual-purpose nature of the study.  
% Although longitudinal, was length enough time to study long-term activity like PIM? Future work: longer time. \textit{Carry out over longer-term}
It can also be argued that the main study could be extended longitudinally. Several aspects of PIM were relatively sporadic (e.g. folder-based organization), so more data would be useful.  This is difficult however, as the study already involved significant commitment on the part of the participants.  % Secondly, the studies were limited in the number of participants involved and thus the results are not statistically significant.  % However it is hoped they suggest interesting routes for future studies with greater resources at hand. % EXAMPLE

%%%%%%%%%%%%%%%%%%%%
% user selection
%%%%%%%%%%%%%%%%%%%%
% pros: Getting access to personal data/privacy issues
% cons: Representative? number BOTH CH4 and CH6: 
% USER SELECTION: novice users, strangers (no friends)
%%%%%%%%%%%%%%%%%%%%%%
% STUDY: NATURE:IDIO
%%%%%%%%%%%%%%%%%%%%%%
% More tools -- to reflect variation in users? So many users -- so few tools.  Encourage variation in tools to match variation in practice.  MORE ATTENTION TO NOVICE/NON-GEEKS: Universal Usability
A final limitation of both studies was the number and selection of participants. The studies focused on a small technically experienced group of users.  However, it must be stated that this criticism can be applied to the vast majority of work in this area.   Different behaviour and problems may well be expected for other types of user, leading to different user needs.  
%%%%%%%%%%%%%%%%%%
% NATURAL DATA
%%%%%%%%%%%%%%%%%%
% NATURAL CONYTEXT: Argue that the field study matches aim of studying . Impact of typicality.
% A primary aim of the study component was to collect real-world data based on usage in natural contexts. It was therefore decided to use a field study method in preference to a controlled study.  Although a correspondingly rich data set was collected, the downside was the sheer wealth of data collected, and the difficulty of generalizing such an idiosyncratic activity across users.
% CHALLENGES: : Organizational challenges but collected rich set of data. 
% Overall success of field study in lieu of significant organizational/logistical challenges
% Numerous logistical challenges were overcome, especially in the field study component reported in \textbf{Chapter~\ref{chapter:main-study}} when users at two different academic institutions, in different parts of London, were involved in testing WM.
%%%%%%%%%%%%%%%%%%%%%%%%%%%%%%%%%%%%%%%%%%%%%%
% INTERVENTION: Impact of the Study - ecological validity}
%%%%%%%%%%%%%%%%%%%%%%%%%%%%%%%%%%%%%%%%%%%%%%
% In other words, it can be argued that the ecological validity was impacted.  Relative influences of study and design on practice. Can this ever be avoided (Dix)? Hawthorne-like effect/or ecological validity - study changed behaviour.  Always going to be a factor - how I chose to reduce impact.
% Future work: watch over longer-term without interventions of design and interviews, i.e. focus on objective data.  Or do in background (e.g. base on user logs?) \textit{Repeat without intervention. But ethical question of non-clear intervention!}
% MAIN STUDY
%%%%%%%%%%%%%%%%
% TOWARDS FUTURE WORK
%%%%%%%%%%%%%%%%
% In \textbf{Section~\ref{conclusion:future-work}}, an ideal study is laid out, dealing with some of the above limitations.
\textbf{Section~\ref{conclusion:future-work-empirical}} describes follow-up studies which could be performed to deal with these limitations.

% Furthermore, this work can be criticized as most participants were known to the investigator.  This is defended on the count that a key challenge of work in this area is overcoming privacy issues to get access to user data.

% Although the work set out to focus on PIM-integration, it is noted that many of the contributions relate to PIM in general. In particular, the empirical findings from \textbf{Chapters~\ref{chapter:exploratory_study}} and \textbf{\ref{chapter:main-study}} offer contributions regarding changes in organizing strategy, and the supporting nature of PIM.  
% Over the course of the thesis, the work shifted away somewhat from a pure focus on PIM-integration.

%However, it is acknowledged that this work in many ways represents the scratching of the surface.  As~\citet{Whittaker-rta:00} note there is room for a multitude of studies in this area.  


%%%%%%%%%%%%%%%%%%%%%%%%%%%%%%%%%%%%%%%%%%%%%%%%%%%%%%%%%%%%%%%%%%%%%%%%%%%%%%%%%%%%%%%%%%%%%%%%%%%%%%%%%%%%%%%%%
%%%%%%%%%%%%%%%%%%%%%%%%%%%%%%%%%%%%%%%%%%%%%%%%%%%%%%%%%%%%%%%%%%%%%%%%%%%%%%%%%%%%%%%%%%%%%%%%%%%%%%%%%%%%%%%%%
%%%%%%%%%%%%%%%%%%%%%%%%%%%%%%%%%%%%%%%%%%%%%%%%%%%%%%%%%%%%%%%%%%%%%%%%%%%%%%%%%%%%%%%%%%%%%%%%%%%%%%%%%%%%%%%%%
%%%%%%%%%%%%%%%%%%%%%%%%%%%%%%%%%%%%%%%%%%%%%%%%%%%%%%%%%%%%%%%%%%%%%%%%%%%%%%%%%%%%%%%%%%%%%%%%%%%%%%%%%%%%%%%%%

%%%%%%%%%%%%%%%%%%
% Design
%%%%%%%%%%%%%%%%%%
%%%%%%%%%%%%%%%%%%%%%%%%%%%%%%%%%%%%%%%%
\subsection{Design and Evaluation Component}
\label{disc:methodological-discussion:design}
%%%%%%%%%%%%%%%%%%%%%%%%%%%%%%%%%%%%%%%%
% \item NB: this could be moved to CHAPTER 5 INTRO DISCUSSION
%%%%%%%%%%%%%%%%%%%%%%%%%%%%%%%%%%%%%%%%
% Design-based research: an attempt to close the research-practice gap
%%%%%%%%%%%%%%%%%%%%%%%%%%%%%%%%%%%%%%%%

% This section reflects on the design component of the thesis.
%%%%%%%%%%
% RECAP
%%%%%%%%%%
% Firstly, the design methods are reviewed.  
% Consider including: use of participatory design
% Consider including: paper prototyping, high-fidelity prototyping
%%%%%%%%%%%%%%%%%%%%%%%%%%%%%%%%%%
% LIMITED ROUNDS OF DEVELOPMENT: 
%%%%%%%%%%%%%%%%%%%%%%%%%%%%%%%%%%
It is argued that the design component of the study served the purposes of the researcher well.  Firstly, it resulted in a novel PIM-integration mechanism, which was subsequently evaluated.  The evaluation identified strengths and weaknesses of folder-mirroring, expressed in terms of the trade-off between organizational consistency and organizational flexibility.  This lead to a moderation of the author's initial view of integration being a purely desirable design route, to one with both pros and cons.  Based on the evaluation findings, a number of design and methodological guidelines were developed for the wider field. In particular, the author argued the need for designers to carefully assess the pros and cons of integration, both from a cross-tool and tool-specific perspective. 

% Successfully designed tool and carried out long-term field trial evaluation.
% ROBUST: Design component of thesis was successful in terms of developing a robust, useful tool.
% Overall, the incremental design approach employed was successful in terms of generating a novel tool, which was robust enough to withstand long-term evaluation and generate a rich set of related data.
% Although tool feedback was mixed, the promise of mirroring top-level folders was illustrated.  However, the 

% INCREMENTAL: Choice of incremental design approach. Why?  Easy to evaluate.
% Consider incremental design approach (consider pros and cons):
% Theoretical perspective: incremental, anti revolutionary design
% Incremental worked: ease of implementation, success at bootstrapping for users (build on current tools). All users tried out tool and over half used over the long-term.
The incremental design approach lead to a number of benefits.  Firstly, it meant that a robust prototype could be developed within the resource constraints available.  Secondly, systematic evaluation was also promoted by making a limited set of changes~\citep{newman:95}.  A final logistical benefit was that test users were able to install the tool within their existing PIM-tool environment with minimum disruption.  Since WM was installed on top of existing folder structures, the testers could try out folder-mirroring without any need to reorganize their data.  Although incremental design is often recommended in HCI, e.g.~\citep{Carroll:00}, it has rarely been used in the field of PIM.  One reason may be that the limitation of \textit{local optimization} puts off designers with ambitious aims.  However, it is argued that the overall benefits of incremental design as described above, outweigh this limitation.
Based on these experiences, a move away from revolutionary design towards an incremental approach is recommended in future research.  In this thesis, it has been shown how this approach is particularly appropriate when resources are limited. % Indeed, some of the empirical feedback suggested that incremental design changes can achieve significant gains.  M5 noted that subtle changes in strategy can result in a large impact on user satisfaction. %: ``Subtle changes, big effect?''
% limiting the changes permissible from current technology means that ground-breaking advances are unlikely as new design iterations inherit many of the limitations of their predecessors.  
%%%%%%%%%%%%%%%%%%%%%%%%%
% local optimization
%%%%%%%%%%%%%%%%%%%%%%%%%
% \item Consider Carroll's potential limitations of this approach (myopic, local optimization, bound to local situation), BUT HERE FAR-REACHING BENEFITS OBSERVED - plus also USEFUL AS A RESEARCH VEHICLE % others in green sleeve

% , primarily that 
One acknowledged limitation of the design work in this thesis is that it was not possible to redesign the tool based on the feedback reported in \textbf{Section~\ref{main-study:results:themes-design-recs}}.  The most promising extensions are discussed as possible future work in \textbf{Section~\ref{conclusion:future-work}}.


%In particular, strengths from a cross-tool perspective must be carefully weighed against possible weaknesses in a tool-specific context.  This represents a shift from the author's initial bias towards PIM-integration being a highly desirable design aim.  Such a view ties in with much of the design sentiment in this area. % convinced author that he was perhaps being overly naive in making similar assumptions to that of other PIM designers
% Although there are opportunities to improve PIM-tool integration, designers embarking on such a programme must carefully 



%%%%%%%%%%%%%%%%%%%%%%%%%%%%%%%%%%%%%%%%%%%%%%%%%%%%%%%%%%%%%%%%%%%%%%%%%%%%%%%%%%%%%%%%%%%%%%%%%%%%%%%%%%%%%%%%%
%%%%%%%%%%%%%%%%%%%%%%%%%%%%%%%%%%%%%%%%%%%%%%%%%%%%%%%%%%%%%%%%%%%%%%%%%%%%%%%%%%%%%%%%%%%%%%%%%%%%%%%%%%%%%%%%%
%%%%%%%%%%%%%%%%%%%%%%%%%%%%%%%%%%%%%%%%%%%%%%%%%%%%%%%%%%%%%%%%%%%%%%%%%%%%%%%%%%%%%%%%%%%%%%%%%%%%%%%%%%%%%%%%%
%%%%%%%%%%%%%%%%%%%%%%%%%%%%%%%%%%%%%%%%%%%%%%%%%%%%%%%%%%%%%%%%%%%%%%%%%%%%%%%%%%%%%%%%%%%%%%%%%%%%%%%%%%%%%%%%%


%%%%%%%%%%%%%%%%%%
% EValuation
%%%%%%%%%%%%%%%%%%
%%%%%%%%%%%%%%%%%%%%%%%%%%%%%%%%%%%%%%%%
% \subsection{Evaluation Component}
%\label{disc:methodological-discussion:evaluation}
%%%%%%%%%%%%%%%%%%%%%%%%%%%%%%%%%%%%%%%%
% This section appraises the evaluation component of the thesis: the field-study based assessment of the WM prototype, as reported in \textbf{Chapter~\ref{chapter:main-study}}.  The naturalistic context was deliberate to ensure evaluation took place under real-world conditions, and in the presence of realistic information needs.  Qualitative feedback was collected through interviews and diaries, and snapshots of folder structures were also taken to track the timing of snapshot events.
%%%%%%%%%%
% RECAP
%%%%%%%%%% 
%Evaluation methods are reviewed as follows.  reported the evaluation of the WM prototype over the course of a longitudinal field study: natural context, feedback through questionnaires, interviews, diary, also objective data.  

% IN GENERAL:RELATE TO EARLIER CH6 DISCUSSION
During evaluation, the author chose to promote the ecological validity of results by employing a field-study method.  This choice lead to several logistical challenges, including the need to install and support the prototype on a number of computers across London.  Furthermore, significant commitment (and some bravery!) was required from the participants. However, the field study approach also had a number of advantages.   Firstly, it gave participants time to try the tool in an unpressured context.  Furthermore, it also provided opportunity for behaviour to emerge as the tool was used, and to assess whether use was long-term.  Also, the fact that participants F3 and M5 both changed their mind regarding the usefulness of WM, illustrates that first impressions as revealed in shorter-term studies may not be correct.  Another retrospective benefit from evaluating over the long-term was that organizing was performed infrequently.  However, it can be argued that the study was not long enough, as several participants performed scant organization, or mostly performed PIM on different computers.  These were all inherent challenges of performing research in the real world.

%%%%%%%%%%%%%%%%%%%%%%%%%%%%%%%%%
% PRO: longitudinal evaluation
%%%%%%%%%%%%%%%%%%%%%%%%%%%%%%%%%
% TIME: Evaluation was over the long-term.  Important as people do not organize very frequently, and that is where the WM design was used.
% SPARSITY: challenge of evaluating in real-world, phenomena of interst may be very sparse, e.g. organizing frequency. Not foreseen.  But lead to insights in other ways.
% Evaluation was carried out over the long-term.

% This turned out to be crucial as participants turned out not to organize very frequently.  The  -- folder-based organizing -- is bursty/sparse.
% Firstly, it depends on the nature of work being performed -- for example, users tend to create more folders if they are starting projects.  Furthermore, the evaluation was centred on one computer, self-selected by the participants as their primary work computer.  







%%%%%%%%%%%%%%%%%%%%%%%%%%%%
% Individual differences
%%%%%%%%%%%%%%%%%%%%%%%%%%%%
% REC: Reporting: consider case studies if a wide user response expected.
Eight participants took part in the longitudinal studies and responded to the tool in eight different ways, highlighting the idiosyncratic nature of PIM.  Despite the small set of participants, the investigation was successful in obtaining a wide range of feedback, some of which was unanticipated (e.g. the promotion of reflection on PIM, as a result of the prompting from WM). % Although some generalizations could be made -- resolving the wide range of different attitudes and usages was highly challenging.  




%%%%%%%%%%%%%%%%%%%%%%%%%%%%
% PROB: user selection
%%%%%%%%%%%%%%%%%%%%%%%%%%%%
% Possible bias is acknowledged but avoided.  
% Argue that an established trust basis leading to straightforward access to personal data outweighed the chance of bias. Ideally in future work would avoid this.
As in the studies, participant selection was somewhat unusual due to the reliance on friends and colleagues. Ideally participants should be randomly selected and not known by the researcher to avoid bias.  In this case, participant selection is defended due to the privacy problems that arise when inspecting users' personal file, email and bookmark collections.  The existing familiarity between participant and researcher lead to an established trust basis, making access to personal data less problematic.  % It is argued that the benefits of high level of trust outweighed the possibility of bias.  Note that participants were not paid.  It is suggested that this may be a useful route to get access to strangers' personal information.

\textbf{Section~\ref{conclusion:future-work}} now moves on to consider possible directions for future work, many of which are driven by the limitations identified in this section.

% so as to build on the findings presented in this thesis.

%%%%%%%%%%%%%%%%%
% challenges - user selection
%%%%%%%%%%%%%%%%%
% One challenge was findings participants with the required combination of tools to assess WM: (Windows 2000/XP and Outlook). The sheer range of PIM-tools available, means that although this was the most common tool configuration observed in the exploratory study, many other otherwise willing participants had to be ruled out.  The eventual participation of several participants was limited by various tool incompatibilities. HOwever, all reported that they fully understood the folder-mirroring functionality.



% The evaluation reported here focused on assessing long-term usefulness and satisfaction, rather than short-term efficiency-based measures. Indeed, it can be argued that the design intervention slowed down the process of creating a folder by inserting an extra demand on the user to decide whether to mirror it or not. % indeed hard to get timing data


%%%%%%%%%%%%%%%%%%%%%%%%%%%%%%%%%%%%%%
% Incremental design - easier eval?
%%%%%%%%%%%%%%%%%%%%%%%%%%%%%%%%%%%%%%
%INCR DESIGN:Made small adjustments to existing tools by performing incremental design -- therefore easier to evaluate. % % Still effin complicated though )-:
%But stil complicated





%%%%%%%%%%%%%%%%
% RECOMMEND
%%%%%%%%%%%%%%%%
% The following recommendations are made, based on lessons learned in the study. What have we learned in evaluating WorkspaceMirror? Can we make evaluation recommendations?:


%%%%%%%%%%%%%
% SUMMARY
%%%%%%%%%%%%%
% \subsection{Summary}



%%%%%%%%%%%%%%%%%%%%%%%%%%%%%%%%%%%%%%%%%%%%%%%%%%%%%%%%%%%%%%%%%%%%%%%%%%%%%%%%%%%%%%%%%%%%%%%%%%%%%%%%%%%%%%%%%
%%%%%%%%%%%%%%%%%%%%%%%%%%%%%%%%%%%%%%%%%%%%%%%%%%%%%%%%%%%%%%%%%%%%%%%%%%%%%%%%%%%%%%%%%%%%%%%%%%%%%%%%%%%%%%%%%
%%%%%%%%%%%%%%%%%%%%%%%%%%%%%%%%%%%%%%%%%%%%%%%%%%%%%%%%%%%%%%%%%%%%%%%%%%%%%%%%%%%%%%%%%%%%%%%%%%%%%%%%%%%%%%%%%
%%%%%%%%%%%%%%%%%%%%%%%%%%%%%%%%%%%%%%%%%%%%%%%%%%%%%%%%%%%%%%%%%%%%%%%%%%%%%%%%%%%%%%%%%%%%%%%%%%%%%%%%%%%%%%%%%

%%%%%%%%%%%%%%%%%%%%%%%%%%%%%%
% FIN@ CHAPTER 7 CRITICAL REFLECTION
%%%%%%%%%%%%%%%%%%%%%%%%%%%%%%
