%%%%%%%%%%%%%
% THE CHANGE
%%%%%%%%%%%%%
% OR CHANGE: Shift towards providing guidance for the developers of PIM-tools, in particular the development of PIM-integration mechanisms.
% This section charts the course of this research.
%\textit{The work represents something of a shift on the part of the author}
%%%%%%%%%%%%%%%%%%%%%%%%%%%%%%%%%%%%%%%%%%%%%%%%%%%%%%%%%%%%%%%%%%%%%%%%%%%%%%%%%%%%%%%%%%%%%
%\begin{itemize}
%\item \textit{Shift of focus on WM as a potential solution to PIM problems, to a partial solution enabling quest for further understanding. WM as a research vehicle. NOT a solution? }
%\item \textit{In addition that PIM-integration is not an entirely positive thing.}
%\item \textit{Re-emphasised to author, that work just focusing on one part of the puzzle. Have only dealt with a small part of the problem.  Admit that I learned a key lesson, primarily the complexity of work in this area.  Speculate about reasons for lack of progress to date and how to deal with them in the future.}
%\end{itemize}
%%%%%%%%%%%%%%%%%%%%%%%%%%%%%%%%%%%%%%%%%%%%%%%%%%%%%%%%%%%%%%%%%%%%%%%%%%%%%%%%%%%%%%%%%%%%%
% Over the course of the thesis, the work shifted away somewhat from a pure focus on PIM-integration.
%In particular, the empirical findings from \textbf{Chapters~\ref{chapter:exploratory_study}} and \textbf{\ref{chapter:main-study}} offer contributions regarding several areas of PIM. These includes investigation of changes in PIM strategy and ongoing user experience.  
% One reason for the shift away from PIM-integration was the sheer complexity of the phenomena being investigated. Although a new PIM-integration design, WorkspaceMirror, was proposed and successfully trialled, the field study evaluation acted as much as a research vehicle to gain more understanding, than to evaluate a definite solution.  



%%%%%%%%%%%%%%%%%%%%%
% REFLECTION ON WM
%%%%%%%%%%%%%%%%%%%%%
%Starting the PhD, the candidate had high hopes of developing a brand new alternative approach to PIM design.  However, he soon realised that this was not going to be possible.  New PIM tools can involve man-centuries of development effort (look at the rate of progress of Microsoft WinFS). 
%incmrental design - users want more!


%%%%%%%%%%%%%%%%%%%
FUTURE STUDY
%%%%%%%%%%%%%%%%%%%

%%%%%%%%%%%%%%%%%
% FOCUS ON ORG: 
%%%%%%%%%%%%%%%%%
% Were all aspects of PIM covered?
As discussed above, PIM is a huge area, and some constraining of research scope was necessary.  As well as the focus in terms of PIM-tools, some focus was implicit in the choice of research methods employed.  In particular, the form of data collected (snapshots of folder structures) lead to a focus on the organizing PIM sub-activity .  The ``ideal study'' laid out in \textbf{Section~\ref{conclusion:future-work}} identifies routes for collecting more data on other PIM sub-activities (such as retrieval) through improved instrumentation and more controlled studies. Currently data on other PIM sub-activities consisted of the qualitative feedback from the participants.

\textbf{Section~\ref{conclusion:future-work}} discusses an ideal long-term study method without any such intervention.  This however raises privacy issues -- one positive aspect of the design intervention was that it reminded the participants that they were being watched.
This however raises privacy issues -- one positive aspect of the design intervention was that it reminded the participants that they were being watched.

%%%%%%%%%%%%%%
% PIM EVALUATION IS TRICKY
%%%%%%%%%%%%%%
% PIM is irrational, idiosyncratic, discretionary, supporting -- so not necessarily a matter of efficiency measures. 
The challenge of evaluating high-level, discretionary activities such as PIM have been identified by several researchers, in particular the difficulty in defining appropriate usability measures~\citep{ad:01}.  
%%%%%%%%%%%%%%%%%%%%%%%%
% Not simple usability measure
%%%%%%%%%%%%%%%%%%%%%%%%
% MEASURES: Pre-defined evaluation measures were avoided. 
% Users may have different measures of success. Users were allowed to define their own measures of usability. Several unanticipated responses, and benefits (e.g. promotion of reflection). EXPLORATORY
A key aim of the study was to explore appropriate evaluation criteria, based on participant feedback.  Some progress was made in this area, including the exploration of criteria based on \textit{user experience-related measures}~\cite{Friedman:03} such as satisfaction in management strategies, and tidiness.

