%%%%%%%%%%%%%%%%%%%%%%%%%%%%
% CHAPTER 8 - CONCLUSION - Contributions
%%%%%%%%%%%%%%%%%%%%%%%%%%%%
%%%%%%%%%%%%%%%%%%%%%%%%%%%%%%%%%%%%%%%%%%%%%%%%%%%%%%%%%%%%%%%%%%%%%%%%%%%%%%%%%%%%%%%%%%%%%
% Richard Boardman PhD Thesis: Improving Tool Support for Personal Information Management
%%%%%%%%%%%%%%%%%%%%%%%%%%%%%%%%%%%%%%%%%%%%%%%%%%%%%%%%%%%%%%%%%%%%%%%%%%%%%%%%%%%%%%%%%%%%%

\newpage
%%%%%%%%%%%%%%%%%%%%%%%%%%%%%%
%%%%%%%%%%%%%%%%%%%%%%%%%%
\section{Contributions}
\label{conclusion:contributions}
%%%%%%%%%%%%%%%%%%%%%%%%%%
%%%%%%%%%%%%%%%%%%%%%%%%%%%%%%
%	\item Contributions (substantive and methodological)
% Set out contributions of this chapter towards overall thesis -- consider achievements in light of objectives.
%%%%%%%%%%%%%%%%%%%%%%%%%%%%%%%%%%%%%%%%%%%%%%%%%%%%%%%%%%%%%%%%%%%%%%%
%\item Consider achievements in light of objectives
%\item to sum up/tie together/pull together conclusions and contributions from this work
% \item \textit{THINK: ensure contributions here match with those in introduction}
%%%%%%%%%%%%%%%%%%%%%%%%%%%%%%%%%%%%%%%%%%%%%%%%%%%%%%%%%%%%%%%%%%%%%%%
%\item Relate to aims. Do the contributions of the thesis meet those aims? Relate progress/findings towards tackling aims (this is a repeat of the contributions, no?) Have I tackled it? No, not fully - but hopefully taken some steps towards doing so ... (-; Honest, realistic, assessment of my contributions
%%%%%%%%%%%%%%%%%%%%%%%%%%%%%%%%%%%%%%%%%%%%%%%%%%%%%%%%%%%%%%%%%%%%%%%
%\item expand on how they contribute to the HCI knowledge base and why they are valid
%%%%%%%%%%%%%%%%%%%%%%%%%%%%%%%%%%%%%%%%%%%%%%%%%%%%%%%%%%%%%%%%%%%%%%%
%\item What are the implications of the now validated thesis?
%%%%%%%%%%%%%%%%%%%%%%%%%%%%%%%%%%%%%%%%%%%%%%%%%%%%%%%%%%%%%%%%%%%%%%%
% This section reviews the contributions made over the preceding chapters.
Firstly, key contributions are highlighted in \textbf{Section~\ref{conclusion:contributions:key}}. The next two sections detail all the contributions made in this thesis.  \textbf{Section~\ref{conclusion:contributions:understanding}} presents contributions resulting from the study components of the thesis.  Then, \textbf{Section~\ref{conclusion:contributions:design}} discusses contributions resulting from the design, implementation and evaluation of the WorkspaceMirror prototype.

%%%%%%%%%%%%%%%%%%%%%%%%%%%%%%%%%%%%%%%%%%%%%
\subsection{Summary of Key Contributions}
\label{conclusion:contributions:key}
%%%%%%%%%%%%%%%%%%%%%%%%%%%%%%%%%%%%%%%%%%%%%
This thesis offers a variety of contributions.  Here the most significant contributions are highlighted.

The study components of the thesis offer a significant advance over previous work through the investigation of PIM (1) across multiple tools, and (2) over time, thus enabling the development of richer understanding. \textbf{Chapter~\ref{chapter:exploratory_study}} compared PIM behaviour across file, email and bookmark collections, and developed \textit{improved classifications of management strategies}. In particular, it was shown how many users employ multiple organizing strategies, both within and between distinct PIM-tools. In turn,  \textbf{Chapter~\ref{chapter:main-study}} collected data over time, leading to the \textit{development of an improved model of changes in organizing strategies}. The knowledge cultivated in the two studies lead in \textbf{Chapter~\ref{chapter:discussion}} to \textit{the enhancement of Barreau's PIM model to accommodate the cross-tool, supporting and ongoing nature of PIM}.

Other contributions result from the design component of the thesis.  The key contributions in this area include the \textit{design of the WorkspaceMirror prototype}, in \textbf{Chapter~\ref{chapter:design}}, driven by observations of folder overlap in  \textbf{Chapter~\ref{chapter:exploratory_study}}. The \textit{evaluation of the design} in \textbf{Chapter~\ref{chapter:main-study}} confirmed the promise of mirroring top-level folders. This represents one of the few systematic, longitudinal evaluations of a PIM-tool performed to date. The evaluation resulted in the \textit{development of a series of design recommendations}, primarily the importance of recognizing that a PIM-integration mechanism may have strengths and weaknesses in different tool contexts. In addition, a number of \textit{methodological recommendations} were developed including the need to evaluate all PIM designs in both tool-specific and cross-tool perspectives.

The next two sections provide more detail on the contributions from the study and design/evaluation components of the thesis.

% \textbf{Table~\ref{table:conclusion:contributions}} presents an overview of the contributions.
%%%%%%%%%%%%%%%%%%%%%%%%%%%%
% Contribution categories
%%%%%%%%%%%%%%%%%%%%%%%%%%%%
%THINK: two alternatives structures here: substantive versus methodological or divide up in terms of the main offerings:
%Firstly, \textbf{Section~\ref{conclusion:substantive-contributions}} details the substantive contributions, and then  \textbf{Section~\ref{conclusion:methodological-contributions}} details the methodological contributions.
% \item Main contributions, divide as substantive/methodological, major/minor, and audience (researchers/designers):
%The contributions are discussed in terms of the following two categories:
%\begin{enumerate}
% \item Theory/model building to describe/explain subsets of the empirical findings.
%\item Improving understanding of PIM (see )
%\item Design, implementation and evaluation of a novel PIM-integration mechanism (see )
%\end{enumerate}

% The following sections focus on each of these categories in turn.



%%%%%%%%%%%%%%%%%%%%%%%%%%%%%%%%%%%%%%%%%%%%%%%%%%%%%%%%%%%%%%%%%%%%%%%%%%%%%%%%%%%%%%%%%%%%%%%%%%%%%%%%%%%%%%%%%
%%%%%%%%%%%%%%%%%%%%%%%%%%%%%%%%%%%%%%%%%%%%%%%%%%%%%%%%%%%%%%%%%%%%%%%%%%%%%%%%%%%%%%%%%%%%%%%%%%%%%%%%%%%%%%%%%
%%%%%%%%%%%%%%%%%%%%%%%%%%%%%%%%%%%%%%%%%%%%%%%%%%%%%%%%%%%%%%%%%%%%%%%%%%%%%%%%%%%%%%%%%%%%%%%%%%%%%%%%%%%%%%%%%
%%%%%%%%%%%%%%%%%%%%%%%%%%%%%%%%%%%%%%%%%%%%%%%%%%%%%%%%%%%%%%%%%%%%%%%%%%%%%%%%%%%%%%%%%%%%%%%%%%%%%%%%%%%%%%%%%

%%%%%%%%%%%%%%%%%%%%%%%%%%%%%%%%%%%%%%%%%%%%%
\subsection{Improved Knowledge of PIM}
\label{conclusion:contributions:understanding}
%%%%%%%%%%%%%%%%%%%%%%%%%%%%%%%%%%%%%%%%%%%%%
%Theory/models to explain/describe PIM
%\begin{itemize}
%\item Extension of Barreau's model to produce framework encompassing cross-tool, supporting, ongoing aspects of PIM   
%\item Model of multiple strategies in TS/CT contexts
%\item Model of changes in PIM strategies (applied to explaining so rate of take-up of many PIM-tools)
%\item Model of user experience.
%\end{itemize}

This section details contributions that relate to the first aim of the thesis: to develop increased understanding of PIM behaviour.  \textbf{Table~\ref{table:conclusion:contributions:understanding}} provides a summary of the empirical, methodological and theoretical contributions in this area.

%%%%%%%%%%%%%%%%%%%%%%%%%%%%%%%%%%%%%%%%%%%%%%%%%%%%%%
% TABLE: contributions providing increased understanding of PIM
%%%%%%%%%%%%%%%%%%%%%%%%%%%%%%%%%%%%%%%%%%%%%%%%%%%%%%
\begin{table}[hbtp]
\begin{center}
\begin{footnotesize}
\setlength{\extrarowheight}{2pt}
\begin{tabular}{|c|p{9cm}|p{3cm}|}
\hline
{\bf Chapter} & {\bf Contribution} & {\bf Type of contribution} \\
\hline
         2 & Definitions and PIM conceptual framework & theoretical \\
\hline
         3 & Critical review of previous work & theoretical \\
\hline
         4 & Comparison of PIM behaviour between file, email and bookmark collections & empirical findings \\
\hline
         4 & New classifications of organizing strategies in file, email and bookmark collections & empirical findings \\
\hline
         4 & Comparison of organizing strategies between files, email and bookmarks & empirical findings \\
\hline
         4 & Development and application of method to analyse folder structures in terms of \textit{organizational dimensions} & methodological, empirical findings \\
\hline
         4 & Development and application of method to compare folder structures in terms of \textit{folder overlap} & methodological, empirical findings \\
\hline
         4 & Model of multiple PIM strategies within tool-specific and cross-tool contexts & theoretical model \\
\hline
         6 & Insights into longitudinal PIM behaviour: (1) PIM strategy changes, and (2) supporting nature of PIM & empirical findings \\
\hline
         6 & Model of incremental changes in organizing strategy & theoretical \\
\hline
         7 & Extended conceptual framework of PIM illustrating its cross-tool, supporting, ongoing nature & theoretical \\
\hline
         7 & Discussion of qualitative measures of PIM user experience, including ``satisfaction with current strategies'' & theoretical \\
\hline
\end{tabular}  
\end{footnotesize}
\caption{Contributions providing increased understanding of PIM}
\label{table:conclusion:contributions:understanding}
\end{center}
\end{table}
\normalsize

Two studies of PIM behaviour were reported.  Firstly, \textbf{Chapter~\ref{chapter:exploratory_study}} reported an exploratory study in which a semi-structured interview methodology was employed to investigate PIM behaviour across three tools: files, email and bookmarks.   Secondly, \textbf{Chapter~\ref{chapter:main-study}} reported a follow-up longitudinal field study, again across the three tools. As well as investigating PIM behaviour over time, the field study also acted as a research vehicle to evaluate the WorkspaceMirror prototype.  Both studies are distinguished from previous work by their cross-tool nature.

% \subsubsection{Contributions from Chapter 4}
% \item The primary aim of the study was to investigate PIM behaviour between the three tools to develop an empirical foundation for the design of PIM-integration mechanisms).  
% \item Tool-specific insights, e.g. strategy classifications
% in both \textit{tool-specific} and \textit{cross-tool} contexts.  
The primary contribution from \textbf{Chapter~\ref{chapter:exploratory_study}} was \textit{a comparison of PIM-behaviour between files, email and bookmarks} in the terms of four PIM sub-activities: acquisition, organization, maintenance, and retrieval. A number of similarities were noted between the tools including: (1) a preference for browsing-based retrieval over search in all three tool contexts, and (2) the importance of older items for many participants\footnote{These contributions, although minor, add to existing HCI knowledge in each tool-specific context, as well as contributing new cross-tool knowledge.}.  However, some major differences in behaviour were also noted.  These included: (1) the relative unimportance of the bookmark collection, and (2) the uncontrolled nature of acquisition in email.

The rest of \textbf{Chapter~\ref{chapter:exploratory_study}} focused on the organizing sub-activity.  The next contribution was \textit{a comparison of each individuals' organizing strategies across the three tools}. Previous strategy analysis has been limited to specific tools such as email and this is the first attempt to analyse strategies from a cross-tool perspective.  The comparison of organizing strategies between files, email and bookmarks firstly necessitated the characterization of strategies in individual tools. Multiple low-level organizing strategies were observed for many participants in all three tool contexts. Three \textit{new tool-specific strategy classifications} were offered to reflect this behaviour.  Previous classifications were criticised for not capturing this level of detail.  Then, at the cross-tool level, \textit{multiple organizing strategies} were observed for many individuals, as in specific tools.  This confirmed that users do not employ consistent strategies across different collections of personal information.  A number of causal factors were identified including the relative value of the information in each collection -- since files are valued most highly, and are most likely to be retrieved, they are seen to be most worth organizing.  \textit{A theoretical model of multiple strategies} was developed to describe the tool-specific and cross-tool observations of multiple strategies. % , and improve conceptualization of the term ``PIM strategy''. % Previous classifications of user behaviour were criticised for only providing abstractions of actual behaviour. 

% However, a number of \textit{tool-specific contributions} were also made.   % Also, the description of users as ``spring cleaners'' was shown to be inaccurate, as this strategy was only observed in combination with other strategies.
% Study Results. Findings from combined Cross-tool Studies- theoretical insight into what PIM is. Provide further understanding of PIM from a holistic "cross-tool" perspective. 
% , and perhaps the most significant, was the comparison of PIM behaviour between the three tools, incorporating all the data collected above.
%\item 
%\end{itemize}

% Finally, a method of \textit{cross-tool strategy profiling} was proposed and employed to correlate organizing behaviour between the three tools. Previous strategy analysis has been limited to specific tools such as email and this is the first attempt to analyse strategies form a cross-tool perspective.  Again it was seen that many users employ \textit{multiple strategies} in terms of organizing behaviour, and a number of causal factors were identified.
% The next three contributions from \textbf{Chapter~\ref{chapter:exploratory_study}} focus on comparing organizing behaviour between files, email and bookmarks:

%\begin{itemize}
%
%\item 
\textbf{Chapter~\ref{chapter:exploratory_study}} also reported the development and application of two new techniques for comparing organizing behaviour between collections:

\begin{itemize}

\item \textit{A technique for analysing folder structures in terms of the organizational dimensions} -- An analysis of the aggregated data characterized the most common dimensions in each tool context.  Although projects and roles were the most common dimensions observed, a wide range of organizational dimensions were employed.  This lead to the criticism of prototypes such as the \textit{Personal Role Manager}~\citep{Shneiderman:94}, which enforce a dominant organizational dimension. This data suggests that such designs would require users to significantly change their current behaviour.

\item \textit{A technique for assessing the level of \textit{folder overlap} between a participant's file, email and bookmark structures} --  Significant overlap was observed for many participants, particularly between the file and email folder structures.  This observation provided the key empirical motivation for the development of the WM prototype in \textbf{Chapter~\ref{chapter:design}}.  % itemize relating to the design and evaluation of WM are discussed in \textbf{Section~\ref{conclusion:contributions:design}}.

\end{itemize}

%%%%%%%%%%%%%%%%%%%%%%%%%%%%%%%%%%%%%%%%%%%%
% \subsubsection{Contributions from Chapter 6}
%%%%%%%%%%%%%%%%%%%%%%%%%%%%%%%%%%%%%%%%%%%%

% Changes in PIM	%%%%%%%%%%%%%%%%%%%%%%%%%%%%%%%%
% Two main empirical contributions from CHAPTER 6
%%%%%%%%%%%%%%%%%%%%%%%%%%%%%%%%%%%%%%%%%%%%%%%%%%%
The field study in \textbf{Chapter~\ref{chapter:main-study}} offered two further empirical contributions:

\begin{itemize}

\item \textit{The observation of the incremental nature of changes in organizing strategy} -- Tracking PIM behaviour over time revealed that few study participants made significant changes in how they organized information in any of the collections.  Only two participants were observed making strategy changes, which were of a subtle, incremental nature.  Previous theory in the area was criticised for portraying strategy changes in terms of global swings in overall organizing behaviour~\citep{ob:97}.  This data was combined with the model of multiple strategies to develop \textit{a model of changes in organizing strategy} which reflected their incremental nature.  

\item \textit{The identification of the supporting, background nature of PIM} -- This was based on participants' comments that they devoted relatively little attention to PIM compared to their work tasks.  Both the study and design interventions caused many participants to report increased reflection on their PIM strategies.

\end{itemize}

%%%%%%%%%%%%%%%%%%%%%%%%%%%%%%%%%%%%%%%%%%%%
% \subsubsection{Contributions from Chapter 7}
%%%%%%%%%%%%%%%%%%%%%%%%%%%%%%%%%%%%%%%%%%%%

The discussion presented in \textbf{Chapter~\ref{chapter:discussion}} lead to two final contributions in this area:

\begin{itemize}

\item \textit{An extended theoretical framework describing the cross-tool, supporting, and ongoing nature of PIM} --  This framework offers several advances over previous theory in the area, including: (1) a conceptualization of PIM-integration mechanisms which bridge between distinct PIM-tools, and (2) the representation of the relationship between PIM and the production activities it supports.  The framework was used to identify routes for theoretical development in the area.

\item \textit{Discussion of PIM user experience} -- A number of qualitative measures of PIM user experience were offered as alternatives to traditional performance-focused evaluation metrics, which it was argued are inappropriate for activities such as PIM.  A focus was taken on ``dissatisfaction in strategies'' as a key source of poor user experience. Several participants expressed dissatisfaction with their current strategies, but did not change them because of the perceived costs involved.  This data was used to illustrate the forms that such dissatisfaction could take.  

% This and a number of other   % A model of the relationship between user experience and strategy changes was proposed and lead to the identification of a number of \textit{user-experience based evaluation criteria}.

\end{itemize}

%%%%%%%%%%%%%%%%%%%%%%%%%%%%%%%%%%%%%%%%%%%%%%%%%%%%%%%%%%%%%%%%%%%%%%%%%%%%%%%%%%%%%%%%%%%%%%%%%%%%%%%%%%%%%%%%%
%%%%%%%%%%%%%%%%%%%%%%%%%%%%%%%%%%%%%%%%%%%%%%%%%%%%%%%%%%%%%%%%%%%%%%%%%%%%%%%%%%%%%%%%%%%%%%%%%%%%%%%%%%%%%%%%%
%%%%%%%%%%%%%%%%%%%%%%%%%%%%%%%%%%%%%%%%%%%%%%%%%%%%%%%%%%%%%%%%%%%%%%%%%%%%%%%%%%%%%%%%%%%%%%%%%%%%%%%%%%%%%%%%%
%%%%%%%%%%%%%%%%%%%%%%%%%%%%%%%%%%%%%%%%%%%%%%%%%%%%%%%%%%%%%%%%%%%%%%%%%%%%%%%%%%%%%%%%%%%%%%%%%%%%%%%%%%%%%%%%%

%%%%%%%%%%%%%%%%%%%%%%%%%%%%%%%%%%%%%%%%%%%%%
\subsection{Design, Implementation and Evaluation}
\label{conclusion:contributions:design}
%%%%%%%%%%%%%%%%%%%%%%%%%%%%%%%%%%%%%%%%%%%%%

% Grounded design and implementation of WorkspaceMirror as research vehicle to investigate issues involved in improving integration between PIM-tools. 
% Cross-tool design of WorkspaceMirror and its implementation. 
% Knowledge in artifact implementation. Explicitation of knowledge as claims? Other designs?
The second main area of contribution resulted from the design, implementation, and evaluation of the WorkspaceMirror prototype (abbreviated to WM).  \textbf{Table~\ref{table:conclusion:contributions:design}} provides an overview.


%%%%%%%%%%%%%%%%%%%%%%%%%%%%%%%%%%%%%%%%%%%%%%%%%%%%%%
% TABLE: design/implementation contributions
%%%%%%%%%%%%%%%%%%%%%%%%%%%%%%%%%%%%%%%%%%%%%%%%%%%%%%
\begin{table}[hbtp]
\begin{center}
\begin{footnotesize}
\setlength{\extrarowheight}{2pt}
% Table generated by Excel2LaTeX from sheet 'DESIGN-EVAL'
\begin{tabular}{|c|p{9cm}|p{3cm}|}
\hline
{\bf Chapter} & {\bf Contribution} & {\bf Type of contribution} \\
\hline
         5 & Design and implementation of WorkspaceMirror (WM) & design and implementation \\
\hline
         5 & Results from the initial evaluation of WM & empirical findings \\
\hline
         6 & Results from the field-study longitudinal evaluation of WM & empirical findings \\
\hline
      7 & Design implications from the WM evaluation: (1) the trade-off between organizational consistency and organizational flexibility, and (2) the pros and cons of integration mechanisms & design guidelines \\
\hline
         7 & A series of methodological recommendations to guide the design and evaluation of PIM-tools based on the cross-tool, ongoing, and supporting aspects of PIM & methodological recommendations \\
\hline
\end{tabular}  
\end{footnotesize}
\caption{Contributions relating to the design of WorkspaceMirror}
\label{table:conclusion:contributions:design}
\end{center}
\end{table}
\normalsize

The first contribution is the WM prototype itself, as described in \textbf{Chapter~\ref{chapter:design}}. WM is offered as \textit{a novel, empirically-grounded form of PIM-integration}.  It enables the user to mirror changes made to folder structures between three tools: files, email and bookmarks.  In contrast to much of the technological prototyping carried out in the area, WM is an example of \textit{incremental} design, and offered as a case-study of the benefits of this approach.  As well as suiting the limited development resources available, this approach enabled straight-forward incorporation of the prototype onto user's computers with minimum disruption.  %The case study illustrates how modest design ambitions can help facilitate effective evaluation, and reveal a range of rich data.
The initial evaluation of WM, reported in \textbf{Section~\ref{design:feasibility-study}}, resulted in a number of design improvements before more extensive evaluation was performed. These included setting WM to operate in prompted mode by default so that users can retain control over mirroring.

The follow-up field study, reported in \textbf{Chapter~\ref{chapter:main-study}}, resulted in two further contributions: (1) an assessment of WM as a specific form of PIM-integration, and (2) a number of design implications for designers working in this area:

\begin{itemize}

\item \textit{Results from the evaluation of WM} -- The evaluation of WM represents one of the few that have been carried out of a PIM-integration mechanism.  A wide range of behaviour was observed, illustrating the idiosyncratic nature of PIM.  Although participants welcomed the ability to mirror entire folder structures, most participants indicated that folder mirroring was appropriate for top-level folders only. For the majority of participants, their requirements for organizational flexibility between tools, outweighed the benefits of organizational consistency offered by WM.  Therefore, the study suggests that the potential to share folder structures between PIM-tools is only partial.  
The study was highly successful in obtaining a wide range of suggestions for improving the design. The most common request was to make mirroring more selective, e.g. limiting it to top-level folders only.  \textbf{Section~\ref{discussion:design-guidelines-discussion}} offered a theoretical explanation of why only certain folders were worth mirroring based on the concept of tool-specific and cross-tool production activities.

% it was successful in terms of raising design issues for the development of guidelines.  % The specific design evaluation may also be useful for designers of similar mechanisms.,

% Design recommendations that can be generalized.
\item \textit{A series of design guidelines based on the WM evaluation} -- Firstly, the designers of mechanisms which offer integration of organizing functionality were urged to pay attention to the trade-off between organizational consistency and organizational flexibility.  This was generalized to an implication for the wider PIM-integration design genre: that any PIM-integration design will have both pros and cons.  Two examples were provided: WM and email attachments.  Furthermore, the positive and negative effects of integration may be distributed across tools, and over time, emphasising the need for careful evaluation.  % \textbf{Section~\ref{discussion:design-guidelines-discussion}} closed by offering a number of potential routes for PIM-integration.


\end{itemize}

% My Cross-tool Research Perspective/Analysis/Model of PIM. Provides theoretical framework for analysing user activity, user needs, discussing design, evaluation/study methodologies and results. Analysis of use of perspective.
% Operationalization as Cross-tool Methodology. Analysis of use/validity of cross-tool method.
%Results from longitudinal evaluation of WM through field study.  One of few examples of this performed to date. Findings from combined Cross-tool Study/Evaluation - formative evaluation of WorkspaceMirror.
%\begin{itemize}
%	\item Case-study of how to do it? Implications for future work (see methodology below)
%\end{itemize}
% Thesis as a case study in incremental design. \textbf{Chapter~\ref{chapter:discussion}} compared and contrasted with previous work.
The final contribution was presented in \textbf{Section~\ref{discussion:methodological-discussion}}: \textit{a set of methodological recommendations for the design and evaluation of PIM-tools}.  These were structured in terms of the extended theoretical framework developed in \textbf{Section~\ref{discussion:theoretical-framework}}.  Firstly, it was argued that PIM-tool designers should consider cross-tool issues during requirements gathering, design, and evaluation.  Cross-tool scenarios were recommended as a potential technique to achieve this.  Examples were provided to illustrate the importance of such a cross-tool perspective in both tool-specific and cross-tool design contexts. The importance of evaluating PIM designs over the long-term was also highlighted, as illustrated by the main study participants who changed their opinion towards WM over time.  Finally, \textbf{Section~\ref{discussion:methodological-discussion}} highlighted the dilemma of designing for supporting activities such as PIM: encouraging users to spend more time on PIM, may result in them spending less time on their production activities.  A long-term PIM-design goal was suggested: that tools should help users achieve a balance between PIM and their production activities.  %  were also made in \textbf{Chapter~\ref{chapter:discussion}} for conceptualizing, designing and evaluating PIM-integration mechanisms.





%%%%%%%%%%%%%%%%%%%%%%%%%%%%%%%%%%%%%%%%%%%%%%%%%%%%%%%%%%%%%%%%%%%%%%%%%%%%%%%%%%%%%%%%%%%%%%%%%%%%%%%%%%%%%%%%%
%%%%%%%%%%%%%%%%%%%%%%%%%%%%%%%%%%%%%%%%%%%%%%%%%%%%%%%%%%%%%%%%%%%%%%%%%%%%%%%%%%%%%%%%%%%%%%%%%%%%%%%%%%%%%%%%%
%%%%%%%%%%%%%%%%%%%%%%%%%%%%%%%%%%%%%%%%%%%%%%%%%%%%%%%%%%%%%%%%%%%%%%%%%%%%%%%%%%%%%%%%%%%%%%%%%%%%%%%%%%%%%%%%%
%%%%%%%%%%%%%%%%%%%%%%%%%%%%%%%%%%%%%%%%%%%%%%%%%%%%%%%%%%%%%%%%%%%%%%%%%%%%%%%%%%%%%%%%%%%%%%%%%%%%%%%%%%%%%%%%%

%%%%%%%%%%%%%%%%%%%%%%%%%%%%%%%%%%%%%%%%%%%%%
% \subsection{Theory/model building}
%%%%%%%%%%%%%%%%%%%%%%%%%%%%%%%%%%%%%%%%%%%%%





%%%%%%%%%%%%%%%%%%%%%%%%%%%%%%
%%%%%%%%%%%%%%%%%%%%%%%%%%
%\subsection{Methodological contributions}
%\label{conclusion:methodological-contributions}
%%%%%%%%%%%%%%%%%%%%%%%%%%
%%%%%%%%%%%%%%%%%%%%%%%%%%%%%%

% Stress building on lack of previous work! Opened up new routes for research.
% Recs for future study, design and evaluation as laid out in \textbf{Chapter~\ref{chapter:discussion}}.


%%%%%%%%%%%%%%%%%%%%%%%%%%%%%%%%%%%%%%%%%%%%%%%%%%%%%%%%%%%%%%%%%%%%%%%%%%%%%%%%%%%%%%%%%%%%%%%%%%%%%%%%%%%%%%%%%
%%%%%%%%%%%%%%%%%%%%%%%%%%%%%%%%%%%%%%%%%%%%%%%%%%%%%%%%%%%%%%%%%%%%%%%%%%%%%%%%%%%%%%%%%%%%%%%%%%%%%%%%%%%%%%%%%
%%%%%%%%%%%%%%%%%%%%%%%%%%%%%%%%%%%%%%%%%%%%%%%%%%%%%%%%%%%%%%%%%%%%%%%%%%%%%%%%%%%%%%%%%%%%%%%%%%%%%%%%%%%%%%%%%
%%%%%%%%%%%%%%%%%%%%%%%%%%%%%%%%%%%%%%%%%%%%%%%%%%%%%%%%%%%%%%%%%%%%%%%%%%%%%%%%%%%%%%%%%%%%%%%%%%%%%%%%%%%%%%%%%