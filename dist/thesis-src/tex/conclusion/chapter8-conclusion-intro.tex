%%%%%%%%%%%%%%%%%%%%%%%%%%%%
% CHAPTER 8 - CONCLUSION - intro / Research problem and approach
%%%%%%%%%%%%%%%%%%%%%%%%%%%%
%%%%%%%%%%%%%%%%%%%%%%%%%%%%%%%%%%%%%%%%%%%%%%%%%%%%%%%%%%%%%%%%%%%%%%%%%%%%%%%%%%%%%%%%%%%%%
% Richard Boardman PhD Thesis: Improving Tool Support for Personal Information Management
%%%%%%%%%%%%%%%%%%%%%%%%%%%%%%%%%%%%%%%%%%%%%%%%%%%%%%%%%%%%%%%%%%%%%%%%%%%%%%%%%%%%%%%%%%%%%
%%%%%%%%%%%%%%%%%%%%%%%%%%%%%%
%Chapter overview:
%%%%%%%%%%%%%%%%%%%%%%%%%%%%%%
%\begin{enumerate}
%	\item Overall Conclusions
%	\item Review of thesis
%	\item Critical analysis of work (here or discussion?)
%	\item Research Vistas for future work
%	\item Closing words
%\end{enumerate}
%%%%%%%%%%%%%%%%%%%%%%%%%%%%%%%%%%%%%%%%%%%%%%%%%%%%%%%%%%%%%%%%%%%%%%%%%%%%%%%%%%%%%%%%%%%%%
% \textit{NEED: good discussion of EXPLORATORY-ness of the work}
%%%%%%%%%%%%%%%%%%%%%%%%%%%%%%%%%%%%%%%%%%%%%%%%%%%%%%%%%%%%%%%%%%%%%%%%%%%%%%%%%%%%%%%%%%%%%
%What has PhD done towards easing poor epistemic state of PIM/HCI? Modest: hope thaht some useful progress
%%%%%%%%%%%%%%%%%%%%%%%%%%%%%%%%%%%%%%%%%%%%%%%%%%%%%%%%%%%%%%%%%%%%%%%%%%%%%%%%%%%%%%%%%%%%%
%Relate key high-level implications ... key message, pros and cons of integration
%%%%%%%%%%%%%%%%%%%%%%%%%%%%%%%%%%%%%%%%%%%%%%%%%%%%%%%%%%%%%%%%%%%%%%%%%%%%%%%%%%%%%%%%%%%%%
%Future: is PIM as an activity that will fade away a la bookmark management (when we have a google on the desktop)? High level of interest says that it isn't; Bring work up to date: Special interest group co-organized by the author, Signs of increased research interest? -- KFTF, Bergman etc.~\cite{Bergman:03}, RTA (back to basics), number of papers in CHI04, Signs of developer interest, both open-source (Chandler), and commercial (Longhorn)
%%%%%%%%%%%%%%%%%%%%%%%%%%%%%%%%%%%%%%%%%%%%%%%%%%%%%%%%%%%%%%%%%%%%%%%%%%%%%%%%%%%%%%%%%%%%%

%%%%%%%%%%%%%%%%%%
%%%%%%%%%%%%%%%
% LEAD-IN INTRO
%%%%%%%%%%%%%%%
%%%%%%%%%%%%%%%%%%
% This chapter summarises the work presented over the previous seven chapters. %\footnote{\textit{This draft of Chapter 8 CONCLUSION was printed \today}}.
\textbf{Section~\ref{conclusion:recap}} reviews the aims and methodology of the research,  \textbf{Section~\ref{conclusion:contributions}} discusses the contributions offered over the previous seven chapters, and \textbf{Section~\ref{conclusion:critical-reflection}} offers a critical reflection of the thesis.  Finally, \textbf{Section~\ref{conclusion:future-work}} considers future work possibilities.




%%%%%%%%%%%%%%%%%%%%%%%%%%%%%%%%%%%%%%%%
\section{Revisiting the Research Problem and Approach}
\label{conclusion:recap}
%%%%%%%%%%%%%%%%%%%%%%%%%%%%%%%%%%%%%%%%

%%%%%%%%%%%%%%%
% RW PROBLEM
%%%%%%%%%%%%%%%
% Problems in the real world.
% Restate the real-world problem: users have problems managing their information.  
This research has been aimed at improving the HCI knowledge base for the design of the next generation of PIM-tools.  Today's computer users encounter a wide range of problems in managing information, and consequently there is a need to develop improved interfaces to better support this everyday activity.
% ongoing interest in developing improved interfaces.  %The potential of such design is to improve user's productivity and satisfaction in this fundamental aspect of computer-based activity.
%%%%%%%%%%%%%%%%%%%%
% RESEARCH PROBLEM
%%%%%%%%%%%%%%%%%%%%
% However the research that has been carried out in this area has been limited. In particular there has been no systematic investigation of user needs across all three tools, and the potential to unify the tools.  
% As well as a lack of empirical groundwork regarding many aspects of PIM, there has been a particular lack of systematic research in the area of PIM-integration.  
% Thus there is little available guidance for the designers of such tools.  
% There is also a lack of evaluation metrics 
% One possible contributing factor in the lack of evaluation is the scarcity of evaluation methodology appropriate for this type of tool.
The research focused on one specific area of ongoing design interest, that of improving integration between PIM-tools.  As discussed in \textbf{Chapter~\ref{chapter:review}}, previous research relating to this area has been limited.  Although many studies of PIM behaviour have been carried out, few have considered user needs beyond the boundaries of specific tools such as email.  Therefore, there is a lack of empirical foundation for \textit{cross-tool} design work aimed at improving PIM integration.  Consequently, much of the design work in this area has been technologically motivated rather than grounded in user requirements.  However, many of the innovative prototypes that offer new forms of integration have not been evaluated.  Since designers' claims have not been empirically validated, they offer little research value beyond indicating possible routes for design.  % A scarcity of metrics for assessing PIM designs may be one factor contributing to the lack of evaluation.  

%%%%%%%%%%%%%%%%%%%%%%%
% AIM OF THE THESIS
%%%%%%%%%%%%%%%%%%%%%%%
% \textit{Restate the aims and approach of the thesis.} This thesis has sought to investigate how the design of PIM-tools can be improved. Explored potential to unify PIM (must take care with XYZ). Potential of cross-tool approach/methods.
% SET OUT AIMS AS PRE-DEFINED: This thesis was aimed as follows:
\newpage
After assessing the state of prior research in the area, the objectives of this research programme were defined as follows:
\begin{enumerate}

%%%%%%%%%%%%%%%%%%%%
% UNDERSTANDING
%%%%%%%%%%%%%%%%%%%%
% \item To improve understanding of PIM through user studies, to provide empirical guidance for the designers of integration mechanisms.  \textbf{Chapter~\ref{chapter:review}} argued that cross-tool investigation of PIM behaviour was required to provide a thorough empirical foundation for the design of PIM integration mechanisms.
\item \textit{To develop increased understanding of PIM behaviour and needs} -- In particular, the researcher set out to investigate behaviour across multiple PIM-tools, to improve the empirical foundation for the design of PIM-integration mechanisms.   A secondary aim was to develop theoretical models to describe and explain empirical observations.

%%%%%%%%%%%%%%%
% TOOL DESIGN
%%%%%%%%%%%%%%%
% To develop and evaluate a novel PIM-integration mechanism, grounded in empirical motivation.
\item \textit{To design, implement and evaluate a novel PIM-integration mechanism} -- One of the author's key motivations was to develop software to alleviate user problems.  A further intention was to avoid the methodological limitations of previous work by emphasising empirical grounding and evaluation.

 %%%%%%%%%%%%%%%%%
% METHODOLOGICAL
%%%%%%%%%%%%%%%%%
% Methodological: Explore potential of cross-tool approach for both researchers and designers of cross-tool approach. Evaluation: explore ways of doing effective evaluation.  To explore issues related to the evaluation of PIM designs, particularly those directed at improving PIM integration.
% regarding appropriate methodology
% devise appropriate methodology to support the above objectives of investigating PIM, and designing and evaluating PIM-integration mechanisms.  Furthermore, based on these experiences, to produce a set of recommendations regarding the design of PIM interfaces, particularly those aimed at improving PIM integration.
\item \textit{To develop methodological recommendations for future design work} -- The final objective was to provide guidance for future design and evaluation in this area. It was envisaged that the experience of designing and evaluating the PIM-integration prototype would allow the development of such guidelines.

%  provide methodological guidance for future work, derived from the experience of pursuing this course of research.  In particular, \citet{Whittaker-rta:00} note the need for the identification of evaluation metrics.

\end{enumerate}

To achieve these aims a the research methodology was structured on a three-stage process of user-centred design: (1) requirements gathering, (2) design, and (3) evaluation:  

% (1) requirements gathering through an exploratory study, (2) design and implementation, and (3) a 
 
\begin{enumerate}

%%%%%%%%%%%
% APPROACH
%%%%%%%%%%%
%%%%%%%%%%%%%%
% FIRST STAGE
%%%%%%%%%%%%%%
% \item The cross-tool investigation of PIM behaviour across three PIM-tools -- files, email and bookmarks -- reported in \textbf{Chapter~\ref{chapter:exploratory_study}}
%, managed within the file system, email tool, and web browser respectively.
% DIANE: not too much detail here
% Tio provide evidence/description
% Requirements will be gathered through a cross-tool study to build cross-tool understanding of PIM, and so provide an empirical foundation for cross-tool design work aimed at improving PIM integration. The study methodology will consist of semi-structured interviews as used in previous empirical work in this area. The aim is to compare how individuals manage a range of types of personal information, to investigate the effectiveness of current forms of integration, and so as support the generation of empirically-grounded ideas of how further integration can be provided.
\item \textit{Requirements gathering} -- The exploratory study, reported in \textbf{Chapter~\ref{chapter:exploratory_study}}, investigated user behaviour across three PIM-tools -- files, email and bookmarks.  This enabled the comparison of behaviour between the tools, and the exploration of  potential routes for integration.  % \textbf{Chapter~\ref{chapter:exploratory_study}} details the study which compares PIM practices across 3 collections of personal information: personal files, email, and web bookmarks. 

%%%%%%%%%%%%%%%%%%
% SECOND STAGE
%%%%%%%%%%%%%%%%%%
% \item The design and implementation of a new PIM-integration mechanism based on the concept of folder-mirroring (reported in \textbf{Chapter~\ref{chapter:design}}),
\item \textit{Design and prototyping} -- \textbf{Chapter~\ref{chapter:design}} described the design of a PIM-integration mechanism which allows the user to share folder structures between different collections. The design was motivated by the observation of significant folder overlap for many participants in the exploratory study. Also, the design approach was deliberately \textit{incremental} to facilitate systematic evaluation, and cause minimum disruption to users.  % ,   % Findings from the exploratory study are used to motivate the design and implementation of a prototype PIM-integration mechanism.  % The prototype was proposed as a research vehicle to enable the investigation of general issues relating to PIM integration during a field study.
% The design was prototyped so as to embody specific design hypotheses that express intended improvements.
% The findings from the exploratory study will provide the grounding for the design of an interface providing enhanced PIM integration. 
% the design may be a new form of integration or modify an existing form
%% contrast with unification systems that unify and propose a new organizational paradigm. I just do one

%%%%%%%%%%%%%%%%%%
% THIRD STAGE
%%%%%%%%%%%%%%%%%%
%  \item A field study investigation of PIM, encompassing the longitudinal evaluation of the folder-mirroring mechanism (reported in \textbf{Chapter~\ref{chapter:main-study}}).   % \textit{Recap justification of the approach.}
\item \textit{Evaluation} -- \textbf{Chapter~\ref{chapter:main-study}} reported the third stage of the research, the field-study based evaluation of the designed prototype.  The evaluation facilitated the assessment of the specific design, as well as the development of guidelines for the wider PIM design genre.  The field study also enabled the investigation of long-term user behaviour such as changes in organizing strategies over time. 
% Note that the field study also facilitates the investigation of long-term aspects of PIM such as changes in strategy over time. Such longitudinal aspects of PIM have received little attention to date.
% Since no evaluations of this type have been carried out, an important part of this work is the development of appropriate methodology, both for evaluating PIM-tools in general, but also for evaluating means of integration.
% Finally the main study also allowed the investigation of appropriate methodology for evaluating PIM tools.
% Do these really go together? Maybe bring out in the discussion - rather than as an up-front aim!
% Finally the effectiveness of the design will be investigated through a field study-based evaluation.
%% to assess impact of the design intervention
% \item The iterative process of study, design and evaluation provided a platform for theory-building.

\end{enumerate}


% This approach both matched the research aims, and also allowed him to get a first-hand appreciation of real-world design issues.


%%%%%%%%%%%%%%%%%%%%%%%%%%%%%
%\section{Thesis tour/summary}
%%%%%%%%%%%%%%%%%%%%%%%%%%%%%
%
%Firstly, \textbf{Chapter~\ref{chapter:bg}} did X and Y.




