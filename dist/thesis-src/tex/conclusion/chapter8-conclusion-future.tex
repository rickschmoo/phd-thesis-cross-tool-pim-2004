
%%%%%%%%%%%%%%%%%%%%%%%%%%%%
% CHAPTER 8 - CONCLUSION - Future work/Research Vistas
%%%%%%%%%%%%%%%%%%%%%%%%%%%%
%%%%%%%%%%%%%%%%%%%%%%%%%%%%%%%%%%%%%%%%%%%%%%%%%%%%%%%%%%%%%%%%%%%%%%%%%%%%%%%%%%%%%%%%%%%%%
% Richard Boardman PhD Thesis: Improving Tool Support for Personal Information Management
%%%%%%%%%%%%%%%%%%%%%%%%%%%%%%%%%%%%%%%%%%%%%%%%%%%%%%%%%%%%%%%%%%%%%%%%%%%%%%%%%%%%%%%%%%%%%

\newpage
%%%%%%%%%%%%%%%%%%%%%%%%%%%%%%%%
%%%%%%%%%%%%%%%%%%%%%%%%%%%%
\section{Future Work}
\label{conclusion:future-work}
%%%%%%%%%%%%%%%%%%%%%%%%%%%%
%%%%%%%%%%%%%%%%%%%%%%%%%%%%%%%%
% \textit{Development of any of any speculative discussion points findings that can be explored in greater depth}
% MANY worthwhile directions/Avenues/vistas for future work based on limitations/issues/ideas include the following:
% uild on ideas raised/anticipated in discussion, suggest ideas/solutions to more tractable ones}. 
This section identifies avenues for future research.  \textbf{Section~\ref{conclusion:future-work-empirical}} outlines promising directions for empirical work, 
\textbf{Section~\ref{conclusion:future-work-design}} discusses prospects for design and evaluation, and 
\textbf{Section~\ref{conclusion:future-work-methodological}} highlights the need to develop the theoretical foundation on PIM.
% Firstly, \textbf{Section~\ref{conclusion:future-work-empirical}} outlines promising directions for empirical work.  Secondly, \textbf{Section~\ref{conclusion:future-work-design}} outlines routes for design and evaluation.  Finally,   \textbf{Section~\ref{conclusion:future-work-methodological}} examines the scope to improve theoretical foundations in this area -- including the need to understand the relationship between PIM and the production tasks which it supports.



%%%%%%%%%%%%%%%%%%%%%%%%%%%%%%%%%%%%%%%%%
\subsection{Further Studies to Improve Understanding of PIM}
\label{conclusion:future-work-empirical}
%%%%%%%%%%%%%%%%%%%%%%%%%%%%%%%%%%%%%%%%%

% As~\citet{Whittaker-rta:00} observe, there is a clear need for more studies of PIM. 
This study has opened up two new avenues for empirical investigation of PIM: (1) across multiple tools, and (2) over time. 
% Extension of work to different user group/tools/extended workspace.  
% OTHER FUTURE RESEARCH THAT NEEDS DOING
% \item This research has started to explore the distribution of PIM across multiple PIM-tools.  Promising routes for future work include more research to improve understanding (e.g. other tools)
% This research took steps towards widening the scope of PIM-based empirical study beyond single PIM-tools.
Since PIM is a complex, and highly idiosyncratic phenomenon, there is a clear need for follow-up studies along both routes.
% On the desktop, the cross-tool perspective could be extended .  % Finally, there is also scope for investigating information use across the digital and physical domains, to build up a unified picture of user needs.
% SET OUT IDEAL STUDY: 
% More long-term studies. Combo of methods? Track production tasks (explore relationship).  Investigate how multiple tools are used together in support of them.
Assuming no resource constraints, a ``wish list'' for a follow-up study is laid out as follows.  The study would be carried out over a full year, using a combination of data collection methods as in \textbf{Chapter~\ref{chapter:main-study}}.  
% More tools would be monitored as well: email, bookmarks, multiple areas of the file system, the calendar, and to-do manager as appropriate.
The number of monitored PIM-tools would be extended to encompass multiple areas of the file system, calendars and to-do lists as appropriate.  Improved instrumentation would be developed to make it possible to track specific acquisition, organization, and retrieval events in each PIM-tool.  Beyond the desktop, there is a need to investigate PIM behaviour across multiple devices, and into the physical domain (for instance, are users who are highly-organized in the digital domain, also highly organized in the physical domain?).  Attention would also be paid to a user's high-level production tasks -- the work and personal activities which provide the information needs to drive PIM.  One interesting angle would be to track a large project from start to finish and relate that to PIM activity, and needs for integration.  Since this 
rich data set would allow the researcher to build up a detailed picture of participants' lives, privacy issues would need to be carefully handled.


% \textbf{Section~\ref{conclusion:future-work}} discusses an ideal long-term study method without any such intervention.  

% In addition to tracking PIM behaviour, 

%\textit{FOCUS ON ORG: Some aspects of PIM were focused on, e.g. organization, over
%others, e.g. retrieval.  Future work: Use of objective studies: to focus on
%certain aspects of PIM, e.g. retrieval. Or use of objective studies combined
%with field trial to investigate retrieval. Use of more advanced logging tools.
%\textit{Use of objective studies combined with field trial to investigate
%retrieval. ONLY PART OF PROBLEM}}


%%%%%%%%%%%%%%%%%%%%%%%%%%
% DIFFERENT USER GROUPS
%%%%%%%%%%%%%%%%%%%%%%%%%%
% Add consideration of non-professional, social users? 
% Towards this end, we are currently extending both study and evaluation to users with less technical know-how. 
One general criticism that can be levelled at most PIM-related research to date, including this thesis, is that it has focused on professional users, people who manage information in a work context. The author calls for the field to devote increased attention to the needs of ``social'' users -- people who use their computers for personal rather than work activities.  It is envisaged that many of these users, especially those with less technical know-how, will have different needs and problems to those encountered in previous studies.  % Since there are millions of such users this is an area in need of attention.
% \textit{However NB: are advanced users are where social users will be one day? And therefore should research focus on them as pathfinders?}
% \textit{Choice of friends as users is not representative. Possible bias but suitable choice in this context to get access to data, also unpaid so bias limited?  Number of users but acceptable, compare to other studies. Future work: novice users, larger user groups.}






		
%%%%%%%%%%%%%%%%%%%%%%%%%%%%%%%%%%%%
\subsection{Designing More Effective PIM tools}
\label{conclusion:future-work-design}
%%%%%%%%%%%%%%%%%%%%%%%%%%%%%%%%%%%%
% \textit{THINK: this may be a (partial) repeat of parts of Chapter 5?}
% \textit{THINK: DIAGRAM to summarise future design space}
%%%%%%%%%%%%%%%%%%%%%%%%%%%%%%%%%%%
% Need for improved evaluation
%%%%%%%%%%%%%%%%%%%%%%%%%%%%%%%%%%%

The author reiterates the importance of evaluation within the design process.  Without evaluation it is not clear whether a new design is ``better or just different''~\citep{newman:95}   In particular, the author would encourage published evaluations of innovative PIM approaches such as \textit{attribute-centred organization} as proposed in systems such as \textit{Presto}~\citep{dourish:99a}.  Such insights would prevent the designers of similar systems, e.g. \textit{MS-Longhorn}, from repeating earlier mistakes which may have been made.


% A number of directions are identified for design-based research.
%\textit{More speculative: Different desktop hierarchies (e.g. tools)}
% Reimplement on different OS
Design is never-ending~\citep{Carroll:00}. A number of design improvements for WM were suggested by the evaluation participants. The most common request for follow-up design would be to make folder-mirroring more selective, e.g. ``mirror top-level folders from email to files only''.
%  We are also interested in how the scope of WorkspaceMirror can be widened beyond the desktop to encompass on-line tools such as web-based email and document management. 
% \textit{Web services dimension a la Microsoft's My Categories}
Two further design routes are raised by the author.  Firstly, the scope of folder-mirroring could be widened beyond the desktop to include other devices, and web-based PIM services.  Another potential of folder-mirroring is within the PIM-tool suites offered by websites such as ``Yahoo!''. %which allow the users to manage distinct collections of email, note archives, contact managers, photos and documents.  Many of these permit folder-based organization, but each must currently be organized separately.  % One challenge is the issue over where such a mechanism would be managed from, and consequent privacy concerns.
A second intriguing, design direction is inspired by the work of \citet{Chaffee:00}.  They investigated how user-defined categories could be used to structure sets of search results.  It may be revealing to investigate whether a user's mirrored folder structure could be used in this way, thus taking a step towards the unification of information management and information retrieval. %\textit{Extended Workspace - Integration with ontology-based retrieval engine~\cite{Chaffee:00}, on-line services}


% AND EVALUATE (EMPIRICAL SUGGESTIONS AS ABOVE): Repeat/extended study/evaluation of WorkspaceMirror at time of workspace set-up, or with non-professional users, new users. Different measures?
% One particular area of interest for the author would be to explore its use with different types of user.  
As in the area of empirical studies, the author encourages more design attention to be paid to ``non-technical'' users.  There is a need for the design of specialized interfaces for specific user groups, e.g. home users, older people. Currently such users have to use the same PIM-tools as technically-experienced users.  In terms of WM, the author hypothesises that such ``non-technical'' users may find the cross-tool consistency introduced by folder-mirroring helpful.  However, two study participants argued a contrasting point of view -- that novice users (both participants used ``My Mum'' as an example) would find it confusing.  An evaluation of WM with such users is called for to assess the utility of folder-mirroring.
% There is already some evidence of specialized PIM interfaces for home users with large easy-to-read interfaces for browsing music, games and videos.  
% In addition, several study participants suggested that WorkspaceMirror-like functionality would be most appropriate at the time of setting up a newly bought computer, or starting a new job. Unfortunately the execution of such an evaluation is beyond the scope of this work.
% ALSO SCOPE FOR MORE AMBITIOUS SCHEMES: This is after all just one form of integration.  But need to evaluate.
% Evaluation of more ambitious schemes such as Longhorn etc.
%Within the PIM design community in general
%A general design route for the future is to pay more attention to the need of ``social users''. 
% http://news.bbc.co.uk/1/hi/technology/3378519.stm.

% METHODS
%  METHODS: Crucially, improve evaluation.  Cross-tool extensions:  Integration/formalization in existing methods (e.g. Reference Task Agenda) -- to encompass cross-tool dimension. In other words, develop methodological support for PIM evaluation.
% FUTURE WORK:




\newpage
%%%%%%%%%%%%%%%%%%%%%%%%%%%%%%%%%%%%%%%
\subsection{Developing a Theoretical Foundation for PIM Research}
\label{conclusion:future-work-methodological}
%%%%%%%%%%%%%%%%%%%%%%%%%%%%%%%%%%%%%%%

% Improve theoretical groundwork.  
% DESCRIPTIVE TERMINOLOGY
% IMPROVED MODELS and validation of the models I have proposed
A final route for future work is to develop the theoretical foundation on PIM.  \textbf{Section~\ref{discussion:theoretical-framework}} has outlined the groundwork for this, extending previous theory to encompass the supporting, ongoing, cross-tool nature of PIM.  However, the author acknowledges that this work is preliminary, and only represents the first steps towards a more complete descriptive theory.
% SUPPRTING/PRODUCTION: investigation of relationship between production goals and supporting tasks.  Use of an activity framework such as Activity Theory or Distributed Cognition.
Theoretical frameworks such as Activity Theory~\citep{bn-at:95} and Distributed Cognition~\citep{dk:01} may be appropriate foundations on which to develop more extensive models of PIM.  One key area to accommodate in such a model is the relationship between \textit{production} and \textit{supporting} activities.  This break-down might also be usefully applied in theoretical descriptions of other supporting activities such as security practice.
% Improved theory may be a good basis for developing more effective evaluation criteria for discretionary, ongoing tasks such as PIM.

%The complex nature of PIM was highlighted in \textbf{Chapter~\ref{chapter:discussion}}. 
An alternative route to developing a ``monolithic'' theory may be to combine theories which operate at distinct analytical levels~\citep{barnard:00}, e.g. theories of short-term versus long-term behaviour.  Also, a theory of information retrieval could be accommodated within a more general PIM theoretical framework.  % Appropriate levels for component theories may include:
%\begin{itemize}
%\item Short-term
%\item 	
%\end{itemize}
% But care -- must be in a form that designers can make use of.
%% therefore can be contrasted with tools such as GOMS for constructing models of routine behaviour

%%%%%%%%%%%%%%%%%%%%%%%%%%%%%%%%%%%%%%%%%%%%%%%%%%%%%%%%%%%%%%%%%%%%%%%%%%%%%
% applying information foraging theory to personal information management
%%%%%%%%%%%%%%%%%%%%%%%%%%%%%%%%%%%%%%%%%%%%%%%%%%%%%%%%%%%%%%%%%%%%%%%%%%%%%
Another possibility would be to extend Information Foraging Theory~\citep{pirolli:99} to encompass PIM behaviour.  This is explained as follows using an evolutionary metaphor similar to that employed by Pirolli and Card.  Information Foraging Theory allows the modelling of information seeking, which in evolutionary terms equates to ``deciding what food [information] to take back to the cave'' (where the cave is one's personal information environment).  A model of PIM must be more encompassing, and should reflect what happens to the information after it has been foraged.  Is it stored in the cave or quickly discarded?  Is it looked at again?  Is it used to decorate the walls of the cave?  Is it arranged in neat piles or hurled into a heap?  Furthermore, Information Foraging Theory assumes that the user has no prior knowledge of the environment from which information is retrieved.  Therefore, in order to model the retrieval PIM sub-activity (``the foraging of information that has already been foraged''), some measure of prior familiarity with the information environment must be included.
%%  Do KFTF talk about this?
%% Talk about in terms of Economics of information. See below. 



The author emphasises that whatever form a theory of PIM takes, it must be accessible to designers, or it will be ignored outside of the research domain.
% The interworking of different theories at the levels of short-term one-off events, and long-term ongoing behaviour, information retrieval and high-level information needs can be envisaged.

% Another alternative would be to extend information foraging theory~\citep{pirolli:99}.  An evolutionary metaphor is employed to describe one possible route. If information foraging is decided what to take back to the cave based on its scent, then a PIM model must encapsulate ``taking back to the den to decorate walls with, its storage in the cave, and its later access''. Information foraging assumes that the user has no knowledge of the information environment. In order to model the foraging of information that has already been foraged clearly some measure of prior familiarity with that knowledge must be embedded in the model.



%%%%%%%%%%%%%%%%%%%%%%%%%%%%%%%%%%%%%%%%%%%%%%%%%%%%%%%%%%%%%%%%%%%%%%%%%%%%%%%%%%%%%%%%%%%%%%%%%%%%%%%%%%%%%%%%%
%%%%%%%%%%%%%%%%%%%%%%%%%%%%%%%%%%%%%%%%%%%%%%%%%%%%%%%%%%%%%%%%%%%%%%%%%%%%%%%%%%%%%%%%%%%%%%%%%%%%%%%%%%%%%%%%%
%%%%%%%%%%%%%%%%%%%%%%%%%%%%%%%%%%%%%%%%%%%%%%%%%%%%%%%%%%%%%%%%%%%%%%%%%%%%%%%%%%%%%%%%%%%%%%%%%%%%%%%%%%%%%%%%%
%%%%%%%%%%%%%%%%%%%%%%%%%%%%%%%%%%%%%%%%%%%%%%%%%%%%%%%%%%%%%%%%%%%%%%%%%%%%%%%%%%%%%%%%%%%%%%%%%%%%%%%%%%%%%%%%%

%%%%%%%%%%%%%%%%%%%%%%%%%%%%%%%%%%%%%%%%%%%%%%%%%%%%%%%%%%%%%%%
%\textit{This draft of Chapter 8 CONCLUSION was printed \today}
%%%%%%%%%%%%%%%%%%%%%%%%%%%%%%%%%%%%%%%%%%%%%%%%%%%%%%%%%%%%%%%
%%%%%%%%%%%%%%%%%%%%%%%%%%%%
% CHAPTER 8 - CONCLUSION - fin
%%%%%%%%%%%%%%%%%%%%%%%%%%%%
%%%%%%%%%%%%%%%%%%%%%%%%%%%%%%%%%%%%%%%%%%%%%%%%%%%%%%%%%%%%%%%%%%%%%%%%%%%%%%%%%%%%%%%%%%%%%
% Richard Boardman PhD Thesis: Improving Tool Support for Personal Information Management
%%%%%%%%%%%%%%%%%%%%%%%%%%%%%%%%%%%%%%%%%%%%%%%%%%%%%%%%%%%%%%%%%%%%%%%%%%%%%%%%%%%%%%%%%%%%%

%\newpage
%%%%%%%%%%%%%%%%%%%%%%%%%%%%%%%%%%%%%%%%%%%%%
%%%%%%%%%%%%%%%%%%%%%%%%%%%%%%%%%%%%%%%%%%%
%\section{Closing Discussion: buzz-phrase}
%\label{conclusion:closing}
%%%%%%%%%%%%%%%%%%%%%%%%%%%%%%%%%%%%%%%%%%%%
%%%%%%%%%%%%%%%%%%%%%%%%%%%%%%%%%%%%%%%%%%%%%%
% End on a triumphant albeit realistic high-note!

%What has PhD done towards easing poor epistemic state of PIM/HCI? Modest: hope thaht some useful progress
%
%
%Relate key high-level implications ... key message, pros and cons of integration
%
%Future: is PIM as an activity that will fade away a la bookmark management (when we have a google on the desktop)? High level of interest says that it isn't; Bring work up to date: Special interest group co-organized by the author, Signs of increased research interest? -- KFTF, Bergman etc.~\cite{Bergman:03}, RTA (back to basics), number of papers in CHI04, Signs of developer interest, both open-source (Chandler), and commercial (Longhorn)

%%%%%%%%%%%%%%%%%%%%%%%%%%%%%%%%%%%%%%%%%%%%%%%%%%%%%%%%%%%%%%%%%%%%%%%%%%%%%%%%%%%%%%%%%%%%%%%%%%%%%%%%%%%%%%%%%
%%%%%%%%%%%%%%%%%%%%%%%%%%%%%%%%%%%%%%%%%%%%%%%%%%%%%%%%%%%%%%%%%%%%%%%%%%%%%%%%%%%%%%%%%%%%%%%%%%%%%%%%%%%%%%%%%
%%%%%%%%%%%%%%%%%%%%%%%%%%%%%%%%%%%%%%%%%%%%%%%%%%%%%%%%%%%%%%%%%%%%%%%%%%%%%%%%%%%%%%%%%%%%%%%%%%%%%%%%%%%%%%%%%
%%%%%%%%%%%%%%%%%%%%%%%%%%%%%%%%%%%%%%%%%%%%%%%%%%%%%%%%%%%%%%%%%%%%%%%%%%%%%%%%%%%%%%%%%%%%%%%%%%%%%%%%%%%%%%%%%