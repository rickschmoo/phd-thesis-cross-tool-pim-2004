The difficulties that individuals encounter in Personal Information Management (PIM) have been widely discussed in the HCI literature, and there is much evidence that this fundamental activity is poorly supported by current interfaces. A number of innovative technologies have been proposed to enhance PIM support by consolidating the handling of different types of information resource (such as documents, email and web bookmarks) within a unified interface.  However these systems can be criticised for a lack of empirical foundation and evaluation. The lack of systematic research into this area means that the potential of PIM unification remains unconfirmed.
\\
\\
The research to be reported in this thesis seeks to investigate the potential of the unification approach for improving PIM support in the desktop context.  The work follows a design-based research methodology, and is grounded in an exploratory study of users' PIM practices across a range of tools including file managers, email tools, and web browsers.  Based on findings from this study, a theoretical framework is developed for analyzing PIM from a cross-tool perspective.  Methodological extensions are proposed to incorporate cross-tool issues in HCI design and research.
\\
\\
The remainder of the thesis puts the cross-tool methodology into practice. Informed by our empirical data, the design of WorkspaceMirror, a PIM-unification prototype, is described. The impact of WorkspaceMirror on PIM practice is explored through an extended fieldwork-based evaluation. The thesis concludes with a summary of implications for future PIM-related research and design, as well as general insights into the nature of PIM provided by our empirical work.


% ONE-SENTENCE SUMMARY: This thesis is aimed at improving interface support for 
% DEFINITION: Personal Information Management (PIM)

% My particular concern is with the problems that result from the distribution of PIM across a range of poorly-integrated tools such as file managers, email tools, and web browsers. As well as the overheads introduced by managing distinct collections of information, users must deal with other cross-tool issues including lack of integration and poor consistency between tools. 

% and develops the thesis that many of the most-pressing PIM-related issues that face computer users can only be addressed by an approach that bridges multiple tools.

% A research prototype WorkspaceMirror was designed, implemented and evaluated in the context of real user workspaces.  

% Design and evaluation -- invent the future. Prototype novel designs to alleviate cross-tool problems, and evaluate them in the context of real user workspaces. An additional contribution is the development of appropriate evaluation criteria to reflect the complex, ongoing nature of PIM;

% Methodology and recommendations. Recommend guidelines for developers carrying out PIM-related work, to enable them to better deal with cross-tool issues in key stages of HCI research and design.

% The research seeks to explore the potential of the design strategy of PIM unification: in which the management of multiple types of information is amalgamated within an integrated interface.  

% In contrast, this work is based on a cross-tool perspective, advocating that many of the most pressing PIM-related issues that face computer users can only be addressed by an approach that bridges multiple tools. 

% Through a series of user studies, develop insights into the distributed nature of PIM - and highlight the cross-tool problems encountered by both professional and social users. One key contribution is the development of a cross-tool model to explore PIM-related problems;

% more grounded foray into the design-space of PIM unification

