%%%%%%%%%%%%%%%%%%%%%%%%%%%%%%%%%%%%%%%%%%%%%%%%%%%%%%%%%%%%%%%%%%%%%%%%%%%%%%%%%%%%%%%%%
% ABSTRACT
%%%%%%%%%%%%%%%%%%%%%%%%%%%%%%%%%%%%%%%%%%%%%%%%%%%%%%%%%%%%%%%%%%%%%%%%%%%%%%%%%%%%%%%%%
% Richard Boardman PhD Thesis
% Improving Tool Support for Personal Information Management
%%%%%%%%%%%%%%%%%%%%%%%%%%%%%%%%%%%%%%%%%%%%%%%%%%%%%%%%%%%%%%%%%%%%%%%%%%%%%%%%%%%%%%%%%
% ONE-SENTENCE SUMMARY: This thesis is aimed at improving interface support for 
% DEFINITION: Personal Information Management (PIM)
%%%%%%%%%%%%%%%%%%%%%%%%%%%%%%%%%%%%%%%%%%%%%%%%%%%%%%%%%%%%%%%%%%%%%%%%%%%%%%%%%%%%%%%%%

Personal Information Management (PIM) describes the acquisition, organization, and retrieval of information by an individual computer user. Studies have shown that many users struggle to manage the volume and diversity of information that they accumulate.  Much design activity has been aimed at improving integration between different PIM tools, such as file and email managers. However, in terms of making a systematic contribution to HCI knowledge, much of this cross-tool design can be criticised for a lack of empirical grounding and evaluation.

The research described in this thesis employs a user-centered design methodology to deepen understanding of PIM, and in particular to provide guidance for PIM-integration design. The research is grounded in an exploratory study of file, email and bookmark management, which is differentiated from previous studies by its cross-tool nature. The study offers several contributions including observations of participants' multiple organizing strategies -- in both tool-specific and cross-tool contexts. Also, many participants had significant numbers of overlapping folders that appeared in multiple tool contexts. This finding informs the design of WorkspaceMirror, a novel PIM-integration prototype, which allows a user to mirror changes between their file, email and bookmark folders. 

The final stage of the research is a dual-purpose field study, aimed at (1) evaluating WorkspaceMirror, and (2) investigating PIM behaviour over time. Participant feedback indicates that mirroring is more appropriate for top-level folders, and illuminates a trade-off between organizational consistency and organizational flexibility. The study also reveals the incremental nature of changes in organizing strategy, and highlights the supporting nature of PIM.  These and other empirical findings are used to improve previous descriptive models of PIM behaviour. Furthermore, a number of design and methodological guidelines are developed. In particular, the author emphasizes the importance of assessing the strengths and weaknesses of PIM designs from both tool-specific and cross-tool perspectives. 
