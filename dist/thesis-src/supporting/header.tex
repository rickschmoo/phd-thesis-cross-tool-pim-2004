% header.tex
% A general header file for
% cool LaTeX documents.
% 
% ldk 3/2/98

\usepackage{epsfig}
% \usepackage{paralist}
% \usepackage{amsmath}
\usepackage{booktabs}
\setlength{\abovetopsep}{2ex}
%\usepackage{topcapt} 
%\usepackage{pstricks}
\usepackage{remreset}
\usepackage[hyperref]{ntheorem}
%THE NEXT 4 LINES SOLVES SOME PROBLEMS OF THE hyperred AND ntheorem PACKAGES INTERFERENCES:
\makeatletter
\providecommand*{\toclevel@defin}{0}
\providecommand*{\toclevel@propos}{0}
\makeatother

% %%%%%%%%%%%%
% New Commands
% %%%%%%%%%%%%
% HOw to do a side-label
% \marginlabel{\small{\textit{My defn: basis on control?}}}

\usepackage{url}
\newcommand{\urlBiBTeX}[1]{\url{#1}}

% Rick testing minitoc's
% \usepackage{minitoc}

%%%%%%%%%%%%%%%%%%%%%%%%%%%%%%%%
% for extrarowheight in tables
%%%%%%%%%%%%%%%%%%%%%%%%%%%%%%%%
\usepackage{array}

% Use aesthetic fonts for main text (pandora) and
% math symbols (euler).
\usepackage{ae}
\usepackage{aecompl}     
\usepackage{utopia}

% \usepackage{euler}
% wasysym provides some unusual symbols (e.g. `\checked' for a tick and `\phone'.
\usepackage{wasysym}

% amssymb provides extended math symbols
%\usepackage{amssymb}
% pifont contains ZapfDingBat characters.
%\usepackage{pifont}
% Use Type 1(?) encoding to ensure proper
% characters generated.
%\usepackage{t1enc}
%\usepackage[T1]{fontenc}
%\usepackage{textcomp}

% Nice wide page, please.
\usepackage{a4wide}

% Graphics almost certainly required.
\usepackage{graphics}

% fancyhdr for fancy page headers(!)
\usepackage{fancyhdr}
% some rotated tables required
\usepackage{rotating}
% http://www.manicai.net/comp/latex/latex_tricks.html
% \begin{sidewaystable}
% \end{sidewaystable}

% setspace allows variable line-spacing.
% Handy for thesis.
\usepackage{setspace}
%\usepackage{doublespace}

% makeidx permits the creation
% of an index file.
% \usepackage{makeidx}

%%%%%%%%%%%%%%%%%%%%%%%%%%%%%%%
% cite sanitises citations
%%%%%%%%%%%%%%%%%%%%%%%%%%%%%%%
% \usepackage{cite}
% \usepackage{chicago}
\usepackage[round]{natbib}

% booktabs allows the creation of 'publication quality' tables
\usepackage{booktabs}

% pstricks allows colours to be used
%\usepackage{color}
%\definecolor{rltbrightred}{rgb}{1,0,0}
%\definecolor{rltred}{rgb}{0.75,0,0}
%\definecolor{rltdarkred}{rgb}{0.5,0,0}
%\definecolor{webred}{rgb}{0.5,0.25,0}
%\definecolor{rltbrightgreen}{rgb}{0,0.75,0}

% \usepackage{pstricks}
% Beginning of the JLC definitions

%\newtheorem{definition}{Definition}
%\newtheorem{example}[definition]{Example}
%\newtheorem{application}[definition]{Application}
%\newtheorem{theorem}[definition]{Theorem}
%\newtheorem{exercise}[definition]{Exercise}
%\newtheorem{lemma}[definition]{Lemma}
%\newtheorem{notation}[definition]{Notation}
%\newtheorem{observation}[definition]{Observation}
%\newtheorem{proposition}[definition]{Proposition}
%\newtheorem{proposal}[definition]{Proposal}
%\newtheorem{remark}[definition]{Remark}
%\newtheorem{result}[definition]{Result}
%\newtheorem{conjecture}[definition]{Conjecture}
%\newtheorem{claim}[definition]{Claim}
%\newtheorem{assumption}[definition]{Assumption}
%\newtheorem{corollary}[definition]{Corollary}
%\makeatletter
%\renewcommand{\@begintheorem}[2]{ % not in italics
%  \trivlist\item[\hskip\labelsep{\bf #1\ #2}]}
%\renewcommand{\@opargbegintheorem}[3]{\trivlist
%  \item[\hskip \labelsep{\bf #1\ #2\ (#3)}]}
%\makeatother
%\newtheorem{proof}{Proof}
%\renewcommand{\theproof}{}  % no numbers on proofs
% End of the JLC definitions
